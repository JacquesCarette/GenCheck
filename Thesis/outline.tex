\documentclass[11pt]{amsart}
\usepackage{geometry}                % See geometry.pdf to learn the layout options. There are lots.
\geometry{letterpaper}                   % ... or a4paper or a5paper or ... 
%\geometry{landscape}                % Activate for for rotated page geometry
%\usepackage[parfill]{parskip}    % Activate to begin paragraphs with an empty line rather than an indent
\usepackage{graphicx}
\usepackage{amssymb}
\usepackage{epstopdf}
\DeclareGraphicsRule{.tif}{png}{.png}{`convert #1 `dirname #1`/`basename #1 .tif`.png}

\title{Gord's thesis outline}
\author{Gord Uszkay}
%\date{}                                           % Activate to display a given date or no date

\begin{document}
\maketitle


%Chapter 1
\section{Introduction}
\begin{enumerate}
\item testing is an experiment: predicted vs. observable behavior, 
sample, cost, supporting hypotheses, interpretation
\item test case selection, representative sample, sampling methods, stratification, test strategy,
heuristics, context dependent, subject of constant research
\item automated test case generation: syntactic sampling, 
lots of low dev cost, low value tests vs. expensive carefully selected tests
\item property based testing: algebraic specification, axioms are invariants, properties,
allows implicit testing of unobservable values, algebraic data types, well suited to autogen
\item plethora of PBT systems, tightly coupled execution and test strategies, 
popular concept, unnecessarily restrictive for test generation
\item modular PBT framework decouples test case generation, evaluation, reporting;
supports test program development and development / research of test strategies,
draws from sampling theory to provide standardized approaches
\item enumerations provide a solution for generating samples of alg.d.t.
with a solid theoretical background but also reasonable performance characteristics
\item contributions: test strategies, enumerative generators, modular framework / interfaces
\item structure of thesis
\end{enumerate}

%Chapter 2
\section{Property Based Testing}

Why partition and sample the domain of a property (in the context of PBT+purity)

\begin{description} 
\item[Testing] {
A test is a special form of experiment comparing 
predicted behaviors or characteristics to those observed in a sample of the population.
Informally describe a generic test and how sampling is the basis for the value of the evidence.

\begin{enumerate}
\item an hypothesis developed over a model of the population
\item (semi-formal) Def'n of test, testable, observation , test hypothesis
\item why is sampling needed (cost, infinite), cost, representative, diversity, evidence
\item {sampling methods: random, systematic, stratified}
\item {verdict reject or accept hypothesis, 
supporting hypotheses for test results to form argument,
confidence in verdict,
statistical sampling as ex. (keep short)}
\end{enumerate} }


\item[Software Testing]{
A software test is a special form of test with some distinct characteristics,
refine the definitions to be software specific.
Establish criterion for selecting test data based on cost vs. confidence.

\begin{enumerate}
\item behaviors, specification, implementation, observable, testable, test case, error detection
\item{selecting test data, representative test cases, partitioning, infinite domains, complexity} 
\item test costs: development vs. execution, repetition, automated test case generation
\item {standard supporting hypotheses for software testing, 
reporting, coverage analysis, test strategies, heuristics}
\end{enumerate} }

\item[Algebraic Specification] {
Property based testing is a specialization of software testing for 
implementing an algebraic specification.
This establishes the observable, testable condition (property),
and the population to be tested (domain of the property function).

\begin{enumerate}
\item{def'n algebraic specification as a theory presentation with sorts and equalities}
\item{ model is implementation, test hyp is implementation models spec }
\item{ property is computable model of axiom from specification }
\item{ observability problem of universal quantification (mention existential quantification) }
\item{ context: ref transparency + typed and why this matters (supporting hypotheses)}
\end{enumerate} }

\item[Property Based Test]{
A property based test evaluates a property, 
a computable implementation of one of the axioms of the specification.
Most axioms include universally quantified variables,
which are implemented as an input argument to the property function.
Test cases are a single substitution of values for the axiom
modeled as a single input argument to the property function,
and the verdict of the test is a monoidal computation over the test results.

\begin{enumerate}
\item {properties: implementation $\forall a.a \rightarrow B$
conditional properties $\forall a. Cond a => a \rightarrow B$ (like a type class)}
\item test cases, results: test case 1 value of domain, test verdict 1 value of property at 1 test case,
	test result is (case,verdict) pair, fail = (verdict of false), success (verdict of true);
\item tests as collection of cases to collection of results, test verdict (true / false)
\item a test program implements a single test to produce a verdict
\item test context - basic PBT supporting hypotheses, motivate further refinement of argument
\end{enumerate} }

\item[Existing PBT Systems]{
Critique existing systems, Highlight sampling methods and lack of distinct strategies, tight coupling.

\begin{description}
\item[Interface] {property function, simple default generators very popular, 
API with different levels of control good idea}
\item[Generators]{Generators built using combinators mirroring type constructor algebra works well; QC makes termination difficult to build into generators;}
\item[sampling]{Each package presents different testing strategy: choice of strategy open area of research,are important, why lock tester into single test strategy, FEAT breaks mold but does not allow compositional strategies}
\item[execution modes]{Tight coupling of sampling method with execution model prevents flexibility in selection methods}
\item[reporting]{Little control over reporting, run to first error not always appropriate, designed for error detection not coverage analysis}
\item[repeatable testing]{"Random" testing should not be random like in QC, needs to be repeatable, should have option of generating a new stream of seeds or keeping same one}
\end{description} }


\section{Sampling Algebraic Data Types}

\begin{description}
\item[Test Case Selection] { use cases, black box, white box, development vs. evaluation cost, syntactic domain analysis, brute force, easier to automate, generate many cases, each low value, complementary
}

\item[Algebraic Data Types]{
Detailed definitions of the type system.

\begin{enumerate}
\item base types, type constructors, data constructors, 
\item products, disjoint unions, �
\item polymorphism
\item �composition / substitution
\item �(mutual) recursion
\end{enumerate} }
\end{description}

\item[Sampling Algebraic Data Types] { finite type sampling methods, stratified sampling, complexity,� sampling infinite recursive types,�composition vs. substitution


\begin{description}
\item[Stratifying by Complexity (Rank)] substitution complexity; other measures
\item[Random] �probability distribution functions from enumeration
\item[Exhaustive and Uniform]
\item[Boundary / Extreme] depend on index function being representative of the
\item[Test Strategies]{ define generators by sampling method, allocate test cases to each rank / generator}
\end{description}
}

\item[Test Case Generators]{
Definition of a generator: select, (optionally) prioritize and instantiate test cases, may use sampling method.

\begin{enumerate}
\item{base generators for simple scalar types}
Specialized Generators for specializations of types (e.g. people's names, address strings)
Finite and infinite valued generators.  Duplicate values in generators.
\item{part generators for complex scalar types reps.}
Part Generators: ad hoc partitioning based on type specific issues related to testing (e.g. IEEE double)
\item{rank generators for recursive types}
\item{complexity ranked generators: maximal recursion depth, total compositions}
\item {applying sampling methods to algebraic data types; generator implements one sampling method}
\item{generator composition/substitution for type systems
ranked composition vs. unranked substitution;
ranking based on substitution set size (as a complexity measure)}
\end{enumerate} }

\item [Test Strategies]{
Defined test strategy as an implementation of a sampling strategy for algebraic data types.
\begin{enumerate}
\item {test strategies as implementations of sampling strategies,  
generator implements one sampling method}
\item {applying sampling methods to base types, parts (random, exhaustive, uniform, boundary)}
\item{stratification by rank; complexity measures for algebraic data types, complexity ranked generator}
\item{regularity hypothesis, small scope hypothesis as supporting hypotheses}
\item{sampling strategy defines test case allocation to each part / rank and sampling method;
test strategy retrieves allocated number of test cases from each generator.}
\item test cases from non-generated sources
\item PBT test context schema template: regularity hypothesis, small scope + exhaustive sampling, other
\end{enumerate} }

\item[Enumerative Generators] {
Demonstrate that enumerative generators are an effective method for sampling 
algebraic types for property based testing

I think this should really be a separate chapter,
with more attention to the construction and sampling of enumerations.

\begin{enumerate}
\item{ formal definition: count, index and selection functions}
\item{base, partitioned and ranked enumerations (by complexity)}
\item{enumerative generator combines enumeration, data constructor and sampling method applied to $[1..n]$}
\item{enumerative test strategy defined by rank, sampling method, number of test cases}
\item{defining sampling methods and test case allocations for different ranks / parts of domain}
\item{combinatorial constructions, ogf existence proof of enumerations}
\item{iterative algorithms for construction of enumerations}
\end{enumerate}
}
\item[Test Context] {formal argument for correctness, exhaustive testing, refining context, regularity hypothesis, uniformity hypothesis, small scope hypothesis, heuristics, Komolgrov complexity(?), pbt hypothesis schema}

\end{description}

%%Chapter
\section{Property Based Testing Systems}

Formally define the structure of a property based test system,
allowing for module specialization.
Expand on the definitions of the property based test to include operational considerations.
Establish the requirements for a modular framework with independent 
testing strategies, execution, and reporting modules.

\begin{description}
\item[Evaluation Function] {
\begin{description}
\item[property] $p : a \rightarrow Bool$, conditional property $p_C : a \rightarrow Bool \lor \bot$
\item[test case]  (a, metadata)�(optional)
\item[evaluation function] property(a) -> (verdict, evaluation info (optional))
\item[test result] (test case, verdict, evaluation info (optional))
\item[verdict] = Pass, Fail, Invalid, Non-terminating, Not evaluated
\end{description} 
}
\item[Test Program] {
\begin{description}
\item[TestSuite] the collection of test cases
\item[ResultSet] the collection of test results
\item[test program]�schedules and applies evaluation function over TestSuite to produce ResultSet(test result)
\item[test�monoid] test cases, results and verdicts as monoidal computations
\item[report] presents verdict and summary or  details of ResultSet to user, coverage analysis
\end{description}
}
\item[Test System] {
Define the user interface, formally define test strategies, generators, generator dictionaries, 
component libraries, evaluation modes, report constructors, specialization

\begin{enumerate}
\item {Generator defined by type, sampling method, (finite or infinite);�
base generator(type,sample): C(a), ranked generator(type,sample): Nat -> C(a)}
\item {Test strategy: $base = \{(sampling method, no.cases)\}, ranked = \{(sampling method, rank, no.cases)\}$ }
\item { �generator dictionary: sample -> generator(type,sample) (base or ranked)
	� �standard sampling strategies = $\{Exhaust, Random, Uniform n, Boundary\}$, could add others}
\item {Test Suite = $Build (Test strategy (sample, (rank), cases), sample -> generator(type, sample))$�}
\end{enumerate} 
}

\item[Requirements] {
For modular framework for PBT. �

\begin{description}

\item[modular infrastructure] {
Decouple generation, evaluation and reporting functions, allowing for specialization.

Default test is constructed by creating evaluation function for property and 
applying to datum of each test case to get result set.

Test suite containers present interface that allows traversal, partitions, labels, merging,
computing verdict using monoidal computation structure, store / retrieve for repeatable testing.

Test strategies capture allocation of test cases across partitions of property domain,
test cases created by generators,
allow generic sampling strategies over standardized generators,
specialized versions of generators for subtypes or optimization.

specializations of scheduler may access metadata in test cases, 
specializations of evaluation functions may record system information during evaluation
specializations of reporting may use meta data or evaluation information.
}
\item[Properties, Cases, Evaluation function, Results] {
Test cases consist of input to property (datum) and optionally metadata such as rank, 
datum must be accessible without knowledge of metadata;

test results consist of test case, verdict and optionally 
information about property evaluation such as CPU time; 
verdict must be accessible without knowledge of evaluation information;
Success, Fail, Invalid, Time Out, Not Evaluated

}
\item[Tests, Suites, Verdicts, Reports] {
Test applies evaluation function to each case in a test suite.
Containers need efficient default processing for pure and sequential test when 
test case / result containers are the same;
need to be able to merge containers of cases / results to grow test, allow multiple results, etc.;
for large test sets, want to group and sort test cases / results by metadata (e.g. rank) for scheduling, reporting;
need to be able to save and retrieve test data sets for repeating tests.

Scheduling is the sequencing of the evaluations on the available computing resources and 
any required reorganization or marshaling to get results into requested organization;
e.g. pure test is map across test cases; 
sequential test is a fold / traversal in priority order until first failure; 
testing on�multiple systems�requires scheduling / evaluation / marshaling of result;
early termination based on evaluation result.

Reporting always includes verdict, may also include summary or details of test results / cases.
specializations can use metadata, optional evaluation information to enhance reports.

}
\end{description} 
}
\end{description}

%%%Chapter
%\section{Constructing Combinatorial Enumerations}
%Present the theory behind enumerations using combinatorial constructions;
%explain relevance of ogf proof of existence,
%provide iterative construction algorithms.
%Finish by mentioning non-regular recursive structures,
%non-algebraic data types, and species
%as additional justification for using this approach.
%
%\begin{description}
%
%\item[Combinatorial Constructions]{
% (of Salvy et al)
%
%\begin{enumerate}
%\item{definitions of polynomial constructions, systems of constructions, recursive systems}
%\item{admissible systems, epsilon-freeness}
%\item{ogf exists so count is available}
%\item{number of atoms is good complexity measure, defines cardinality of substitutions into a structure}
%\end{enumerate} }
%
%\item[Algorithm for Enumerating Algebraic Data Types] {
%complexity measure is number of atoms (constants are atoms, not 0 weight $\epsilon$ constructions),
%
%\begin{enumerate}
%\item counting function 
%\item index construction
%\item selection function
%\item non-regular recursive types cannot be enumerated this way
%\item enumerating over compositions (variable complexity)
%\item enumerating over substitutions (fixed complexity)
%\item generator composition vs. enumerating compositions
%\end{enumerate} }
%
%
%\end{description} 
%
%% Chapter
\section{GenCheck}

\begin{description}

\item[Architecture] {
\begin{enumerate}
\item Test strategy and test case generation. 
\item Scheduling and evaluation of test cases.
\item Reporting of verdict, results.
\end{enumerate}
}

\item[Evaluation Interface] {

Basic test interfaces for test cases, results, evaluation functions and
containers for multiple cases / results, verdicts.

\begin{description}
\item[property]
\item[datum] test case -> a
\item[verdict] test result -> verdict
\item[eval] property -> test case -> test result
\item[test] eval -> test suite -> test result
\end{description}

}

\item[Test Interface, Labelled Partitions] {

Scheduling (concurrent, sequential), termination conditions, parallel testing

Labelled partitions are a container to hold test suite / results;
a container of labelled containers of test cases / results.
Functor and traversable (foldable) (C(a) -> [a]) for concurrent and sequential testing;
Monoidal (empty container no label, merge preserving labels and duplicates) 
with an injection function (singleton container with label) for merge and verdict.

Note that labelled partitions are also containers, so can have multiple levels of labels (categories) for cases or results

Test System:�evalp ->�LP(Lbl, test cases) -> LP'(Lbl', test results) schedules and evaluates property at each test case, stores result in same or different container;
}
\item[Reporting] {

Report : LP(Lbl, test results) -> IO (LP(test results)); 

verdict only, failures only, complete report

use labels to categorize reports, verdicts for each label, groups of labels, etc.

reports / verdicts can be combined because monoidal containers

reports specialized using meta data or evaluation information from other specialized components.
}

\item[Test Strategies, Generators, Enumerations] {

\begin{description}
\item[base generator] { }
\item[ranked generator]
\item[enumeration]
\item[enumerative generator]
\item[sampling strategy]{$base = \{(sampling method, no. cases)\}, ranked = \{(sampling method, rank, no.cases)\}$ }
\item[Test Suite]{$ Build (Test strategy (sample, (rank), cases), sample \rightarrow generator_(type, sample))�$}
\item[Custom, Library Generators]{ generator(type,sample);
standard library generators:$ sample \rightarrow generator(type, sample)$�(generator dictionary);
can only use class system dictionary to allocate if single form of sampling is used
can provide supplemental generator dictionary and override with custom generators}
\item[Generator Dictionary] {$sample \rightarrow generator(type,sample)$ (base or ranked);
standard sampling strategies = $\{Exhaust, Random, Uniform n, Boundary\}$, could add others;
enumerative generator builders : $enumeration_type \rightarrow sample \rightarrow generator_(type,sample)$;
standardizing sampling schemes and enumerations allow type independent test strategies}
\end{description} }

\item[Implementation Details] {
 (Haskell specific issues, nits)

\begin{description}
\item[Lists] { are very efficient structures in Haskell, monoids and functors, so are good containers. �Generators for list types should not be built using the enumerative modules, they are custom built.}
\item[Exhaustive Generators]{should not use enumerative generators for exhaustive samples, 
this is inefficient.}
\end{description} 
}
%
\item[SimpleCheck]{ strategy as a sample test system built atop GC. }

\item[QC, SC, FEAT Compatibility]{can all be implemented atop GC too.}
\end{description} 

\end{document}  