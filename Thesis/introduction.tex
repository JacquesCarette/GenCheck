%\chapter{ Introduction }

Testing is, at the very least, a necessary evil 
in the development of software and software intensive systems.
Frequently treated as an art as opposed to a science,
software testing might appear to be the ugly cousin to theorem proving,
lacking the elegance of mathematical proofs and
offering only to manage the risk of failure, not eliminate it.
Theorem proving, however, exists only the realm of mathematical abstractions,
and cannot address the real world in which the actual program will be functioning,
a world fraught with unexpected and pernicious challenges.
If there is to be any meaningful claim that
the software solves the \emph{real world} problem it was invented to address,
it will be necessary to invoke an installed instance of the program 
on an actual computing device under the conditions that it is expected to operate:
in other words it must be tested.
Perhaps the best summary of the relationship between theorem proving and testing 
is provided by Donald Knuth (\cite{KnuthProvedNotTested1977}):

\begin{quote}
Beware of bugs in the above code; I have only proved it correct, not tried it.
\end{quote}

\noindent
Since testing is a necessity, 
it behooves practitioners who appreciate the elegance of theorem proving
to attempt to apply the same rigor and discipline to the testing process.

Driven by time and resource limitations
in the face of the staggering complexity of modern software systems,
software practitioners naturally look to software tools
to assist, manage and automate the testing process.
One area of particular interest to programmers is 
software that automatically generates and evaluates test cases.
The hope is that these tools would generate large numbers of test cases
with little effort on the part of a test developer, 
and lead to both quicker and more thorough testing.

Property based testing is a powerful form of unit testing that is
particularly well suited to automatic test case generation for functional programming languages
such as Haskell (\cite{Haskell98}, \cite{Haskell2010}),
Erlang (\cite{armstrong1993Erlang}, \url{http://erlang.org/} ) and OCaml(\url{http://ocaml.org/})
(\cite{Fink1997}, \cite{Papadakis2011}, \cite{OSullivan2008}).
A number of software packages supporting
automated test case generation for property based testing have been released,
providing different approaches to selecting and generating test cases.
While test developers benefit from having
these different test case selection strategies,
it does raise a number of questions:
\begin{itemize}
\item Which test selection strategy is best for this test?
\item Are there other test selection strategies to consider?
\item Will the selected test cases provide a good trade off between uncovering errors and the cost of running the test?
\item Perhaps most importantly,
is it really necessary to have a separate software package
for each test selection strategy?
\end{itemize}

The \GC project was initiated to 
support the research and development of property based test case generation
by establishing a theoretical basis for test case selection and
providing a single, extensible, software package that 
would support a wide variety of such strategies.
It explores the relationship between test case generation,
formal \emph{sampling theory} (\cite{Stuart1968}, \cite{Cochran1977}),
the complexity of test terms,
and combinatorial enumerations as the basis for sampling Haskell data types.
A modular framework with independent interfaces for 
evaluation,  automated test case generation and reporting
allows these strategies to be used for 
projects of varying nature and complexity.
The intent of this work is not to provide a single canonical test case selection algorithm,
but instead to act as the foundation for research into this area
and provide a pragmatic tool for automatic test case generation for \pbt.

\section{Property Based Testing}
A property is computable boolean valued function,
implemented using the module(s) to be tested,
implementing an equivalence relationship  from the module's specification.
A \emph{property based test}(\pbt)%
\footnote{Not to be confused with \emph{property testing}, \cite{Ron00}.}
consists of evaluating the property function
over a subset of its allowable inputs (test cases).
If the property does not evaluate to true for one or more of the allowable inputs,
then the module is incorrect.
Typically a specification will consist of a number of such equivalence relationships,
with each relationship being implemented as a distinct property and evaluated independently.

%Another question that is raised is the relationship between
%the generation of test cases and their evaluation for the test verdict.
%\QC, \SC and the other published \pbt packages 
%(chapter \ref{pbtsystems} will present a detailed critique)
%sequentially generate and evaluate test cases,
%terminating when an error (or a fixed number of errors) is found.
%While a single developer unit testing a simple module may be satisfied with this approach,
%a more sophisticated software project might 
%execute test cases in parallel over multiple cores and computers,
%and require a full report of all of the test results for analysis.
%The nature and complexity of a software projects differ widely,
%so it is unlikely that any one approach for evaluating and reporting will suit all needs.
%Could a test system provide flexibility in 
%the scheduling, evaluation and reporting of test results
%while offering the same kinds of test case generation provided in these packages?

The first step in defining a property based test is
to identify the properties of the specification.
As an example, consider an algebraic specification for polymorphic lists
equipped with constructors (nil and cons) and a concatenation operation
that is defined by the equations of the specification.

\begin{figure*}
%\fbox {
\begin{minipage}[t]{.3\linewidth}
Operators:
\begin{align*}
& \nil : \specL_{\alpha} \\
& \cons : \alpha \sprod \specL_{\alpha} \mapsto \specL_{\alpha} \\
& \concat : (\specL_{\alpha} \sprod \specL_{\alpha}) \mapsto \specL_{\alpha} \\
& =_{\specL_{\alpha}} : (\specL_{\alpha} \sprod \specL_{\alpha}) \mapsto \boolean \\
& (\text{assumes} =_{\alpha} \text{defined}).
\end{align*}
\end{minipage}
\begin{minipage}[t]{.2\linewidth}
\end{minipage}
\begin{minipage}[t]{.5\linewidth}
Equations:
\begin{align*}
&\forall x \in \specL_{\alpha}.x \concat \nil =_{\specL_{\alpha}} x\\
&\forall x \in \specL_{\alpha}.\nil \concat x =_{\specL_{\alpha}} x\\
&\forall x, y, z \in \specL_{\alpha}. 
    ((x \concat y) \concat z) =_{\specL_{\alpha}} ( x \concat (y \concat z)) \\
&\forall a \in \alpha. \forall x,y \in \specL_{\alpha}. 
    (\cons (a, x \concat y)) =_{\specL_{\alpha}} (\cons (a, x) \concat y)
\end{align*}
\end{minipage}
%} % end xbox
\caption[Algebraic specification of a list]
{An algebraic specification of a list with concatenation.}
\label{intro_spec_ex}
\end{figure*}

An implementation of this specification in Haskell
would consist of a module with a type constructor for the new list structure,
and a binary operator over that type to implement concatenation.
An example of such an implementation is provided below%
\footnote{This concatenation operator will conflict with that of the Prelude when compiled,
so either the prelude version must be excluded or the operator renamed}.
A traditional approach to unit testing the concatenate operator would have
the test developer to choose pairs of lists
and compare the result of using the concatenation operator in the module
to the expected result.
This would require the test developer to:
\begin{itemize}
\item select pairs of lists representing a variety of likely operands
\item determine the concatenation of each pair \emph{without} using the module being tested,
\item encode the test pairs and resulting concatenations correctly in a test program,
\item compare the concatenation produced by the module to the prediction
and report any discrepancies as errors
\end{itemize}

\begin{figure*}
\begin{lstlisting}
module List (List(), (++)) where

data List a = Nil | Cons a (List a)
  deriving Show

(++) :: List a -> List a -> List a
(++) (xs) (Nil) = xs
(++) (Nil) ys = ys
(++) (Cons x xs) ys = Cons x (xs ++ ys)
\end{lstlisting}
\caption{Implementation of polymorphic list concatenation}
\label{intro_spec_impl}
\end{figure*}

Consider two such test cases,
with manually derived results:
\begin{align*}
\cons(1, \cons (2, \nil)) \concat \nil & = \cons(1, \cons (2, \nil))\\
\cons(1, \cons (2, \nil)) \concat \cons(3, \cons (4, \nil)) & = \cons(1, \cons (2, \cons(3, \cons (4, \nil))))
\end{align*}
\noindent
These inputs and results must then be coded into a test program
and compared to the actual result for the module:
\begin{lstlisting}
import List
x = Cons 1 (Cons 2 Nil)
y = Cons 3 (Cons 4 Nil)
xy = Cons 1 (Cons 2 (Cons 3 (Cons 4 Nil)))

test :: Boolean
test =   ( x ++ Nil ) == x  && ( Nil ++ x ) == x && ( x ++ y ) == xy
\end{lstlisting}

\noindent
Determining the predicted result of an operation
without using the implementation to be tested,
and manually encoding the inputs and expected results,
is both arduous and error prone.

\pbt generalizes this approach by encoding the specified relationships
and testing for internal consistency,
without explicitly knowing the results of an operation.
Here the list specification equations are encoded as Haskell properties,
with the specification variables as parameters:

\begin{figure*}
\begin{lstlisting}
import List

prop_leftunit, prop_rightunit :: List a -> Bool
prop_leftunit    xs = (Nil ++ xs) == xs
prop_rightunit xs = (xs ++ Nil) == xs

prop_assoc_list :: (List a, List a, List a) -> Bool
prop_assoc_list (xs, ys, zs) = (xs ++ (ys ++ zs)) = ((x ++ ys) ++ zs)

prop_assoc_cons :: (a, List a, List a) -> Bool
prop_assoc_cons (x, ys, zs) = (Cons x (ys ++ zs)) = ((Cons x ys) ++ zs)
\end{lstlisting}
\caption{Haskell properties of list concatenation.}
\label{intro_haskprop_ex}
\end{figure*}

\noindent
A test case for these properties is then an input of the appropriate type,
and a test is the conjugate of the Boolean result of each of these property function evaluations
over a collection of such test cases.
Removing the need to supply an explicit result for a test case
makes test case generation much simpler as only the arguments to the property function must be generated.

\section {Automated Test Case Generation for \pbt}

The promise of software tools providing comprehensive, inexpensive testing
by automatically generating test cases is a compelling one.
A tool for automating property based tests has several functions :

\begin{enumerate}
\item generate (or otherwise acquire) test cases
\item evaluate the property over test cases until all are successful or at least one fails
\item return the verdict of the test and (optionally) report the results of the test cases evaluations
\end{enumerate}

\subsection{Test Case Selection and Generation}
In a traditional software testing situation, 
\emph{domain knowledge} experts select 
a relatively small number of test cases that are representative of
the \emph{standard uses} of the program,
and possibly some that are expected to pose additional risks.
Identifying classes of system behaviors and representative test cases from those classes
is a \emph{heuristic} test strategy that creates an efficient test as each case is meaningful.
It is also time consuming, requires significant expertise on the part of the test developers,
and is error prone as unexpected behaviors are unlikely to be addressed.

The definition of the property function
allows test case generation to be based on
the syntactic definition of the argument types,
instead of the semantics of the specification
or the implementation of the module to be tested.
The automated test system partitions the property argument values
and select representatives from each partition,
based solely on the data type of the argument.
This approach discards any a priori knowledge of the system to be tested,
but compensates by greatly reducing the cost of generating more test cases.
These techniques are particularly useful in finding
errors relating to the implementation or an ``irrational'' use of the system,
which might not occur to a human test developer,
and for testing complicated modules which might defy domain expertise.
While one could argue in favor of either approach,
it is very reasonable to decide that they are instead complementary
and should both be supported in any practical test framework
(although this work will focus almost exclusivley on the generation of test cases using syntactic strategies).

Automated test generators are able to use syntactic strategies
generate very large numbers of test cases inexpensively,
reducing the time to develop a test,
but at the expense of evaluation costs.
This creates the essential compromise of testing:
there must be enough test cases to create confidence,
but the cost of the test must be reasonable given the value of the system.
Many more test cases may be needed, however,
to achieve the same level of confidence as
a test suite developed by a human expert
who will include fewer redundant test cases based on their experience.
The strategies that define these classes and
the number and complexity of test cases
are a critical characteristic of a test system.

There are a number of existing \pbt packages available to the Haskell community
using this syntactic approach to test case generation but 
offering different approaches to test case selection.
For the most part, the focus of these packages is on how to select test cases
from recursive algebraic data types,
so the complexity of the input values is often referred to as the ``size''.
\begin{itemize}
\item \QC generates a stream of random test cases,
initially favoring simpler test cases (smaller data structures) and then allowing more complex ones;
\item \SC exhaustively generates all structures up to a set complexity (``size''),
selecting scalar values from a small range;
\item \EC (\cite{EasyCheck2008}) generates a mixture of
simple and complex data structures to ensure small test suites contain a variety of complexity.
\item \FEAT(\cite{Duregard2012}) includes both of the above but
also allows a selection over uniform intervals (discussed further in \ref{}).
\end{itemize}
\noindent
Each of these packages partitions the property's domain (possible input values)
and then selects a representative set of the cases to evaluate.
The interface to each of these packages is very similar,
allowing testers to move back and forth between them,
to use the different test strategies as needed.
This then poses a dilema for the tester:
which test selection strategies will provide
the highest degree of confidence
relative to the investment in generating and running the tests?
How are these different packages and in particular their test selection criteria 
to be compared and judged?

\subsection{Sampling Strategies}
In the broader scientific community,
this kind of partitioning and selection of elements from a population,
without human expertise guiding or biasing the selection process,
are considered forms of \emph{sampling}.
Sampling theory (\cite{Stuart1968}, \cite{Cochran1977}) is 
the branch of statistics that studies
the selection of a representative sample of elements 
that can be tested to draw conclusions about a large population.

There are two concepts in particular that are valuable in discussing test case generation:

\begin{enumerate}
\item A sampling methodology is an unblased approach to selecting members of a group
\item Stratification is the partitioning of a population into similarly behaving groups or classes,
each of which should be sampled independently, and possibly using distinct sampling methodologies
\end{enumerate}

\noindent
Stratifying a population and then selecting different sampling methodologies
is a useful technique for managing the pragmatic aspects of testing hypotheses,
and in particular is useful when different parts of a population are much more costly to sample.
For example, consider a module managing height balanced trees,
and a \pbt of the ``node insertion'' capabilities.
It would be desirable to exhaustively test the properties 
over inserting a node into all ``small'' trees (say up to depth of 4),
and then test a small sample of larger trees, perhaps with a few being extremely large.
This suggests that the population of height binary balanced trees be partitioned into small and large,
with all small trees (exhaustive sampling) and a random sample of large trees
being included in the test suite.

Statistics and probability theory are used to provide 
guidance for selection strategies and 
estimates of the bias and accuracy of tests over selected samples.
While many of the assumptions regularly used in statistics 
are not applicable to software testing,
we propose that the standard concepts behind sampling methods 
provide a starting point for structuring and evaluating test case selection strategies.
In particular, we would propose that test systems should provide
stratified test strategies to allow different sampling strategies for different classes of test cases,
the discussion of which will be a significant focus of this document.

The choice of test case selection strategies is 
a subjective but important area of ongoing research.
Existing \pbt packages mostly generate test cases based on a single sampling method
\footnote{\FEAT being the exception as discussed in the next section},
and since they tightly couple test case generation, evaluation and reporting,
they cannot easily be adapted to other sampling methods.
Factors such as stopping conditions,
sequencing of the test cases,
and the presentation of the test result details
are all influenced by the sampling choice.
This both hampers the comparison of the different methodolgies
and is an impediment to the practitioner wishing to use different test strategies
when validating a module.
We would propose that a proper evaluation of sampling strategies can only be carried out
using a platform that supports stratification of the test domain,
and a variety of sampling methodologies including allowing new methodologies to be defined,
and allows different sampling methodologies within different parts of the domain.

\section{Enumerative Generators}

Sampling methods generally require some sort of organization of the values being sampled,
in this case is the allowable inputs for the property being tested.
Ideally there would be a single interface / approach that would support 
all of the standard sampling methods over
any of the data types likely to be arguments for the properties.
For Haskell, this includes recursively defined algebraic data types,
which are particularly challenging to sample as the represent
infinitely many values of arbitrary complexity.

One of the major contributions of this thesis is
demonstrating that \emph{combinatorial enumerations} (\cite{FlSa95}) provide 
a canonical organization suitable for standard sampling methods
of the most common Haskell types including recursive and mutually recursive algebraic data types..
An enumeration  of a data type provides 
the number of values represented
and a selection function based on an index over those values.
Sampling methods can then be generalized to apply to the index,
creating a set of arguments to be mapped through the selection function to obtain the selected values.
A family of enumerations, further indexed (or ranked) by the complexity of the terms,
allows stratified sampling of (mutually) recursive algebraic types%
\footnote{Only regular recursive types are supported, as discussed in \ref{chp:testgen}}.
A generalized sampling method is combined with
the combinatorial enumeration of a data type to create an
\emph{enumerative sampling generator} for automating test case generation
(chapter \ref{chp:enumgen}).

The development of ranked enumerations of algebraic data types
is based on the \emph{combinatorial structures} of \cite{FlajoletSedgewick2009},
providing a strong theoretical basis for the existence of these enumerations.
Additional works in this area (\cite{FlSaZi91}, \cite{FlajoletZC94}, \cite{FlSa95})
direct the details of their implementation
and support the efficiency of the algorithms.
This theory also extends beyond algebraic data types to
sets (bags), cycles, restricted size constructions,
suggesting future research in testing a broader family of data structures
than can currently be specified in languages such as Haskell.
The development of these enumerative generators was a primary motivation
behind the development of the \GC framework.

\section{The \GC Framework}

The proliferation of \QC like \pbt frameworks
suggest that while automatically generating test cases
based on a data type declaration is a popular approach,
no one test case selection strategy is going to be entirely satisfactory.
In addition,
the level of concurrency in test evaluation,
test termination conditions and the desired reporting
are all dependent on the project context
\footnote{but independent of selection method}.
Given that the complexity of software development projects
may vary from simple, single data type, single module algebras to
complex multi-programmer, multiple platform, multi-user systems,
it is hard to imagine a single approach to evaluating and reporting test results
could be optimal in all situations in which
automated test case generation for \pbt will be used.
The goal of this research is provide  a modular, flexible platform for developing \pbt programs,
providing standard interfaces and a library of components for 
test case generation, property evaluation and reporting results,
instead of cloning and modifying testing software for each new situation.

The \GC framework (chapter \ref{chap:GCimplement}) provides
the interfaces and components to build \pbt programs for Haskell modules,
incorporating a facility for both custom and mechanically derived 
automated test case generation using a wide range of sampling strategies.
A \GC test program will include:

\begin{itemize}
\item the properties to be tested,
\item one or more test case generators,
\item a test case selection strategy,
\item a test case evaluator,
\item a reporting function,
\item and the implementation modules to be tested.
\end{itemize}

\noindent The architecture decouples test case generation, 
evaluation and reporting,
and the library provides interchangeable implementations of these components.
SimpleCheck, a package similar to \QC, 
provides an example of a \GC test program,
providing several interfaces with varying  levels of control of the test process.
The interfaces between components are abstract and documented to provide 
support for developing new components,
hopefully forming the base for an extensible, open source environment to 
support ongoing research into property based testing methodologies and tools.

\section{ Contributions }

The contributions documented in this thesis are:

\begin{enumerate}

\item {Provide a critical review of \pbt systems,
identifying the valuable, shared functionality and 
the limitations of each of these systems. 
}
\item { Identify the importance of sampling methods and 
test case generation strategies based on term complexity as
a unifying concept behind property based testing systems
(chapter \ref{chp:propertytesting}). 
}
\item{Applying the 
and the importance of these hypotheses in evaluating test strategies. 
}
\item{ Providing detailed definitions for
term complexity based sampling generators for algebraic data types,
generator composition and substitution for mutually recursive and polymorphic types
}
\item { Implement an approach to developing
type independent test strategies
using standardized sampling generators. 
}
\item{ Introduce the use of enumerative sampling generators for
use in automated test case generation.
This includes adapting the background theory of combinatorial structures
for the algebraic data types commonly found in functional programming languages.
Adapt these algorithms for constructing enumerations,
based type definitions, for use in Haskell programs. 
}
\item{ Develop requirements for property-based testing software
with automated test case generation for Haskell,
including the decoupling of test case generation, evaluation and reporting components. 
}
\item{ Concrete realization of the \GC test framework in Haskell (and published to Hackage).
}
\end{enumerate}


\section{Thesis Layout}

This thesis motivates and defines \GC,
a modular framework for the development of property based test programs for Haskell modules,
using standard interfaces and components for automated test case generation,
test evaluation and reporting.
It is based on the application of sampling theory to algebraic specifications,
a critique of existing testing packages,
and the theory of combinatorial structures (\cite{FlajoletSedgewick2009}).
It provides a common model for complexity based test case generation for algebraic data types,
a library of generator combinators,
and a module to construct \emph{enumerative} generators
for the most common Haskell types 
for a variety of sampling methods.
It also provides a model for constructing type independent test strategies
using standard sampling methods
over the default enumerative generators or customized replacements.
This framework allows the creation of property based tests and testing systems;
an example called SimpleCheck is provided,
as well as a \GC compatible version of \QC.
The results of this thesis should also apply to 
other similar functional languages such as the ML family,
and to a lesser extent other types of programming languages.

The thesis is laid out in chapters as follows:

Chapter 1: introduction.

Chapter 2: defines property based testing and 
considers the interpretation of the test results
relative to the test selection strategy applied.

Chapter 3: a critique of some 
existing property based testing systems
incorporating automated test case generation.

Chapter 4: how formal sampling methods can be applied to algebraic data types,
and how these can be combined to create a wide variety of test case selection strategies.
Stratification by term complexity is presented as
an appropriate way to sample recursive types,
and several measures of complexity are presented and compared.
Post generation substitution is introduced as 
a way to independently generate the shape and elements of data structures
with different sampling methods.

Chapter 5: introduces combinatorial enumerations and
enumerative sampling generators as
a standardized way to sample and generate the most common Haskell data types.
This includes a theoretical basis for these enumerations,
a proof of their existence for systems of (mutually) recursive algebraic data types,
and an efficient means for iteratively constructing enumerations
which can be mechanically derived from the definition of an algebraic data type.

Chapter 6: provides a formal definition of a property based testing system,
and the requirements for developing a modular framework,
a set of interfaces and a library of standardized components
to support their development.

Chapter 7: describes the \GC framework for the development of property based testing systems,
including the interfaces, sample evaluation and reporting components,
an implementation of enumerative generators,
and some example property based testing systems
built on this framework that demonstrate how 
the \QC family of packages are superseded by this framework.

