%\section{Haskell Types}

This section is a brief overview of the Haskell types%
\footnote{It is assumed that the reader is familiar with the Haskell type system,
or will turn to \cite{Haskell2010} or other Haskell references as required.} 
that can be generated in the \GC framework
and the other \pbt systems discussed in Chapter \ref{pbtsystems}.
These represent a broad range of commonly used Haskell types,
but certain categories of types have been excluded from this work,
and in particular it does not address any extensions to the Haskell type system
that are not part of the core language definition.

Haskell types consist of a small number of built-in \emph{base types} (e.g. Char, Int),
functions,  and algebraic types.
Type constructors for lists and tuples are included in the language definition.
Standard algebraic data types are provided through the Haskell Prelude and library modules,
but new algebraic data types may be defined in a module,
including recursive and mutually recursive types.


\subsection{Type Constructors}
Recall that algebraic data types in Haskell are defined as (\cite{Haskell2010}):

\begin{align*}
\text{data} T\ a_1\ ...\ a_n =\ & C_1\ t_{1,1}\ ...\ t_{1,n_1}\ \\
  \vert\ & C_2\ t_{2,1}\ ...\ t_{2,n_2}\  \\
  \vert\ & ...\ \\
  \vert\ & C_k\ t_{k,1}\ ...\ t_{k,n_k}
\end{align*}
\noindent
where 
\begin{itemize}
\item $a_i$ are defined types or type (valued) variables,
\item $t_{i,j}$ are either defined type constants or one of the $a_i$ arguments to the type constructor,
\item each $C_j$ is a \emph{data constructor},
\item and $ \vert $ represents the (disjoint) union of the constructors.
\end{itemize}
\noindent
Data constructors are reified as functions

\begin{equation*}
C_j :: \ t_{j,1} \ra ...\ \ra\ t_{j,n} \maps T\ a_1\ ...\ a_n
\end{equation*}

\noindent
The results of each data constructor can be considered a labelled product of its arguments,
and $C_j$ must be a unique label within the module.

\noindent
There are two other ways that new types can be defined:
\begin{description}
\item[type synonym] provides a new name for an existing type,
but is purely syntactic and is interchangeable%
\footnote{with minor exceptions that are not significant to generation} with it;
\item [newtype] creates a new type by labelling instances of an existing type
using a single data constructor,
meaning the new values are distinguishable from the originals.
\end{description}
\noindent
In both cases, there is a one to one correspondence between the values of the original and new type.
For test case generation,
|newtype| constructors can be considered equivalent to
a new algebraic data type constructor with a single data constructor,
and type synonyms can be ignored as a syntactic convenience.

\subsubsection{Built-in Type Constructors}
Haskell provides built-in type constructors for lists and n-tuples.
Lists and tuples could be generated using
their algebraic equivalents,
but it is more convenient to treat them as special cases.

\subsubsection{Functions}
While it is possible to generate a limited class of functions,
they have a fundamentally different semantic interpretation
which poses special challenges for test case generation
and are out of the scope of this thesis.
\gordon{is this sufficient?}

\subsection{Polymorphism}

An Haskell type is polymorphic if
it represents a type constructor ranging over one or more type variables.
Modules may (and usually do) contain expressions and functions that have a polymorphic type,
and even be compiled in this state,
but these must subsequently be refined to grounded types before execution.

Properties, as a kind of Haskell function, 
can be defined over polymorphic types.
For example, typically the involution property for a list reverse function
would be defined over lists with an arbitrary element type.
The test cases generated for this property, however,
must each have a concrete element type
(and this choice must satisfy the class constraints).
The selection of which types to test is 
is part of defining the \emph{sampling frame} (section \ref{sec:sampling_theory}),
and the impact of restricting the test to these types
must be addressed as a supporting hypotheses
of the test context (section \ref{sub:context}).

\subsection{Other types}\label{nonregrecurs}
One of the important assumptions made to this point 
about the algebraic data types being generated
is that every value has a finite representation.
It is possible in Haskell to define and use types with 
no finite representation,
the simplest example of which is the stream:
\begin{lstlisting}
data Stream a = Stream a (Stream a)
\end{lstlisting}
\noindent
It is not possible to instantiate such a structure using proper values,
but Haskell's lazy evaluation makes it possible to use these types.
These kinds of coalgebraic data types are beyond the scope of this thesis.

Another assumption that has been made to this point is that
recursively defined structures are \emph{regularly} recursive,
meaning that the recursion occurs over the type itself.
Nested, or non-uniform recursive data type is a parameterized type that
includes a recursive reference to the type but with modified or different type arguments.
For example, a nested pairing type defines the type of arbitrarily deep pairings:

\begin{lstlisting}

data NestedP a = Node a | Nest (NestedP (a, a))

\end{lstlisting}

While we anticipate it is possible to handle these, we leave them to future work.
The interested reader may consult
\cite{BlampiedThesis} and \cite{Joha07}.


