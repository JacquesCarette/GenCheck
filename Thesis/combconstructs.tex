%\section{Combinatorial Constructions}

In order to use an enumerative generator for a \pbt,
an enumeration must exist, be computable,
and for infinite types be associated with a ranking function.
Combinatorial constructions  (\cite{FlajoletSedgewick2009}) provide the tools needed.
In particular, combinatorial structures have well defined \emph{generating functions},
providing a calculus for computing the number of structures 
with a specified number of atoms,
the combinatorial complexity (section \ref{complexgen}) of those structures.
(\cite{FlSaZi91}) shows that these counts can be obtained efficiently,
and the selector functions can be constructed using the same techniques,
as implemented in Maple's |combstruct| module 
(\cite{FlSa95}).

Combinatorial constructions describes a class of structures significantly larger than those that can be implemented in Haskell,
including data structures constructed with sets, multi sets, sequences and cycles.
A subset of the class of combinatorial constructions,
called the \emph{polynomial} constructions,
provides a useful model of the types that concern us here.
The combinatorial structures provide a model of type constructors,
with the atoms representing the type arguments.
The complexity measure is the size of the structure underlying the term,
namely the number of atoms that must be substituted with values%
\footnote{This is a particularly useful measure of complexity 
for substitution generators, as the number of atoms is the number of elements in the structure.}.
This model is developed throughout this section,
restating the relevant findings of
(\cite{FlSaZi89b}, \cite{FlajoletZC94}, \cite{FlSa95} and \cite{FlajoletSedgewick2009}),
restricted to the combinatorial structures required to model to Haskell's data types.

It also includes the combinatorial construct of rank $0$, called $\epsilon$.
Certain systems of combinatorial equations that incorporate $\epsilon$
result in infinitely many constructs of a fixed rank,
so are not well-defined with respect to creating enumerations
and must be avoided for our purposes.
Systems that avoid this issue, described by the authors as ``epsilon-freeness'',
are well aligned with the requirement that each rank of structures 
must include a finite number of instances in order to effectively sample them
for automated test case generation.
Demonstrating that nullary constructors (or constants) in a type
(e.g. the |Nil| construct in a list) should be modelled as an atom of rank $1$,
instead of a rank $0$ $\epsilon$ was an important outcome of this modelling.

\subsection{Systems of Polynomial Constructions}
A \emph{class of combinatorial constructions} contains elements that 
can be partitioned by their size:

\begin{df}[A Combinatorial Class]
is a set $\set{S}$ where for $x \in \set{S}$ a size function $\phi(x)$ is defined, 
such that :
\begin{enumerate}
\item $\phi(x) \ge 0$ 
\item $\forall n . \setcard{ \{ x \mid \phi(x)=n \} } \text{ is finite} $.
\end{enumerate}
\end{df}

\noindent
Note that these are the same restrictions as those required for a ranked partition.

The class of polynomial constructions can be 
defined with a specification language consisting of 
the primitives $0, 1, + \text{ and } x$.
This algebra is defined here
(with $\spec{F}, \spec{G}$ as variables ranging over constructions):

%\begin{df}[Admissible Polynomial Construction Algebra]
\begin{tabular}[b]{ll}
Symbol(s) &  Definition  \\
$\specO$ & the unique empty construction, $ \spec{A} \siso \spec{A} \splus \spec0 \siso \spec0 \splus \spec{A} $\\
$\specI$ & Neutral, the unique construction on 0 atoms, $ \spec{A} \siso \spec{A} \sprod \specI \siso \specI \sprod \spec{A} $ \\
$\specX, \specX_{a}$ & Atom \\
$\spec{F} \splus \spec{G}$ & Disjoint union $(\circ \sprod \spec{F}) \union (\bullet \sprod \spec{G})$ \\
$\spec{F} \sprod \spec{G}$ & $\{ (f, g) \lvert f \in \spec{F}, g \in \spec{G} \}$ 
\end{tabular}
%\end{df}

\noindent
Note there may be multiple instances of the neutral structure  with different names
but all of these instances are isomorphic constructions.
They play a similar role in recursive construction definition 
as that of nullary data constructors (constants) in Haskell,
providing a terminal end point without content%
\footnote{Keep in mind that, despite this similarity of purpose,
we recommend modeling Haskell nullary constructors
as atoms over a sort with a single value to give it a size of 1 instead of
the $0$ size of the neutral element
to ensure that all constructions have finitely many structures for any given size.}

Atoms are the unique singleton-set construction over an underlying finite sort of elements.
A multi-sorted construction will have one atom type per sort,
distinctly labelled (here by convention with a subscript).

Note that n order to guarantee disjointness,
$+$ labels it's constructions with a distinct symbols
(shown as  $\circ$ and $\bullet$ in the grammar above) 
that differentiate what might otherwise be identical elements;
sums in Haskell are similarly labelled.

The $\spec0$ construction is the isomorphic group of constructions which permit no members.
For our purposes, this can be considered an error condition, and will not be discussed further.

\subsubsection{Size of Constructions}
The size of a construction is the number of atoms 
and is computed as expected:

\begin{df}[Construction Size]
The size of a construction $h \in \spec{H}$, denoted $\setcard{h}$, is given by
\begin{align}
 \setcard{h \in \specI } & = 0 \\
\setcard{h \in \specX}& = 1 \\
\setcard{h \in (\spec{F} \splus \spec{G})} & =
      \begin{cases} \setcard{f} & h = (\circ \sprod f \in \spec{F})\\
                              \setcard{g} & h = (\bullet \sprod g \in \spec{G}) 
       \end{cases}\\
\setcard{(f,g) \in (\spec{F} \sprod \spec{G})} & = \setcard{f} + \setcard{g} 
\end{align}
\end{df}

\noindent
The set of constructions $\specF$ of a fixed size $n$ is denoted $\specn{F}{n}$.

\subsubsection{Structures from Admissible Constructions}
A \emph{combinatorial structure} is the result of replacing each atom in a construction
with a particular element of the sort it represents.

\begin{df}[Combinatorial Structure]
For a single sorted construction $\specF$, 
the collection of structures that can be created by substituting 
elements of a set $\setA$ for the atoms of $\specF$ is denoted $\specset{F}{A}$.
This definition extends to multi-sorted constructions,
with each atom being replaced by an element of the appropriate sort. 
\end{df}

Note that in order to have a well defined deterministic substitution operation,
it is necessary to have a total order over the atoms in the construction
and order the elements of each sort that are to be substituted.
The substitution ordering may be accomplished by matching explicit labels or
implicitly through the order inherent in the substitution operation%
\footnote{This requirement for ordered substitutions is realized explicitly in 
the substitution method of the Structure class (Substitution module) of the \GC implementation
(and Structure2, Structure3 for multi-sorted types).}.
For admissible polynomial constructions a canonical ordering exists,
labeling the atoms ``depth-first'' as if the construction were a tree,
using the inherent left to right order of the operands in the product operator term.
Although any ordering will suffice, it will be assumed that 
atoms in a polynomial construction are labelled in this order unless otherwise stated.

\subsubsection{ Composition of Admissible Constructions }

\begin{df}[Composition]
The composition of admissible constructs $\specG$ into $\specF$ is formally defined as (\cite{FlajoletSedgewick2009}, pg. 87)

$$\specF \scomp \specG = \sum_{k \ge 0} \specn{F}{k} \sprod ( \spec{G}^k ) $$
where $\specG^k = \specG \sprod \specG \sprod ... \sprod \specG$ is the $k$-fold product.
\end{df}

\noindent
The \emph{size} of a $f_k \scomp (g_1, g_2, ..., g_k) \in \specF \scomp \specG$ construct is therefore given by

$$\forall f_k \in \specn{F}{k}.\forall g_i \in \specG.\setcard{f_k \scomp (g_1, g_2, ..., g_k)} = \sum_{i = 1}^{k} \setcard{g_i}$$

where $f_k$ and the $ g_i$ are $\specF, \specG$ structures respectively.


\noindent
The class $\specF \scomp \specG$ can be intuitively interpreted as
structures where the atoms of a $\specn{F}{k}$ construction are substituted for  
the $k$ elements of an \emph{ordered} collection of $\specG$ constructions.
The size of such a structure is given by
the sum of the sizes of the $\specG$ structures.
Only the $\specG$ structures have atoms in a composed construction,
the $\specF$ atoms having been replaced with $\specG$ structures.

An alternate but equivalient definition of composition
is given by considering the construction to be defined over a finite set of elements $\setA$
that will be substituted for the atoms in the construction to create a structure.

\begin{df}[Composition via Partitions]\label{def:subbypart}
Let $\pi$ be a partition of a finite set $\setA$,
and denote
For constructions $\specF, \specG$
composition over a finite set $\setA$ is may also be defined as

$$(\specF \scomp \specG)[\setA] = \sum_{\pi of \setA} \specF[\pi] \sprod \prod_{p \in \pi} \specG[p]$$

where $\specG[p]$ is a $\specG$ structure over the part $p \text{ of } \pi$;
each $(\specF \scomp \specG)[p]$ structure is of size $\setcard{p}$.
\end{df}

\noindent
These two definitions of the construction are equivalent (\cite{FlajoletSedgewick2009}),
but this latter definition emphasizes that the composition
is a new class of constructions.

In order for this operation to be well defined under either definition,
there must be some unique order, or \emph{labeling},
of the $\specG$ populated atoms in the $\specF$ class such that
the $i^{th}\ \specG$ structure is consistently assigned to the $i{th}$ atom of $\specF$
(\cite{FlajoletSedgewick2009}, pg. 87).

\subsubsection{Systems of Polynomial Constructions }

Multiple combinatorial classes can be defined with respect to each other
by defining a system of construction classes.
This is the class of combinatorial constructions required for modelling Haskell types,
including mutually recursive type definitions.

\begin{df}[System of Polynomial Constructions]

A system (or specification) of mutually dependent $r$ combinatorial classes,
by convention called the \emph{specification} of  $\specF^{(1)}$,
is a collection of $r$ class specifications

\begin{array}[b]{rcl}
\specF^{(1)} & = & \Phi_{1}(\specF^{(1)}, ..., \specF^{(r)}) \\
\specF^{(2)} & = & \Phi_{2}(\specF^{(1)}, ..., \specF^{(r)}) \\
 & ... & \\
\specF^{(r)} & = & \Phi_{r}(\specF^{(1)}, ..., \specF^{(r)})
\end{array}

\noindent
where each $\Phi_{k}$ is a term built from the $\specF$ classes and the polynomial algebra.

\end{df}

\begin{df}[Polynomial Class]
A polynomial combinatorial class is the solution to a polynomial system%
\footnote{This is also known as the class of context-free specifications (\cite{FlajoletSedgewick2009}).}.
\end{df}

\noindent
For example, a planar tree with an arbitrary number of ordered branches
is a polynomial class described by

\begin{align*}
\spec{PT} & = \specX + (\spec{L} \scomp \spec{PT})\\
\spec{L} & = \specX \sprod \specL
\end{align*}

\subsubsection{Triangular Systems}
Simple systems of constructions do not exhibit any cyclic dependencies in their specification.
Any such system may be reorganized so that 
its specification forms a \emph{triangular} system,
where each class is dependent only on earlier specifications.
Triangular systems  may be reduced to a single construction,
and are always well-defined (\cite{FlajoletSedgewick2009}):

\begin{df}[Triangular Systems]
A system of polynomial construction classes is \emph{triangular}
if each $\specF^{(k)} (k \ge 2)$ is dependent only on the definitions of $\specF^{(i)}, 1 \le i < (k-1)$
with $\specF^{(1)}$ being defined strictly in the primitives and operators of the algebra.
\end{df}
\noindent

\begin{prop}[Triangular Systems Reduction] \label{prop:iterativeOGF}
A specification given by an triangular system can be expressed as a single polynomial.
\end{prop}
\begin{proof}
The proof for this follows simply from the substitution of the definitions
from $\specF^{1}$ into $\specF^{2}$, $\specF^{2}$ into $\specF^{3}$, etc.
\end{proof}

\begin{corollary}[Triangular Systems are Admissible]
If $(\specF^{1}, ..., \specF^{n})$ is specified as an triangular system,
then each $\specF^{i}$ is  \emph{admissible}.
\end{corollary}

\subsubsection{General Systems}
Combinatorial classes with cyclic dependencies (such as the planar tree defintion)
cannot be expressed as a triangular system.

\begin{df}[Recursive System of Polynomial Combinatorial Classes]
A system of construction classes is \emph{general} if it is not triangular.
\end{df}
\noindent
General systems require more sophisticated mathematics to prove they are admissible:
this will be demonstrated through the use of generating functions.

\subsection{Ordinary Generating Functions of Constructions}

\begin{df}[OGF of a Combinatorial Construction]
The ordinary generating function of a construction $\specG$ is a formal power series of the form 
$$ G(x) = \ogf{g}{n} $$
where the coefficients ($g_n$) are the number of G structures over a set of $n$ elements
(this is also called the ``counting sequence'' in \cite{FlajoletSedgewick2009}).
\end{df}
\noindent
If a construction has a well-defined OGF,
then there are a finite number of constructions for any given size,
and therefore the system is admissible.

The ordinary generating functions of polynomial constructs can be computed recursively,
where  $F(x), G(x)$ are the OGFs of constructions $\specF$ and $\specG$
(see \cite{FlajoletSedgewick2009}  for the proofs).

\begin{tabular}[b]{lll}\label{countseq}
Symbol(s)& OGF  & Counting Sequence  \\
$\specO$ & $0$ & $0$ \\
$\specI$ & $1$ & $A_{0} = 1, A_{n} = 0 \text{ for } n > 0$ \\
$\specX$ & $x$ & $A_{1} = 1, A_{n} = 0  \text{ for }  n \neq 1$ \\
$\spec{F} \splus \spec{G}$ & $ (F + G)(x) = \displaystyle{\ogf{(a+b)}{n}}$ &  $(F + G)_{n} = F_{n} + G_{n} $ \\
$\spec{F} \sprod \spec{G}$ & 
                   $F(x) \sprod G(x) = \displaystyle{\sum_{n=0}^{\infty} (\sum_{k=0}^{n} a_k b_{n-k}) x^n}$ &  
                   $ (F \sprod G)_{n} = \sum\limits_{k=0}^{n} F_{k} G_{n-k} $ \\
$\spec{F} \scomp \spec{G}$  & $(F \scomp G)(x) = F(G(x))$ & $(F \scomp G)_n = (F_{n}) ( G(x)^n )$
\end{tabular}

\noindent 
Note that for the composition to be well defined,
either the construction is triangular, so the generating function has finitely many terms,
or $\specn{G}{0} = 0$,
because the composition of formal power series 
is only defined where $G(0) = 0$ 
(\cite{WilfGeneratingFunctionology}, \cite{FlajoletSedgewick2009} appendix A.5, pg 731).

\subsubsection{Simple Substitution}

One important special case of composition
is the substitution of values from a base set for the atoms of a construction.
This kind of substitution can be modeled using a singleton set as an intermediating structure, i.e..

\begin{equation}
(\specF \scomp \specX) [\setA]  =  \sum_{k \ge 0} \specn{F}{k} \sprod (\specX [\setA])^k
\end{equation}

The OGF of the substitution will depend on
whether the same value can be substituted multiple times into the new construct,
and if there are other restrictions on the selection process.
The number of structures of a size $k$ populated from a set of cardinality $n$ will be 
the number of constructions of size $k$ multiplied by 
the number of subsets of size $k$ of the underlying set allowed for the substitution.
The number of subsets of a set is a standard problem in combinatorics.
If repetition is allowed, the OGF for the substitution of $k$ values from 
a type with cardinality $n$ is 

$$ \setcard{(\specF \scomp \specX)_{k}[\setA]} = F_k (\setcard{\setA})^k $$

However, if each element from the type can be drawn only once for each substitution,
ie. without repetition, the ogf is

\begin{equation}
F_n[A](x) = \sum_{k \ge 0} f_{k} \binom{n}{k}
\end{equation}

The number of populated structures of a given size
is important for property based testing
as it represents the size of the population to be sampled for the test.

\subsubsection{Existence of OGF for Admissible Polynomial Classes}
Any system with an OGF that can be computed from the specification
has a finite number of constructions for each size of structure,
and therefore is admissible (well-defined).

\begin{df}[Admissible Constructions]

Let $\spec{A} = \Phi( \spec{B}^1, \spec{B}^2, ..., \spec{B}^m )$ be a specification over admissible constructions,
then the $m-ary$ construction $\Phi$ is \emph{admissible} if the ordinary generating function of \spec{A}
is only dependent on the OGFs of $\spec{B}^i, 1 \le i \le m$.

\end{df}
\noindent

\begin{theorem}[OGF of Admissible Polynomial Constructible Class]\label{OGFPolynomialClass}
The ordinary generating function of an admissible polynomial specification $\specA$ is an \emph{algebraic function}, 
i.e. there exists a bivariate polynomial such that

$$P(x, y) \in \field{C}[x,y] . P(x, F(x)) = 0$$

and that this can be solved using algebraic elimination to produce 
a polynomial equation $F(X)$ for its generating function.

\end{theorem}
\begin{proof}
\cite{FlajoletSedgewick2009} (proposition 1.17, pg.80)%
\footnote{They actually provide the proposition over a wider class of combinatorial constructions,
but the theorem and proof are not dependent on the availability of these other operators.}.
\end{proof}

\noindent
It should be noted that in \cite{FlajoletSedgewick2009} 
it is unclear that only well defined structures
should be included in this class,
but that this is clearly required in the earlier work \cite{FlSaZi91}.

