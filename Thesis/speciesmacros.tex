\newcommand{\AJ}{Andr\'{e} Joyal}
% spec marks the argument as a species, and some common species get their own commands
\newcommand{\spec}[1]{\ensuremath{\mathcal{#1}}}
% specn is for species over fixed size sets, e.g. \specn{E;2} is a set (bag) of two elements.
\newcommand{\specn}[2]{\ensuremath{\spec{#1}_{#2}}}
\newcommand{\specnum}[1]{\ensuremath{\mathbf{#1}}}

\newcommand{\specO}{\specnum{0}}   % using O for 0, the non-existent structure
\newcommand{\specI}{\specnum{1}}  % using I for 1, the structure over the empty set
\newcommand{\specX}{\ensuremath{\mathcal{X}}}  % singletons
\newcommand{\specE}{\ensuremath{\mathcal{E}}}  % set (bag)
\newcommand{\specC}{\spec{C}}  % cycle
\newcommand{\specL}{\spec{L}}  % list
\newcommand{\specA}{\spec{A}}  % trees - arbres
\newcommand{\specF}{\spec{F}}  % generic species labelled F
\newcommand{\specG}{\spec{G}}  % generic species labelled G
\newcommand{\specZ}{\spec{Z}}  % singleton, but in Flajolet et al syntax

\newcommand{\field}[1]{\ensuremath{\mathbb{#1}}}
\newcommand{\nat}{\field{N}}
\newcommand{\integer}{\field{Z}}
\newcommand{\boolean}{\field{B}}
\newcommand{\True}{T}
\newcommand{\False}{F}
\newcommand{\reals}{\field{R}}
\newcommand{\absval}[1]{\ensuremath{\lvert#1\rvert}}

% sfrak gives a species with a different font
\newcommand{\specroot}[1]{\ensuremath{\mathfrak{#1}}}
\newcommand{\specfrak}[1]{\ensuremath{\mathfrak{#1}}}
\newcommand{\specname}[1]{\ensuremath{\mathit{#1}}}    % names longer than one letter

\newcommand{\specwgt}[2]{\ensuremath{\mathcal{#1}_{\field{#2}}}}

% sets of elements have no structure, the species is applied over a set 
\newcommand{\set}[1]{\ensuremath{\mathit{#1}}}
\newcommand{\setA}{\set{A}}
\newcommand{\setcard}[1]{\ensuremath{\lvert#1\rvert}}
\newcommand{\setcardn}[2]{\ensuremath{\lvert#1\rvert}_{#2}}
\newcommand{\specset}[2]{\ensuremath{\spec{#1}[\set{#2}]}}
\newcommand{\specsetpar}[2]{\ensuremath{(\spec{#1})[\set{#2}]}}
\newcommand{\specnset}[3]{\ensuremath{\spec{#1}_{#2}[\set{#3}]}}
\newcommand{\specfrakset}[2]{\ensuremath{\mathfrak{#1}[\set{#2}]}}
\newcommand{\sunion}[2]{\ensuremath{\bigcup_{#1}^{#2}}}

% Species operators
\newcommand{\splus}{\ensuremath{+}}           % disjoint union
\newcommand{\sprod}{\ensuremath{\cdot}}       % product
\newcommand{\scomp}{\ensuremath{\circ}}       % composition (substitution)
\newcommand{\sdiff}{'}      % differentiation
\newcommand{\spt}{\ensuremath{^{\bullet}}}    % pointing
\newcommand{\scart}{\ensuremath{\times}}      % Cartesian product
\newcommand{\sfunccomp}{\ensuremath{\Box}}
\newcommand{\speccard}[2]{\ensuremath{\lvert#1[\set{#2}]\rvert}}
\newcommand{\specring}{\ensuremath{\mathfrak{Spe}}}
\newcommand{\virtring}{\ensuremath{\mathfrak{Virt}}}

%L species operators
\newcommand{\sprodL}{\ensuremath{\sprod_{O}}}
\newcommand{\sinteg}{\ensuremath{\int}}

% Species equivalence operators
\newcommand{\sequiv}{\ensuremath{=}}		% equivalence
\newcommand{\sequip}{\ensuremath{\equiv}}    % equipotence  
\newcommand{\siso}{\ensuremath{\simeq}}	% isomorphic

%Species transformations or meta operators
%\newcommand{\toset}{\ensuremath{\mu}}  % forgetful functor, maps F to E for all F, \mu is arbitrary decision

%Generating series
\newcommand{\specgs}[1]{\spec{#1}(x)}   % formal power series
\newcommand{\factorial}{\!}
\newcommand{\coeff}[2]{\ensuremath{\[ {#1}^{#2} \]}}  % coefficient  of formal power series

\newcommand{\semiring}[2]{#1\llbracket #2 \rrbracket}

%Categories
\newcommand{\categoryfnt}[1]{\ensuremath{\mathbb{#1}}}
\newcommand{\catname}[1]{\categoryfnt{#1}}
\newcommand{\categorywrd}[1]{\ensuremath{\mathfrak{#1}}}
\newcommand{\catB}{\categoryfnt{B}}
\newcommand{\catE}{\categoryfnt{E}}
\newcommand{\catL}{\categoryfnt{L}}
\newcommand{\ordset}[1]{\ensuremath{\mathit{#1}}}
\newcommand{\lspecset}[2]{\specset{#1}{#2}}
\newcommand{\symgroup}{\ensuremath{\mathfrak{G}}}
\newcommand{\speccat}{\catE^{\catB}}
\DeclareMathOperator{\catid}{\mathit{id}}
\DeclareMathOperator{\join}{\nabla}
\DeclareMathOperator{\meet}{\triangle}
\newcommand{\banana}[1]{\ensuremath{\llparenthesis #1 \rrparenthesis}}

% Logic terms
%\newcommand{\implies}{\Rightarrow}	% implication

% Functional programming terms
\newcommand{\algspec}[1]{\ensuremath{\mathfrak{#1}}}

\DeclareMathOperator{\map}{map}
\DeclareMathOperator{\fold}{fold}
\DeclareMathOperator{\unfold}{unfold}
\DeclareMathOperator{\build}{build}
\DeclareMathOperator{\hylo}{hylo}
\DeclareMathOperator{\foldl}{foldl}
\DeclareMathOperator{\foldr}{foldr}
\DeclareMathOperator{\add}{add}
\DeclareMathOperator{\remove}{remove}
\DeclareMathOperator{\boldheart}{\pmb{\heartsuit}}

\newcommand{\ra}{\ensuremath{\rightarrow}}
\newcommand{\maps}{\ra}
\newcommand{\la}{\ensuremath{\leftarrow}}
\newcommand{\kind}{\ensuremath{\star}}
\newcommand{\mkfunc}[2]{#1-\texttt{\textgreater}#2}
\newcommand{\textfunc}{\texttt{TC}}
\newcommand{\unafunc}{\mkfunc{\textfunc}{\textfunc}}
\newcommand{\binfunc}{\mkfunc{\unafunc}{\textfunc}}
\newcommand{\typerep}[1]{\ensuremath{\text{#1}^{\scomp}}}
\newcommand{\discrim}{\ensuremath{\bigtriangledown}}
\newtheorem{combeq}{Combinatorial Equation}


%\newenvironment{comment}
%{\textbf{Comments: }\ \begin{slshape}}
%{\end{slshape}}
\newcommand{\rebelheart}{\ensuremath{\sideset{_{\swarrow}}{^{\swarrow}}{\boldheart}}}


% Generic programming brackets
\newcommand{\leftb}{\{\!|}
\newcommand{\rightb}{|\!\}}
\newcommand{\genfunc}[2]{#1\leftb#2\rightb}
\newcommand{\encode}[1]{\genfunc{\mathit{encode}}{#1}}
\newcommand{\gfold}[1]{\genfunc{\mathit{fold}}{#1}}
\newcommand{\concat}{\ensuremath{+\!\!\!+}}
\newcommand{\funcin}[1]{\ensuremath{\text{IN}_{#1}}}

% Miscellaneous
\newcommand{\Joyal}{Andr\'{e} Joyal}
\newcommand{\combdex}[1]{\ensuremath{\mathbb{I}_{#1}}}
