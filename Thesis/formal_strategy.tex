%\section{Sampling and Test Strategies}
%\gordon{Formally define 'strategy' and how it relates to sampling}

\subsection{Generators}
\gordon{ defined by type, sampling method, (finite or infinite) }
\gordon{ already gave formal definitions of generators, but should have done that here I think }

Test case generators were described in section \ref{???}
as the basic mechanism for generating test cases.
A \emph{generator} is a function that instantiates 
a subset of the possible input values for a property for use as test cases.
The selection of test cases forms a \emph{sample} of the property's domain,
or a part of the property's domain,
and possibly invalid test cases from outside the domain for conditional properties.
The generator also provide an order to the test cases they generate,
interpreted as the priority of the values for testing,
although it may simply be an artifact of the implementation of the generator.


\begin{df}[Valid Generator]
A generator for a conditional property is valid if it only creates valid test cases,
%i.e. $\gen{\dom{\property}}$ :: $\[\powerset{\dom{\property}}\]$
\end{df}


It was recommended earlier that the test framework support
the construction of generators using the four standard strategies
(exhaustive, boundary / extreme, uniform and random);
a test suite strategy can be defined as a collection of these strategies.
These should be implemented in a data type generic way.

Should also be possible to construct custom generators that 
are arbitrarily assigned as satisfying one of those sampling strategies.

Should be able to add other generator sampling strategies 
to be included in the test suite strategies.

There should be no requirement that generators produce finite lists of test cases,
as Haskell's lazy evaluation permits infinite lists to be defined.
For example, the random generators from \QC are not finite,
and as can be seen in the discussion of substitution and composition below,
infinite generators can be quite useful.
Test suite building should function with both finite and infinite generators,
and test programs should ensure termination through 
carefully selecting a fixed number of values from a generator,
or only selecting all values when there is an expectation of finiteness.

�%base generator_(type,sample): C(a), ranked generator_(type,sample): Nat -> C(a)

\subsection{Test strategy}
%base = \{(sampling method, num. cases)\}, ranked = \{(sampling method, rank, num. cases)\}

\begin{df}[Sampling Strategy]
blah
\end{df}

\subsection{Generator Dictionaries}
Access generators for a type given a sampling method
Standardized sampling methods allows standardized strategies.

%generator dictionary: sample -> generator_(type,sample) (base or ranked)
%	� �standard sampling strategies = {Exhaust, Random, Uniform n, Boundary}, could add others

\subsection{Test Suite Builders}
%Build (Test strategy (sample, (rank), cases), sample -> generator_(type, sample))�
There is a ``strategy'' in the construction of a test suite,
and that strategy will consist of instructions to automated test case generators,
loading cases from a file, interactively requesting cases etc.
Test suites might be composed of 
multiple collections of test cases from different sources.
It should be possible to merge these sources into a single test suite
without biasing the scheduling and evaluation of these test cases.
There should be a language to construct test suites
in the same way there is a language to construct test case generators.
The language will decouple the strategy for building the test suite
from the provision of the actual test cases,
to allow test suite strategies to be researched independently.

\subsubsection{Traversing a Test Suite}
\gordon{must define these standard operations}
