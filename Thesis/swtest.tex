%\section{Software Testing}

The variables that may affect the outcome of an experiment
must be controlled in some way in the data selection process.
There are five categories of options for dealing with factors in an experiment (\cite{Holt1982}):
\begin{description}
\item[randomized]{Randomly select values for this variable 
from some probability distribution.}
\item[systematically varied]{Deliberately select a variety of values for the variable.}
\item[measured]{If the variable cannot be controlled but can be measured,
the measurement can be included as part of the test result,
and a reasonable diversity of values can be argued post hoc.}
\item[fixed]{A single value for this factor can be selected and maintained for the entire test,
so that variations in it do not confound the results.}
\item[ignored]{If the factor cannot be set, measured or controlled, 
it must be ignored and acknowledged as a weakness in the test.}
\end{description}
\noindent
Those factors that are expected to directly affect the observations
will generally be varied as part of the experiment,
or will be measured under sufficiently different conditions
to create a diversity of values for the observations.
Factors that are assumed not to impact the system's behavior,
or cannot be directly controlled,
will either be fixed and documented as part of the experiment,
or measured and recorded as a possible confounding factor.

These factors can be applied to software testing,
and in particular to ``real world'' testing,
as opposed to the abstract formalisms of the previous section.
In an properly performed actual software test,
the choices of harware, operating system, software tools, etc.
may be fixed and measured (all tests performed on the same system)
or systematically varied (repeated on muiltple platforms with different configurations).
Alternately, the configuration of the test platform can be ignored,
with consequences well known to experienced practitioners.
Similarly, these factors can be considered in the choice of test cases.

\subsection{Selection Criteria}
The \emph{population} of a software test experiment consists of 
all possible \emph{valid} test cases,
i.e. all combinations of inputs and system states allowed by the specification.
Test cases are composed of one or more component values,
and these values can usually be controlled and varied as part of the test.
These values may be systematically or randomly varied
and combined to create a diversity of test cases.
\emph{Invalid} test cases may also produce a behavior in the system,
but since the behavior is unspecified, 
invalid cases should not be considered in the verdict.
The subset of the test case population that is 
evaluated during the test is called the \emph{test set} 
(or \emph{test suite}).

A test is \emph{exhaustive} if every case is tested,
i.e. the test set is the entire population.
A successful exhaustive test is effectively a point-wise proof
that the implementation is correct,
when combined with the hypotheses supporting the validity of the test process.
Unfortunately such a test is only possible for the simplest of specifications
due to the large number of possible inputs for even a moderately complex system.
Note, however, that exhaustive testing can be performed on a subdomain of the population,
which is a particular effective strategy according to the small scope hypothesis (\ref{def:smallscope}).

When exhaustive testing is not feasible,
a \emph{selection criterion} is applied to the population of test cases
to define a test set that is feasible to evaluate.
The selection criterion should inspire confidence that 
the software is correct over the untested cases,
but this claim is generally subjective and often contentious.
An intuitive starting point for defining the adequacy of a test case selection criterion
is provided by \cite{GoodenoughGerhart1975}:
\begin{df}[Reliable Selection]
A selection criterion is \emph{reliable} if 
all tests that satisfy the selection criteria will produce the same test result.
\end{df}

\begin{df}[Valid Selection]
The selection criterion is \emph{valid} if 
for any error the selection will include a test that shows that error.
\end{df}

\begin{theorem}[Fundamental Theorem of Testing]
There exists a reliable, valid selection criterion and a test data set satisfying it 
such that that all errors in a program will be detected.
\end{theorem}

\noindent
Unfortunately, \cite{Howden1976} demonstrated that
there was no computable way in general to provide a reliable selection criterion.
\gordon{review Howden, expand on this thought}

In the absence of a computable, theoretically ideal, test selection criteria,
research has focused on practical approaches to providing adequate coverage
(\cite{ZhuHallMay1997}).
As stated earlier, the complementary goals of testing are to find errors
but also to create confidence that a successful test implies the program is correct.
Confidence is intangible and difficult to quantify,
but in general a diversity of evidence positively contributes to it.
This is supported by  \cite{Cartwright1981}:

\begin{quote}
A sophisticated formal testing system should ... 
show that the generated data is sufficiently diverse to 
establish that the program is almost certainly correct.
\end{quote}

There are many approaches to selecting test data sets,
and many approaches to evaluating those selection criteria.
\cite{WeyukerEtal1991} provides a survey of test selection criteria,
with a comparison of cost and effectiveness of different approaches.
\cite{ZhuHallMay1997} presents a survey of 
\emph{test selection adequacy criteria},
which determine the adequacy of a selection criterion 
with respect to a particular specification, implementation or interface.
In the absence of a guaranteed approach to constructing a representative test,
this is an active area of research.

\subsection{Partitioning}
One approach to strengthening the claim that 
a test selection criterion is adequate
is to partition the possible test cases into ``similar'' groups,
and then select representatives from each part for the test.
The expectation is that if there is an error in the program,
it will be exhibited over many related test cases.
If the test cases are grouped carefully,
testing the representatives should expose any errors,
or at least provide a higher degree of confidence that any errors have been found.

For example, one common approach to test case selection is
for an expert to define classes of \emph{standard uses} of a system,
and unusual situations that are expected to be problematic.
Test cases are the selected from each of these classes,
with the sample allocation (percentage of test cases per class)
reflecting the complexity and perceived risk.
This approach is commonly used in commercial software development,
and in particular for systems with a substantial amount of user interaction.
\gordon{citation}
The software engineering technique of use-cases \gordon{citation} is 
a popular form of this type of usage partitioning.
This sort of analysis is characterized by relatively high development costs,
because of the human expertise involved to create each test case,
but can result in an efficient test set with a small number of high value test cases.
It is also limited by the human ability to comprehend complex systems,
even when supported by tools and methodologies,
as noted in \cite{HamletTaylor1990}:
\begin{quote}
Quantitative results about partition testing are counterintuitive because
our intuition is untrained in confidence testing.
To guarantee high confidence in even medium-scale software requires very large test sets ....
and for this situation our intuition fails.
\end{quote}

Another approach to partitioning is to group the test cases
strictly on the values as terms, i.e. the syntax of the input case,
without reference to the specification or implementation.
\cite{ZhuHallMay1997}) refers to this as \emph{interface based} testing,
but it is also referred to as \emph{type based} testing for 
typed specifications (\cite{Hieronsetal2009}).
Intuititvely, it would seem that ignoring 
both the semantics of the specification and 
the information contained in the implementation
would result in an inferior test set.
This approach, however, is more amenable
to automated test case generation
and can be used to create much larger test sets
than more sophisticated partitioning strategies
\emph{relative to the cost}.
The cost of these approaches are weighted more heavily to system resources
than to human developer time, which is generally preferable.
Large test sets using selection criterion that are unbiased by
an interpretation of the specification or a particular implementation,
such as random test case selection,
have been shown through empirical studies to be as good as
more rigorously defined partition based tests 
(\cite{DuranNtafos1981}, \cite{HamletTaylor1990}).

When an input variable takes on heterogeneous values,
it may be appropriate to partition those values into like groups,
and then select samples according to the importance of those groups.
For example,
the special values |Nan, Infty, NInfty| etc. of the IEEE floating point types
are likely to expose different errors than regular numbers;
similarly, values with extreme exponents might
expose different errors than those with extreme mantissas.
If a data type has some sort of internal ``type'' value,
such as a record in a personnel data base that is either a ``manager'',
``employee'' or ''contractor'',
it may make sense to partition the domain to ensure each is sampled.

\subsection{Infinitely Large Test Domains}
One of the significant challenges of software testing
is that specifications often describe infinitely many possible inputs,
such as arbitrarily large integers or recursive data structure of arbitrary size.
Any feasible test must include only a finite number of elements
in its sampling frame.
Furthermore, a ``real'' system cannot represent arbitrarily large elements,
so there must be a limit on the complexity of the terms that can be tested.
These problems arise frequently in Haskell programming and other functional languages
in which recursive data structures are commonly used.

In general, it is less costly to test smaller or simpler cases,
and there are fewer of them.
Fortunately, experience suggests that
most errors in a program occur with relatively simple test cases,
the \emph{small scope hypothesis}.
This hypothesis assumes that
there is some measure of \emph{complexity}
that can be applied to the input values of a test,
and that any errors are likely to occur in the terms of lower complexity.
A natural extension of this heuristic is that
there must be a level of complexity such that
if the program is correct for all simpler inputs,
it can be assumed to be correct for all possible terms.
This is described formally in section (\ref{pbt})
as a \emph{regularity} hypothesis (\ref{sub:regularity}, \cite{BernotGaudelMarre1991}).
Since for any finite test set and measure of complexity
there must be a maximum complexity to the test set,
a supporting hypothesis of this sort is generally necessary
in the interpretation of a test verdict.

The necessity of a regularity hypothesis to manage infinite property domains,
and the benefit in focusing on less complex elements of the domain,
suggest that an automated test generation strategy should incorporated a complexity measure
to partition the test cases and determine sampling methodologies.
\GC enumerative generators (chapter \ref{chp:enumgen}) provide that complexity measure
which can be used to select sampling methodologies (\ref{sec:sampling_theory}).