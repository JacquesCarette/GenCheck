
There are a number of differences between the \FEAT (\cite{Duregard2012} ) enumerations
and those described in this chapter:

\begin{enumerate}
\item \FEAT uses the number of constructors embedded in the term as it's size,
instead of the number of atoms in the construct.
\item the enumeration is over concrete terms, meaning that all element types are incorporated into the overall enumeration
as opposed to supporting post enumeration substitution;
\item  a single sampling method is applied to the enumeration to create the test data set
(although the test can be repeated independently with each of the three supported sampling methods);
\item \FEAT does not support explicit substitution of independently sampled values into structures;
\item a different enumeration algebra is supported - 
they explicitly discuss \cite{FlSa95} and their reasons for not using this approach.
\end{enumerate}

Their approach, while providing a single source of random, uniform and exhaustive test case generators,
does not allow for more complicated hybrid sampling methodologies
nor for specialized generators for specific domains (e.g. a ``person name generator'' or ``physical address generator'')
as an alternate sampling methodology.
The main focus of \FEAT was to provide computationally efficient enumerations,
so further analysis of their enumerations is left to the chapter \ref{chap:source}.
\gordon{I better make sure I actually did this.}

One interesting detail is that the issue of epsilon-freeness in the theory of combinatorial structures
presents itself as the need to assign an explicit cost to recursion in the calculation of term size
(in the paper this is referred to as the part identifier).
In their enumeration part cost function (equivalent to rank in this thesis),
constants (called singletons) are given a cost of $0$.
The authors point to a representation of the natural numbers

\begin{code}
data Nat = Z | S Nat
\end{code}

and note that without an explicit cost for each recursive call,
all of the natural numbers would end up with the same cost,
thus violating the requirement that each part of the population must be finite.
This specification, as stated, cannot be represented in the theory of combinatorial constructions,
but must instead be altered so |S| is a product :

$$ \spec{N} = \specI \splus (\spec{S} \sprod \spec{N}) $$

where $\specI, \spec{S}$ are constants with weight $1$.
In effect, the natural numbers are modelled as lists of constants,
and the need to provide a weight of $1$ to the $\spec{S}$ constant to keep the structure epsilon free
is equivalent to the need for the recursion cost in the \FEAT enumeration.


