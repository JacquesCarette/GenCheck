\documentclass[11pt]{report}
\usepackage{jrl_thesis}
\usepackage{fancyheadings}
\usepackage{natbib}
\usepackage{setspace}               % this package defines the spacing of your thesis and should be included
\usepackage[latin1]{inputenc}
\usepackage{xspace} % to get better space control in macros


% Math package and custom definitions
\usepackage{amsmath}
\usepackage{amsfonts} % all sorts of extra fonts including math ones
\usepackage{amssymb} % some more symbols and fonts
\usepackage{bbm}   % used forindicator function, replaces characteristic


\newcommand{\cross}{\ensuremath{\times}}      % Cartesian product
\newcommand{\union}{\cup}
\newcommand{\intersection}{\cap}

\newcommand{\refines}{\geq}

\newcommand{\true}{\mathbb{T}}
\newcommand{\false}{\mathbb{F}}
\newcommand{\xor}{\mathbin{\oplus}}

\renewcommand{\phi}{\varphi}
\renewcommand{\labelenumi}{(\theenumi)}

\newtheorem{theorem}{Theorem}
\newtheorem{corollary}{Corollary}
\newtheorem{lemma}{Lemma}
\newtheorem{proposition}{Proposition}
\newtheorem{remark}{Remark}
\newtheorem{example}{Example}
\newtheorem{definition}{Definition}
\newtheorem{algorithm}{Algorithm}

\newtheorem{thm}{Theorem}[chapter]
\newtheorem{cor}[thm]{Corollary}
\newtheorem{lem}[thm]{Lemma}
\newtheorem{prop}[thm]{Proposition}
\newtheorem{rem}[thm]{Remark}
\newtheorem{eg}[thm]{Example}
\newtheorem{df}[thm]{Definition}
\newtheorem{alg}[thm]{Algorithm}
\newtheorem{interface}[thm]{Interface}
\newtheorem{implementation}[thm]{Implementation}

\newenvironment{proof}{\par\noindent{\bf Proof\ \ }}{$\Box$}

\newcommand{\urlpart}[1]{\mbox{\texttt{#1}}\linebreak[0]}


% The collection of symbols and labels for the description of a test system

%\newcommand{\scart}{\ensuremath{\times}}      % Cartesian product
%\DeclareMathOperator{\join}{\nabla}

% property related tags

\newcommand{\property}{\emph{P}}
\newcommand{\propertyrestrict}{\emph{\ensuremath{P^{*}}}}
\newcommand{\dom}[1]{\ensuremath{\mathop{Dom}(#1)}}
\newcommand{\boolbot}{\ensuremath{\mathbb{B}_{\bot}}}      % requires bum package
\newcommand{\charac}[1]{\ensuremath{\mathbbm{1}_{#1}}}    
\newcommand{\specification}{\emph{Spec}}
\newcommand{\impl}{\emph{Impl}}

\newcommand{\gen}[1]{\ensuremath{\emph{G_{#1}}}}
\newcommand{\powerset}[1]{\ensuremath{\mathbbm{2}_{#1}}}    

\newcommand{\case}[1]{\ensuremath{\tau_{#1}}}
\newcommand{\extcase}[2]{\ensuremath{\tau^*_{#1,#2}}}
\DeclareMathOperator{\datum}{datum}

\newcommand{\testsuite}[1]{\emph{\ensuremath{T_{#1}}}}
\newcommand{\exttestsuite}[1]{\emph{\ensuremath{T^*_{#1}}}}
\newcommand{\labels}{\mathfrak{L}}

\DeclareMathOperator{\verdict}{verdict}
\newcommand{\verdictset}{\mathbbm{V}}
\newcommand{\success}{\emph{Pass}}
\newcommand{\fail}{\emph{Fail}}
\newcommand{\nonterm}{\emph{Incomplete}}
\newcommand{\invalid}{\emph{Invalid}}
\newcommand{\noteval}{\emph{NotEval}}

\newcommand{\eval}[1]{\ensuremath{\varphi_{#1}}}
\newcommand{\evaluncon}[1]{\ensuremath{\varphi^*_{#1}}}

\newcommand{\result}[1]{\ensuremath{r_{#1}}}
\newcommand{\resultset}[1]{\ensuremath{\mathfrak{R}_{#1}}}
\newcommand{\test}[1]{\ensuremath{T_{#1}}}

\newcommand{\cntnr}[1]{\ensuremath{C(#1)}}
\newcommand{\ogf}[2]{\ensuremath{\sum_{{#2} = 0}^{\infty} {#1}_{#2} x^{#2}}}
\newcommand{\ogfone}[2]{\ensuremath{\sum_{{#2} = 1}^{\infty} {#1}_{#2} x^{#2}}}

\usepackage{tikz}
\usepackage{pgf}
\usetikzlibrary{arrows,shapes.geometric,shapes.misc,decorations,positioning}

\usepackage[compact]{fancyvrb1}
\DefineShortVerb{\|}
\DefineVerbatimEnvironment{code}{Verbatim}{xleftmargin=1em,fontsize=\small}

\usepackage{listings}
\lstset{language=Haskell, basicstyle=\small, stringstyle=\ttfamily}

%\usetikzlibrary{arrows,shapes.geometric,shapes.misc,decorations,positioning}
\newcommand{\minibox}[2]{\begin{minipage}[c]{#1}\begin{center}{#2}\end{center}\end{minipage}}

\usepackage{url} % wrap urls to avoid bad formatting

%
%
%% Use the listings package for the code display; these are the options suggested by the Haskell Wiki
%\usepackage{listings}
%\lstloadlanguages{Haskell}
%\lstnewenvironment{code}
%    {\lstset{}%
%      \csname lst@SetFirstLabel\endcsname}
%    {\csname lst@SaveFirstLabel\endcsname}
%    \lstset{
%      basicstyle=\small\ttfamily,
%      flexiblecolumns=false,
%      basewidth={0.5em,0.45em},
%      literate={+}{{$+$}}1 {/}{{$/$}}1 {*}{{$*$}}1 {=}{{$=$}}1
%               {>}{{$>$}}1 {<}{{$<$}}1 {\\}{{$\lambda$}}1
%               {\\\\}{{\char`\\\char`\\}}1
%               {->}{{$\rightarrow$}}2 {>=}{{$\geq$}}2 {<-}{{$\leftarrow$}}2
%               {<=}{{$\leq$}}2 {=>}{{$\Rightarrow$}}2 
%               {\ .}{{$\circ$}}2 {\ .\ }{{$\circ$}}2
%               {>>}{{>>}}2 {>>=}{{>>=}}2
%               {|}{{$\mid$}}1               
%    }


% Include the comment macros
\newcommand{\gordon}[1]{{\textit{Gordon says:} #1 \textit{(???)}}}
\newcommand{\jacques}[1]{{\textit{Jacques says:} #1 \textit{(???)}}}

\newcommand{\pbt}{\textit{PBT}\xspace}
\newcommand{\pbting}{\textit{PB testing}\xspace}
\newcommand{\algdt}{\textit{AlgDT}\xspace}
\newcommand{\typecon}{\ensuremath{\mathbb{T}}}
\newcommand{\role}{r\^{o}le}
\newcommand{\GC}{\textit{GenCheck}\xspace}
\newcommand{\QC}{\textit{QuickCheck}\xspace}
\newcommand{\SC}{\textit{SmallCheck}\xspace}
\newcommand{\LSC}{\textit{Lazy SmallCheck}\xspace}
\newcommand{\EC}{\textit{EasyCheck}\xspace}
\newcommand{\GAST}{\textit{GAST}\xspace}
\newcommand{\HOLTG}{\textit{HOL-TESTGEN}\xspace}
\newcommand{\FEAT}{\textit{FEAT}\xspace}
\newcommand{\DAISTS}{\textit{DAISTS}\xspace}

\DeclareMathOperator{\nil}{nil}
\DeclareMathOperator{\cons}{cons}

\newcommand{\lblhat}[2]{\ensuremath{\stackrel{#2}{#1}}}

\newcommand{\AJ}{Andr\'{e} Joyal}
% spec marks the argument as a species, and some common species get their own commands
\newcommand{\spec}[1]{\ensuremath{\mathcal{#1}}}
% specn is for species over fixed size sets, e.g. \specn{E;2} is a set (bag) of two elements.
\newcommand{\specn}[2]{\ensuremath{\spec{#1}_{#2}}}
\newcommand{\specnum}[1]{\ensuremath{\mathbf{#1}}}

\newcommand{\specO}{\specnum{0}}   % using O for 0, the non-existent structure
\newcommand{\specI}{\specnum{1}}  % using I for 1, the structure over the empty set
\newcommand{\specX}{\ensuremath{\mathcal{X}}}  % singletons
\newcommand{\specE}{\ensuremath{\mathcal{E}}}  % set (bag)
\newcommand{\specC}{\spec{C}}  % cycle
\newcommand{\specL}{\spec{L}}  % list
\newcommand{\specA}{\spec{A}}  % trees - arbres
\newcommand{\specF}{\spec{F}}  % generic species labelled F
\newcommand{\specG}{\spec{G}}  % generic species labelled G
\newcommand{\specZ}{\spec{Z}}  % singleton, but in Flajolet et al syntax

\newcommand{\field}[1]{\ensuremath{\mathbb{#1}}}
\newcommand{\nat}{\field{N}}
\newcommand{\integer}{\field{Z}}
\newcommand{\boolean}{\field{B}}
\newcommand{\True}{T}
\newcommand{\False}{F}
\newcommand{\reals}{\field{R}}
\newcommand{\absval}[1]{\ensuremath{\lvert#1\rvert}}

% sfrak gives a species with a different font
\newcommand{\specroot}[1]{\ensuremath{\mathfrak{#1}}}
\newcommand{\specfrak}[1]{\ensuremath{\mathfrak{#1}}}
\newcommand{\specname}[1]{\ensuremath{\mathit{#1}}}    % names longer than one letter

\newcommand{\specwgt}[2]{\ensuremath{\mathcal{#1}_{\field{#2}}}}

% sets of elements have no structure, the species is applied over a set 
\newcommand{\set}[1]{\ensuremath{\mathit{#1}}}
\newcommand{\setA}{\set{A}}
\newcommand{\setcard}[1]{\ensuremath{\lvert#1\rvert}}
\newcommand{\setcardn}[2]{\ensuremath{\lvert#1\rvert}_{#2}}
\newcommand{\specset}[2]{\ensuremath{\spec{#1}[\set{#2}]}}
\newcommand{\specsetpar}[2]{\ensuremath{(\spec{#1})[\set{#2}]}}
\newcommand{\specnset}[3]{\ensuremath{\spec{#1}_{#2}[\set{#3}]}}
\newcommand{\specfrakset}[2]{\ensuremath{\mathfrak{#1}[\set{#2}]}}
\newcommand{\sunion}[2]{\ensuremath{\bigcup_{#1}^{#2}}}

% Species operators
\newcommand{\splus}{\ensuremath{+}}           % disjoint union
\newcommand{\sprod}{\ensuremath{\cdot}}       % product
\newcommand{\scomp}{\ensuremath{\circ}}       % composition (substitution)
\newcommand{\sdiff}{'}      % differentiation
\newcommand{\spt}{\ensuremath{^{\bullet}}}    % pointing
\newcommand{\scart}{\ensuremath{\times}}      % Cartesian product
\newcommand{\sfunccomp}{\ensuremath{\Box}}
\newcommand{\speccard}[2]{\ensuremath{\lvert#1[\set{#2}]\rvert}}
\newcommand{\specring}{\ensuremath{\mathfrak{Spe}}}
\newcommand{\virtring}{\ensuremath{\mathfrak{Virt}}}

%L species operators
\newcommand{\sprodL}{\ensuremath{\sprod_{O}}}
\newcommand{\sinteg}{\ensuremath{\int}}

% Species equivalence operators
\newcommand{\sequiv}{\ensuremath{=}}		% equivalence
\newcommand{\sequip}{\ensuremath{\equiv}}    % equipotence  
\newcommand{\siso}{\ensuremath{\simeq}}	% isomorphic

%Species transformations or meta operators
%\newcommand{\toset}{\ensuremath{\mu}}  % forgetful functor, maps F to E for all F, \mu is arbitrary decision

%Generating series
\newcommand{\specgs}[1]{\spec{#1}(x)}   % formal power series
\newcommand{\factorial}{\!}
\newcommand{\coeff}[2]{\ensuremath{\[ {#1}^{#2} \]}}  % coefficient  of formal power series

\newcommand{\semiring}[2]{#1\llbracket #2 \rrbracket}

%Categories
\newcommand{\categoryfnt}[1]{\ensuremath{\mathbb{#1}}}
\newcommand{\catname}[1]{\categoryfnt{#1}}
\newcommand{\categorywrd}[1]{\ensuremath{\mathfrak{#1}}}
\newcommand{\catB}{\categoryfnt{B}}
\newcommand{\catE}{\categoryfnt{E}}
\newcommand{\catL}{\categoryfnt{L}}
\newcommand{\ordset}[1]{\ensuremath{\mathit{#1}}}
\newcommand{\lspecset}[2]{\specset{#1}{#2}}
\newcommand{\symgroup}{\ensuremath{\mathfrak{G}}}
\newcommand{\speccat}{\catE^{\catB}}
\DeclareMathOperator{\catid}{\mathit{id}}
\DeclareMathOperator{\join}{\nabla}
\DeclareMathOperator{\meet}{\triangle}
\newcommand{\banana}[1]{\ensuremath{\llparenthesis #1 \rrparenthesis}}

% Logic terms
%\newcommand{\implies}{\Rightarrow}	% implication

% Functional programming terms
\newcommand{\algspec}[1]{\ensuremath{\mathfrak{#1}}}

\DeclareMathOperator{\map}{map}
\DeclareMathOperator{\fold}{fold}
\DeclareMathOperator{\unfold}{unfold}
\DeclareMathOperator{\build}{build}
\DeclareMathOperator{\hylo}{hylo}
\DeclareMathOperator{\foldl}{foldl}
\DeclareMathOperator{\foldr}{foldr}
\DeclareMathOperator{\add}{add}
\DeclareMathOperator{\remove}{remove}
\DeclareMathOperator{\boldheart}{\pmb{\heartsuit}}

\newcommand{\ra}{\ensuremath{\rightarrow}}
\newcommand{\maps}{\ra}
\newcommand{\la}{\ensuremath{\leftarrow}}
\newcommand{\kind}{\ensuremath{\star}}
\newcommand{\mkfunc}[2]{#1-\texttt{\textgreater}#2}
\newcommand{\textfunc}{\texttt{TC}}
\newcommand{\unafunc}{\mkfunc{\textfunc}{\textfunc}}
\newcommand{\binfunc}{\mkfunc{\unafunc}{\textfunc}}
\newcommand{\typerep}[1]{\ensuremath{\text{#1}^{\scomp}}}
\newcommand{\discrim}{\ensuremath{\bigtriangledown}}
\newtheorem{combeq}{Combinatorial Equation}


%\newenvironment{comment}
%{\textbf{Comments: }\ \begin{slshape}}
%{\end{slshape}}
\newcommand{\rebelheart}{\ensuremath{\sideset{_{\swarrow}}{^{\swarrow}}{\boldheart}}}


% Generic programming brackets
\newcommand{\leftb}{\{\!|}
\newcommand{\rightb}{|\!\}}
\newcommand{\genfunc}[2]{#1\leftb#2\rightb}
\newcommand{\encode}[1]{\genfunc{\mathit{encode}}{#1}}
\newcommand{\gfold}[1]{\genfunc{\mathit{fold}}{#1}}
\newcommand{\concat}{\ensuremath{+\!\!\!+}}
\newcommand{\funcin}[1]{\ensuremath{\text{IN}_{#1}}}

% Miscellaneous
\newcommand{\Joyal}{Andr\'{e} Joyal}
\newcommand{\combdex}[1]{\ensuremath{\mathbb{I}_{#1}}}
 % all of the nomenclature for species and categories


%% End of configuration information

\title{Property Based Testing \\ with \\ Automated Test Case Generation}

\author{Gordon J. Uszkay}

\prevdegreeone{B.Math (Applied Math and Computer Science)\\ University of Waterloo, Waterloo ON, Canada}
\prevdegreetwo{M.Sc. Computer Science\\ McMaster University, Hamilton ON, Canada}

\submitdate{Sept. 1st, 2011}

\copyrightyear{2011}

\principaladviser{Dr. J. Carette}
               % your information goes here

\begin{document}

%\beforepreface                      % command to create the parts of your thesis that come before the preface like title and etc.
%
%  \input{dedication}
%  \prefacesection{Abstract}

This thesis develops a modular framework for developing 
property based tests of Haskell modules
using sampling strategies to automatically generate test cases.
A formal argument for program correctness is defined
to provide the basis for selecting test cases and interpreting successful tests as evidence.
Combinatorial enumerations of Haskell data types
are presented as a technique for supporting 
type independent strategies for sampling and generating test cases,
along with a comparison of other techniques already in use.
The \GC framework provides a modular architecture and Haskell library for 
integrating different sampling methods, evaluation strategies and reporting techniques to support
both research into property based testing and practical use by Haskell developers.


%  \input{acknowledgements}
%  \input{notation}
%
\afterpreface                      % command to create the parts of your thesis that come after your preface like contents and etc.

%------------------------------------------------------------------------------
\setcounter{figure}{0}
\setcounter{equation}{0}
\setcounter{table}{0}
\chapter{Introduction}
%\chapter{ Introduction }

Testing is, at the very least, a necessary evil 
in the development of software and software intensive systems.
Frequently treated as an art as opposed to a science,
software testing might appear to be the ugly cousin to theorem proving,
lacking the elegance of mathematical proofs and
offering only to manage the risk of failure, not eliminate it.
Theorem proving, however, exists only the realm of mathematical abstractions,
and cannot address the real world in which the actual program will be functioning,
a world fraught with unexpected and pernicious challenges.
If there is to be any meaningful claim that
the software solves the \emph{real world} problem it was invented to address,
it will be necessary to invoke an installed instance of the program 
on an actual computing device under the conditions that it is expected to operate:
in other words it must be tested.
Perhaps the best summary of the relationship between theorem proving and testing 
is provided by Donald Knuth (\cite{KnuthProvedNotTested1977}):

\begin{quote}
Beware of bugs in the above code; I have only proved it correct, not tried it.
\end{quote}

\noindent
Since testing is a necessity, 
it behooves practitioners who appreciate the elegance of theorem proving
to attempt to apply the same rigor and discipline to the testing process.

Driven by time and resource limitations
in the face of the staggering complexity of modern software systems,
software practitioners naturally look to software tools
to assist, manage and automate the testing process.
One area of particular interest to programmers is 
software that automatically generates and evaluates test cases.
The hope is that these tools would generate large numbers of test cases
with little effort on the part of a test developer, 
and lead to both quicker and more thorough testing.

Property based testing is a powerful form of unit testing that is
particularly well suited to automatic test case generation for functional programming languages
such as Haskell (\cite{Haskell98}, \cite{Haskell2010}),
Erlang (\cite{armstrong1993Erlang}, \url{http://erlang.org/} ) and OCaml(\url{http://ocaml.org/})
(\cite{Fink1997}, \cite{Papadakis2011}, \cite{OSullivan2008}).
A number of software packages supporting
automated test case generation for property based testing have been released,
providing different approaches to selecting and generating test cases.
While test developers benefit from having
these different test case selection strategies,
it does raise a number of questions:
\begin{itemize}
\item Which test selection strategy is best for this test?
\item Are there other test selection strategies to consider?
\item Will the selected test cases provide a good trade off between uncovering errors and the cost of running the test?
\item Perhaps most importantly,
is it really necessary to have a separate software package
for each test selection strategy?
\end{itemize}

The \GC project was initiated to 
support the research and development of property based test case generation
by establishing a theoretical basis for test case selection and
providing a single, extensible, software package that 
would support a wide variety of such strategies.
It explores the relationship between test case generation,
formal \emph{sampling theory} (\cite{Stuart1968}, \cite{Cochran1977}),
the complexity of test terms,
and combinatorial enumerations as the basis for sampling Haskell data types.
A modular framework with independent interfaces for 
evaluation,  automated test case generation and reporting
allows these strategies to be used for 
projects of varying nature and complexity.
The intent of this work is not to provide a single canonical test case selection algorithm,
but instead to act as the foundation for research into this area
and provide a pragmatic tool for automatic test case generation for \pbt.

\section{Property Based Testing}
A property is computable boolean valued function,
implemented using the module(s) to be tested,
implementing an equivalence relationship  from the module's specification.
A \emph{property based test}(\pbt)%
\footnote{Not to be confused with \emph{property testing}, \cite{Ron00}.}
consists of evaluating the property function
over a subset of its allowable inputs (test cases).
If the property does not evaluate to true for one or more of the allowable inputs,
then the module is incorrect.
Typically a specification will consist of a number of such equivalence relationships,
with each relationship being implemented as a distinct property and evaluated independently.

%Another question that is raised is the relationship between
%the generation of test cases and their evaluation for the test verdict.
%\QC, \SC and the other published \pbt packages 
%(chapter \ref{pbtsystems} will present a detailed critique)
%sequentially generate and evaluate test cases,
%terminating when an error (or a fixed number of errors) is found.
%While a single developer unit testing a simple module may be satisfied with this approach,
%a more sophisticated software project might 
%execute test cases in parallel over multiple cores and computers,
%and require a full report of all of the test results for analysis.
%The nature and complexity of a software projects differ widely,
%so it is unlikely that any one approach for evaluating and reporting will suit all needs.
%Could a test system provide flexibility in 
%the scheduling, evaluation and reporting of test results
%while offering the same kinds of test case generation provided in these packages?

The first step in defining a property based test is
to identify the properties of the specification.
As an example, consider an algebraic specification for polymorphic lists
equipped with constructors (nil and cons) and a concatenation operation
that is defined by the equations of the specification.

\begin{figure*}
%\fbox {
\begin{minipage}[t]{.3\linewidth}
Operators:
\begin{align*}
& \nil : \specL_{\alpha} \\
& \cons : \alpha \sprod \specL_{\alpha} \mapsto \specL_{\alpha} \\
& \concat : (\specL_{\alpha} \sprod \specL_{\alpha}) \mapsto \specL_{\alpha} \\
& =_{\specL_{\alpha}} : (\specL_{\alpha} \sprod \specL_{\alpha}) \mapsto \boolean \\
& (\text{assumes} =_{\alpha} \text{defined}).
\end{align*}
\end{minipage}
\begin{minipage}[t]{.2\linewidth}
\end{minipage}
\begin{minipage}[t]{.5\linewidth}
Equations:
\begin{align*}
&\forall x \in \specL_{\alpha}.x \concat \nil =_{\specL_{\alpha}} x\\
&\forall x \in \specL_{\alpha}.\nil \concat x =_{\specL_{\alpha}} x\\
&\forall x, y, z \in \specL_{\alpha}. 
    ((x \concat y) \concat z) =_{\specL_{\alpha}} ( x \concat (y \concat z)) \\
&\forall a \in \alpha. \forall x,y \in \specL_{\alpha}. 
    (\cons (a, x \concat y)) =_{\specL_{\alpha}} (\cons (a, x) \concat y)
\end{align*}
\end{minipage}
%} % end xbox
\caption[Algebraic specification of a list]
{An algebraic specification of a list with concatenation.}
\label{intro_spec_ex}
\end{figure*}

An implementation of this specification in Haskell
would consist of a module with a type constructor for the new list structure,
and a binary operator over that type to implement concatenation.
An example of such an implementation is provided below%
\footnote{This concatenation operator will conflict with that of the Prelude when compiled,
so either the prelude version must be excluded or the operator renamed}.
A traditional approach to unit testing the concatenate operator would have
the test developer to choose pairs of lists
and compare the result of using the concatenation operator in the module
to the expected result.
This would require the test developer to:
\begin{itemize}
\item select pairs of lists representing a variety of likely operands
\item determine the concatenation of each pair \emph{without} using the module being tested,
\item encode the test pairs and resulting concatenations correctly in a test program,
\item compare the concatenation produced by the module to the prediction
and report any discrepancies as errors
\end{itemize}

\begin{figure*}
\begin{lstlisting}
module List (List(), (++)) where

data List a = Nil | Cons a (List a)
  deriving Show

(++) :: List a -> List a -> List a
(++) (xs) (Nil) = xs
(++) (Nil) ys = ys
(++) (Cons x xs) ys = Cons x (xs ++ ys)
\end{lstlisting}
\caption{Implementation of polymorphic list concatenation}
\label{intro_spec_impl}
\end{figure*}

Consider two such test cases,
with manually derived results:
\begin{align*}
\cons(1, \cons (2, \nil)) \concat \nil & = \cons(1, \cons (2, \nil))\\
\cons(1, \cons (2, \nil)) \concat \cons(3, \cons (4, \nil)) & = \cons(1, \cons (2, \cons(3, \cons (4, \nil))))
\end{align*}
\noindent
These inputs and results must then be coded into a test program
and compared to the actual result for the module:
\begin{lstlisting}
import List
x = Cons 1 (Cons 2 Nil)
y = Cons 3 (Cons 4 Nil)
xy = Cons 1 (Cons 2 (Cons 3 (Cons 4 Nil)))

test :: Boolean
test =   ( x ++ Nil ) == x  && ( Nil ++ x ) == x && ( x ++ y ) == xy
\end{lstlisting}

\noindent
Determining the predicted result of an operation
without using the implementation to be tested,
and manually encoding the inputs and expected results,
is both arduous and error prone.

\pbt generalizes this approach by encoding the specified relationships
and testing for internal consistency,
without explicitly knowing the results of an operation.
Here the list specification equations are encoded as Haskell properties,
with the specification variables as parameters:

\begin{figure*}
\begin{lstlisting}
import List

prop_leftunit, prop_rightunit :: List a -> Bool
prop_leftunit    xs = (Nil ++ xs) == xs
prop_rightunit xs = (xs ++ Nil) == xs

prop_assoc_list :: (List a, List a, List a) -> Bool
prop_assoc_list (xs, ys, zs) = (xs ++ (ys ++ zs)) = ((x ++ ys) ++ zs)

prop_assoc_cons :: (a, List a, List a) -> Bool
prop_assoc_cons (x, ys, zs) = (Cons x (ys ++ zs)) = ((Cons x ys) ++ zs)
\end{lstlisting}
\caption{Haskell properties of list concatenation.}
\label{intro_haskprop_ex}
\end{figure*}

\noindent
A test case for these properties is then an input of the appropriate type,
and a test is the conjugate of the Boolean result of each of these property function evaluations
over a collection of such test cases.
Removing the need to supply an explicit result for a test case
makes test case generation much simpler as only the arguments to the property function must be generated.

\section {Automated Test Case Generation for \pbt}

The promise of software tools providing comprehensive, inexpensive testing
by automatically generating test cases is a compelling one.
A tool for automating property based tests has several functions :

\begin{enumerate}
\item generate (or otherwise acquire) test cases
\item evaluate the property over test cases until all are successful or at least one fails
\item return the verdict of the test and (optionally) report the results of the test cases evaluations
\end{enumerate}

\subsection{Test Case Selection and Generation}
In a traditional software testing situation, 
\emph{domain knowledge} experts select 
a relatively small number of test cases that are representative of
the \emph{standard uses} of the program,
and possibly some that are expected to pose additional risks.
Identifying classes of system behaviors and representative test cases from those classes
is a \emph{heuristic} test strategy that creates an efficient test as each case is meaningful.
It is also time consuming, requires significant expertise on the part of the test developers,
and is error prone as unexpected behaviors are unlikely to be addressed.

The definition of the property function
allows test case generation to be based on
the syntactic definition of the argument types,
instead of the semantics of the specification
or the implementation of the module to be tested.
The automated test system partitions the property argument values
and select representatives from each partition,
based solely on the data type of the argument.
This approach discards any a priori knowledge of the system to be tested,
but compensates by greatly reducing the cost of generating more test cases.
These techniques are particularly useful in finding
errors relating to the implementation or an ``irrational'' use of the system,
which might not occur to a human test developer,
and for testing complicated modules which might defy domain expertise.
While one could argue in favor of either approach,
it is very reasonable to decide that they are instead complementary
and should both be supported in any practical test framework
(although this work will focus almost exclusivley on the generation of test cases using syntactic strategies).

Automated test generators are able to use syntactic strategies
generate very large numbers of test cases inexpensively,
reducing the time to develop a test,
but at the expense of evaluation costs.
This creates the essential compromise of testing:
there must be enough test cases to create confidence,
but the cost of the test must be reasonable given the value of the system.
Many more test cases may be needed, however,
to achieve the same level of confidence as
a test suite developed by a human expert
who will include fewer redundant test cases based on their experience.
The strategies that define these classes and
the number and complexity of test cases
are a critical characteristic of a test system.

There are a number of existing \pbt packages available to the Haskell community
using this syntactic approach to test case generation but 
offering different approaches to test case selection.
For the most part, the focus of these packages is on how to select test cases
from recursive algebraic data types,
so the complexity of the input values is often referred to as the ``size''.
\begin{itemize}
\item \QC generates a stream of random test cases,
initially favoring simpler test cases (smaller data structures) and then allowing more complex ones;
\item \SC exhaustively generates all structures up to a set complexity (``size''),
selecting scalar values from a small range;
\item \EC (\cite{EasyCheck2008}) generates a mixture of
simple and complex data structures to ensure small test suites contain a variety of complexity.
\item \FEAT(\cite{Duregard2012}) includes both of the above but
also allows a selection over uniform intervals (discussed further in \ref{}).
\end{itemize}
\noindent
Each of these packages partitions the property's domain (possible input values)
and then selects a representative set of the cases to evaluate.
The interface to each of these packages is very similar,
allowing testers to move back and forth between them,
to use the different test strategies as needed.
This then poses a dilema for the tester:
which test selection strategies will provide
the highest degree of confidence
relative to the investment in generating and running the tests?
How are these different packages and in particular their test selection criteria 
to be compared and judged?

\subsection{Sampling Strategies}
In the broader scientific community,
this kind of partitioning and selection of elements from a population,
without human expertise guiding or biasing the selection process,
are considered forms of \emph{sampling}.
Sampling theory (\cite{Stuart1968}, \cite{Cochran1977}) is 
the branch of statistics that studies
the selection of a representative sample of elements 
that can be tested to draw conclusions about a large population.

There are two concepts in particular that are valuable in discussing test case generation:

\begin{enumerate}
\item A sampling methodology is an unblased approach to selecting members of a group
\item Stratification is the partitioning of a population into similarly behaving groups or classes,
each of which should be sampled independently, and possibly using distinct sampling methodologies
\end{enumerate}

\noindent
Stratifying a population and then selecting different sampling methodologies
is a useful technique for managing the pragmatic aspects of testing hypotheses,
and in particular is useful when different parts of a population are much more costly to sample.
For example, consider a module managing height balanced trees,
and a \pbt of the ``node insertion'' capabilities.
It would be desirable to exhaustively test the properties 
over inserting a node into all ``small'' trees (say up to depth of 4),
and then test a small sample of larger trees, perhaps with a few being extremely large.
This suggests that the population of height binary balanced trees be partitioned into small and large,
with all small trees (exhaustive sampling) and a random sample of large trees
being included in the test suite.

Statistics and probability theory are used to provide 
guidance for selection strategies and 
estimates of the bias and accuracy of tests over selected samples.
While many of the assumptions regularly used in statistics 
are not applicable to software testing,
we propose that the standard concepts behind sampling methods 
provide a starting point for structuring and evaluating test case selection strategies.
In particular, we would propose that test systems should provide
stratified test strategies to allow different sampling strategies for different classes of test cases,
the discussion of which will be a significant focus of this document.

The choice of test case selection strategies is 
a subjective but important area of ongoing research.
Existing \pbt packages mostly generate test cases based on a single sampling method
\footnote{\FEAT being the exception as discussed in the next section},
and since they tightly couple test case generation, evaluation and reporting,
they cannot easily be adapted to other sampling methods.
Factors such as stopping conditions,
sequencing of the test cases,
and the presentation of the test result details
are all influenced by the sampling choice.
This both hampers the comparison of the different methodolgies
and is an impediment to the practitioner wishing to use different test strategies
when validating a module.
We would propose that a proper evaluation of sampling strategies can only be carried out
using a platform that supports stratification of the test domain,
and a variety of sampling methodologies including allowing new methodologies to be defined,
and allows different sampling methodologies within different parts of the domain.

\section{Enumerative Generators}

Sampling methods generally require some sort of organization of the values being sampled,
in this case is the allowable inputs for the property being tested.
Ideally there would be a single interface / approach that would support 
all of the standard sampling methods over
any of the data types likely to be arguments for the properties.
For Haskell, this includes recursively defined algebraic data types,
which are particularly challenging to sample as the represent
infinitely many values of arbitrary complexity.

One of the major contributions of this thesis is
demonstrating that \emph{combinatorial enumerations} (\cite{FlSa95}) provide 
a canonical organization suitable for standard sampling methods
of the most common Haskell types including recursive and mutually recursive algebraic data types..
An enumeration  of a data type provides 
the number of values represented
and a selection function based on an index over those values.
Sampling methods can then be generalized to apply to the index,
creating a set of arguments to be mapped through the selection function to obtain the selected values.
A family of enumerations, further indexed (or ranked) by the complexity of the terms,
allows stratified sampling of (mutually) recursive algebraic types%
\footnote{Only regular recursive types are supported, as discussed in \ref{chp:testgen}}.
A generalized sampling method is combined with
the combinatorial enumeration of a data type to create an
\emph{enumerative sampling generator} for automating test case generation
(chapter \ref{chp:enumgen}).

The development of ranked enumerations of algebraic data types
is based on the \emph{combinatorial structures} of \cite{FlajoletSedgewick2009},
providing a strong theoretical basis for the existence of these enumerations.
Additional works in this area (\cite{FlSaZi91}, \cite{FlajoletZC94}, \cite{FlSa95})
direct the details of their implementation
and support the efficiency of the algorithms.
This theory also extends beyond algebraic data types to
sets (bags), cycles, restricted size constructions,
suggesting future research in testing a broader family of data structures
than can currently be specified in languages such as Haskell.
The development of these enumerative generators was a primary motivation
behind the development of the \GC framework.

\section{The \GC Framework}

The proliferation of \QC like \pbt frameworks
suggest that while automatically generating test cases
based on a data type declaration is a popular approach,
no one test case selection strategy is going to be entirely satisfactory.
In addition,
the level of concurrency in test evaluation,
test termination conditions and the desired reporting
are all dependent on the project context
\footnote{but independent of selection method}.
Given that the complexity of software development projects
may vary from simple, single data type, single module algebras to
complex multi-programmer, multiple platform, multi-user systems,
it is hard to imagine a single approach to evaluating and reporting test results
could be optimal in all situations in which
automated test case generation for \pbt will be used.
The goal of this research is provide  a modular, flexible platform for developing \pbt programs,
providing standard interfaces and a library of components for 
test case generation, property evaluation and reporting results,
instead of cloning and modifying testing software for each new situation.

The \GC framework (chapter \ref{chap:GCimplement}) provides
the interfaces and components to build \pbt programs for Haskell modules,
incorporating a facility for both custom and mechanically derived 
automated test case generation using a wide range of sampling strategies.
A \GC test program will include:

\begin{itemize}
\item the properties to be tested,
\item one or more test case generators,
\item a test case selection strategy,
\item a test case evaluator,
\item a reporting function,
\item and the implementation modules to be tested.
\end{itemize}

\noindent The architecture decouples test case generation, 
evaluation and reporting,
and the library provides interchangeable implementations of these components.
SimpleCheck, a package similar to \QC, 
provides an example of a \GC test program,
providing several interfaces with varying  levels of control of the test process.
The interfaces between components are abstract and documented to provide 
support for developing new components,
hopefully forming the base for an extensible, open source environment to 
support ongoing research into property based testing methodologies and tools.

\section{ Contributions }

The contributions documented in this thesis are:

\begin{enumerate}

\item {Provide a critical review of \pbt systems,
identifying the valuable, shared functionality and 
the limitations of each of these systems. 
}
\item { Identify the importance of sampling methods and 
test case generation strategies based on term complexity as
a unifying concept behind property based testing systems
(chapter \ref{chp:propertytesting}). 
}
\item{Applying the 
and the importance of these hypotheses in evaluating test strategies. 
}
\item{ Providing detailed definitions for
term complexity based sampling generators for algebraic data types,
generator composition and substitution for mutually recursive and polymorphic types
}
\item { Implement an approach to developing
type independent test strategies
using standardized sampling generators. 
}
\item{ Introduce the use of enumerative sampling generators for
use in automated test case generation.
This includes adapting the background theory of combinatorial structures
for the algebraic data types commonly found in functional programming languages.
Adapt these algorithms for constructing enumerations,
based type definitions, for use in Haskell programs. 
}
\item{ Develop requirements for property-based testing software
with automated test case generation for Haskell,
including the decoupling of test case generation, evaluation and reporting components. 
}
\item{ Concrete realization of the \GC test framework in Haskell (and published to Hackage).
}
\end{enumerate}


\section{Thesis Layout}

This thesis motivates and defines \GC,
a modular framework for the development of property based test programs for Haskell modules,
using standard interfaces and components for automated test case generation,
test evaluation and reporting.
It is based on the application of sampling theory to algebraic specifications,
a critique of existing testing packages,
and the theory of combinatorial structures (\cite{FlajoletSedgewick2009}).
It provides a common model for complexity based test case generation for algebraic data types,
a library of generator combinators,
and a module to construct \emph{enumerative} generators
for the most common Haskell types 
for a variety of sampling methods.
It also provides a model for constructing type independent test strategies
using standard sampling methods
over the default enumerative generators or customized replacements.
This framework allows the creation of property based tests and testing systems;
an example called SimpleCheck is provided,
as well as a \GC compatible version of \QC.
The results of this thesis should also apply to 
other similar functional languages such as the ML family,
and to a lesser extent other types of programming languages.

The thesis is laid out in chapters as follows:

Chapter 1: introduction.

Chapter 2: defines property based testing and 
considers the interpretation of the test results
relative to the test selection strategy applied.

Chapter 3: a critique of some 
existing property based testing systems
incorporating automated test case generation.

Chapter 4: how formal sampling methods can be applied to algebraic data types,
and how these can be combined to create a wide variety of test case selection strategies.
Stratification by term complexity is presented as
an appropriate way to sample recursive types,
and several measures of complexity are presented and compared.
Post generation substitution is introduced as 
a way to independently generate the shape and elements of data structures
with different sampling methods.

Chapter 5: introduces combinatorial enumerations and
enumerative sampling generators as
a standardized way to sample and generate the most common Haskell data types.
This includes a theoretical basis for these enumerations,
a proof of their existence for systems of (mutually) recursive algebraic data types,
and an efficient means for iteratively constructing enumerations
which can be mechanically derived from the definition of an algebraic data type.

Chapter 6: provides a formal definition of a property based testing system,
and the requirements for developing a modular framework,
a set of interfaces and a library of standardized components
to support their development.

Chapter 7: describes the \GC framework for the development of property based testing systems,
including the interfaces, sample evaluation and reporting components,
an implementation of enumerative generators,
and some example property based testing systems
built on this framework that demonstrate how 
the \QC family of packages are superseded by this framework.

                  % property based testing, quickcheck et al


%------------------------------------------------------------------------------
\setcounter{figure}{0}
\setcounter{equation}{0}
\setcounter{table}{0}
\chapter{Property Based Testing}\label{chp:propertytesting}

Software testing can be seen as a sequence of experiments
designed to test the \emph{hypothesis} that
a software system works correctly.
While this is a special kind of experiment,
the general scientific method can still be considered
to provide the foundations for the design and interpretation of software tests.
Scientific methods, and in particular statistical sampling theory,
can provide insights and guidance to developing
test case selection strategies and 
the interpretation of test results.

This chapter establishes a conceptual framework
for the discussion of property based testing systems
and test case selection strategies.
It provides formal definitions for properties, property based testing,
and the interpretation of test results.
The relationship between general experiments,
general software testing and  property based testing
is explored to provide the context for
the use of formal sampling methods in test case selection.
Finally a formal test hypothesis is presented that defines
the assumptions necessary for accepting a software artifact as correct
given only a finite set of test cases.

\section{Testing in Science and Software}
Experiments provide \emph{empirical} evidence
supporting or refuting the correctness of a theoretical model of a system \cite{Holt1982} .
A test is a form of experiment in which 
a \emph{test hypothesis} is formed from the model,
relating antecedent variables (known values) to 
consequent variables (results).
The system is observed in conditions representing
a variety of inputs (antecedent values),
and compared to the predicted output (consequent) variables.
The hypothesis is \emph{testable} if it is possible to 
either control or measure the conditions represented by the antecedent variables,
and observe and compare those represented by the predicted consequent variables.
The comparison identifies each observation as a supporting or refuting result,
forming the \emph{verdict} of the experiment.
This empirical evidence is combined with 
assumptions about the validity of the experiment
to form an argument for or against the test hypothesis.

The collection of all combinations of the antecedent variables
allowed by the model is called the population of the system.
Theoretical models are powerful in the sense that
they easily represent infinitely many conditions, 
including conditions that are difficult or impossible to achieve in the real world.
Real world experiments are constrained by 
what is possible and the resources available to conduct them,
and so constrain the conditions that can be observed and compared.
When it is infeasible to test 
all possible conditions to which the test hypothesis may be applied,
a \emph{sample} of the \emph{population} is selected instead.

The results of the experiment will be extrapolated to untested values,
so the conditions tested should be \emph{representative} of those covered by the hypothesis.
The choice of test cases must also minimize the impact of \emph{confounding} factors,
those aspects of the real system that are 
not anticipated by the model and not controlled as part of the experiment.
The comparison must be \emph{valid} and \emph{unbiased},
meaning that it will correctly identify 
observations that refute the hypothesis (no false positives),
but correctly identify acceptable observations (no false negatives).

The interpretation of experimental evidence in science
requires careful scrutiny of both the results and 
these assumptions about the validity of the experiment,
called the \emph{supporting hypotheses} of the experiment.
In general, adding more test cases 
increases the credibility of the experiment but also the cost,
so test case selection must balance these two factors
by both accurately reflecting the real world
and selecting cases relevant to testing the predictions of the hypothesis (\cite{Holt1982}).

A software test is a special case of an experiment
but follows the same general structure.
The software test hypothesis is that a software system behaves as specified,
with the specification playing the \role\ of the underlying theory.
The specified (predicted) and observed behaviors of an actual instance of the system
(an installed program running on a suitable computing device)
are compared over a collection of inputs.
Each comparison is a Boolean valued \emph{test result},
and the \emph{verdict} of the test is the conjunction of the results,
indicating whether the test supports or refutes the test hypothesis.
If the comparison is computable,
this emph{oracle} can be incorporated into a \emph{test program} that
computes the verdict of the test,
given a finite collection of test cases.
If the observed behavior is ``close enough'' to the predicted results, i.e. there are no errors,
the verdict of the test is that the system is a correct implementation of the specification.
Assuming that the test is correctly implemented and
the tested inputs are sufficiently diverse to represent the actual usage of the system,
the program is accepted as correct.

There are different kinds of specified behaviors,
including subjective evaluation (e.g. ``attractive''or ``easily learned'')
and measurements  (e.g. completion time, memory usage, physical device movement)
which might require sophisticated observations and comparisons to validate.
Only observable behaviors can be tested, 
i.e. either the outputs of the system or some aspect of the behavior that can be measured.
Fortunately for software engineers, 
the specified behavior is often 
just a relation over discrete input and output values
that allows for reliable and automated comparisons,
but the translation of these relations from the theoretical specificatino
to the implementation can pose significant challenges.

It must be possible to predict the system's behavior using the specification, 
and then compare those to observed actual behaviors
for the specification to be testable.
This specification could consist of informal documents in a natural language,
but these require human interpretation to establish predictions
and to write or perform tests.
A more rigorous approach is to define a \emph{formal} specification,
based on formal language and system of reasoning
to define the expected behaviors of the system.
\cite{ZhuHallMay1997}  and \cite{Hieronsetal2009} present surveys of
different classifications of formal specification languages
and approaches to their validation.
One common advantage of these formal specifications is that
they support the creation of computable comparison functions,
called \emph{oracles},
necessary to create an automated test.

The specification and implementation are entirely separate entities,
with distinct collections of values and operations,
i.e. they exist in distinct \emph{universes of discourse} \cite{boole_laws}.
In order to compare the specification and implementation,
there must be some translation, or mapping, between them.
This is another advantage of formal specification languages:
this translation can be rigorously and unambiguously defined,
and in some cases automated by software tools.

The remainder of this chapter will focus on
how choices made in software test case selection and generation
can improve the credibility of the interpretation of test results
by directly addressing the supporting hypotheses of the assumption of correctness.
We review the concept of a software test \emph{context},
consisting of a test set, oracle and supporting hypotheses,
and discuss common classes of these supporting hypotheses.
We then review general sampling methodologies and
identify how these are related to these standard hypotheses,
with the goal of improving their subjective credibility.

\section{Properties and Property Based Testing}\label{pbt}
%\section{Property Based Testing}

This section provides a formal definition of property based testing
based on the definition of an algebraic specification as a theory presentation,
and an implementation as an interpretation of that presentation.
Algebraic specifications were selected as they are well suited to 
implementation in a functional programming language,
but definitions of properties could be derived for other specification languages.
This definition was motivated by Goguen's \cite{GoguenIBM1977},
but also by the more general works (\cite{BernotGaudelMarre1991'}, { \cite{Bernot1991})
defining a formal specification as a form of Burtall and Guguen institution (\cite{BurstallGoguen1977}),
which provide the basis for a formal definition of a test context.

\subsection{Algebraic Specification as a Theory Presentation}\label{formal_spec}


%\begin{df}[Formal Specification]
%Given a formal syntax and semantics:
%\begin{description}
%\item[syntax:] { a signature $\Sigma$ and a set of valid sentences $\Phi_{\Sigma}$
%that contains all of the well formed formulas built on
%$\Sigma$, variables, atomic predicates and logical connectors.
%}
%\item[semantics:] {an associated class of $\Sigma$-models $(Mod_\Sigma)$ with 
%a satisfaction predicate on $Mod_\Sigma \cross \Phi_{\Sigma}$ denoted by $\models$,
%such that $\forall M \in Mod_{\Sigma}, \phi \in \Phi_{\Sigma}. M \models \phi$ is defined
%and when true denotes $M$ satisfies $\phi$.
%}
%\end{description}
%A \emph{formal specification} is given by a pair $SP = (\Sigma, Ax)$,
%where $Ax \subset \Phi_{\Sigma}$ are the finite \emph{axioms}.
%$Mod(SP) = \{ M \in Mod_{\Sigma} \vert \forall x \in Ax. M \models a \}$ 
%is the class of models that satisfies (or validates) $SP$.
%\end{df}

\subsection{Algebraic Specification}

\begin{df}[Signature]
A (multi-sorted) signature is a pair $\Sigma = <S,\Omega>$ where
\begin{itemize}
\item S is a set of sort names (also denoted $sorts(\Sigma)$)
\item $\Omega$ is a $(S^* \scart S)$ sorted set of operation names (also denoted $ops(\Sigma)$).
\end{itemize}
Allow the following notations:
\begin{itemize}
\item S* denotes the set of finite sequences of elements of S, including the empty sequence
\item $f: s_1 \scart \cdots \scart s_n \mapsto s$ indicates 
$s_1,\ldots,s_n \in S^*, s \in S\ and\ f \in \ \Omega_{s_1,\ldots,s_n,s}$
\item $f:s$ is a short form for $f:\epsilon \mapsto s$, i.e. $f$ is a constant operator
\item $arity(f) = s_1\ldots s_n$, or $f$ has $arity$ of $s_1\ldots s_n$
\item the \emph{result sort} of $f$ is $s$.
\end{itemize}

\end{df}

\begin{df}[Generators]
$s \in S$ valued operators (elements of $\Omega$ with result $s$) are 
collectively called the \emph{generators} of $s$.
\end{df}

\begin{df}[Constructors]
A minimal set of the generators of $s \in S$
that can still generate every object in $s$ are the \emph{constructors} of $s$.
\end{df}

\begin{df}[$\Sigma$ Algebra]
A $\Sigma$ Algebra consists of
\begin{itemize}
\item an S-sorted set $\setcard{A}$ of carrier sets, the elements of which are called values
\item for all $f: s_1 \scart \cdots \scart s_n \mapsto s \in \Omega$, a function
$f_A: \setcard{A}_{s_1} \scart \cdots \scart  \setcard{A}_{s_n} \mapsto \setcard{A}_s$ 
\end{itemize}
\end{df}

equations / equality, satisfaction

\begin{df}[Presentation]
A presentation is a pair $<\Sigma, \epsilon\>$ where
$\epsilon$ is a set of $\Sigma$ equations, called the axioms of the presentation.
\end{df}

\begin{df}[Algebraic Interpretation]
Is a model that satisfies axioms .
\end{df}

\subsection{Properties} \label{formal_pbt}

A property is the implementation of one of the axioms of the presentation,
a computable Boolean valued predicate with an input parameter of the appropriate sort
for each of the universally quantified variables in the axiom.
These predicates are called the \emph{properties} of the specification
to distinguish them from the axioms defined in the specification language.
The constants and operators in the predicate must be supplied by the implementation.
If it does not provide all of them the property cannot be tested,
and  the implementation is said to be \emph{inadequate}
(an \emph{adequacy} hypothesis is part of the test context for algebraic specifications).

An implementation satisfies a property if
the property evaluates to true for all valid inputs,
namely (the interpretations of) the valid subtitutions for those variables.
This collection of values is called the \emph{property's domain}.
The implementation satisfies the specification if 
all of the axioms of the specification are implemented as properties,
and all of the properties are satisfied according to the test context.

\subsection{Formal Definition of a \pbt}

The property based test hypothesis states that the implementation is correct
if each property holds true over all allowed variable substitutions.
The elements of the property domain that are evaluated
are called the \emph{test cases},
and the value of the property is the \emph{test result} for that case.
The \emph{verdict} of the test is the conjugate of the results,
and indicates whether the test supports ($True$) or refutes ($False$) the test hypothesis.
Note that the test verdict can objectively refute the hypothesis,
but only support, not prove, that the implementation is correct;
the subjective conclusion that can be drawn from the result must be considered in the \emph{test context},
discussed in section \ref{test_context}.

\begin{df}[Property Based Test (\pbt)]
A \emph{test} of property $p:Dom(p) \ra \boolean$ is a triple $\pbt = (p, T, v)$ where:

\begin{description}
\item[test case] is an element of the property domain, i.e. $t \in T$.
\item[test set] is the collection of test cases $T \subseteq Dom(p)$ 
\item[test verdict] determines whether the test supports or refutes the test hypothesis
$$ v = (\wedge (\forall t \in T) . p(t) ) \in \boolean $$
\noindent
i.e. the conjugate of evaluation of the property over every input in the test set.
\end{description}
\end{df}


\subsection{Example: Simple Stack Module}\label{sub:ListSpec}

The implementation of an algebraic specification must be an interpretation.
A Haskell module, for example,
provides a type context that represents the sorts as types of values,
and defines constants and functions for the operators as 
exported entry points of a module.

The implementation must also satisfy the semantics of the specification,
meaning that the properties of the specification must hold over their domains.
This interpretation of the semantics is referred to as 
``loose'' semantics (\cite{MeseguerGoguen1986})/
There are a number of different algebraic semantics which can be used,
but this interpretation is sufficient for the purposes of testing.

\begin{figure*}
%\fbox {
\begin{minipage}[t]{.3\linewidth}
Sorts: Stack; Nat; Bool;
\begin{align*}
& Operators \\
& new : Stack \\
& empty : Stack \mapsto \boolean \\
& push : Stack \sprod Nat \mapsto Stack \\
& pop : Stack \mapsto Stack\\
& top : Stack \mapsto Nat
\end{align*}
\end{minipage}
\begin{minipage}[t]{.1\linewidth}
\end{minipage}
\begin{minipage}[t]{.6\linewidth}
\begin{align*}
& Axioms \\
& empty( new ) = True \\
&empty ( push( s, n ) ) = False\  \forall s:Stack, n:Nat\\
& pop( new ) = new \\
& pop( push( s, n ) ) = s \  \forall s:Stack, n:Nat\\
& top( new ) = 0 \\
& top ( push( s, n ) ) = n\  \forall s:Stack, n:Nat
\end{align*}
\end{minipage}
%} % end xbox
\caption[Algebraic specification of a stack]
{A full algebraic specification of a stack with new, top, pop and empty.}
\label{full_list_spec_ex}
\end{figure*}

One possible \emph{implementation} of this specification 
is a Haskell module exporting the operators from the specification.

\begin{code}
-- A definition of Nat is assumed to be available in the context, but is provided here for clarity.
data Nat = Zero | Succ Nat 

module Stack (Stack(..), new, empty, push, pop, top) where

data Stack = NewStack | Stk Nat Stack 

new :: Stack
new = NewStack

empty:: Stack -> Bool
empty NewStack = True
empty _ = False

push:: Stack -> Nat -> Stack
push s n = Stk n s

pop ::  Stack -> Stack
pop (Stk n s) = s
pop NewStack = NewStack

top :: Stack -> Nat
top (Stk n s) = n
top NewStack = Zero

\end{code}

A \emph{test} of the implementation against the specification 
consists of an interpretation of the predicates in Haskell 
that can be evaluated over the module, i.e. the properties,
and a collection of \emph{substitutions} for
the universally qualified variables defined as input variables
for the properties.
For example, one possible test for the properties around empty are

\begin{code}
propNewEmpty = empty (NewStack)
propPushNotEmpty n s  = (empty (push n s) == False)

testEmpty = propNewEmpty
    && and (zipWith (propPushNotEmpty.Stk) ns ss)
    where ns = [Zero, Succ Zero, Succ (Succ Zero)]
                 ss = [ NewStack, Stk (Succ Zero) NewStack, Stk (Succ (Succ Zero)) (Stk (Succ Zero) NewStack) ]
                 
\end{code}

\noindent There is only one value for the empty NewStack property,
so it is possible to confirm that the implementation is correct.
For the Push not empty property, however,
there are infinitely many possible test cases consisting of 
any stack with at least one element.
By necessity, a sample of values is tested,
and the test is considered successful if
the property evaluates to true at each test case
\footnote{The ``and'' function is the conjugate over a list of Booleans
and zipWith pulls arguments from multiple lists in sequence.}.
How much confidence can be associated with this conclusion,
given the small number of cases actually tested?

\subsection{ Test Context }\label{sub:context}

The verdict of a property based test
is \emph{evidence} that the implementation is or is not correct,
but does not in itself \emph{prove} this to be the case.
The verdict must always be interpreted relative to
the underlying assumptions about the correctness of the test procedures,
but also to the criterion use to select the test set.
A supporting hypothesis is required to justify 
the claim that a program is correct for any untested elements of its domain.
A \emph{testing context} (\cite{BernotGaudelMarre1991}, \cite{Bernot1991})
formalizes the relationship between
the specification, program, test set and the supporting hypotheses required
to draw conclusions about the correctness of the implementation
based on the verdict of the test.

\begin{df}[test context]
Given a program $P$ and its formal specification $SP = (\Sigma, Ax)$ 
a \emph{test context}
\footnote{\cite{Bernot1991} distinguishes between 
a testing context and the existence of an oracle;
these two definitions have been combined here for the sake of brevity}
is a triple $(H, T, O)$ consisting of

\begin{description}
\item[H] { a set of hypotheses about the model $M_P$ associated with P;
the class of models \emph{satisfying} H is denoted $Mod(H) \subset Mod(\Sigma_P)$.
}
\item[T] {the test set $T \subset Ax \subset \Phi_\Sigma$}
\item[$O_P$] {an oracle is a partial predicate on $\Phi_\Sigma$ making use of the implementation P;
$\forall \phi \in T. O_P(\phi) \implies M_P \models \phi$ 
(the oracle may be undefined outside of T).
} 
\end{description}
\end{df}

\noindent
In this definition, a test case is a single axiom expressed
as the equality between two ground terms.
If the specification of the axiom includes variables,
then the test case is the axiom after an allowable substitution of ground terms for the variables.
In \pbt, the oracle is the property written using the program P,
and the test cases are pairs consisting of 
a property function and a single input argument
$(p: Dom(p) \ra \boolean, t)$ 
representing the model of an axiom and a variable substitution.

In order to form an argument for correctness
the test context must be \emph{valid},
i.e. accept all correct programs,
and \emph{unbiased}, i.e.reject incorrect programs.

\begin{df}[Valid Test Context]
The test context $(H,T,O_P)$ is \emph{valid} if 
$(M_P \models H) \implies ( (M_p \models T \implies M_p \models Ax) \land
(\forall \phi \in \Phi_\Sigma. O_P(\phi) \implies M_P \models \phi)$
\end{df}

\begin{df}[Unbiased Test Context]
The test  context $(H,T,O_P)$ is \emph{unbiased} if 
$(M_P \models H) \implies ( (M_p \models Ax \implies M_p \models T)
\land (\forall \phi \in \Phi_\Sigma. M_P \models \phi) \implies O(\phi)$
i.e. if the program correctly implements the specification, 
it will be successful over the test data set.
\end{df}

In order for testing context to be of use, 
it must also be feasible to conduct the test in a reasonable period of time.
A reasonable, valid, unbiased test context,
called a \emph{practicable} test context in \cite{BernotGaudelMarre1991},
is a short form for a formal argument for the correctness of an implementation.

\begin{df}[Formal Argument for Correctness]
A test context $(H, T, O_P)$ is a formal argument for 
the correctness of an implementation $P$ of specification $SP(\Sigma, Ax)$ if:
$M_{p} \models H \land \forall \phi \in T . O(\phi) \land 
((M_P \models H \land M_P \models T) \implies M_P \models Ax)) $.
\end{df}

An exhaustive test is one in which each axiom of the specification is verified,
which in the case of a property based test means the entire domain is tested.
An exhaustive test context provides the base set of hypotheses
for a point wise proof of correctness.

\begin{df}[Exhaustive Test Context]
$(H_{0}, Ax, O_P)$ is an exhaustive test context where
$H_{0}$ are a set of base hypotheses.
\end{df}

These base hypotheses are just those required to 
confirm that the oracle and test cases are developed using the implementation.
This includes (but is not limited to):

\begin{itemize}
\item the test oracle must correctly compare the observed and predicted behavior;
\item the test cases are correct interpretations of the inputs they represent;
\item the evaluation of the test and computation of the verdict are correct; and
\item if the same test were evaluated at a later time, the verdict would be the same
\end{itemize}
\noindent
These assertions are necessary for the interpretation of
any software test result.
In a broader ``real world'' context,
these hypotheses would also include assumptions
about the scope of the argument for correctness
such as repeatability of the test or implementation on multiple platforms.
The development of this base set of hypotheses is beyond the scope of this thesis,
so the definition of $H_0$ will implicitly be those hypotheses necessary 
to establish the validity of the exhaustive test context.

At the other extreme, the program is assumed to be correct without any testing.

\begin{df}[Trust Test Context] 
$(H_\top, \emptyset, undefined)$ where $M_P \models Ax \in H_\top$
\end{df}

\noindent In the trust context, 
testing is unnecessary and the definition of the oracle is not relevant,
but the very strong hypothesis ``the program is correct'' makes 
the argument very weak
(although sadly it is sometimes used in practice).
This context is included here as it represents the opposite of the exhaustive context,
thereby defining the range of test contexts.

Where exhaustive testing is infeasible because the test data set is too large, 
some compromise must be struck between 
the ``it works'' (Trust)  and ``test everything'' (Exhaustive) contexts.
One approach to developing practicable test contexts
is to start with the exhaustive (also called the \emph{canonical}) test context, 
which is known to be valid and unbiased,
and refine it in a way that preserves those characteristics
until the test data set is of a reasonable size,
so the test context is practicable.
\cite{BernotGaudelMarre1991}, defines a refinement preorder :

\begin{df}[refinement preorder]
Given two test contexts $TC_1 = (H_1, T_1, O_1), TC_2 = (H_2, T_2, O_2)$,
$TC_2$ \emph{refines} $TC_1$ (denoted $TC_2 \refines TC_1$) if

\begin{enumerate}
\item $ Mod(H_2) \subset Mod(H_1)$ (or $H_2 \implies H_1$)
\item $ \forall M_P \in Mod(H_2) . (M_P \models T2 \implies M_P \models T1)$
\item $ \forall M_P \in Mod(H_2) . Dom(O_1) \subset Dom(O_2)$
\item $ \forall M_P \in Mod(H_2) . \forall \phi \in D(O_1). O_2(\phi)  \implies O_1(\phi))$
\end{enumerate}

\end{df}

\noindent
A more \emph{refined} test context will generally have 
stronger hypotheses and fewer test cases,
but still must detect any errors that would have been caught using a less refined context.
The refined oracle must have at least the same domain and 
return the same results as the unrefined for any test case in both oracles' domains.
Successive refinements to the test context build up a set of supporting hypotheses
each of which justifies the assumption of correctness over a set of untested cases
based on a subset of those test cases that are tested.

The formal refinement preorder is introduced in this document
to provide a framework and a vocabulary for discussing
different test systems and test case selection strategies.
The fundamental concept behind this preorder is
the recognition that for each class of test cases that is removed from the test set,
there should be one or more explicitly identified supporting hypotheses
justifying the conclusion of program correctness over those untested values
in the refined test context.  
Explicitly and formally defining those hypotheses,
even in the abstract setting of these formalized specifications and implementations,
helps establish subjective confidence in 
the verdict of the test and the arguement for program correctness.

\subsection{Supporting Hypotheses}

Two types of supporting hypotheses 
(other than the base $H_0$ set required to justify even exhaustive testing
in the ``real world'')
are particularly important
in establishing practicable test contexts:
\emph{regularity} hypotheses and \emph{uniformity} (also \emph{stratification}) hypotheses.
These are discussed in some detail here,
as they form the basis for the credibility of 
sampling based automated test case generation for \pbt,
and in particular the test strategies of the \GC system,
as will be seen in chapter \ref{chp:enumgen}.

\subsubsection{Regularity Hypotheses}\label{sub:regularlity}
A regularity hypothesis states that 
if a program is correct over all of the test cases up to a certain complexity,
the it will be correct over the remainder of the domain.

\begin{df}[Regularity Hypothesis]
Let $\phi(v) \in \Phi_\Sigma$ be a well-formed formula with variable $v$ of a sort S.
Let $\| t \|_S$ be a complexity measure (a natural valued function) on the terms,
and $k$ a positive natural value.  A regularity hypothesis w.r.t. $\phi$ and $v$ is
$\forall t \in W_\Sigma . (\| t \|_s \leq k \implies M_P \models \phi(t)) \implies
\forall t \in W_\Sigma . M_P \models \phi(t)$
where $W_\Sigma$ is the set of ground $\Sigma-terms$.
\end{df}

\noindent 
Regularity hypotheses are particularly important for 
testing the properties of algebraic specifications,
which may have axioms with infinite domains over recursively defined sorts.
This hypothesis is also consistent with the small scope hypothesis (\ref{def:smallscope}), 
the fundamental principle behind model checking finite state automata.

A popular form of the regularity hypothesis is the \emph{small scope hypothesis}:
\begin{df}[Small Scope Hypothesis]\label{def:smallscope}
If there are any errors in a program,
there will be errors in the program's handling of relatively simple inputs
(\cite{JacksonDamon1996}).
\end{df}
\noindent
This hypothesis has been examined for 
a number of different programming paradigms (\cite{Andoni2003}),
and forms the basis for model checking tools (\cite{ClarkePeled1999}).
This research supports the notion that
testing resources should be more heavily allocated to simple test cases,
or put more simply: ``most bugs are shallow''.

Evidence can be presented to support the regularity (or small scope) hypothesis
and the choice of maximum complexity constant
by analyzing the the implementation code (white box testing).
Alternately, the specification properties can be analyzed 
to determine what complexity of test case should be redundant.
Generally neither of these approaches  ``prove'' such a hypothesis, however,
as noted in \cite{ZhuHallMay1997} 

\begin{quote}
This (ed. Regularity) hypothesis captures the intuition of 
inductive reasoning in software testing, but it cannot be proved formally 
(at least in its most general form) nor validated empirically. 
Moreover, there is no way to determine the complexity k such
that only the test cases of complexity less than k need to be tested.
\end{quote}

\noindent so the credibility of the hypothesis remains subjective.
There is little choice, however, but to accept some sort of limit
on the complexity and size of arbitrarily large terms.
Regularity hypotheses and this definition of complexity measure, 
have proven to be very useful in property based testing tools 
such as \QC, \SC, etc. which generate test cases up to a maximum ``size'' (complexity),
and form the basis of complexity based test case generation in the GenCheck framework.

\subsubsection{Uniformity Hypotheses}\label{sub:uniformity}

A \emph{uniformity} hypotheses is used when the property domain is 
partitioned and a proper subset of each part is tested.
A formal version of this concept was introduced in \cite{WeyukerOstrand1980},
as \emph{revealing subdomains}
and then used by \cite{HamletTaylor1990}, 
\cite{GoodenoughGerhart1975}, and \cite{BernotGaudelMarre1991}.
The uniformity hypothesis claims that
the partition consists of revealing subdomains (also called equivalence classes)
so a single test case drawn from each part forms a valid and unbiased test.
Note that the definition below does not state that there is such a partition,
only that a claim of this nature can be described as a uniformity hypothesis.

\begin{df}[Uniformity Hypothesis]
If $(H, T_1 \union T_2, O_P)$ is a valid, unbiased test context,
then $H_{U2} = \exists \phi^{*} \in T_2. O_P(\phi^{*}) \implies \forall \phi \in T_2. O_P(\phi)$ is a uniformity hypothesis,
and $(H \union\{H_{U2}\}, T_1 \union \{\phi^{*}\}, O_P)$ is a valid, unbiased test context.
\end{df}
\noindent

This formal partitioning of the test domain is
generally based on a formal specification,
an analysis of the implementation,
or a combination of the two.
For example, \cite{BernotGaudelMarre1991} details a constructive implementation analysis for 
establishing equivalence classes of test cases using the following criteria:
\begin{enumerate}
\item each individual branching condition
\item each potential termination condition (e.g. overflow)
\item very variable correctly partitioned
\item every condition implied by the specification
and conditions resulting from the data structures and operation of the program
\item predicates must be independent w.r.t. testing order
\end{enumerate}
\noindent 
Any process of trying to identify equivalence classes using a \emph{semantic} analysis of this nature
will be complicated and time consuming when applied to any practically useful program
(the paper provides a lengthy explanation of how to actually perform this task).

Software tools can assist in the decomposition into revealing subdomains.
For example, \cite{Brucker2012} introduces the HOL-TestGen system,
which partitions the domain of the property into equivalence classes
with the help of the Isabelle/HOL theorem proving system
\url{http://www.brucker.ch/projects/hol-testgen/}, \cite{HOL-testgen-UG}, \cite{Brucker2009}),.
HOL-TestGen decomposes the specification and test hypothesis 
into a collection of subdomains,
and generates a representative test case for each.
It also creates explicit the supporting hypotheses for this selection process,
based on the inductions used to arrive at the partition.

This approach to testing relies heavily on
correctly partitioning the possible inputs into revealing subdomains.
\cite{ZhuHallMay1997} provides a survey of research into test selection criterion, 
either as strategies for selecting an adequate test set,
or quantitative measures of the adequacy of a test set,
for a given specification and / or implementation.
Unfortunately, regardless of the tools or methodology used,
it is quite difficult to be sure that
all of the possible input values are represented in the test data 
(\cite{Cartwright1981}).
\cite{HamletTaylor1990} includes the following observation:
\begin{quote}
When looking for failures, it is appropriate to use peculiar test inputs. ...
But if the right subdomains are the ones that find the unknown faults,
those subdomains are equally unknown, and no system can necessarily find them.
\end{quote}

\subsubsection{Stratification Hypotheses}\label{sub:stratificationhyp}

If formal notion of equivalence classes is relaxed
by replacing the guarantee of uniformity of the test cases
with a subjective argument suggesting the test cases are similar
and likely to be affected by and expose the same errors,
it becomes a basis for using \emph{stratified sampling} methods 
(see \ref{stratified_sampling}).
Testing multiple representatives from each of the strata (groups of similar test cases)
using different sampling methods provides greater confidence
that the program is correct over the entire partition,
because it allows for unknown sources of errors within each part.
For example, \cite{HamletTaylor1990} suggests 
partitioning the domain according to a static analysis of
the specification and / or implementation, 
and then systematically (uniformly) sampling the subdomains.
It is also the justification behind random sampling.
This relaxed version of a uniformity hypothesis
will be referred to as a \emph{stratification hypothesis}.

A stratification hypothesis is similar to the kind of supporting hypotheses
used in experiments supported by statistical inference.
The two definitions below formalize 
the assumptions that are typically made 
to generalize the verdict of a software test
to the untested parts of a property's domain.
They are somewhat similar to 
the \emph{hypothesis schemes} described in \cite{Romeyn2004},
which provides a formal analysis of
the relationship between statistical partition, 
statistical hypotheses and Carnapian inductive logic.

\begin{df}[Strong Stratification Hypothesis]
If $(H, T_1 \union T_2, O_P)$ is a valid, unbiased test context,
and $T_2^{*} \subset T_2$ is a proper subset, then 

$$H_{U2} = \forall \phi \in T_2^{*}. O_P(\phi) \implies \forall \phi \in T_2. O_P(\phi)$$

\noindent
is a stratification hypothesis,
and $(H \union\{H_{U2}\}, T_1 \union T_2^{*}, O_P)$ is a valid, unbiased test context.
\end{df}
\noindent
This version of the stratification hypothesis is ``strong'' because
the local verdict formed from the strata's sample is independent
of the stratification and verdicts from the other samples.
A weaker version of the stratification hypothesis requires
corroboration of correctness from the localized verdicts of other strata;
this correctness may in itself be a deduction based on other stratification hypotheses.

\begin{df}[Weak Stratification Hypothesis]
If $(H, T_1 \union ... \union T_{n-1} \union T_{n}, O_P)$ is a valid, unbiased test context,
and $T_n^{*} \subset T_n$ is a proper subset, then 

$$H_{Un} = \forall \phi \in T_i, 1 \le i < n. O_P(\phi) \implies (\forall \phi \in T_n^{*}. O_P(\phi) \implies \forall \phi \in T_n. O_P(\phi))$$ 

\noindent
is a weak stratification hypothesis,
and $(H \union\{H_{U2}\}, T_1 \union ... \union T_{n-1} T_n^{*}, O_P)$ 
is a valid, unbiased test context.
\end{df}

The small scope hypothesis is a form of weak stratification hypothesis:
the verdict of the test over ``simpler'' test cases
supports reduced testing of more complex test cases.

Regularity hypotheses are an extreme form of stratification hypothesis:
if the program is correct up to the set input complexity,
it is correct everywhere.
This kind of stratification hypothesis is a basic,
although frequently implicit, principle behind 
many test case selection strategies.
Test systems using syntactic approaches to generating test cases 
(e.g. \QC, \SC, \GC) rely on stratification hypotheses rather than uniformity hypotheses
because there is no attempt to justify the partitioning of the property domain 
into equivalence classes,
just an assertion that test cases of similar complexity and content will uncover the same errors.



\section{Test Case Selection}\label{swtest}
%\section{Software Testing}

The variables that may affect the outcome of an experiment
must be controlled in some way in the data selection process.
There are five categories of options for dealing with factors in an experiment (\cite{Holt1982}):
\begin{description}
\item[randomized]{Randomly select values for this variable 
from some probability distribution.}
\item[systematically varied]{Deliberately select a variety of values for the variable.}
\item[measured]{If the variable cannot be controlled but can be measured,
the measurement can be included as part of the test result,
and a reasonable diversity of values can be argued post hoc.}
\item[fixed]{A single value for this factor can be selected and maintained for the entire test,
so that variations in it do not confound the results.}
\item[ignored]{If the factor cannot be set, measured or controlled, 
it must be ignored and acknowledged as a weakness in the test.}
\end{description}
\noindent
Those factors that are expected to directly affect the observations
will generally be varied as part of the experiment,
or will be measured under sufficiently different conditions
to create a diversity of values for the observations.
Factors that are assumed not to impact the system's behavior,
or cannot be directly controlled,
will either be fixed and documented as part of the experiment,
or measured and recorded as a possible confounding factor.

These factors can be applied to software testing,
and in particular to ``real world'' testing,
as opposed to the abstract formalisms of the previous section.
In an properly performed actual software test,
the choices of harware, operating system, software tools, etc.
may be fixed and measured (all tests performed on the same system)
or systematically varied (repeated on muiltple platforms with different configurations).
Alternately, the configuration of the test platform can be ignored,
with consequences well known to experienced practitioners.
Similarly, these factors can be considered in the choice of test cases.

\subsection{Selection Criteria}
The \emph{population} of a software test experiment consists of 
all possible \emph{valid} test cases,
i.e. all combinations of inputs and system states allowed by the specification.
Test cases are composed of one or more component values,
and these values can usually be controlled and varied as part of the test.
These values may be systematically or randomly varied
and combined to create a diversity of test cases.
\emph{Invalid} test cases may also produce a behavior in the system,
but since the behavior is unspecified, 
invalid cases should not be considered in the verdict.
The subset of the test case population that is 
evaluated during the test is called the \emph{test set} 
(or \emph{test suite}).

A test is \emph{exhaustive} if every case is tested,
i.e. the test set is the entire population.
A successful exhaustive test is effectively a point-wise proof
that the implementation is correct,
when combined with the hypotheses supporting the validity of the test process.
Unfortunately such a test is only possible for the simplest of specifications
due to the large number of possible inputs for even a moderately complex system.
Note, however, that exhaustive testing can be performed on a subdomain of the population,
which is a particular effective strategy according to the small scope hypothesis (\ref{def:smallscope}).

When exhaustive testing is not feasible,
a \emph{selection criterion} is applied to the population of test cases
to define a test set that is feasible to evaluate.
The selection criterion should inspire confidence that 
the software is correct over the untested cases,
but this claim is generally subjective and often contentious.
An intuitive starting point for defining the adequacy of a test case selection criterion
is provided by \cite{GoodenoughGerhart1975}:
\begin{df}[Reliable Selection]
A selection criterion is \emph{reliable} if 
all tests that satisfy the selection criteria will produce the same test result.
\end{df}

\begin{df}[Valid Selection]
The selection criterion is \emph{valid} if 
for any error the selection will include a test that shows that error.
\end{df}

\begin{theorem}[Fundamental Theorem of Testing]
There exists a reliable, valid selection criterion and a test data set satisfying it 
such that that all errors in a program will be detected.
\end{theorem}

\noindent
Unfortunately, \cite{Howden1976} demonstrated that
there was no computable way in general to provide a reliable selection criterion.
\gordon{review Howden, expand on this thought}

In the absence of a computable, theoretically ideal, test selection criteria,
research has focused on practical approaches to providing adequate coverage
(\cite{ZhuHallMay1997}).
As stated earlier, the complementary goals of testing are to find errors
but also to create confidence that a successful test implies the program is correct.
Confidence is intangible and difficult to quantify,
but in general a diversity of evidence positively contributes to it.
This is supported by  \cite{Cartwright1981}:

\begin{quote}
A sophisticated formal testing system should ... 
show that the generated data is sufficiently diverse to 
establish that the program is almost certainly correct.
\end{quote}

There are many approaches to selecting test data sets,
and many approaches to evaluating those selection criteria.
\cite{WeyukerEtal1991} provides a survey of test selection criteria,
with a comparison of cost and effectiveness of different approaches.
\cite{ZhuHallMay1997} presents a survey of 
\emph{test selection adequacy criteria},
which determine the adequacy of a selection criterion 
with respect to a particular specification, implementation or interface.
In the absence of a guaranteed approach to constructing a representative test,
this is an active area of research.

\subsection{Partitioning}
One approach to strengthening the claim that 
a test selection criterion is adequate
is to partition the possible test cases into ``similar'' groups,
and then select representatives from each part for the test.
The expectation is that if there is an error in the program,
it will be exhibited over many related test cases.
If the test cases are grouped carefully,
testing the representatives should expose any errors,
or at least provide a higher degree of confidence that any errors have been found.

For example, one common approach to test case selection is
for an expert to define classes of \emph{standard uses} of a system,
and unusual situations that are expected to be problematic.
Test cases are the selected from each of these classes,
with the sample allocation (percentage of test cases per class)
reflecting the complexity and perceived risk.
This approach is commonly used in commercial software development,
and in particular for systems with a substantial amount of user interaction.
\gordon{citation}
The software engineering technique of use-cases \gordon{citation} is 
a popular form of this type of usage partitioning.
This sort of analysis is characterized by relatively high development costs,
because of the human expertise involved to create each test case,
but can result in an efficient test set with a small number of high value test cases.
It is also limited by the human ability to comprehend complex systems,
even when supported by tools and methodologies,
as noted in \cite{HamletTaylor1990}:
\begin{quote}
Quantitative results about partition testing are counterintuitive because
our intuition is untrained in confidence testing.
To guarantee high confidence in even medium-scale software requires very large test sets ....
and for this situation our intuition fails.
\end{quote}

Another approach to partitioning is to group the test cases
strictly on the values as terms, i.e. the syntax of the input case,
without reference to the specification or implementation.
\cite{ZhuHallMay1997}) refers to this as \emph{interface based} testing,
but it is also referred to as \emph{type based} testing for 
typed specifications (\cite{Hieronsetal2009}).
Intuititvely, it would seem that ignoring 
both the semantics of the specification and 
the information contained in the implementation
would result in an inferior test set.
This approach, however, is more amenable
to automated test case generation
and can be used to create much larger test sets
than more sophisticated partitioning strategies
\emph{relative to the cost}.
The cost of these approaches are weighted more heavily to system resources
than to human developer time, which is generally preferable.
Large test sets using selection criterion that are unbiased by
an interpretation of the specification or a particular implementation,
such as random test case selection,
have been shown through empirical studies to be as good as
more rigorously defined partition based tests 
(\cite{DuranNtafos1981}, \cite{HamletTaylor1990}).

When an input variable takes on heterogeneous values,
it may be appropriate to partition those values into like groups,
and then select samples according to the importance of those groups.
For example,
the special values |Nan, Infty, NInfty| etc. of the IEEE floating point types
are likely to expose different errors than regular numbers;
similarly, values with extreme exponents might
expose different errors than those with extreme mantissas.
If a data type has some sort of internal ``type'' value,
such as a record in a personnel data base that is either a ``manager'',
``employee'' or ''contractor'',
it may make sense to partition the domain to ensure each is sampled.

\subsection{Infinitely Large Test Domains}
One of the significant challenges of software testing
is that specifications often describe infinitely many possible inputs,
such as arbitrarily large integers or recursive data structure of arbitrary size.
Any feasible test must include only a finite number of elements
in its sampling frame.
Furthermore, a ``real'' system cannot represent arbitrarily large elements,
so there must be a limit on the complexity of the terms that can be tested.
These problems arise frequently in Haskell programming and other functional languages
in which recursive data structures are commonly used.

In general, it is less costly to test smaller or simpler cases,
and there are fewer of them.
Fortunately, experience suggests that
most errors in a program occur with relatively simple test cases,
the \emph{small scope hypothesis}.
This hypothesis assumes that
there is some measure of \emph{complexity}
that can be applied to the input values of a test,
and that any errors are likely to occur in the terms of lower complexity.
A natural extension of this heuristic is that
there must be a level of complexity such that
if the program is correct for all simpler inputs,
it can be assumed to be correct for all possible terms.
This is described formally in section (\ref{pbt})
as a \emph{regularity} hypothesis (\ref{sub:regularity}, \cite{BernotGaudelMarre1991}).
Since for any finite test set and measure of complexity
there must be a maximum complexity to the test set,
a supporting hypothesis of this sort is generally necessary
in the interpretation of a test verdict.

The necessity of a regularity hypothesis to manage infinite property domains,
and the benefit in focusing on less complex elements of the domain,
suggest that an automated test generation strategy should incorporated a complexity measure
to partition the test cases and determine sampling methodologies.
\GC enumerative generators (chapter \ref{chp:enumgen}) provide that complexity measure
which can be used to select sampling methodologies (\ref{sec:sampling_theory}).

\section{Sampling Theory}\label{sec:sampling_theory}
%\section{Sampling Theory}

If a software test is a kind of experiment,
then \emph{sampling theory},
the study of establishing selection criteria
in scientific and engineering experiments,
may be of use in establishing test selection criteria.
This section provides a brief summary of sampling theory
based on an introduction for non-statisticians provided by \cite{Stuart1968};
additional references are made to the more technical \cite{Cochran1977} as needed.
Although the statistical theory supporting these methods may not always apply,
it provides valuable guidance in understanding
the implications of using \emph{sampling methods}
to select software test cases.

In statistical theory, 
an experiment that estimates the
proportion of a population that
exhibits a certain value or range of values for a characteristic
is called a \emph{proportion} test (\cite{Stuart1968}).
A property based software test
estimates the proportion of the property's domain
that would evaluate to false, 
i.e. the proportion of inputs that would result in an error.
In general, the expected proportion of erroneous inputs should be $0$,
but more sophisticated and complex software development strategies
may have a higher tolerance for error.

Statisticians have developed sophisticated estimates of
accuracy, bias and variance of the test results
based on the sampling methods used for an experiment.
Unfortunately the assumptions required to provide these estimates
do not really apply to software tests.
In particular, it is difficult to imagine a situation in which
software errors were independently distributed throughout the input cases
and not correlated amongst similar test cases,
or even have any predictable prior probability distribution that can be assessed.
While this reduces the confidence that can be acheived using statistical sampling techniques,
they are still a useful starting point for defining and evaluating test selection criteria in software testing.

\subsection{Populations and Sampling Frames}

The \emph{population} of an experiment is
the collection of all possible conditions, individuals or elements
to which the hypothesis can be applied.
The population of a software test is in part defined by
all combinations of inputs and system states allowed by the specification.
In general, however, the sample will be drawn from
a subset of the population based on fixing certain factors in the test.
Where it is undesirable, or impossible, to select values from the entire population,
test cases may be drawn from a subset called the \emph{sampling frame} of the experiment.
For example,
a chemical reaction might be observed over a variety of temperatures,
but with the humidity held constant as it is not expected to impact the results.
In this case, 
the \emph{sampling frame} of the experiment 
is the range of temperatures and the fixed value of the humidity (as well as any other factors).
\begin{df}[Sampling Frame]
The sampling frame is the subset of the population
that will be considered by the test selection criterion (\cite{Stuart1968}).
\end{df}

In a software test,
the sampling frame is a combination of the external ``real'' factors that will be fixed for the duration of the test
(e.g. the test system) and the subset of the possible inputs that will be considered as possible test cases.
This will include limits on test case size and composition,
but any and all factors that might cause or be correlated with errors
must be considered in defining this population,
not just the values of the input and state variables.
The hardware platform, operating system versions and configurations,
compiler versions, other uses of the test system,
may all impact the results of a test.
These factors may be fixed, 
or may be varied by performing the test on different systems,
but they must be considered as forming distinct dimensions in the property domain
and the restriction to a single test system as part of the sampling frame.

For the sake of clarity (the word test is very fungible in the English language)
a single evaluation of a test program,
on a particular test environment and with a defined sampling frame,
will be called a \emph{test run} or \emph{test pass}:

\begin{df}[Test Run (Test Pass)]
The evaluation of a given test oracle over a test set,
drawn from a single sampling frame,
against a single installed instance of the software to be tested.
\end{df}

\noindent
The final verdict of the test is then
based on the results from each of the test passes,
much as a scientific study with multiple independent experiments,
to achieve sufficient diversity in those factors to establish
confidence in the results.

The remainder of this section describes various sampling methodologies
that can be applied to a sampling frame to generate test cases for an experiment,
but will not discuss the ``real world'' factors further.

\subsection{Random Sampling}
The most basic approach to establishing an unbiased sample is \emph{random} sampling,
where each element in the sampling frame is assigned a known, 
non-zero probability for selection.
Elements are selected ``at random'' based on its assigned \emph{probability},
the ratio of its selection weight to the accumulated weight of the population.

One of the important advantages of random sampling is that 
it prevents \emph{selection} bias, 
and in particular provides a good way of sampling unknown confounding factors.
It allows for uneven weighting
given to each element in the frame,
but the weighting must be known and objectively assigned.
A disadvantage of random sampling is that it  either allows duplicate test values (sampling with replacement),
which must then be evaluated or detected and removed,
or requires a more expensive algorithm for avoiding duplicates (sampling without replacement).

Establishing and maintaining known weightings for the random selection process 
can be very challenging in some contexts,
especially when sampling without replacement.
A \emph{pseudo-random} sampling method allows
the relative weights of selecting the individual elements 
to be unknown or change over the course of the sampling process.
Pseudo-random sampling methods may exhibit a \emph{systemic bias},
where a subset of elements will have a very small or no probability of being selected,
reducing the value of tests based on these samples%
\footnote{Note that this use of the term ``pseudo-random'' is distinct from
acknowledging the fundamentally deterministic nature of 
software based random generators.}.

\subsection{Systematic Sampling}
A sample is selected \emph{systematically} by
establishing some sort of structure or order over the population such that 
close elements are similar in some way, 
and distant elements are dissimilar.

There are different methods of sampling from such a structure, such as:

\begin{description}
\item[exhaustive] the entire domain is included in the test;
\item[uniform] pick values at uniform intervals over the index or structure;
\item[near-uniform] similar to uniform sampling, 
but randomizing the interval between selected elements to avoid systemic bias%
\footnote{This might be considered a pseudo-random sampling method,
but would only be a true random sampling strategy if 
all elements had a known non-zero probability of selection.};
\item[boundary] select some or all of 
the values near the boundaries of the structure or index
(may also be called extreme).
\end{description}

\noindent
Non-exhaustive systematic sampling may introduce an exclusion bias,
because certain values may have no possibility of being included in the sample.
For example, consider the specification $\forall n \in [1 .. 100]\ .\ f(n) = n$ 
and implementation |f n = (n `div' 2) * 2|;
if the sample is selected by uniformly taking every second element,
i.e. |[2,4,...,100]|, the test will fail to identify problems which will occur for odd integers.

Systematic sampling avoids the duplication inherent in random sampling,
which is particularly important when the sample size is large with respect to the domain.
Uniform and near-uniform sampling methods also guarantee a \emph{diversity}
of values from a sample, something not guaranteed from random sampling.
Systematic sampling techniques often embody valuable heuristics
based on engineering experience,
such as the recognition that boundaries and extreme values are 
more likely to expose errors.

\subsection{Stratified Sampling}\label{sub:stratified_sampling}

\emph{Stratified} sampling involves \emph{partitioning} the sampling frame such that 
test cases of the same stratum are expected to behave in a similar way.
Representatives are selected from each of the strata independently,
using either a random or systematic sampling methods.
The strata can, but need not be, represented equally in the sample,
although the weight given to each value in a given stratum should be roughly equal.
The ratio of the number of elements drawn from a stratum
to the total size of the sample is called its \emph{sampling fraction};
the distribution of the sample across the strata is called the \emph{sampling allocation}.
The choice of sampling allocation is part of the stratified sampling method;
\cite{Podgurski1999} suggests \emph{proportional allocation} is the best approach,
but other approaches include setting the sampling fractions
according to the perceived level of risk or heterogeneity in the strata.

The value of independently sampling the stratum
is dependent on the similarity of the elements of each stratum,
and the distinctiveness between the different strata.
Stratified sampling allows different parts of the population (sampling frame)
to have different sampling fractions and sampling methods.
As an example, 
\cite{Podgurski1999} presents the results of applying
stratified random sampling in the allocation of test cases
during an interactive beta test,
with the stratifying based on traces of the implementation.
Stratification generally improves the quality of the test,
according to \cite{Stuart1968} (pg 51):

\begin{quote}
 Stratification with uniform sampling fraction
 \footnote{taking equal numbers of elements from each stratum}
 almost always increases precision.
 \end{quote}
 
 \noindent 
The general impact of stratification on statistical estimators 
can be seen in \cite{Cochran1977}.

\subsubsection{Other Sampling Methods}

Any selection criterion may be described as a sampling method.
Several alternative sampling methods (\cite{Cochran1977}, pg. 10) are :

\begin{description}
\item[purposive] the expertise of the practitioners suggests 
the choice of test cases to form a representative sample;
\item[opportunistic] sample consists of existing observations,
as opposed to those performed as part of an experiment;
\item[haphazard] in the absence of any established selection criterion,
test cases are arbitrarily included in the sample
without regard to representation.
\end{description}

\noindent
Although these methods reduce the benefit of statistical analysis,
they can be quite valuable depending on the nature of the experiment.
Opportunistic and haphazard sampling are most often used in tests with high costs,
and are more common in fields other than software testing.
Purposive testing is probably the most common form of commercial software testing,
as it includes the manual writing of test cases,
and it should be noted that any test system should allow for the inclusion of purposive test cases
(e.g. allow a file of hand written test cases to be added to the test set).

\section{Sampling Strategies}
A test selection criterion is used to identify a test set from a sampling frame,
and a sampling based test selection criterion
uses one or more formal sampling methods to do this.
The test set, however, is a theoretical construction,
with no order or organization to the test cases it contains,
factors that are quite significant in a system that supports the creation and evalution of tests.
An implementation of an actual test selection criterion within a test program
will be called its \emph{sampling strategy}.

\begin{df}[Sampling Strategy]
A sampling strategy is an implementation of a test selection criterion within a test system
and is a combination of:

\begin{itemize}
\item a generator capable of creating the test cases in the \emph{sampling frame},
\item a \emph{selection criterion} consisting of one or more sampling methods for selecting test cases,
\item the \emph{organization} of the test cases to support an analysis of the results,
\item and their \emph{priority} for evaluation.
\end{itemize}
\end{df}

\noindent
The test case selection criterion is
applied to the sampling frame established for the test
in order to define the test set.
The priority of the test cases may form the order 
in which the test cases are stored and / or evaluated,
which is important if the test will be terminated once an error is identified,
but should not otherwise influence the results.
The way in which the test results are organized
is important in the \emph{coverage analysis},
the demonstration that the test data set is representative of
the sampling frame of the test.
This is generally the most important and contentious component
of the argument for program correctness,
so the sampling strategy needs to align with the coverage analysis.








%------------------------------------------------------------------------------
\setcounter{figure}{0}
\setcounter{equation}{0}
\setcounter{table}{0}

\chapter{\pbt Systems}\label{pbtsystems}
The previous chapter developed 
a formal definition for a property based test of a specification,
and a formal argument for program correctness 
using the evidence from such tests with supporting hypotheses.
This chapter uses these definitions as the basis for
an overview and critique of some existing property based testing systems.
Three generally contrasting approaches to property based testing are considered:

\begin{enumerate}
\item assertion based testing;
\item tools that analyze the specification and / or implementation to 
either derive minimal test contexts or ensure minimal coverage of the code by the test;
\item tools that use sampling methodologies 
and generic programming strategies to 
automatically generate test suites.
\end{enumerate}
\noindent
This last category,
which includes \QC (\cite{Claessen2000}) and related testing packages,
is the focus of the thesis
and so will receive a more comprehensive treatment.
The first two categories have been addressed 
to highlight alternatives to interface based \pbt.

\section{Assertion Based Testing }
\gordon{ Does this add anything?  I never really refer back to this section.}
In assertion based testing (\cite{Hoare1969}),
invariants and pre and post conditions from the formal specification
are inserted directly into the implementation as executable \emph{assertions}.
Assertions are conditional statements that are evaluated at run time, 
raising an exception and terminatng if the assertion fails
\footnote{These assertions are not to be confused with 
the supporting hypotheses of a test context,
which is not part of the test program.}.
The process of translating the axioms of the specification 
into assertions for run-time checking is 
similar to that for properties in a property based test.
This is also called \emph{design by contract},
because the invariants and pre/post conditions form a ``contract'' to be fulfilled by each operation.

Assertions  are part of the implementation,
while in other forms of \pbt the properties are 
in an independent test program entirely separate from the implementation.
This results in a number of differences between the two styles:

\begin{enumerate}
\item embedding the assertions in the code ensures programmers will 
see the pre/post conditions and invariants as they work,
while for property based testing these are (usually) in separate modules;
\item properties can only reference 
their input arguments and the components of the specification
but assertions can also incorporate the implementation's \emph{internal state} values;
\item assertion test cases are generated by the regular use of the system,
while test cases must be separately generated for property based testing
\item in some programming languages (e.g. Java),
assertion statements can also be used for error handling,
blurring the line between testing and regular use
\item assertions either continue to incur 
run-time checking costs during regular operations,
or a version of the system without the run-time checks must be produced,
weakening the value of the testing and possibly introducing errors.
\item If the assertions are dependent on persistent internal state variables,
the testing must be designed in such a way as to indirectly manipulate those values,
which may be challenging.
\end{enumerate}

\subsection{Example: xUnit framework}

The xUnit test framework (\url{http://sourceforge.net/projects/xunit/})is 
a language independent interface to support assertion based testing.
Originally based on the work of Kent Beck for the Smalltalk language
(\url{http://www.xprogramming.com/testfram.htm}),
variants of xUnit have been implemented for a number of languages, 
including JUnit for Java and HUnit for Haskell.

Test cases are written in 
the native language of the module or program to be tested,
using the language specific implementation of 
the xUnit interface to implement the assertions.
The programming interface supplies functions to initialize and clean up
the underlying state of the system before each test.
A test is successful if it does not throw an exception,
and any reporting of the successful test cases must be
provided by the test developer,
or observed directly from the test program source.

xUnit is useful for repeating tests and testing across multiple platforms.
Combined with a source code repository (such as GitHub or SVN) it
can also be used to share test programs between project participants.
It does not, however,  assist in developing the test cases or provide any information 
about test coverage other than identifying the failing test case that caused the exception to be raised.

\subsection{Eiffel}
\gordon{If this section gets left in, then I need to review how Eiffel enforces program correctness based on the specification, and mention that here.}

Eiffel (\url{www.eiffel.com}) is a programming language built around 
the ``design by contract'' programming paradigm.
Pre-conditions, post-conditions and invariants 
are explicitly defined in the language as part of the class definition
(Eiffel is also object oriented).
Sub-classes inherit these conditions and invariants,
and can only modify them to weaken pre-conditions
and strengthen post-conditions (invariants of the superclass must be respected).
These conditions and invariants are labelled 
with unique names for tracking purposes.
Unlike the xUnit example,
the different kinds of assertions are clearly distinguished in Eiffel,
and specifications are clearly distinguished from error handling.


\section{Integrating Static Analysis and Test Systems}

Static analysis refers to the examination of 
the specification or implementation of a system
to determine an optimal test set.
Theorem provers may be used both to 
identify a decomposition of the property's domain
into equivalence classes (page \pageref{sub:uniformity}),
and proofs supporting that decomposition.
Code analyzers examine what parts of the implementation
(source code, state variables, conditions) are executed during a test
to ensure a minimal coverage of the implementation's components,
for example ensuring that every line of code or each branch is run at least once.
This analysis can be done independently,
but some test systems have incorporated these concepts
to validate the test data set or even generate the test cases.

Both of these approaches provide a useful insight into the value of a test.
However, as noted in section \ref{swtest}, 
the value of a specification analysis must 
always be viewed with some skepticism
due to the differences between the idealized program and
the real world systems that implement it:
\cite{JacksonDamon1996} noted that ``good provers tend to be bad refuters''.
Code analyzers are particularly useful for 
ensuring implementation details not addressed by the specification
are covered by the test,
but are specific to an implementation and may not be applicable
to any modifications to that code.
Despite these limitations,
these approaches can be useful as one part of a testing strategy,
and can be incorporated into \emph{stratified sampling} methods.

\subsection{Example: \DAISTS}\label{sub:DAISTS}

The Data Abstraction, Implementation, Specification and Testing System (\DAISTS) (\cite{GannonEtAl1981})
was one of the earliest property based testing tools,
and also incorporated code analysis to evaluate the test.
It allows the algebraic specification of an abstract data type
to be augmented with axioms written in the \DAISTS specification language,
and set of test values for each of the axioms.
The implementation (the data type representation and methods),
the specification axioms and the test cases (``test points'') are stored in the same source file,
which is compiled into a test program (``test driver'')
that evaluates each property over each test point.
The coded axioms are written as boolean valued properties,
and express an equality relation between 
the composition of two or more functions and an expression.

The authors stress that successful tests provide 
limited confidence in program correctness,
and that this confidence is directly associated with test coverage.
Since the test points were manually written by the test developer,
\DAISTS implements structural test criteria
to analyze the implementation and test values to ensure:
\begin{itemize}
\item all lines of code in the method implementations are executed in the property test
\item all branch conditions in the axiom are evaluated
\item all sub-expressions and variables in the implementation and axioms also change 
at some point during each test
\end{itemize}
\noindent This attention to test coverage,
as opposed to the mechanics of defining and evaluating the tests,
is a distinctive feature of \DAISTS.

The strength of this system was in providing the specification as 
a collection of mechanically verifiable properties and test cases that
the implementation must satisfy.
In general, they concluded that 
testing with \DAISTS avoided common errors in writing test programs,
and that it did not take any longer to write the modules with \DAISTS because
the additional time was offset by not having to debug the test programs.
In particular , their case study suggested that the structural analysis of the axioms and implementations
forced the participants to include boundary and extraordinary cases in their tests,
which was shown to be productive in finding errors.
The authors noted two weaknesses: axioms were limted to equalities where 
the left hand side was a composition of functions,
and the implementation had to be complete
instead of allowing parts of the specification to be tested independently.

\subsection{Example: \HOLTG }

\emph{\HOLTG} (\cite{Brucker2012}) is an interactive tool for 
automatically generating test cases for a property written in higher order logic (HOL).
It uses the Isabelle/HOL theorem prover to 
partition the property's domain into equivalence classes (section \ref{sub:uniformity}),
and then creates and runs a test case for each.
It also explicitly creates the regularity, uniformity and independence hypotheses
\footnote{A test is \emph{independent} if the result is independent of the order of the test cases.})
from the proof strategies used by Isabelle/HOL to obtain the equivalence classes.
Concrete representatives for each of the classes
are generated and written into a test script,
which is compiled into a test program consisting of:

\begin{itemize}
\item the program to be tested,
\item a test script (written in SML) containing 
the test data and the specification as a computable predicate (the oracle)
\item and a driver that provides an interface between
the program and the compiled test script.
\end{itemize}

The test execution is encapsulated in a \emph{state error monad},
with each test evaluated in turn maintaining any necessary state,
and interrupting execution if a test fails.
The test driver makes use of the foreign language interfaces of SML
to allow calls to programs written in many different languages.

The authors state as a theorem that  \HOLTG is a complete testing procedure:

\begin{theorem}

$$TS \implies TC_1 \land \cdots  \land TC_n \land H_1 \land \cdots \land H_n$$

where $TC_i$ are the test cases and $H_j$ are the hypotheses.
\end{theorem}

\noindent The  approach of developing a customized  \emph{test hypothesis} to 
explicitly address details of the property being tested
\footnote{The authors use the colorful term ``logical massage'' to describe this process.}
was based on \cite{Gaudel1995},
and results in a specific version of the kind of 
test theory described in section \ref{formal_spec}.
The Isabelle/HOL theory library is extended with this test theory,
which models the application domain using conservative extensions,
so the resulting \emph{test context} (section \ref{sub:context}) is a formally correct test plan.

\HOLTG provides an good starting point for generating test cases,
ensuring that all of the identified subdomains of each property are tested at least once.
This approach also guarantees the test specification is sound and
provides an explicit, customized version of the test hypotheses
to create a verifiable, formal, proof of correctness.
Although only one representative from each subdomain is created and tested,
the authors note that the process required to generate the test cases
can be quite long.
This approach is well suited for systems that are near ideal,
but the heavy dependency on the equivalence classes of the program,
based solely on the specification,
makes the correctness argument fragile in a real world environment.

\section{Sampling Generators: The *Check Family}\label{sec:samplegens}

Sampling \pbt systems automatically generate large test data sets
from the definition of the property's domain,
without analysis of the specification or implementation.
A collection of software test packages that use different sampling methodologies
were developed for use by the functional language community.
Even though this approach produces test cases of lower value
than the more sophisticated static analysis based approaches,
many more test cases can be generated and evaluated with
significantly less effort from the test developer.

\QC  (\citep{Claessen2000} is a popular \pbt tool for Haskell
that automatically generates random test cases.
A number of other similar packages were
based on \QC (or developed concurrently):

\begin{description}
\item[\SC (\citep{Runcimanetal2008}) ] exhaustively generates all test cases up to the requested size;
\item[GAST (\cite{GAST2002})],  developed concurrently with \QC,
uses exhaustive test case generation but randomly orders the values
and truncates the test case sequence at a specified number of cases;
\item[\EC (\cite{EasyCheck2008}] generates test cases using a
``randomized, diagonal traversal'' to balance small and large algebraic data structures in the test suite;
\item[feat (\cite {Duregard2012}] (developed concurrently with \GC) 
provides efficient enumerations of Haskell types to support exhaustive, uniform interval and random sampling.
\end{description}

A quick note: throughout this chapter and the remainder of the thesis,
the uniformity and probabilistic nature of the ``random'' generators 
provided by software libraries such as Haskell's will be accepted
as sufficiently random for the purposes of test case generation
without further discussion.

This work is largely inspired by and based on the \QC family of testing packages,
and so they are discussed at length there.
This discussion will include comparisons of their test selection strategy,
but will also delve into the architecture and design of the packages:
user interface, property definition, test case generation, test case evaluation, termination conditions and reporting capabilities.
The design of the \GC package presented in chapter \ref{chap:source} 
resulted from this analysis.

\subsection{QuickCheck}\label{sub:quickcheck}

\QC has become the de facto testing platform in the Haskell community,
and the standard for testing publicly released Haskell packages 
through the Hackage library (\url{www.hackage.haskell.org}).
As such, a critique and comparison of \QC and the other related packages
will be useful for establishing the requirements for 
automated test case generation for property based testing.
\QC will be described in depth
as many of the features of the other packages are either 
derived from or in response to that package;
the others will receive a more cursory review.
It is assumed the readers are somewhat familiar with these packages;
those unfamiliar and interested in them should consult the original papers 
as many otherwise important details have been omitted.
In particular, this analysis was based on early version of \QC (1.2.0.0) and
does not reflect the current state of this project.

\subsubsection{Specification and Properties}
\QC is a property based testing tool for Haskell modules,
generating \emph{random} test values 
based on the input type of a property.
More formally, given a function |f| over a data-type |a|, 
equipped with a \emph{generator} of random instances of |a|,
and a property | p | relating | a | and |f a| for all | a |, 
\QC will verify that | p | holds over a ``random'' sample of values of | a |.
As an example, 
consider a new polymorphic function $reverse$ that 
reverses the order of the elements in a list.
Recalling the formal specification for the polymorphic list type $\specL_{\alpha}$ (section \ref{sub:ListSpec}),
the specification for $reverse$ consists of an interface and these properties (figure \ref{reverse_spec_ex}).

\begin{figure*}
%\fbox {
\begin{minipage}[t]{.3\linewidth}
Axioms
\begin{align*}
& (type) : \\
& unit  property: \\
& involution property: \\
& concat property: 
\end{align*}
\end{minipage}
\begin{minipage}[t]{.1\linewidth}
\end{minipage}
\begin{minipage}[t]{.6\linewidth}
\begin{align*}
& reverse :  \specL_{\alpha} \ra \specL_{\alpha} \\
& \forall l \in \specL_{\alpha}.  reverse (cons(x, empty)) =_{\specL_{\alpha}} cons( x, empty) \\
& \forall l \in \specL_{\alpha}.     reverse (reverse(\ l\ ))  =_{\specL_{\alpha}}l \\
& \forall l_1,l_2 \in \specL_{\alpha} .  reverse(\ l_1 \concat\ l_2\ )  =_{\specL_{\alpha}} reverse( l_2 ) \concat reverse( l_1)
\end{align*}
\end{minipage}
%} % end xbox
\caption[Axioms for the reverse function.]
{An algebraic specification of a reverse function for the polymorphic list.}
\label{reverse_spec_ex}
\end{figure*}

\noindent This example specification is used throughout this section
to illustrate the different facets of \QC.

The example in figure \ref{reverse_prop_ex}  (\cite{Claessen2000}) shows 
the translation of the $reverse$ predicates
into Haskell properties (Boolean valued functions).

\begin{figure*}
\begin{code}
propRevUnitEq :: (Eq a) => a -> Bool
propRevUnitEq x = reverse (Cons x Nil) == (Cons x Nil)

propRevInvEq :: (Eq a) => List a -> Bool
propRevInvEq xs = (reverse.reverse) xs == xs

propRevConcatEq :: (Eq a) => List a ->  List a -> Bool
propRevConcatEq xs ys = reverse (xs ++ ys) == (reverse ys) ++ (reverse xs)
\end{code}
\caption[Haskell properties for the reverse function.]
{The Haskell properties of a reverse function for the polymorphic list.}
\label{reverse_prop_ex}
\end{figure*}

\noindent
Each of the three properties has a different domain:
unit of type |a|, involution |List a| and concatenation the pairs |(List a, List a)|.
In order to actually run a test for each of these, 
each must be assigned a concrete type,
for example replacing the |a| type variable with |Int|.
\QC must be able to generate the selected data types,
either using default generators supplied with the package (such as for |Int|),
generators built with the supplied combinators (such as the list generator combinator),
or custom built generators.

Concatenation is slightly different because
the property has two arguments and the property is implemented in its \emph{curried} form.
\QC treats curried properties as families of properties,
and tests a random selection of that family by
generating the first instance of the first argument, 
testing the property resulting from applying that argument as if it were a single argument property,
and repeating the process with additional instances of the first argument.
Alternately, the property could have been defined as

\begin{code}
propRevConcatEq :: (Eq a) => (List a, List a) -> Bool
propRevConcatEq (xs, ys) = reverse (xs ++ ys) == (reverse ys) ++ (reverse xs)
\end{code}

\noindent which would be treated as a property with a single input argument and 
require a generator for pairs of lists of type |a|.

If a property does not apply to all of the values of it's input type,
i.e. has an associated precondition,
the | (==>) | operator guards the equality so it is only tested with satisfactory values.
This is illustrated below with an example property called | propMax | (unrelated to reverse).
which is conditional on the first argument being less than the second;
test cases not satisfying this condition will be rejected from the test.

\begin{code}

max :: (Ord a) => a -> a -> Bool
propMax :: (Ord a) => (a, a) -> Property
propMax (x, y) = ( (x < y) ==> ( max x y == y) )

\end{code}

Conditional properties provide an important tool to practical test developement,
but also highlight a significant issue with syntax based automated test case generation.
A property will always have a parameter that is a valid type in the implementing language (in this case Haskell),
but that does not mean that every value of that type will be a valid input for that property.
In other words, the domain of the property may well be a proper subset of the values of it's input type.
This can be addressed in three ways:
\begin{itemize}
\item the test generator can be constructed to only provide valid test cases
\item the test program can filter out invalid test cases
\item the results of invalid test cases can be excluded from the verdict
\end{itemize}
This subject is addressed in chapter \ref{chap:requirements}.

\subsubsection{Test Program Interface}
The default invocation of \QC is through the |quickCheck| function,
supplying the property to be evaluated as input,
generating 100 random test values and returns ``OK'' if 
the property holds for all test cases or ``Falsifiable after $n$ tests''.
|quickCheck| is a call to the |check| function, 
supplying a set of default configuration values;
this approach of providing a general purpose function with
layers of defaulted parameters
is a standard technique in the implementation of programming interfaces,
allowing the user of the library to take as little or as much control of the settings
as their understanding and needs dictate.

|quickCheck|, |check| and the other related functions are instances of the | Testable | class,
guaranteeing a property that will compute a |Result| inside of the |Gen| monad.

\begin{lstlisting}
newtype Property = Prop (Gen Result)
newtype Gen a = Gen (Int -> StdGen -> a)
data Result  = Result { ok :: Maybe Bool, stamp :: [String], arguments :: [String] }
class Testable a where
    property = a -> Property
\end{lstlisting}

\noindent |Result| contains the test verdict, the classification of the test data,
and parameters used during the call to quickCheck.
A |Property| is a |Result| computation that occurs in a |Gen| monad,
which ensures that the unique seed (|StdGen|) values required for random generation
are threaded through the sequential computations.
Properties are encoded as |Testable| Haskell functions.
Typically a property is Boolean valued,
but \QC also supports conditional properties 
(i.e. the domain of the property is defined by its type
and an additional precondition) of type |Property|.

If the input type of a property is an instance of the |Arbitrary| class,
a \emph{generator} for that type exists (see below)
and so a default instance of Testable for that property.
The forAll operator combines a generator and 
a function that produces a |Testable| result into a |Property|.  
This automatic instance and the supplied default Arbitrary instances
hide the complexity of the interface for many common Haskell types,
allowing a user to invoke |quickCheck| 
without specifying or even knowing how the test cases are to be generated.

\begin{lstlisting}
forAll :: (Show a, Testable b) => Gen a -> (a -> b) -> Property
forAll gen body = Prop $
  do a   <- gen
     res <- evaluate (body a)
     return (argument a res)
 where
  argument a res = res{ arguments = show a : arguments res }

instance (Arbitrary a, Show a, Testable b) => Testable (a -> b) where
  property f = forAll arbitrary f

class Arbitrary a where
  arbitrary   :: Gen a
  coarbitrary :: a -> Gen b -> Gen b

\end{lstlisting}

\subsubsection{Generating Data}

Generators of test values of a type |a| are provided through 
the |arbitrary::Gen a| method from the |Arbitrary| class,
based on an integer size and random generator seed (|StdGen|).
Default generators for base types such as |Int| and |Char|
use Haskell's native random generators ,
and are also supplied for common structures such as (polymorphic) lists and tuples.
The |Arbitrary| class can optionally include a dual method |CoArbitrary|
to support the generation of ``random functions'' by 
creating random relations between inputs and outputs%
\footnote{A casual review of instances of Arbitrary provided by
packages available in the Hackage system revealed
few implementations of the coarbitrary method.}.

An |Arbitrary| instance must be provided for user defined algebraic data types.
A generator for an algebraic data type has a weight assigned 
to each of its data constructors.
When a term is generated, 
one of  the data constructors for the type is randomly selected based on that weighting,
and then its arguments are each generated using calls to |arbitrary|.
\GC relies on Haskell's class system to select the correct instance for that type.
This approach works for any algebraic data type, 
including recursive and mutually recursive systems, 
as long as all of the types involved are also instances of |Arbitrary|.

The generator functions are constructed using 
the \QC  product and disjoint union generator combinators.
Some of the provided combinators are:

\begin{description}
\item [oneof] provide an equal probability of picking from a list of constructors.
\item [frequency] specify a weighted probability of picking a constructor from a list.
\item [vector] generate a vector (list) of values of the type
\item [two, three, four] generate 2, 3 and 4 tuples (products) of one type
\end{description}

\noindent
For example, an instance of Arbitrary List can be constructed using 
the generator combinator |frequency|.
This allocates an integer weight to each of the constructors;
the probability a constructor is selected is its proportion of the total weight.
In this case, any given node will be the end of the list (Nil) only 1 out of 8 times,
or with a nominal probability of $12.5\%$.

\begin{code}
instance (Arbitrary a) => Arbitrary (List a) where
   arbitrary = 
          frequency [(1, Nil)
                    (7, liftM2 Cons arbitrary arbitrary]
\end{code}

This naive instance of |Arbitrary| is not guaranteed to terminate
in any given number of recursive iterations,
although the probability of generating any given list
is inversely proportional to its size.
\cite{Claessen2000} recommend explicitly terminating recursion after 
a maximum number of steps based on the size argument
(exposed by the |sized| combinator),
by supplying an internal definition of the generating function:

\begin{code}
instance (Arbitrary a) => Arbitrary (List a) where
   arbitrary = arbList
   
arbList = sized list'
list' 0 = return Nil
list' n | n>0 = 
         frequency [(1, Nil),
                    (7, liftM2 Cons arbitrary (list' (n-1))]
\end{code}

\noindent
This approach guarantees that the structure will be 
no larger than the requested size
which is exposed using the |sized| combinator.
Managing the termination explicitly adds to the complication
of developing \QC generators for recursive types,
and in particular mutually recursive structures.

\subsubsection{Test Execution}

During test evaluation, 
the |Gen| type wraps a function from a positive integer size and 
a Haskell |StdGen| seed value to the type being generated.
|Gen| is defined as a monad and is used to generate  a stream of random values
(although it is not a true monad as it does not satisfy the monad laws)
using the bind and return definitions to manage splitting and passing on the generating seed.
The | check | program binds a sequence of value requests,
passing the random generation seed,
and slowly incrementing the size parameter from 0 using a size modification function
(defaulting to |(+3).(`div` 2)|).
This provides a sampling of the property's domain,
stratified by the \emph{size} of the term,
which is the maximum recursive depth for an algebraic data type
(sizes are ignored for scalar / base types).

Test cases are generated and evaluated sequentially,
and the | check | program terminates:
\begin{enumerate}
\item on the first failed test case,
\item after the requested number of successful test cases are generated,
\item or after generating the maximum allowable test cases that did not satisfy the pre-condition.
\end{enumerate}
\noindent
The test is invoked in the Haskell IO monad,
to obtain a starting random seed (|StdGen|)
and ensure the tests evaluated despite Haskell's lazy evaluation model.

\subsubsection{Reporting}

The default report from a test declares either success and the number of test cases,
or failure, the failed case, and the number of successful tests before the failure.
The test results can be enhanced with
information about the test data values
by using the reporting combinators | collect |, | classify | and | trivial |.
These label and classify the test values used,
providing a simple coverage analysis, 
such as a count of test cases by size.

\begin{lstlisting}
propRevInvSized xs = collect (length xs)$ (reverse.reverse) xs == xs
	  where types = x::Int

Main> quickCheck propRevInvSized
OK, passed 100 tests.
18\% 0.
22\% 1.
13\% 2.
 8\% 3.
...

\end{lstlisting}

\noindent
This simplistic approach to coverage analysis
is suitable for the ``quick check'',
used to ensure a reasonable degree of variety in the test data set.

The test context for a \QC test is based on
testing cases of different sizes up to a maximum size.
This requires a regularity hypothesis to 
justify the maximum size (number of recursions) of the test cases,
and a stratification hypothesis to justify 
the selection of random cases from the domain stratified by size.
The definition of these assertions,
and the interpretation of the test verdict,
is left to the test developer.

\subsection{SmallCheck}\label{sub:easycheck}

\SC (\citep{Runcimanetal2008})  is based on \QC, 
but instead of randomly checking test values,
it exhaustively generates all simple, or ``small'', test values.
Because the two packages are so similar,
only the differences between them are mentioned here.

\subsubsection{Specification and Properties}
Formally, given a function |f| over a data-type |a|,
an exhaustive generator of instances of |a| up to a given ``depth'' (size), 
and a property |p| relating |a| and |f a| for all |a|,  
\SC verifies that | p | holds by creating all elements of | a | to that depth, evaluating | p | at those values.

There is a |Testable| class with |Property| encapsulating the test cases in a list monad
(as opposed to \QC's Gen monad).
The size parameter for generating test cases is explicitly exposed in \SC
as an argument of the property function (the |Int| is the size).

\begin{lstlisting}
data Result = Result {ok :: Maybe Bool, arguments :: [String]}
newtype PR = Prop [Result]
newtype Property = Property (Int -> PR)
	
class Testable a where
  property :: a -> Int -> PR

\end{lstlisting}

Properties are constructed in the same way,
using the |forAll| combinator to combine a test case generator and a | Testable | predicate.
The \SC package also introduces two new features for defining properties:

\begin{description}
\item [existential variables] tests that there is at least one value
within the specified depth that satisfies the property
\item [unique variables] test that there is \emph{only one} satisfactory value to the specified depth
\end{description}

\noindent Existential quantification in a property is a unique feature of \SC and \GAST as
it can be tested more effectively with a (partially) exhaustive testing strategy than 
with other sampling methods.

\subsubsection{Generating Data}
Exhaustive generators are instances of a class | Serial |, 
the analog of \QC 's |Arbitrary| class,
and provides a |series| method for generating lists of values
and |coseries| for generating function types.
The |Series| type encapsulates exhaustive data generation
of values of size $\leq n$, and is the analog of \QC's |Gen| monad.

\begin{lstlisting}

forAll :: (Show a, Testable b) => Series a -> (a->b) -> Property
forAll xs f = Property $ \d -> Prop $
  [ r{arguments = show x : arguments r}
  | x <- xs d, r <- evaluate (f x) d ]

instance (Serial a, Show a, Testable b) => Testable (a->b) where
  property f = f' where Property f' = forAll series f

type Series a = Int -> [a]
class Serial a where
  series   :: Series a
  coseries :: Series b -> Series (a->b)

\end{lstlisting}

Test case generators for algebraic data types can be constructed using the following combinators:

\begin{description}
\item [ $><$ ] |:: Series a -> Series b -> Series (a,b) | creates a generic product generator
\item [ cons0, cons1, etc.] creates generators for data constructors of 0, 1, ... arguments
\item [$\backslash/$] :: |Series a -> Series a -> Series a | creates a disjoint union generator
\end{description}

\noindent There are no weighted generator combinators like \QC's |frequency|
because exhaustive testing includes all of the test cases up to the specified size.

The Serial generator for List is given below as an example.
the simplicity of the interface is evident as the |cons| functions
hide the recursive calls to |series| in the generator.
\SC generators do not have the non-termination problem 
associated with the \QC constructive random generators,
so calling the default |series| method for the type is usually appropriate.

\begin{lstlisting}
instance Serial a => Serial (List a) where
  series = cons0 Nil \/ cons2 Cons
\end{lstlisting}

The interpretation of the \emph{size} of a type is very important
because it directly controls the size of the test suite:

\begin{itemize}
\item for structures, the size is the recursion depth
\item for based types that are instances of the Haskell (Enum) class,
the ``depth'' is interpreted as a range of values  | [ (-d) .. d ] |,
where $d$ is the size of the test case.
\end{itemize}

\noindent
All possible substitutions for each of the elements in a data structure
are made when populating algebraic data types 
defined over one or more type arguments.
The size of the element range is one less than the size of the node of the it occupies,
e.g. if the maximum depth is 4 and the node is at depth 2,
the elements will be |[-1,0,1]|.
All of the ``shapes'' up to the specified depth will be generated,
and the nodes will be populated by all permutations of 
the element values for each of the type arguments into each node of each structure.
This is the default interpretation of the composition of lists in Haskell,
so the implementation is straightforward.

Exhaustive testing by depth leads to an exponential increase in
the number of values when generating tree-like recursive algebraic data types,
because both the number of shapes and substitutions increases rapidly.
\SC provides additional combinators to reduce the size of the generated test suite
by altering the depth or filtering the generator lists,
but of course this makes the testing non-exhaustive.

\subsubsection{Interface and Reporting}

The | smallCheck :: (Testable a) => Int -> a -> IO () | function 
generates and tests all of the values of type |a| up to the specified maximum size
(recursion depth for recursive algebraic data types, a range for base types).
Other variants of the  | smallCheck | function 
allow the tester to decide to proceed to the next size,
or test only the values of a particular size.
The results of the test provide
the first failing case if one is found,
or the number of values successfully tested at each depth.
There is no parallel to the reporting combinators of \QC
because the expected strategy of using partially exhaustive testing
means the number of values tested at each depth is sufficient coverage analysis.
There is no way to determine the test data set
if the generators are filtered so not exhaustive.
%
\subsubsection{Test Execution}
%
| smallCheck | is run in the IO monad,
either from the command line or within a Haskell module.
The result is computed as a |map| of the property over the test data set,
a sequence of lists of test values from size 0 to the maximum depth;
this is implemented as a list comprehension
for most properties (those built over the qualifiers |forAll|, |exists|, or |exists1|).

The \SC test context is again stratified by
the complexity, or size, of the test cases,
but only requires the regularity hypothesis defining 
the maximum size necessary for the test
(or this could be considered a small scope hypothesis).
Since the test is exhaustive over the test cases up to that size,
the test context provides a ``proof'' of program correctness up to that complexity
(at least within the context of an idealized test),
so no stratification hypothesis is necessary.
The regularity hypothesis, however,
is generally much stronger for \SC than for \QC as
the maximum size of the test cases must generally be much smaller
to allow for exhaustive testing in a reasonable time.

\subsubsection{Lazy SmallCheck}

Lazy SmallCheck (\cite{Runcimanetal2008}) is similar to \SC but
works on partially evaluated structures.
It wraps values with a customized |Maybe| that 
identifies a value or a labelled ``hole'' in the structure of a test case, 
combined with a mechanism for filling the hole in the structure.
The result of a test is three possible values:
True, False or a test that can be evaluated over
a the test suite generated by substituting the possible values for that ``hole''.
This \emph{refinement}, or ``drill down'', procedure allows a test 
to terminate if the property is shown to fail over all instances that can be generated from the partial value,
thus avoiding the need to evaluate larger test cases.

There are some disadvantages to \LSC when compared to \SC.
It can be significantly more efficient that \SC,
but in situations where the property cannot benefit from the partial result, 
\LSC is actually less efficient than \SC because 
it creates  and evaluates the test over the partial values as well as the ``full'' ones.
It is restricted to operation on universally qualified properties,
as opposed to \SC which also supports existentially qualified variables,
and \LSC ties test evaluation and data generation more closely
in order to allow test cases to be generated from the partial cases.

\subsection{GAST}

The GAST project described in \cite{GAST2002} is an early version of 
generic automated software testing written,
with version written in both Clean and Haskell.
It is similar to \SC in using exhaustive testing of all values up to a specified size (complexity) as the default strategy,
and the input parameters for properties can be universally or existentially quantified,
but it differs in that it randomizes the order of the test cases
by randomly traversing left or right branches of disjoint unions. 
It uses a universal tree representation type descriptor 
to describe algebraic data types (e.g. the Haskell type constructor) and 
generic programming techniques to generate and execute tests.
An analysis function processes the list of results and provides a verdict for the test:

\begin{description}
\item [Proof] if all of the values in the domain of the property were exhaustively tested;
\item [Passed] if a test data set of the requested number of cases were tested successfully
but the test was not exhaustive; and
\item [Failed] if a counter example was found.
\end{description}

The authors justify exhaustive testing up to the specified size
by noting that a function typically contains 
special cases for the small elements of the input parameter type
and recursive cases for larger elements,
i.e. that the small scope hypothesis applies for algebraic properties.
A representative test data set should therefore have 
all of the small test cases and
a selection of larger values to test the more general cases, or equivalence classes.

Randomizing the order in which values from disjoint unions are generated
removes the order bias from a test data set.
This is important in \GAST test case selection because
the test generation is truncated at the size specified by the user,
and the randomization offers a broader diversity of values in any subset of
the partitions of the domain by size.
They also suggest that randomizing the order will
decrease the expected time to find an error in the test data set (should there be one),
which is important since \GAST terminates the test after finding an error.

\begin{quote}
Moreover, small values which appear close to the root of the generic tree have to be generated first. 
A depth-first traversal will encounter these values too late. 
Finally, a left-to-right strategy will favor values in the left branches and visa versa. 
A bias in any direction is undesirable.
In order to meet all these requirements, 
we use a strategy that uses a random choice at each EITHER in the type tree.
\end{quote}

There are a number of other similarities with \QC:
\begin{itemize}
\item a |forAll| combinator to identify universal quantification;
\item a |Testable| class providing an evaluation of properties 
when a test value generator is available;
\item the conditional implication operator, used to implement pre-conditions,
allows values of the input argument type to be excluded from the test suite 
\item provides reporting combinators to build into the properties that collect / classify test cases
\item uses a stream of random values to make left / right decisions at brand points,
so requires sequential (monadic) processing with a |Gen| like monad to manage the random seed values
\end{itemize}

The authors recognized that generating test cases from 
base types with large sets of  values 
posed a particular challenge in developing efficient tests.
They recommend the use of a \emph{hybrid sampling strategy}
combining exhaustive, boundary and random value generation
(and accepting duplicate values).
%\begin{quote}
%``For finite types like Bool or non-recursive algebraic datatypes 
%we can generate all elements of the type as test-data. 
%For basic types like Real and Int, generating all elements is not feasible. 
%There are far too many elements...
%For these types, we want GAST to generate some common border values, like 0 and 1, 
%as well as random values of the type. 
%Here, preventing duplicates is usually more work than repeating the test. 
%Hence we do not require that Gast prevents duplicates here.''
%\end{quote}
%
%\noindent 
GAST also includes combinators to pull in user supplied test cases,
which seems an obvious addition but is not available in the other packages.
The test data set is truncated to the number of cases requested by the tester,
resulting in an incomplete sample of the stratum of the largest cases tested.
As a result, the test context for \GAST is similar to that of \SC,
but the regularity hypothesis may need to be supplemented by
a stratification hypothesis supporting the partial (truncated random) sampling
of the stratum of the largest test cases.

\subsection{EasyCheck}

EasyCheck (\cite{EasyCheck2008}) is written in Curry, 
and uses Curry's built in narrowing and data refinement to generate test cases.
This provides support for testing partially defined values
and terminating the testing of a particular branch of an algebraic data type if
the property can be conclusively evaluated with just that partial type,
as found in \LSC.
Curry provides support for non-deterministic types and free variables,
encapsulating the non-determinism in a ``search tree'' to provide order to the choices.
Test evaluating is based on refining and expanding the ``search tree'' that
implicitly defines an enumeration over the non-deterministic choices provided by the free variable or type.
Test inputs are defined in the same way as \QC and \SC,
but the selection process takes the diagonal from the enumeration across the branches,
randomizing the selection order of the branches.
This provides a balance between small and large tree-like structures
earlier in the order of the test cases
than the other packages which generate the test cases in order of size (complexity).

The default enumeration provided by Curry is modified by \EC with the goal of
establishing what the authors describe as a \emph{complete, balanced and advancing enumeration}:

\begin{df}[Enumeration Properties  (\cite{EasyCheck2008})]
An enumeration of the values of a type in Curry is :

\begin{description}
\item[complete] if every value is eventually enumerated.
\item[advancing] in order to avoid numerous trivial test cases,
the first node of the $n^{th}$ level of a search tree must have an enumeration index
of less than p(n) where p is a polynomial. 
\item[balanced] if the enumerated values are independent of 
the order of child trees in branch nodes
\end{description}

\end{df}

\noindent Diagonalizing the search trees into a list
that encapsulates the total ordering of the test values 
provides the complete, balanced, advancing enumeration.
The use of the enumeration is fundamental to the EasyCheck testing strategy,
providing the basis for selecting representative test cases from 
anticipated equivalence classes of the test domain.
The authors also note that more work is required to
find an efficient enumeration scheme that is complete, balanced and 
generates sufficiently diverse large test values \emph{early} in the test data set.
This complete, balanced, advancing enumeration is 
the supporting stratification hypothesis for the text context,
which is completed by a regularity hypothesis.
%\begin{quote}
%Finding an efficient enumeration scheme that is complete, balanced and 
%generates sufficiently different large values \emph{early} deserves future work.
%\end{quote}

\subsection{Functional Enumeration of Algebraic Types (FEAT)}

Functional Enumeration of Algebraic Types, or \FEAT, was introduced in \cite{Duregard2012}.
The interface is loosely based on \QC and \SC
and provides similar functionality,
but test data can be either randomly, 
uniformly or exhaustively generated (see section \ref{sec:sampling_theory} for definitions) .
It explicitly provides both ranked and base \emph{enumerations} of Haskell types,
as described in section \ref{sec:enumerations},
providing the number of values (count function),
an order (index function) and a selection function that produces the value at any given index.
For recursively defined algebraic data types
the values are partitioned by size,
which is defined to be the total number of \emph{data constructors} in the term.

\FEAT uses a \emph{selective} approach to random sampling,
as opposed to \QC's \emph{constructive} approach,
basing the selection on an \emph{enumeration} of the types values.
The enumeration specifies the number of structures of a given size,
and so allows a uniform probability distribution for
randomly selecting elements.
This is consistent with the stratified sampling methods discussed in section \ref{sec:sampling_theory}
and the use of \emph{enumerative generators} discussed in section \ref{sec:enumerations}.
Random samples are evaluated as instances of the \QC |Gen| monad,
providing a high degree of compatibility.
Exhaustive and uniformly selected (i.e. every $k^{th}$ value from the index) test data sets
are evaluated using the Haskell |map| function, as in \SC;
this use of separate monads precludes hybrid sampling methods.
The test context is the same as \QC or \SC, respectively.

The focus of the \FEAT package is providing efficient 
and easily implemented enumerations of Haskell types .
The performance is enhanced by extensive optimizations,
including \emph{memoization} of the count, index and select functions.
They claim their enumerations are faster for generating large exhaustive samples than \SC
(some empirical evidence of this is provided in the paper).
They also claim they are easier for test developers to write than \QC generators,
particularly for mutually recursive types.

%Authors notes:
%"We argue that functional enumeration is the only available option for automatically generating useful test cases from large groups of mutually recursive syntax tree types. Since compilers are a very common application of Haskell, Feat fills an important gap left by existing tools."
%"However, it is well known that reasonable QuickCheck generators are really difficult to write for mutually recursive datatypes (such as syntax trees) � sometimes the generator grows as complex as the code to be tested! SmallCheck generators are easier to write, but fail to falsify some properties that Feat can."
%"A noticeable feature of Feat is that it pro- vides random sampling with uniform distribution over a size- bounded subset of a type. This is not just nice for compatibility with QuickCheck, it is genuinely difficult to write a uniform gener- ator even for simple recursive types with the tools provided by the QuickCheck library."

\subsection {Comparison}

These  packages share many characteristics,
some of which should be included in any property based testing package,
while others seem to be weaknesses despite appearing in all of the packages.

The use of properties as the basis for the testing is simple, useful and popular.
The |Property| type in \QC, \SC, and \GAST is used to 
abstract the test result to allow 
additional information to be stored along with the pass / fail verdict.
The specification axioms are initially encoded as a Boolean valued functions,
as expected, but the |Property| type is actually 
a |Result| type embedded in the execution strategy.
This introduces  a number of weaknesses in the implementations
(as opposed to the concept):

\begin{enumerate}
\item it encapsulates the collection of test results but does not include the tested values;
\item the reporting is embedded in the property so it is necessary to rewrite 
the property function to change the coverage analysis;
\item the test generation functions are tightly coupled with the execution monad 
(e.g. the |Gen| monad for \QC, the list monad for \GAST and \SC, 
or the non-determinisic enumeration of \EC) so 
that it is difficult to add, combine or modify the sampling methods within the package.
\end{enumerate}

All of the packages assume a sequential model of test execution,
with the test terminating after the first failure.
Newer versions of \QC allow multiple failures before termination,
but still assume the tests are running sequentially.
Test cases are generated as needed by a \emph{monadic} test evaluator:

\begin{description}
\item [QuickCheck] uses the Gen monad to sequentially generate and evaluate random samples,
\item [SmallCheck, GAST] uses the list monad to order test cases of each depth,
\item [Lazy SmallCheck] uses a custom monad to manage the injection
of simulated partial values and then refine those values,
\item [EasyCheck] uses Curry's monad-like non-deterministic free variables and search trees
to generate and order the test cases,
\item[\FEAT] evaluates test cases in either the list or |Gen| monad,
depending on the sampling method used
(although the enumerations were developed so that
they could be computed in parallel to some extent)
\end{description}

Serializing the order of test case evaluation and 
terminating with the first (or $n_{th}$) error
may be efficient during the early, debugging stage of  the development process,
but:

\begin{itemize}
\item in general, the ordering of the test cases is only a heuristic
to improve the probability that an error will be found early in the test suite;
\item identifying more failing test cases might make fixing the program easier;
\item reporting errors as they occur but continuing to test
until the test is interrupted is clearly a superior solution with little or no impact on the program architecture
\end{itemize}

As the development cycle nears the end,
the focus of the test shifts to collecting and presenting evidence of program correctness,
and so achieving better coverage through more test cases and better reporting becomes the priority.
A ``quick check'' of a simple property is well served by a linear test,
but in later stages of development with much larger test suites,
concurrent test evaluation (with time limits to catch non-terminating test cases) is appropriate.
The packages combine three activities that could be de-coupled:

\begin{enumerate}
\item generating a test set
\item evaluation of properties over a test set
\item providing an appropriate report of the test results to support coverage analysis
\end{enumerate}

\noindent
Since one of the fundamental axioms of testing is that the test cases are independent,
there is no activity better suited to parallel execution than testing.
The increasing availability of multi-processor, multi-core, and multi-system (cloud) platforms
should be considered an opportunity to greatly increase the resources available for testing.
\SC, \GAST and \FEAT use Haskell's |map| function to evaluate test cases,
which might be amenable to a concurrent implementation 
(for example using parallel Haskell),
but the use of the |Gen| monad to carry the seed for the random generators
effectively locks \QC and \FEAT into sequential evaluation.
Decoupling the test case generation and execution would permit
a simple sequential model of test evaluation during the early phases of development,
with a transition to more sophisticated reporting and concurrent test evaluation
using the same properties and test case generation.

The strongest feature of these packages is their ability generate test data.
All provide a \emph{test case generator language}, 
using a datatype generic programming strategy based on 
universal descriptions of algebraic data types
to define generator functions.
Generators for base types and common structures are provided with each package,
with generators for user defined types built from 
combinators for disjoint unions and products of component type generators.
The combinators mirror the type constructor algebra,
so generators can be created (mostly) mechanically from the type definition
(instances of the \QC |Arbitrary| class can be generated using DrIFT for example,
although these instances do not provide the 
modifications required to guarantee termination).
These are accessed through the class system,
which makes building structures with polymorphic arguments seamless.
This approach supports creating generators for most Haskell (regular polynomial) types,
that should form the core of any property based testing for Haskell or other functional language.

All of the packages consider the size, or complexity,
of the test cases to be a controlling factor in
the selection and ordering of test cases,
prioritizing the smaller test cases over the larger.
The size variable acts as a measure of complexity
compatible with use of the regularity hypothesis (section \ref{sub:regularity}),
and the small scope hypothesis (definition \ref{def:smallscope}).
\FEAT uses the total number of constructors in a value as it's size,
while the other packages use the maximum recursion depth of the structure as the size.
Using the total number of constructors instead of the maximum branch depth
avoids a left-weighting bias caused by structure composition,
(which will be discussed in \ref{sec:enumerations}).
In \cite{Duregard2012}, the authors present the results of an experiment
comparing the \SC and \FEAT approach to 
exhaustively generating a complex data type (the Haskell AST),
and demonstrate that \FEAT produces a more appropriate distribution of test cases
\footnote{They specifically note that 
"SmallCheck tends to get stuck in a corner of the space and test large but similar values".}.
It is unclear which of these complexity measures will provide a superior test suite
when used as the basis for stratified sampling.

The interpretation of size for types other than recursive tree-like structures
is different between the packages.
\QC and \EC ignore  size when generating base types,
while \SC and \LSC generate a list of base type values  
bounded above and below by the size
(i.e. $\left(-d,d\right)$ for enumerated (Haskell class |Enum|) base types like |Int|).
\FEAT enumerates base types and extends the enumeration of
any type incorporating them to include that enumeration
(potentially creating unmanageable enumerations in the process).
\GAST does not provide generators for base types,
but suggests (\cite{GAST2002}) that a combination of extreme and random values 
are used for test cases.
While these disparate approaches do not offer much guidance
(the generation of base types seems to largely be an afterthought),
this last suggestion seems to provide the most appropriate direction.
This is particularly evident in exhaustive (or uniform interval) sampling,
where it is unclear what advantage is gained by sampling a small compact interval
of the values of a base type in each position in a recursive tree-like structure.
Providing distinct or hybrid sampling methods for base and recursively defined types
would allow much more efficient test data sets to be generated.

All of the papers agree that automated test case generation is important for
performing tedious, mundane task of generating tests,
and all focus on the way in which test cases are selected and ordered.
Each package uses the size of recursive structures as the basis for generation,
and all except \FEAT provides a single sampling method for test case selection.
Even \FEAT, which supports multiple sampling strategies for test case generation,
does not allow different sampling methods to be used in a single generator,
and cannot because the test evaluations are through distinct monadic evaluators.

While the packages can be used independently to complement each other,
especially since the properties are defined in the same way,
it seems that it would be more appropriate to allow
multiple sampling methods to be combined into 
a common strategy for test case generation
in a single test package.
Earlier work by Uszkay and Carette (presented at IFL 2009 but not published)
showed that \QC and \SC could be effectively merged
allowed exhaustive and random test generation
of the different components of a common type
(e.g. exhaustively generating heaps trees up to a depth of 3, 
but with random integer nodes)
by merging the |Gen| and list monads,
but even this was locked to two sampling methods.
Test developers should be free to combine different sampling methods into test case generators without arbitrary restrictions to support test case evaluation.

All of the packages provide a limited ability to
perform an analysis of the coverage provided by the test cases they generate.
The test programs provide the report of the test result through Haskell's IO monad (or equivalent IO facility).
\QC provides the |trivial|, |collect| and |classify| functions which 
count size 0 cases, test cases of each size, 
and categorize results in a user defined scene respectively.
This is a good but minimal set of report combinators,
suitable for a ``quick check'' but might not provide enough information
about the successful test result to justify a conclusion of program correctness
in more complicated situations.
While generally the strategy of exhaustive testing 
to a given depth used by \SC is self-explanatory,
if the test suite is modified with filters there is no easy way to get coverage analysis;
similarly in \LSC there is no way to know how many test cases were resolved 
through partial evaluations.
\EC and \GAST randomize the order in which test cases are selected
and then truncate the test data set at the requested number of elements,
so it is difficult to determine which values were \emph{not} selected.
\FEAT does not address the issue of coverage analysis.

The authors of \QC acknowledge the weaknesses of the reporting explicitly:

\begin{quote} 
The major limitation of QuickCheck is that there is no measurement of test coverage. (\cite{Claessen2000})
\end{quote}

\noindent
As an example of a problem caused by the lack of coverage analysis, 
both \cite{Claessen2000} and \cite{EasyCheck2008} provide (different) examples
of a test for an insertion function that requires a valid input collection
(a sorted list and a valid heap, resp.).
If a conditional property is tested over a collection of valid and invalid inputs, 
rejecting the invalid cases as part of the test,
the resulting test suite is very heavily slated towards small test cases as 
the arbitrary larger test cases are significantly less likely to be valid.  
Ideally, the tester would write a generator that produces only valid collections
to avoid the bias introduced by filtering,
but the test report must provide sufficient coverage information to identify the problem.
Reporting and coverage analysis was not a significant part of any of the packages,
perhaps appropriately given the nature of the work done by the Haskell community,
but production software testing environments will require
much more sophisticated coverage analysis and reporting tools.

\subsection{Summary}
One of the most attractive features of these systems is that
they require relatively little effort to learn
(\FEAT is arguably more complicated),
are easy to use, and provide results quite quickly.
Even given the different implementation languages and generating strategies, 
these packages share a number of traits:

\begin{enumerate}
\item test individual properties encoded as Boolean valued functions,
\item allow properties to be \emph{polymorphically typed}
\item support \emph{automatic test case generation} based on the property type,
\item \emph{tightly integrate} test case generation with test execution and reporting,
\item provide a \emph{test data generation language} for building test case generating functions,
\item allow default test value generators to be \emph{mechanically derived} from 
the input type of the property, and
\item allow test case generators to be \emph{customized} by the test developer.
\end{enumerate}

\noindent 
The consensus of the packages on the definition of properties and
the use of Haskell classes for integrating generators of different types
suggests that these aspects of their design is appropriate
for any property based testing tools.

The most significant conclusion to be drawn from the differences 
is that the different \emph{sampling strategies} are clearly the focus of research in this area,
but there is little basis to support direct comparison except experimentation.
Given this lack of certainty,
a \pbt system should allow test developers to 
use different sampling methodologies,
and develop new (or hybrid) testing strategies without
requiring the development of an entirely new software package.
The tight coupling of evaluation, reporting and generator functions
prevents this mixing of sampling methodologies,
and also ties the user into a single (sequential) mode of evaluation
and a simple reporting system.
This thesis will demonstrate that these restrictions are unnecessary
and that \pbt systems of this nature can provide a great deal of flexibility
without giving up the usability of the \QC interface.



%------------------------------------------------------------------------------
\setcounter{figure}{0}
\setcounter{equation}{0}
\setcounter{table}{0}
\chapter{Test Case Generators}\label{chp:testgen}

The popularity of \QC and other sampling based \pbt systems (Chapter \ref{pbtsystems})
suggests that sampling based automated test case generation
is an area of some interest to the Haskell community.
As defined earlier, 
a property is a univariate Boolean valued function
which must evaluate to true over its domain,
and the \pbt system evaluates it over a sample of those values.
Automatically generating that sample requires the ability to 
select and construct values of the appropriate type
to pass to the property function.
These systems provide a \emph{data generation language} to 
create test case generator functions (generators)
which are used to select, instantiate and sequence the evaluation of test cases.

The systems reviewed in Chapter \ref{pbtsystems}  
provide generators based on a single sampling method
(\FEAT allows three different kinds of sampling, but each generator is restricted to a single method).
More sophisticated test generation strategies
would stratify a property's domain and use different sampling methods for the different strata
(as discussed in section \ref{sec:sampling_theory}).
While this can be achieved by conducting separate tests using different \pbt systems,
this makes the testing process more complicated.
It also precludes the use of hybrid sampling strategies
in which different components of an algebraic data type are 
sampled using different methods.

This chapter provides general definitions for generators,
and explores the relationship between generators and Haskell types.
This approach supports the development of 
sophisticated type-independent test generation strategies,
using multiple sampling methods over different partitions of a property's domain,
including hybrid sampling methods over composite structures.
Properties implemented in Haskell will be used as concrete examples,
with a focus on sampling \emph{regular recursive algebraic data types} 
(\cite{Okasaki1998}, but called uniformly recursive there)
stratified by complexity measures.
Chapter \ref{chp:enumgen} will follow with a comprehensive and efficient algorithm for constructing
sampling generators for several significant sampling methods,
as implemented in the \GC \pbt framework (Chapter \ref{chap:source}).

\section{Haskell Types}\label{algdatatype}
%\section{Haskell Types}

This section is a brief overview of the Haskell types%
\footnote{It is assumed that the reader is familiar with the Haskell type system,
or will turn to \cite{Haskell2010} or other Haskell references as required.} 
that can be generated in the \GC framework
and the other \pbt systems discussed in Chapter \ref{pbtsystems}.
These represent a broad range of commonly used Haskell types,
but certain categories of types have been excluded from this work,
and in particular it does not address any extensions to the Haskell type system
that are not part of the core language definition.

Haskell types consist of a small number of built-in \emph{base types} (e.g. Char, Int),
functions,  and algebraic types.
Type constructors for lists and tuples are included in the language definition.
Standard algebraic data types are provided through the Haskell Prelude and library modules,
but new algebraic data types may be defined in a module,
including recursive and mutually recursive types.


\subsection{Type Constructors}
Recall that algebraic data types in Haskell are defined as (\cite{Haskell2010}):

\begin{align*}
\text{data} T\ a_1\ ...\ a_n =\ & C_1\ t_{1,1}\ ...\ t_{1,n_1}\ \\
  \vert\ & C_2\ t_{2,1}\ ...\ t_{2,n_2}\  \\
  \vert\ & ...\ \\
  \vert\ & C_k\ t_{k,1}\ ...\ t_{k,n_k}
\end{align*}
\noindent
where 
\begin{itemize}
\item $a_i$ are defined types or type (valued) variables,
\item $t_{i,j}$ are either defined type constants or one of the $a_i$ arguments to the type constructor,
\item each $C_j$ is a \emph{data constructor},
\item and $ \vert $ represents the (disjoint) union of the constructors.
\end{itemize}
\noindent
Data constructors are reified as functions

\begin{equation*}
C_j :: \ t_{j,1} \ra ...\ \ra\ t_{j,n} \maps T\ a_1\ ...\ a_n
\end{equation*}

\noindent
The results of each data constructor can be considered a labelled product of its arguments,
and $C_j$ must be a unique label within the module.

\noindent
There are two other ways that new types can be defined:
\begin{description}
\item[type synonym] provides a new name for an existing type,
but is purely syntactic and is interchangeable%
\footnote{with minor exceptions that are not significant to generation} with it;
\item [newtype] creates a new type by labelling instances of an existing type
using a single data constructor,
meaning the new values are distinguishable from the originals.
\end{description}
\noindent
In both cases, there is a one to one correspondence between the values of the original and new type.
For test case generation,
|newtype| constructors can be considered equivalent to
a new algebraic data type constructor with a single data constructor,
and type synonyms can be ignored as a syntactic convenience.

\subsubsection{Built-in Type Constructors}
Haskell provides built-in type constructors for lists and n-tuples.
Lists and tuples could be generated using
their algebraic equivalents,
but it is more convenient to treat them as special cases.

\subsubsection{Functions}
While it is possible to generate a limited class of functions,
they have a fundamentally different semantic interpretation
which poses special challenges for test case generation
and are out of the scope of this thesis.
\gordon{is this sufficient?}

\subsection{Polymorphism}

An Haskell type is polymorphic if
it represents a type constructor ranging over one or more type variables.
Modules may (and usually do) contain expressions and functions that have a polymorphic type,
and even be compiled in this state,
but these must subsequently be refined to grounded types before execution.

Properties, as a kind of Haskell function, 
can be defined over polymorphic types.
For example, typically the involution property for a list reverse function
would be defined over lists with an arbitrary element type.
The test cases generated for this property, however,
must each have a concrete element type
(and this choice must satisfy the class constraints).
The selection of which types to test is 
is part of defining the \emph{sampling frame} (section \ref{sec:sampling_theory}),
and the impact of restricting the test to these types
must be addressed as a supporting hypotheses
of the test context (section \ref{sub:context}).

\subsection{Other types}\label{nonregrecurs}
One of the important assumptions made to this point 
about the algebraic data types being generated
is that every value has a finite representation.
It is possible in Haskell to define and use types with 
no finite representation,
the simplest example of which is the stream:
\begin{lstlisting}
data Stream a = Stream a (Stream a)
\end{lstlisting}
\noindent
It is not possible to instantiate such a structure using proper values,
but Haskell's lazy evaluation makes it possible to use these types.
These kinds of coalgebraic data types are beyond the scope of this thesis.

Another assumption that has been made to this point is that
recursively defined structures are \emph{regularly} recursive,
meaning that the recursion occurs over the type itself.
Nested, or non-uniform recursive data type is a parameterized type that
includes a recursive reference to the type but with modified or different type arguments.
For example, a nested pairing type defines the type of arbitrarily deep pairings:

\begin{lstlisting}

data NestedP a = Node a | Nest (NestedP (a, a))

\end{lstlisting}

While we anticipate it is possible to handle these, we leave them to future work.
The interested reader may consult
\cite{BlampiedThesis} and \cite{Joha07}.




\section{Generators for Finitely Populated Types}\label{adtgen}
%\section{Generators for Finite Types}: 
In the most general sense,
generators produce an ordered 
collection of values of a given type
based on a test selection criterion
(the formal definition (pg. \pageref{sub:basegen})  
follows this informal discussion of generator characteristics).
A generator plays three complementary roles in generating test cases for a \pbt program:
\begin{enumerate}
\item selects test cases from all or a part of the input type;
\item instantiates those cases as arguments to pass to the property function; and
\item orders the test cases for evaluation
\end{enumerate}
\noindent
This general definition allows several important variations.
A generator may:

\begin{enumerate}
\item  produce a finite or unbounded sequence of values,
\item guarantee unique values or permit duplicates,
\item have a range that does or does not include all of the type's values,
\item may order values according to their priority for testing,
\item guarantee that values are valid test cases,
i.e. are part of a property's domain,
or may allow invalid test cases to be generated 
(which must be detected as part of the test).
\end{enumerate}

\noindent
A wide variety of these generator characteristics can be seen
in the \pbt systems evaluated in chapter \ref{pbtsystems}.
For example:

\begin{itemize}
\item {the random value generators of \QC and \FEAT
provide an unlimited stream of values,
in the order they are to be tested, allowing duplicates,
with every value up to a set ``size'' having a non-zero probability of being generated; }
\item { \SC generates one of each value up to a specified size,
with priority given to smaller cases, 
but within each size orders cases for efficient generation, not test priority; }
\item { \FEAT uniform interval selection picks every $k^{th}$ value from an index once,
with the other values excluded from possible generation,
prioritized by size;}
\item{ A ``hard-coded'' list of test cases is also a form of generator,
providing a fixed and finite list of values (presumably without duplicates),
possibly in an intuitively, but not necessarily technically, significant order.}
\end{itemize}

The order may not be relevant if all of the test cases will be evaluated,
but can influence the perceived efficiency of a test program
that stops after one (or a fixed number) of errors are found.
All of the \pbt systems in chapter \ref{pbtsystems} use such a stopping criterion,
and attempt to order the generation of test cases to find errors earlier
in their sequential evaluation.
\QC, \SC, and \FEAT generate recursive types sequentially 
from smallest (least complex) to largest,
a heuristic based on the small scope hypothesis (\ref{def:smallscope}).
In \EC the size and complexity of the test cases are deliberately mixed,
a heuristic intended to find errors that would only appear in larger test cases earlier.
Each of these packages identifies their test selection and prioritization as 
a significant advantage of their test case generation strategy.

The generation of \emph{invalid} test cases is a pragmatic issue that arises 
in automated test case generation.
As discussed in section \ref{formal_pbt},
a property is a typed, univariate, Boolean valued function
that must hold over the property's domain.
The values of the property's domain must be of the property's input type,
but the input type may include other values over which 
the property is not required to hold, and may even be undefined.
Ideally, the generator would be written to only provide valid test cases,
but it may be more practical to allow invalid values to be generated
and subsequently filter them out.
For example,
a test case that required a sorted list could be 
created with a (general) list generator
and a filter that only allows ordered lists through.
This kind of post-generation filtering might well have negative repercussions for the resulting test set,
as shown in \cite{Claessen2000}),
and should generally be avoided.
A test program that uses filters must ensure that 
invalid cases are clearly reported and excluded from the verdict.
The issue of invalid test case generation will be addressed in more detail
in chapter \ref{chap:requirements} under the heading of \emph{conditional} tests.

\subsection{Base Generators}\label{sub:basegen}
Built in, or base,  types have no decomposable structure,
and so their generators are simply an ordered collection of values:

\begin{df}[Base Generator]
A generator for a scalar (or built-in) type $T$ is a function 

$$g: \nat \ra T$$

\noindent
such that $g$ is either total
or defined over an initial segment $[1,n]$ (where n may be 0, i.e. the empty sequence).
\end{df}
\noindent
Allowing generators to be either finite or unbounded 
creates some challenges in their implementation and use,
but provides the flexibility required to support different sampling methods.
Empty generators are relevant when part of a family of ranked generators (section \ref{complexgen}).

Base generators may equivalently be defined as 
a (possibly infinite) \emph{list} of values of the generated BASEtype, i.e. $[ T ]$.
This definition is convenient,
and \SC, \FEAT, \EC and \GC have all implemented their generators as lists.
These two definitions of generators will be used interchangeably in 
the theoretical parts of this thesis
(and generators of a type $T$ will generally be denoted as $[ T ]$ for brevity),
but the distinction between the function and list definitions
will be made explicit in any discussion of \pbt implementations.

As noted earlier,
generators must select, instantiate and order a sequence of values of the specified type.
The binary encoding for |Char| and |Int| types share the property
that there is an \emph{enumeration} of their values,
namely a bijective functions between the values of the type
and an integer range $[1,n], n \le 2^k$ where $k$ is the number of bits in the encoding of the type.
We call such functions a \emph{selection} function,
and the interval $[1..n]$ forms an \emph{index} of the type's values.
In Haskell this property of a type is represented by the |Enum| class
with the methods | fromEnum :: Int -> a | and | toEnum:: a -> Int|.
If a type may be enumerated,
it may be generated by mapping the selection function over 
an integer valued generator,
providing a type-independent approach to generation.

Any type encoded with a fixed number of bits
could be enumerated in this fashion.
However, for more populous types or complex binary encodings,
such as IEEE floating point types,
a simple linear enumeration may not be useful for generating test cases.
The Haskell Enum instance for the Double type
provides a particularly vivid example of 
the problems with linearly enumerating a complex type:
the selection function maps $[1 .. (2^{31}-1)] \ra [1.0, 2.0, ...., 2.147483647e9]$ 
a rather poor representation of the |Doubles| for a test data set.
A more appropriate approach for generating IEEE floating point types
is to generate pairs consisting of the mantissa and the exponent,
and include an additional base generator for the special values NaN, $\infty$, $- \infty$, etc.

Another approach to generating values of a base type is
to use the lexical rules for that type,
generating the \emph{representation} of the value
as an algebraic data type (as developed through the remainder of this chapter)
instead of as a base type.

\subsection{Parameterized Generators}

Generators may be parameterized depending on the sampling method they apply for the value selection criterion.
Some examples of parameterized generators include:
\begin{itemize}
\item providing an upper and lower bound to the values generated,
\item random generators parameterized by a seed,
\item uniform interval generators parameterized by the sampling interval size.
\end{itemize}

Generators with range parameters can be used to 
sample part of the property's domain
by restricting each generator to provide values for a particular part of the domain.
This allows different weights and sampling methods over different parts of the test domain
when it is sufficient to sample a small number of parts independently.
This is particularly useful for data types that have a few of distinct parts
that might unevenly challenge the implementation:

\begin{enumerate}
\item using separate generators for |Double| special values,
values with large mantissas and values with large exponents;
\item for records that use a field with an enumerated code to represent distinct entities,
such as a list of external vs. internal network connections,
generate each class independently.
\end{enumerate}

\noindent
Section \ref{complexgen} introduces a special kind of parameterized generator,
called \emph{ranked} generators,
each of which samples a finite part of an infinite test domain.
The rank acts as an index into a partition of the domain,
and may be combined with other parameters to allow
finite sampling techniques to be applied to infinite types.

\subsection{Generating Products of Finite Types}
A generator for the product of finite types
may be constructed from generators for each of the components.
This is a fundamental building block in developing a theory of generators.

\subsubsection{Generator Products}
One intuitive approach for generating products is to
use the Cartesian product of finite component generators,
defined as:

\begin{df}[Generator Product]
Give generators $g_1,g_2$ for (finite) types $T = (T_1, T_2)$, 
define their product (denoted $g_1 \sprod g_2$) as

$$ g = [ (x_1, x_2) \mid x _1 \leftarrow g_1, x_2 \leftarrow g_2 ] $$

with a similar definition for n-tuples.
\end{df}

\noindent
Generator products are particularly suitable for 
exhaustive or other dense sampling generators in Haskell,
because the intermediate products are reused,
e.g. an exhaustive n-tuple generator would
share a generated list of (n-1)-tuples with 
each of the first element values.
\SC exhaustive generators are constructed in this fashion.

The product of the component generators will only be suitable for a \pbt
if each of the component generators produces a reasonable number of element values
with respect to the property being tested.
This is particularly an issue with the inclusion of large base types such as |Int|,
where small variations in the values have little or no impact on the outcome of the test,
i.e. are part of the same \emph{equivalence class} of test cases with respect to the property.
The component generators should be defined to provide a list of values
likely to provide an appopriate level of variation in the values of the product type.
For example, \SC addresses this issue by limiting (enumerated) base type values chosen for a test
to a small range of values around the center of the enumeration,
e.g. $g_{Int} = [-2,-1,0,1,2] $, depending on the size of the test case
(unfortunately, this is not a particularly good sample of the |Int| type for testing).

\subsubsection{Step-by-Step Generation}

An alternative approach that allows greater variability in each of the components
is to generate the components one at a time as they are needed,
an approach called ``step-by-step'' here.
Each time a value of component type $a_i$ is needed,
the next value from the generator for type $a_i$ is consumed.
The component generators may be finite or infinite in this approach;
if there are no further values from any one of the component generators,
the product generator is exhausted.

\begin{df}
A generator $g$ for type $T = (T_1, T_2, ..., T_n)$ 
is constructed   \emph{step by step} from the component generators $g_1, g_2, ..., g_n$
if 
$$ g_{T} (k) = (g_1(k_1), g_2(k_2), ..., g_n(k_n)) $$
\noindent
where $g_j(k_j)$ represents the next component from the $j^{th}$ generator.
\end{df}
\noindent
This is the result that would be produced using the |zip| Haskell functions:
e.g.
\begin{lstlisting}
g = zip3 g1 g2 g3
\end{lstlisting}
\noindent
Step-by-step generation is particularly suitable for random and ad-hoc sampling
because the components of the product may be sampled without regard to the other components,
i.e. the sampling method for the tuple may be decomposed into sampling each of the components independently.
\QC uses this approach for its Arbitrary class random generators,
using the class dispatching system to invoke the needed generator
for each value in a product.

If the tuple's component element types are distinct,
their generators must be distinct.
Where multiple elements are of the same type, however,
it may be desireable to use the same generator
for all like typed components in each product value.
The $g_i$ in the definition above would not be distinct,
and the $k_j$ generator indices allow the same generator to supply
multiple elements to the product generator.
For example, a generator for pairs of type $(a, a)$ can be created from a generator for type $a$
by ``pairing up'' the values, i.e.

$$ g_{(a,a)}(k) = (g_a(2k-1), g_a(2k)) $$

or in list form

$$ g_{(a,a)} = [(a_1, a_2), (a_3, a_4), ...] \text{ where } g_a=[a_1, a_2, ...] $$
\noindent
More generally,
if an n-tuple has multiple elements of a given type,
each of those elements can be extracted from the same generator 
in the order they appear in the tuple.
Using a single generator alters the relationship between like typed elements in a product,
and is useful for maintaining the sampling distribution
for products with randomly selected elements.
It is particularly useful, however,
when generating arbitrarily large, recursively defined, types
in which an arbitrary number of elements are required to populate the generated structures,
as will be discussed in section \ref{complexgen}.

\subsection{Substitution Generators}\label{sub:subgenops}
Generator products and step-by-step generators are the opposite extremes of 
how generated elements can be used to populate generated product values.
They both treat the product as a structure with ``holes'' to populate with elements,
and the elements are extracted from element generators.
The difference between the two types of generators is in
how the generated element values are combined
to create the resulting product values:
generator products use all combinations of the element values (exhaustive),
but step-by-step generators use each element only once.
It is possible, however, to define a hybrid of this approach
in which some elements of a product are populated exhaustively
but others are only used once,
or use additional ``substitution strategies'' such as generating 
all permutations or combinations of (like-typed) elements.
We call this broader family \emph{substitution} generators,
and the mapping from the generated elements to generated product values the ``substitution strategy''.

This additional control over substitution strategies 
provides one relatively simple, pragmatic approach to
sampling very large test domains,
such as those based on complicated records,
allowing specific elements or combinations of elements to be more heavily weighted in a test suite.
It is also important for generating recursively defined data types
(as will be discussed in section \ref{complexgen})
where the definition of substitution is extrapolated
to populating the tree-like structures of algebraic data types.
In these cases,
the substitution generator combines a structure generator that 
samples the possible ``shapes'' of the recursive type,
and then applies a substitution strategy to the element generators
to populate the selected structures with elements values.

\subsection{Disjoint Unions}

Generating a type with disjoint unions simply result in 
choices being made between product generators.
There are two intuitive approaches:

\begin{enumerate}
\item Concatenate the results of the product generators
(generally but not necessarily in the order they are defined in the type constructor).
\item Choose one of the products at generation time by generating a ``decision''.
\end{enumerate}
\noindent
These two strategies treat each component of the disjoint union independently,
and thus are the easiest to implement.
\SC exhaustive generators use concatenation,
while \QC arbitrary generators use a potentially weighted choice.

These simple approaches do not allow for a coordinated sampling strategy over the entire value population.
\FEAT and \GC \emph{enumerative} generators instead use combinatorial techniques to
create an index over \emph{all} of the values that could be constructed,
and then uniformly apply a sampling strategy to select values.
Enumerating and indexing the entire population
is particularly advantageous when
the component types of the products being generated have significantly different cardinalities.

This situation arises frequently, for example, in generating trees,
but can also apply to simple product structures used in practical ``real-world'' applications.
Consider a module supporting a directory of local and remote file directories.

\begin{lstlisting}
-- simple directory for file repositories
newtype TCP4 = TCP4 {unTCP::(Int,Int,Int,Int)} -- [0..255] 
newtype FLoc = FileLoc{unFL :: String} 	-- directory
data Protocol = FTP | SFTP | HTTP | HTTPS
data Node = 
    Local FLoc 
  | Remote Protocol TCP4 FLoc 

-- provided generators
tcp4Gen :: Generator TCP4
tcp4Gen = [tcp_1, tcp_2,...,tcp_n] -- node ips

prtclGen :: Generator PrtCol
prtclGen = [ FTP, SFTP, HTTP, HTTPS ]

badLocGen :: Generator FLoc
badLocGen = [bloc_1, bloc_2, ...] -- invalid directories

goodLocGen :: TCP4->Generator FLoc
goodLocGen = [gloc_1, gloc_2, ...] -- valid directory`
 
\end{lstlisting}

The first step in any of the generator construction approaches
is to define |Local| and |Remote| node generators.
It is assumed here that the sampling strategy has already been incorporated
into the component generators.

\begin{lstlisting}
locNdGen :: Generator FLoc -> Generator Node
locNdGen = map Local 

remNdGen :: Generator (PrtCol,TCP4,FLoc) -> Generator Node
remNdGen = uncurry3.Remote

\end{lstlisting}

A stepwise generator for Nodes
would then combine the two component generators
through a stream of choices between the two.
A \QC Arbitrary class random value generator, for example,
would have a weight assigned to picking the local or random node,
and then choose which type of node based on a stream of randomly generated probabilities.
This could be described as a ``weighted zip'' function,
combining the contents of two lists into one but 
non-deterministically staggering the order of the elements.

A generator that concatenates the outputs would first provide all of the (finitely many) local nodes to be generated first
and then provide the remote nodes.  

\begin{lstlisting}
ndGen :: Generator FLoc -> Generator (PrtCol, TCP4, FLoc) 
                                    -> Generator Node
ndGen flg ptfg = (locNodeGen flg) ++ (remNodeGen ptfg)
\end{lstlisting}

The third approach, used by \GC and \FEAT,
is to enumerate all possible values of both the local and remote nodes,
and then construct a selector function |s : Integer -> Node|
that provides an index into all possible values of the |Node| type.
Generators apply a sampling method to the indices $[1..n]$,
where $n$ is the number of possible |Node| values,
and return the mapping of selector function over the sample..


\section{Complexity Ranked Generators}\label{complexgen}
%\section{Complexity Ranked Generators}

In the last example,
there were only finitely many values for the generated types,
allowing generators to apply a sampling method to the entire population of values.
Where there are infinitely many values, however,
it is not possible to ``fairly'' sample the entire population,
so the range of any given generator must be restricted to a finite subset of the values.
One approach is to establish a partition of the type's values
consisting of an infinite but countable number of finite parts,
and define an indexed family of generators that sample each part independently.
The test program may then use as many generators as desired
to support a \emph{stratified sampling strategy},
as discussed in section \ref{sec:sampling_theory}.

\QC originally provided such a solution by
capping the ``size'' of the values to be generated on each call.
Small test cases were generated and evaluated first,
followed by larger values (a heuristic based on the small scope hypothesis).
This required a family of generators parameterized by the size of the structures they produced,
each of which was dependent on generators of smaller structures,
a solution also adopted by the other \pbt systems discussed earlier
(Chapter \ref{pbtsystems}).

\GC addresses this problem by using a class of complexity \emph{ranked} generators,
indexed by a positive integer rank,
each of which samples a finite part of the type's values.
This section presents the definition of \GC ranked generators and
the use of \emph{complexity measures} as the basis for this ranking.
It also identifies the size ranking used in the other reviewed \pbt systems
as forms of complexity ranking,
and compares the complexity \emph{measures} each uses.

\subsection{Ranked Generators}
A \emph{ranked} generator is an indexed generator $g_r : \nat \ra T$
that partitions T into finite subsets.

\begin{df}[Ranked Generator]\label{def:rankedgen}
A ranked generator for a type $T$,
with a rank function $\text{rank} : T \ra \nat$  inducing a partition $T_1, T_2, ...$
is a parameterized generator
$$G : \nat \ra [ T ] $$
such that
$$G(k) = [T_k]$$

\end{df}

\noindent
The goal is to use this partition for stratified sampling (section \ref{sec:sampling_theory}).
Ideally, the rank function would have the following properties:

\begin{enumerate}
\item groups values which are considered alike in some way, 
with respect to the evaluation of the property;
\item justifies setting a maximal rank to limit testing to a finite part of the property's domain;
\item reflect the \emph{priority} of the values for testing,
with lower ranks being higher priority test cases,
although this may be a subjective measure.
\end{enumerate}
\noindent
For example,
one rank function for Haskell's arbitrarily large integer type (|Integer|)
might partition the values by magnitude and sign:

\begin{equation*}
\text{rank}(n) = \frac{2\ log_2 \absval{n}}{ k } + \begin{cases} 1 & \text{k positive} \\ 0 & \text{otherwise}\end{cases}
\end{equation*}
\noindent
resulting in the partition $\{\ [0..(2^{k} - 1)], [-(2^{k} - 1)..-1], [2^{k} .. (2^{2k} - 1)], \ldots\ \}$.

\subsection{Complexity Ranked Generators}
In order to use uniformity, regularity or small scope hypotheses in a test context,
the partition of the type values must be based on the \emph{complexity} of the underlying terms.
A computable measure of term complexity is therefor required,
such as that provided in  (\cite{BernotGaudelMarre1991}) .

\begin{df}[Complexity Measure]
A  complexity measure c must satisfy the following conditions:

\begin{enumerate}
\item must be a non-negative integer;
\item must be strictly monotonic with respect to term inclusion, i.e.
$$ c(\ C\ t_1\ t_2\ ...\ t_n\ ) > max( c(t_1), ..., c(t_n) ) $$ 
where $C$ is a type constructor;
\end{enumerate}

\end{df}

\noindent
Three distinct complexity measures for algebraic data types
that are used in the \pbt systems described in this thesis:

\begin{description}
\item[maximal depth:]
the maximum number of constructors in a branch required to reach any leaf value;
\item [total number of constructors:] the number of data constructors included in the term, including nullary constructors;
\item [combinatorial size:] the number of terminals,
i.e. the number of base type and nullary constructors in the term.
\end{description}

\noindent
Note that these do not form a comprehensive list of possible measures,
but we restrict our attention to these three.

The computation of term complexity is given below (where $c$ is the complexity function):

\begin{tabular}{ l c c c l}
complexity measure & base type & constant & $C t_1 \dots t_n$ & Used By\\
max. depth & $0$ & $0$ & $1 + max ( c(t_1), \dotsc, c(t_n) )$ & QC, SC, EC, GAST\\
total constructors & $0$ & $1$ & $1 + \sum_{i=1}^{n} c(t_i)$ & FEAT\\
combinatorial size & $1$ & $1$ & $\sum_{i=1}^{n} c(t_i)$ & GC
\end{tabular}

\noindent
Complexity measures can be defined on products over any component types
that the measure is defined for,
so supports recursive and mutually recursive types.
Note that disjoint unions do not play a role in term complexity as these measures
are defined on the terms, not types.

Maximal depth and total constructors both count constructors to determine the complexity of the term,
which is a natural approach in a functional programming environment.
Conceptually the combinatorial size complexity measure is quite different,
being based on a count of the number of values
that must be substituted into a term's ``shape''
in order to produce a populated data structure.
The ``weight'' of the structure is in the base types and nullar constants that make up the terminals,
as opposed to the constructors that contain them.
For simple structures,
this provides similar rankings to the total constructors approach,
but for more complex data types this is not the case,
as will be discussed in \ref{sub:cmprcomplex}.

\subsection{Generating Regular Types}\label{sub:gensubst}
The challenge of generating infinitely valued types can now be 
decomposed into a countable sequence of complexity ranked generators,
each generating products of a fixed rank from a finite population of terms.
A ranked generator that uses a complexity measure as its ranking function 
will be called a complexity ranked generator.

Complexity ranked generators can be defined compositionally
by following the structure of the type definition,
i.e. using \emph{generic programming} techniques
(\cite{Bird99}, \cite{Gib-dgp06}, \cite{LPJ03}, \cite{ComparingGP})
The population of a constructed product term $C a_1 a_2 ... a_k $ of a given complexity $n$
may be determined by inverting the measure to proscribe the complexity of the term's components
(where $c$ is the complexity function):

\begin{tabular}{l l}
complexity measure & component complexity\\
max. depth & $ \forall i . c (a_i) \le n - 1  \land \exists j . c (a_j) = n - 1$\\
total constructors &  $1 + \sum c (a_i) = n$ \\
combinatorial size &  $\sum c (a_i) = n$ 
\end{tabular}

\noindent
If the type being generated is the disjoint union of multiple data constructors,
the population of terms of rank $n$ is the union of the terms of that rank for each constructor.

Higher ranked generators can then be defined using lower ranked generators,
terminating with base type and constant generators.
For example, \SC uses the total constructors measure and
generates terms $T\ a_1\ a_2\ ...\ a_n$ of complexity $r$ by
combining all possible terms $a_1, \dots, a_n$ such that the sum of their ranks is $r-1$.
\QC uses maximal recursive depth as its complexity measure,
with higher ranked generators using components from lower ranked generators,
although it is slightly different in that the rank is the upper bound of term complexity instead of a fixed value.
This approach is well suited to the construction of exhaustive and random sampling generators,
but these are the only two sampling methods that can be constructed in this way.
More complicated sampling strategies such as boundary sampling and uniform interval sampling
cannot be arbitrarily decomposed over products,
and so cannot be implemented by calls to lower ranked generators.

The alternative is to use the complexity measure to define the \emph{population} of the terms of a given rank,
and then apply the sampling method to that population.
The definition of these populations,
and the application of the sampling methods,
can also be defined compositionally following the definition of the type,
and are therefore also subject to generic programming methodolgies.
\GC and \FEAT use this approach,
compositionally defining \emph{enumerations} of the terms of a given complexity,
and then sampling those terms;
we refer to these as \emph{enumerative generators}
(chapter \ref{chp:enumgen}).

\subsection{Comparing Complexity Measures}\label{sub:cmprcomplex}

Each of the complexity measures satisfy the rank criteria (definition \ref{def:rankedgen}),
but how well does the measure reflect the complexity with respect to its use as a test case?
The objective of ranking the test cases is to partition the values
according to the complexity of evaluating the tested property over them,
and to support test strategies based on the uniformity and regularity hypotheses.
The complexity measures can be evaluated against these objectives
to provide a partial assessment of their value in \pbt.

Maximal depth, while an intuitive approach that is easy to implement in a Haskell generator function, 
assumes that the complexity of evaluating the property at a value is 
determined by its most complex branch,
ignoring the other parts of the term.
This seems to be contrary to the uniformity hypothesis,
which requires that the terms of a like ranking be of similar complexity with respect to the property being tested.

The total constructors measure assumes that complexity is proportional to the number of constructors in the term,
while the combinatorial size measure associates complexity with the number of terminal nodes in the structure,
Both of these seem to be well suited for a uniformity hypothesis,
and in fact are nearly identical for simple type definitions.
There is a difference for more complicated structures, however,
especially when the type consists of multiple recursive components.
For example, consider a general tree with an arbitrary number of branches and external leaves.

\begin{lstlisting}
data List a  = Null | Cons a (List a)

data GenTree a  = Leaf (a) | Brs (List (GenTree a))
\end{lstlisting}

For the combinatorial measure,
the complexity would be the number of leaves, 
regardless of how the branches were represented.
The ranking of other two complexity measures, howerver,
would be strongly influenced by the representation of the branches within the list,
and would provide different ranks depending on how the leaves were distributed in the branches,
introducing a distinct ``left bias'' to the tree structures of like rank.
The terms of similar combinatorial complexity would be those
that were alike in terms of the abstract general tree,
and therefore provides superior uniformity with respect to the property specification.
The total constructors and maximal depth generators would
produce like ranked terms with respect to the complexity of the concrete representation of the tree,
and providing superior uniformity with respect to the implementation of the type.
For more complex heterogeneous structures of this nature,
the choice of complexity measure will have a significant impact on the nature of the test suite,
but a valid argument can be made for either choice in this case.

The significance of the choice of complexity measure for generated test cases is not yet well understood.
\cite{Duregard2012} provides some empirical evidence of improved error identification
using a total constructor complexity metric (through their enumerative generators),
but each of the \pbt systems described in chapter \ref{pbtsystems}
also provide supporting arguments and emperical evidence for their generator strategies.
The best approach has clearly not been determined, 
and there may not be a single ``best choice'' for complexity measure.
It is clear that more research is required in this area,
and \GC provides a platform to explore this issue by allowing
generators to be defined with different ranking functions.







\section{Sampling in Generators}
%\section{Sampling Generators}

If the selection criterion of a generator is
the application of a generic sampling method (section \ref{sec:sampling_theory})
to the property's domain,
it will be called a \emph{sampling generator}.
Sampling generators do not use information about 
the implementation nor specification to guide test case selection,
distinguishing them from other approaches to test case generation
e.g. white-box testing with \DAISTS, 
specification analysis with \HOLTG, 
or manual preparation using human intuition.
\QC, \SC, \GAST, \EC and \FEAT (chapter \ref{pbtsystems})
all rely on sampling generators,
with different sampling methods,
for test case generation.

There are two main categories of sampling methods,
random and systematic (e.g. exhaustive, uniform interval and boundary)
as discussed in section \ref{sec:sampling_theory}.
Generators for Haskell data types can be classified in the same way,
independently of the specific type being generated,
by \emph{abstracting generation into the application of a sampling method
and a structure the over values being sampled}.
The structure organizes the values of the type,
either implicitly through a traversal strategy,
or explicitly through an index of the values,
and effectively forms an interface for the application of a sampling method.
A base sampling generator applies a sampling method to all of the values that might be generated,
and ranked generators apply the sampling method to the values of like rank independently,
but otherwise there is little difference in how the sampling method is applied.


\subsection{Random Generators}

Generally in software testing  a random value generator 
is a deterministic numerical algorithm (pseudo-random number generator) that 
emulates the production of a sequence of ``random'' values
from an arbitrary starting value (the ``seed'').
They will always produce the same sequence given the same seed,
which allows repeatable testing,
but care must be taken to use different seeds
when collecting multiple sequences from the same generator.
Many compilers provide libraries of random generators:
for example Haskell provides the |System.Random| module.
The randomness of such deterministic algorithms may be challenged,
but for the purposes of this work it will be assumed that system provided
(pseudo-)random number generators provide 
sufficiently random sequences over their range.

Creating a random sample of a given type $\beta$ requires three components:
\begin{enumerate}
\item a function $r: \sigma \ra [ \alpha ]$ that produces 
an unbounded sequence of (pseudo-)random values of a type $\alpha$ given a seed of type $\sigma$
\item a seed value $s \in \sigma$
\item a map $f : \alpha \ra \beta$ or $f': [ \alpha\ ] \ra \beta\ $
\end{enumerate}
\noindent
A generator may provide a one-to-one mapping between the sequence and generated values,
or it may use multiple consecutive values from the sequence 
to generate a single value of the target type.
The former approach is common for simple, finite base types;
the latter is used in \QC for the stepwise generation of recursive algebraic data types (discussed below),
and also in \GC for composite generation of data structures and their elements.

\subsubsection{Generators with Fixed Probability Distributions}
The simplest form of random value generation produces 
a sequence that is \emph{uniformly distributed} over its range,
i.e. assigns a uniform selection probability of $1/n$ to each $n$ values.
Alternatively, the weight given to each element 
can be varied to create a different probability distribution for selection.
System supplied random number generators are generally based on uniform distributions
as they can be used to construct generators for other distributions.
These generators will be called \emph{base} generators here
to reflect their \role as the foundation for constructing other random value generators.

Random generators with non-uniform distributions are defined by 
a weight function that assigns a probability of selection to each element,
combined with a base generator. 
The definitions here are drawn from \cite{Knuth1997:ACP270146}.

\begin{df}[Weight Function]
If $\setA$ is a set with $n$ elements, 
$w$ is a weight function for $\setA$ if

$$w: \setA \ra [0, 1] \subset \reals $$
such that
$$\sum_{a_{k} \in \setA} w(a_k) = 1 $$
and
$$w(a_k) > 0 \forall a_k \in \setA$$
\end{df}

\noindent
For example, the weight function for a uniform distribution over a set of $n$ elements is 
$w(a_k) = frac{1}{n} \forall a_k \in \setA$.

A discrete \emph{cumulative distribution function} (cdf) is derived from a weight function
and an index over the elements:

\begin{df}[Cumulative Distribution Function]
If $\setA$ is a set with $n$ elements and an associated fixed index $I$ over those elements
(e.g. $\setA_{I} = [ a_1, a_2, ..., a_n ]$),
the cumulative distribution function $cdf$ induced by $w$ is:

$$cdf_{(A,I)}: \setA_{I} \ra [0,1] \subset \reals $$
$$cdf_{(A,I)} (a_k) =  \sum_{i=1}^{k} w(a_i)  $$
\end{df}

\noindent
A  selection function for this distribution maps values in the unit interval,
through the cumulative distribution function $cdf$, to the values to be generated:

\begin{df}[Selection Function]
A selection function for the cumulative distribution function $cdf_{(A,I)}$ is given by

$$ sel_{_{(A,I)}}. [0,1) \ra  \setA_{I}$$
$$ sel_{_{(A,I)}} (x) = a_k $$
where 
$$ cdf_{(A,I)}(a_{k-1}) \le x <  cdf_{(A,I)}(a_k)$$
\end{df}

\noindent
The combination of a uniformly distributed random value (base) generator
with such a selection function produces a \emph{distributed} generator,
a random value generator with the distribution defined by the weight function.

\begin{df}[Distributed Generator]
For a uniform distribution (pseudo-)random value generator $r: \sigma \ra [ \alpha ], \alpha \in [0,1)$ 
and a selection function  $sel_{_{(A,I)}}$ over an indexed set $\setA_{I}$,
then

$$ g_{cdf_{(A,I)}} : \sigma \ra [\setA_{I}] $$
$$ g_{cdf_{(A,I)}} (s) = [ sel_{cdf_{(A,I)}}(r_i) ]$$
where
$$r(s) = [r_1, r_2, ...] , r_i \in [0,1)$$ 
\end{df}

Note that both the definition of the cumulative distribution function
and the mapping from a uniform random (base) generator to the distributional generator
require an enumeration of the values being generated
to form the required index.
In theory, it is always possible to define such an order
as by definition generators have discrete and finite domains.
Simple base types such as |Char| and |Int| can be ordered by 
the binary encoding of their values;
in Haskell this order is exposed through the |Enum|  (enumerated) class, 
which includes |toEnum:: Int -> a, fromEnum:: a -> Int| methods.
In practice, more complicated  finite data types,
such as the IEEE floating point values,
are difficult to enumerate effectively and 
it may be more practical to restrict generation to a subset of the type's values
or partition the domain and generate the parts independently.

Infinitely valued types,
such as recursive algebraic data types or arbitrarily large |Integers|,
do not permit such a positive valued weight function.
Instead,
it is necessary to establish a partition,
with countably many finite parts,
define the weight, cdf and selection functions for each part,
and then apply this to a base generator.
\FEAT and \GC (chapter \ref{chp:enumgen}) both provide
functions to enumerate the values in a complexity based partition
of recursive algebraic data types,
providing a practical means to create such families of ranked random generators
for uniform distributions.

\subsubsection{Stepwise Random Generators for Algebraic Data Types }
Stepwise random generators provide one method of 
generating recursive algebraic data types
(the alternative of selecting from an enumerative index over the population is addressed in chapter \ref{chp:enumgen}).
Recall that a regular recursive algebraic data type
can be considered a disjoint union (or `sum') of labelled product types (section \ref{algdatatype}).
In Haskell, algebraic data types are defined using 
a type constructor which is the disjoint union of data constructors,
which construct labelled products of their component arguments.
A stepwise, or constructive, random generator 
randomly picks one of the data constructors,
and then randomly generates each of the component arguments
for the labelled product.
A generator must also exist for each of 
the components of the type,
as these will be called to create the component values;
the generator will call itself recursively as required,
repeating the selection of data constructor and components.
\QC (\cite{Claessen2003}) generators use this approach for recursive types
(although a uniform distribution base generator for simple base types).

Unlike distributed generators,
stepwise random generators consume a sequence of values
from a base random generator during the generation process,
one value for each choice of data constructor,
and one or more for each non-constant product component.
A weight is assigned to each data constructor,
which forms the basis of a distributed generator for that choice,
providing some control over the test case selection.
Each product component generator provides its own weighting,
independently of the other component generators.
One issue with this approach is that there is no guarantee of termination for recursive types.
This can be solved by explicitly setting a maximum number of recursive steps
and only generating node or constant values when that maximum is reached,
as \cite{Claessen2003} suggests.

A more significant drawback of stepwise generation,
when compared to distributional generators over a known cumulative distribution function,
is the lack of control over the probability of selecting values for generation,
and in particular larger recursive structures.
The cumulative effect of the weights applied to each decision point and component,
and the need to provide explicit termination criteria,
may have an unpredictable impact on test case selection over multiple recursive calls.
This may introduce a bias into the selection of test cases,
or will at least make the lack of bias difficult to justify in the test context.
The significance of this deviation from a true random sampling
is unclear in the context of software testing, however,
as the assumptions typically made in statistical sampling theory
do not generally apply, as discussed in \ref{sec:sampling_theory}.

\subsection{Systematic Generators}

A systematic sampling method organizes a population and
deterministically selects the elements for the test
(see section \ref{sec:sampling_theory}).

Exhaustive generators produce a single instance of each element of their domain.
For base types, this is simply a matter of generating each binary encoding in turn
and casting it to the appropriate type.
For algebraic data types,
each value of a disjoint union is generated
by invoking the  data constructor for every possible combination of arguments
(components in the labelled product).
Again, each of those component types must have an associated exhaustive generator.

Recursive types are generated through recursive calls to the exhaustive generator,
as with random generators, 
but termination must be explicitly addressed.
This naturally leads to the use of a complexity measure,
such as maximum recursive depth,
as a way of limiting the size of the recursive values being generated,
and to treating exhaustive generators as complexity ranked generators.

Although random and exhaustive generators appear to be opposites,
they share the common characteristic that there is no ambiguity
about how to generate the arguments for a data constructor,
and no information other than the total complexity (size) of the term being generated is required.
For a random generator,
when needed a single value is randomly generated;
for an exhaustive generator,
all possible combination of values for each data constructor
that will lead to the required complexity must be created.
These two generators represent the only non-trivial generators
for which this is the case
(a constant generator would also have this characteristic).

\subsubsection{Uniform Interval Selection}
Uniform sampling is the application of a methodical selection of values
with the goal of guaranteeing representative coverage of the sampling frame.
Coverage is dependent on their being an observable structure
that meaningfully groups values so that representatives can be selected.
This approach introduces a \emph{selection bias} which can cause the resulting test to be invalid:
a simple example would be selecting every second integer to test an $even:: Int \mapsto Bool$ function.
Since the coverage guarantee is not provided by random sampling,
but random sampling is unbiased,
these two approaches are complementary.
A hybrid approach of using uniform selection with a small random variation applied to the selections can also be used in some cases.

In this setting, methodical selection is equivalent to 
requiring that every $k^{th}$ value be generated
based on some ordering of the values in the domain.
This requires that the generator somehow ``skip'' the other values
while traversing the possible choices.
For base types this can again be managed through the binary encoding,
but for recursive algebraic data types requires more work.
Chapter \ref{chp:enumgen} demonstrates, however,
that for the most  important Haskell data types,
namely base types and regular recursive algebraic data types,
an order of this sort can be constructed mechanically from the type definition.
\emph{Enumerating} recursive algebraic data types supports all four of these sampling methods,
and can be used to create sampling generators with little programmer effort.

\subsubsection{Boundary / Extreme Case}
Boundary or extreme case selection is even more complicated,
requiring that the generator prioritize \emph{unusual} values of the type.
This requires a definition of unusual
as well as a mechanism to identify and instantiate those values.
For base types, 
this will generally be the most extreme values in the range
(e.g. |min::Int, max::Int| for Haskell |Int| type),
but might also include other values (0 for |Int|, unprintable characters for |Char|),
so should be addressed on a type by type basis.
For tree-like recursive algebraic data types, however,
an argument can be made that the most unbalanced structures,
such as the left and right single branched binary trees,
are the ``extreme'' values.
More generally,
the boundary or extreme values of a recursive algebraic data type
are those instances that have the most repetition, or least variety, of component elements.
Generating these structures requires that the most repetitious elements be generated first
(with a higher priority)
followed by elements with slightly more variety,
for the arguments of each data constructor,
which requires more information about the previously generated test cases.
This construction of such generators tends to be type specific,
but an approximation of a boundary generator can be made using
the boundaries of an enumerated index for selection,
e.g. selecting elements $1,\ n,\ 2,\ n-1,\ldots$.





%------------------------------------------------------------------------------
\setcounter{figure}{0}
\setcounter{equation}{0}
\setcounter{table}{0}
\chapter{Enumerative Generators}\label{chp:enumgen}

An enumeration of a type provides a count of the values
and an injective selector function to retreive each of those values.
Certain simple Haskell types, such as |Int| and |Char|, 
are already enumerated through Haskell's |Enum| and |Bounded| classes,
and we show here how these can be used to create generators for those types.
In addition, we provide algortihms to create ranked enumerations of 
infinitely populated algebraic data types and create generators by
applying sampling methods directly to an index over those types.
This approach abstracts the sampling methods from the underlying type,
which is beneficial both in the implementation of test generators
and the assessment of sampling strategies.
A generator based on an enumeration of a type will be called an \emph{enumerative generator},
and these play an important role in \GC property based testing.

The existence of such complexity ranked enumerations,
and efficent algorithms for their use,
was established in \cite{FlSaZi89b}, \cite{FlajoletZC94}, \cite{FlSa95} and \cite{FlajoletSedgewick2009} 
through the mathematics of \emph{combinatorial constructions}.
These works provide the conditions for a system of combinatorial constructions to be well-defined,
and a calculus to determine the ordinary generating function to count the number of constructions
for any given number of \emph{atoms} (terminal nodes in a structure).
A model of mutually (regular) recursive Haskell algebraic data types
as systems of polynomial combinatorial constructions is presented,
along with a discussion of the choices made to ensure 
all valid Haskell data types modelled this way are allowed.
It will be shown that such enumerations can be based on the type's definition and
can also be implemented using generic programming techniques or Template Haskell.
This allows \GC to provide default enumerative generators,
using sampling methods defined over indices,
for all regularly recursive Haskell data types.

\section{Enumerative Generators}\label{sec:enumerations}
%\section{Enumerative Generators}


\subsection{Base Enumerations}

A \emph{base enumeration} provides an index over the values of a finite type,
and a selection function that constructs the values at each index.
These are used to enumerate and generate built-in types (e.g. Int).

\begin{df}[Base Enumeration]

A base enumeration for a type $T$ consists of :
\begin{itemize}
\item $count :: \nat$ the cardinality of the set of values of $T$
\item an \emph{injective} selection function $s :: [1 .. count] \ra T$
\end{itemize}

\end{df}

\noindent
The selection function is total 
and provides an order over the enumerated values.
The binary encoding of such values 
provides a default order, bounds and selection function.

Any type that is an instance of the classes |Bounded| and |Enum|
allows the construction of a base enumeration over that type,
using the |minBound, maxBound| and |toEnum| methods.
Two variants of the selection function are shown,
getBase which tests the input argument and is exposed through the programming interface
and getBaseUnsafe which is used internally within \GC.

\begin{lstlisting}
type BaseSelector a = Count -> a
data BaseEnum a = Base {baseCount::Count, baseSelect :: BaseSelector a }

getBase :: BaseEnum a -> Count -> Maybe a
getBase (Base c s) n | (n > 0)   = if c >= n then Just (s (n-1)) else Nothing
getBase _  _         | otherwise = Nothing

getBaseUnsafe :: BaseEnum a -> Count -> a
getBaseUnsafe  (Base _ s) n = s (n-1)

enumBaseRange :: (Enum a) => (a,a) -> BaseEnum a
enumBaseRange (l,u) = 
  let shift = toInteger (fromEnum l)
      cnt = ((toInteger (fromEnum u)) - shift) + 1
  in makeBaseEnum cnt (\x -> toEnum (fromInteger (x + shift - 1)))

enumBaseInt :: BaseEnum Int
enumBaseInt    = enumBaseRange (minBound::Int, maxBound::Int)

enumBaseChar :: BaseEnum Char
enumBaseChar = enumBaseRange (minBound::Char, maxBound::Char)

\end{lstlisting}

A generator can be constructed from an enumeration of a base type by
applying a sampling method to the index of the enumeration,
namely the integer range $[1 .. count]$,
and then applying the selection function over that resulting sample of index values.
Some examples, from \GC, illustrate the relationship between the sampling method
(called an |EnumStrat|, short for enumeration test strategy),
the enumeration and the resulting generator.
Note that for exhaustive sampling in \GC,
the sampling has been replaced by simply using an exhaustive list of the values as the generator for efficiency,
and for random sampling the selector function randG is based on the Haskell |randomR| function.

\begin{lstlisting}
type EnumStrat = Count -> [Count]  

exhaustG :: EnumStrat
exhaustG u | u > 0 = [1..u]
exhaustG _ | otherwise = []

uniform :: Int -> EnumStrat
uniform n u | n > 2 && u > 1 = 
  let s = (toRational u) / (toRational n)
  in if u > (toInteger n) then take (n+1) $ 1 : (map round (iterate ((+) s) s))
              else exhaustG u
uniform _ u | u > 1 = [1,u]
uniform _ u | u == 1 = [1]
uniform _ _ | otherwise = []

randG :: StdGen -> EnumStrat
randG s = \cnt -> if cnt>=0 then genericTake cnt $ gr s cnt else []
    where gr t cnt = let (x,s') = (randomR (1,cnt) t) in x : (gr s' cnt)

baseGen :: [a] -> Generator a
baseGen xs r = if r == 1 then xs else []
baseEnumGen :: EnumStrat -> BaseEnum a -> Generator a
baseEnumGen strat e r | r ==1 = map (getBaseUnsafe e) (strat (baseCount e))
baseEnumGen   _   _ _ | otherwise = []

genCharAll      = baseGen [(minBound::Char)..(maxBound::Char)]
genCharRnd  s = baseEnumGen (randG s) enumBaseChar
genCharUni   k = baseEnumGen (uniform k) enumBaseChar

\end{lstlisting}

These can be extended to any enumerable base type,
as the sampling methods apply to the index of the values,
not the values themselves.

\subsection{Parameterized Enumerations}

Like generators, 
parameterized families of enumerations can be used to enumerate
subsets of a type's values.

\begin{df}[Parameterized Enumeration]
A enumeration of a type $T$ parameterized by a value $\alpha$ is given by 
\begin{itemize}
\item $\phi_\alpha$ a characteristic function that determines if a value of type $T$ is enumerated for a given $\alpha$
\item $T' = \{t :: T \mid \phi_\alpha(t)\}$ the values of type $T$ that are enumerated
\item count the cardinality ($n$) of $T'$
\item selection function, a bijective function $s :: [1..n] \maps T'$
\end{itemize}

\end{df}

\noindent
One common use of parameterized enumerations within \GC is to enumerate a range of values,
such as enumerating the lower case letters of |Char|, or non-negative |Int|.
These are then used to construct generators for the restricted ranges in the same way.

\begin{lstlisting}
enumDfltChar, enumLowChar, enumDigitChar :: BaseEnum Char
enumDfltChar   = makeBaseEnum 95 (\k -> chr ((32 +) (fromInteger k))) -- ' ' to '~'
enumDigitChar  = makeBaseEnum 10 (\k -> chr ((48 +) (fromInteger k)))
enumUpperChar  = makeBaseEnum 26 (\k -> chr ((65 +) (fromInteger k)))

genDigitCharAll    = baseGen ['0'..'9']
genDfltCharRnd   s = baseEnumGen (randG s) enumDfltChar
genUpperCharUni  k = baseEnumGen (uniform k) enumUpperChar

\end{lstlisting}


\subsection{Ranked Enumerations}
Enumerations provide simple and datatype generic technique for 
constructing generators for simple base types.
For infinitely valued types, such as recursive algebraic data types,
a base enumeration is insufficient as the domain of the selector function
would not be finite and therefore could not be sampled effectively.
The solution, as with ranked generators, 
is to define ranked enumerations that
partition the values of the type and enumerate each part separately.

\begin{df}[Ranked Enumeration]

A ranked enumeration for the values of a type $T$ consists of :
\begin{itemize}
\item a rank function $\rho: T \maps \nat$ that induces the partition $\pi_{T} = [T_1, T_2, \dots]$ of $T$,
where each $T_i$ is finite;
\item $count :: \nat \ra \nat$ such that $count(r)  = \setcard{T_r}$ and $r$ is the rank;
\item $selector :: (r:\nat) \ra [1..count(r)] \ra T_r $ 
\end{itemize}
\end{df}

\noindent
Given a ranked enumeration,
a ranked generator can be constructed by 
applying a sampling method to each rank of the enumeration.
Note that while rank must be a total function over the type,
there is no requirement that there will be values of any given rank.
As a result,  a ranked enumeration can be used for finite types,
which is useful for sampling large or complicated base types
such as the IEEE floating point values, or even records with many components.
The simplicitly of this approach is illustrated by the simple |enumGenerator| function in \GC.

\begin{lstlisting}
enumGenerator :: EnumStrat -> Enumeration c a -> Generator (c a)
enumGenerator strat e rnk =  fmap (Enum.getUnsafe e rnk) (strat (Enum.counter e rnk))
\end{lstlisting}



\section{Combinatorial Constructions}\label{sec:combconstructs}
%\section{Combinatorial Constructions}

In order to use an enumerative generator for a \pbt,
an enumeration must exist, be computable,
and for infinite types be associated with a ranking function.
Combinatorial constructions  (\cite{FlajoletSedgewick2009}) provide the tools needed.
In particular, combinatorial structures have well defined \emph{generating functions},
providing a calculus for computing the number of structures 
with a specified number of atoms,
the combinatorial complexity (section \ref{complexgen}) of those structures.
(\cite{FlSaZi91}) shows that these counts can be obtained efficiently,
and the selector functions can be constructed using the same techniques,
as implemented in Maple's |combstruct| module 
(\cite{FlSa95}).

Combinatorial constructions describes a class of structures significantly larger than those that can be implemented in Haskell,
including data structures constructed with sets, multi sets, sequences and cycles.
A subset of the class of combinatorial constructions,
called the \emph{polynomial} constructions,
provides a useful model of the types that concern us here.
The combinatorial structures provide a model of type constructors,
with the atoms representing the type arguments.
The complexity measure is the size of the structure underlying the term,
namely the number of atoms that must be substituted with values%
\footnote{This is a particularly useful measure of complexity 
for substitution generators, as the number of atoms is the number of elements in the structure.}.
This model is developed throughout this section,
restating the relevant findings of
(\cite{FlSaZi89b}, \cite{FlajoletZC94}, \cite{FlSa95} and \cite{FlajoletSedgewick2009}),
restricted to the combinatorial structures required to model to Haskell's data types.

It also includes the combinatorial construct of rank $0$, called $\epsilon$.
Certain systems of combinatorial equations that incorporate $\epsilon$
result in infinitely many constructs of a fixed rank,
so are not well-defined with respect to creating enumerations
and must be avoided for our purposes.
Systems that avoid this issue, described by the authors as ``epsilon-freeness'',
are well aligned with the requirement that each rank of structures 
must include a finite number of instances in order to effectively sample them
for automated test case generation.
Demonstrating that nullary constructors (or constants) in a type
(e.g. the |Nil| construct in a list) should be modelled as an atom of rank $1$,
instead of a rank $0$ $\epsilon$ was an important outcome of this modelling.

\subsection{Systems of Polynomial Constructions}
A \emph{class of combinatorial constructions} contains elements that 
can be partitioned by their size:

\begin{df}[A Combinatorial Class]
is a set $\set{S}$ where for $x \in \set{S}$ a size function $\phi(x)$ is defined, 
such that :
\begin{enumerate}
\item $\phi(x) \ge 0$ 
\item $\forall n . \setcard{ \{ x \mid \phi(x)=n \} } \text{ is finite} $.
\end{enumerate}
\end{df}

\noindent
Note that these are the same restrictions as those required for a ranked partition.

The class of polynomial constructions can be 
defined with a specification language consisting of 
the primitives $0, 1, + \text{ and } x$.
This algebra is defined here
(with $\spec{F}, \spec{G}$ as variables ranging over constructions):

%\begin{df}[Admissible Polynomial Construction Algebra]
\begin{tabular}[b]{ll}
Symbol(s) &  Definition  \\
$\specO$ & the unique empty construction, $ \spec{A} \siso \spec{A} \splus \spec0 \siso \spec0 \splus \spec{A} $\\
$\specI$ & Neutral, the unique construction on 0 atoms, $ \spec{A} \siso \spec{A} \sprod \specI \siso \specI \sprod \spec{A} $ \\
$\specX, \specX_{a}$ & Atom \\
$\spec{F} \splus \spec{G}$ & Disjoint union $(\circ \sprod \spec{F}) \union (\bullet \sprod \spec{G})$ \\
$\spec{F} \sprod \spec{G}$ & $\{ (f, g) \lvert f \in \spec{F}, g \in \spec{G} \}$ 
\end{tabular}
%\end{df}

\noindent
Note there may be multiple instances of the neutral structure  with different names
but all of these instances are isomorphic constructions.
They play a similar role in recursive construction definition 
as that of nullary data constructors (constants) in Haskell,
providing a terminal end point without content%
\footnote{Keep in mind that, despite this similarity of purpose,
we recommend modeling Haskell nullary constructors
as atoms over a sort with a single value to give it a size of 1 instead of
the $0$ size of the neutral element
to ensure that all constructions have finitely many structures for any given size.}

Atoms are the unique singleton-set construction over an underlying finite sort of elements.
A multi-sorted construction will have one atom type per sort,
distinctly labelled (here by convention with a subscript).

Note that n order to guarantee disjointness,
$+$ labels it's constructions with a distinct symbols
(shown as  $\circ$ and $\bullet$ in the grammar above) 
that differentiate what might otherwise be identical elements;
sums in Haskell are similarly labelled.

The $\spec0$ construction is the isomorphic group of constructions which permit no members.
For our purposes, this can be considered an error condition, and will not be discussed further.

\subsubsection{Size of Constructions}
The size of a construction is the number of atoms 
and is computed as expected:

\begin{df}[Construction Size]
The size of a construction $h \in \spec{H}$, denoted $\setcard{h}$, is given by
\begin{align}
 \setcard{h \in \specI } & = 0 \\
\setcard{h \in \specX}& = 1 \\
\setcard{h \in (\spec{F} \splus \spec{G})} & =
      \begin{cases} \setcard{f} & h = (\circ \sprod f \in \spec{F})\\
                              \setcard{g} & h = (\bullet \sprod g \in \spec{G}) 
       \end{cases}\\
\setcard{(f,g) \in (\spec{F} \sprod \spec{G})} & = \setcard{f} + \setcard{g} 
\end{align}
\end{df}

\noindent
The set of constructions $\specF$ of a fixed size $n$ is denoted $\specn{F}{n}$.

\subsubsection{Structures from Admissible Constructions}
A \emph{combinatorial structure} is the result of replacing each atom in a construction
with a particular element of the sort it represents.

\begin{df}[Combinatorial Structure]
For a single sorted construction $\specF$, 
the collection of structures that can be created by substituting 
elements of a set $\setA$ for the atoms of $\specF$ is denoted $\specset{F}{A}$.
This definition extends to multi-sorted constructions,
with each atom being replaced by an element of the appropriate sort. 
\end{df}

Note that in order to have a well defined deterministic substitution operation,
it is necessary to have a total order over the atoms in the construction
and order the elements of each sort that are to be substituted.
The substitution ordering may be accomplished by matching explicit labels or
implicitly through the order inherent in the substitution operation%
\footnote{This requirement for ordered substitutions is realized explicitly in 
the substitution method of the Structure class (Substitution module) of the \GC implementation
(and Structure2, Structure3 for multi-sorted types).}.
For admissible polynomial constructions a canonical ordering exists,
labeling the atoms ``depth-first'' as if the construction were a tree,
using the inherent left to right order of the operands in the product operator term.
Although any ordering will suffice, it will be assumed that 
atoms in a polynomial construction are labelled in this order unless otherwise stated.

\subsubsection{ Composition of Admissible Constructions }

\begin{df}[Composition]
The composition of admissible constructs $\specG$ into $\specF$ is formally defined as (\cite{FlajoletSedgewick2009}, pg. 87)

$$\specF \scomp \specG = \sum_{k \ge 0} \specn{F}{k} \sprod ( \spec{G}^k ) $$
where $\specG^k = \specG \sprod \specG \sprod ... \sprod \specG$ is the $k$-fold product.
\end{df}

\noindent
The \emph{size} of a $f_k \scomp (g_1, g_2, ..., g_k) \in \specF \scomp \specG$ construct is therefore given by

$$\forall f_k \in \specn{F}{k}.\forall g_i \in \specG.\setcard{f_k \scomp (g_1, g_2, ..., g_k)} = \sum_{i = 1}^{k} \setcard{g_i}$$

where $f_k$ and the $ g_i$ are $\specF, \specG$ structures respectively.


\noindent
The class $\specF \scomp \specG$ can be intuitively interpreted as
structures where the atoms of a $\specn{F}{k}$ construction are substituted for  
the $k$ elements of an \emph{ordered} collection of $\specG$ constructions.
The size of such a structure is given by
the sum of the sizes of the $\specG$ structures.
Only the $\specG$ structures have atoms in a composed construction,
the $\specF$ atoms having been replaced with $\specG$ structures.

An alternate but equivalient definition of composition
is given by considering the construction to be defined over a finite set of elements $\setA$
that will be substituted for the atoms in the construction to create a structure.

\begin{df}[Composition via Partitions]\label{def:subbypart}
Let $\pi$ be a partition of a finite set $\setA$,
and denote
For constructions $\specF, \specG$
composition over a finite set $\setA$ is may also be defined as

$$(\specF \scomp \specG)[\setA] = \sum_{\pi of \setA} \specF[\pi] \sprod \prod_{p \in \pi} \specG[p]$$

where $\specG[p]$ is a $\specG$ structure over the part $p \text{ of } \pi$;
each $(\specF \scomp \specG)[p]$ structure is of size $\setcard{p}$.
\end{df}

\noindent
These two definitions of the construction are equivalent (\cite{FlajoletSedgewick2009}),
but this latter definition emphasizes that the composition
is a new class of constructions.

In order for this operation to be well defined under either definition,
there must be some unique order, or \emph{labeling},
of the $\specG$ populated atoms in the $\specF$ class such that
the $i^{th}\ \specG$ structure is consistently assigned to the $i{th}$ atom of $\specF$
(\cite{FlajoletSedgewick2009}, pg. 87).

\subsubsection{Systems of Polynomial Constructions }

Multiple combinatorial classes can be defined with respect to each other
by defining a system of construction classes.
This is the class of combinatorial constructions required for modelling Haskell types,
including mutually recursive type definitions.

\begin{df}[System of Polynomial Constructions]

A system (or specification) of mutually dependent $r$ combinatorial classes,
by convention called the \emph{specification} of  $\specF^{(1)}$,
is a collection of $r$ class specifications

\begin{array}[b]{rcl}
\specF^{(1)} & = & \Phi_{1}(\specF^{(1)}, ..., \specF^{(r)}) \\
\specF^{(2)} & = & \Phi_{2}(\specF^{(1)}, ..., \specF^{(r)}) \\
 & ... & \\
\specF^{(r)} & = & \Phi_{r}(\specF^{(1)}, ..., \specF^{(r)})
\end{array}

\noindent
where each $\Phi_{k}$ is a term built from the $\specF$ classes and the polynomial algebra.

\end{df}

\begin{df}[Polynomial Class]
A polynomial combinatorial class is the solution to a polynomial system%
\footnote{This is also known as the class of context-free specifications (\cite{FlajoletSedgewick2009}).}.
\end{df}

\noindent
For example, a planar tree with an arbitrary number of ordered branches
is a polynomial class described by

\begin{align*}
\spec{PT} & = \specX + (\spec{L} \scomp \spec{PT})\\
\spec{L} & = \specX \sprod \specL
\end{align*}

\subsubsection{Triangular Systems}
Simple systems of constructions do not exhibit any cyclic dependencies in their specification.
Any such system may be reorganized so that 
its specification forms a \emph{triangular} system,
where each class is dependent only on earlier specifications.
Triangular systems  may be reduced to a single construction,
and are always well-defined (\cite{FlajoletSedgewick2009}):

\begin{df}[Triangular Systems]
A system of polynomial construction classes is \emph{triangular}
if each $\specF^{(k)} (k \ge 2)$ is dependent only on the definitions of $\specF^{(i)}, 1 \le i < (k-1)$
with $\specF^{(1)}$ being defined strictly in the primitives and operators of the algebra.
\end{df}
\noindent

\begin{prop}[Triangular Systems Reduction] \label{prop:iterativeOGF}
A specification given by an triangular system can be expressed as a single polynomial.
\end{prop}
\begin{proof}
The proof for this follows simply from the substitution of the definitions
from $\specF^{1}$ into $\specF^{2}$, $\specF^{2}$ into $\specF^{3}$, etc.
\end{proof}

\begin{corollary}[Triangular Systems are Admissible]
If $(\specF^{1}, ..., \specF^{n})$ is specified as an triangular system,
then each $\specF^{i}$ is  \emph{admissible}.
\end{corollary}

\subsubsection{General Systems}
Combinatorial classes with cyclic dependencies (such as the planar tree defintion)
cannot be expressed as a triangular system.

\begin{df}[Recursive System of Polynomial Combinatorial Classes]
A system of construction classes is \emph{general} if it is not triangular.
\end{df}
\noindent
General systems require more sophisticated mathematics to prove they are admissible:
this will be demonstrated through the use of generating functions.

\subsection{Ordinary Generating Functions of Constructions}

\begin{df}[OGF of a Combinatorial Construction]
The ordinary generating function of a construction $\specG$ is a formal power series of the form 
$$ G(x) = \ogf{g}{n} $$
where the coefficients ($g_n$) are the number of G structures over a set of $n$ elements
(this is also called the ``counting sequence'' in \cite{FlajoletSedgewick2009}).
\end{df}
\noindent
If a construction has a well-defined OGF,
then there are a finite number of constructions for any given size,
and therefore the system is admissible.

The ordinary generating functions of polynomial constructs can be computed recursively,
where  $F(x), G(x)$ are the OGFs of constructions $\specF$ and $\specG$
(see \cite{FlajoletSedgewick2009}  for the proofs).

\begin{tabular}[b]{lll}\label{countseq}
Symbol(s)& OGF  & Counting Sequence  \\
$\specO$ & $0$ & $0$ \\
$\specI$ & $1$ & $A_{0} = 1, A_{n} = 0 \text{ for } n > 0$ \\
$\specX$ & $x$ & $A_{1} = 1, A_{n} = 0  \text{ for }  n \neq 1$ \\
$\spec{F} \splus \spec{G}$ & $ (F + G)(x) = \displaystyle{\ogf{(a+b)}{n}}$ &  $(F + G)_{n} = F_{n} + G_{n} $ \\
$\spec{F} \sprod \spec{G}$ & 
                   $F(x) \sprod G(x) = \displaystyle{\sum_{n=0}^{\infty} (\sum_{k=0}^{n} a_k b_{n-k}) x^n}$ &  
                   $ (F \sprod G)_{n} = \sum\limits_{k=0}^{n} F_{k} G_{n-k} $ \\
$\spec{F} \scomp \spec{G}$  & $(F \scomp G)(x) = F(G(x))$ & $(F \scomp G)_n = (F_{n}) ( G(x)^n )$
\end{tabular}

\noindent 
Note that for the composition to be well defined,
either the construction is triangular, so the generating function has finitely many terms,
or $\specn{G}{0} = 0$,
because the composition of formal power series 
is only defined where $G(0) = 0$ 
(\cite{WilfGeneratingFunctionology}, \cite{FlajoletSedgewick2009} appendix A.5, pg 731).

\subsubsection{Simple Substitution}

One important special case of composition
is the substitution of values from a base set for the atoms of a construction.
This kind of substitution can be modeled using a singleton set as an intermediating structure, i.e..

\begin{equation}
(\specF \scomp \specX) [\setA]  =  \sum_{k \ge 0} \specn{F}{k} \sprod (\specX [\setA])^k
\end{equation}

The OGF of the substitution will depend on
whether the same value can be substituted multiple times into the new construct,
and if there are other restrictions on the selection process.
The number of structures of a size $k$ populated from a set of cardinality $n$ will be 
the number of constructions of size $k$ multiplied by 
the number of subsets of size $k$ of the underlying set allowed for the substitution.
The number of subsets of a set is a standard problem in combinatorics.
If repetition is allowed, the OGF for the substitution of $k$ values from 
a type with cardinality $n$ is 

$$ \setcard{(\specF \scomp \specX)_{k}[\setA]} = F_k (\setcard{\setA})^k $$

However, if each element from the type can be drawn only once for each substitution,
ie. without repetition, the ogf is

\begin{equation}
F_n[A](x) = \sum_{k \ge 0} f_{k} \binom{n}{k}
\end{equation}

The number of populated structures of a given size
is important for property based testing
as it represents the size of the population to be sampled for the test.

\subsubsection{Existence of OGF for Admissible Polynomial Classes}
Any system with an OGF that can be computed from the specification
has a finite number of constructions for each size of structure,
and therefore is admissible (well-defined).

\begin{df}[Admissible Constructions]

Let $\spec{A} = \Phi( \spec{B}^1, \spec{B}^2, ..., \spec{B}^m )$ be a specification over admissible constructions,
then the $m-ary$ construction $\Phi$ is \emph{admissible} if the ordinary generating function of \spec{A}
is only dependent on the OGFs of $\spec{B}^i, 1 \le i \le m$.

\end{df}
\noindent

\begin{theorem}[OGF of Admissible Polynomial Constructible Class]\label{OGFPolynomialClass}
The ordinary generating function of an admissible polynomial specification $\specA$ is an \emph{algebraic function}, 
i.e. there exists a bivariate polynomial such that

$$P(x, y) \in \field{C}[x,y] . P(x, F(x)) = 0$$

and that this can be solved using algebraic elimination to produce 
a polynomial equation $F(X)$ for its generating function.

\end{theorem}
\begin{proof}
\cite{FlajoletSedgewick2009} (proposition 1.17, pg.80)%
\footnote{They actually provide the proposition over a wider class of combinatorial constructions,
but the theorem and proof are not dependent on the availability of these other operators.}.
\end{proof}

\noindent
It should be noted that in \cite{FlajoletSedgewick2009} 
it is unclear that only well defined structures
should be included in this class,
but that this is clearly required in the earlier work \cite{FlSaZi91}.



\section{Enumerating Haskell Algebraic Data Types}
%\section{Enumerating Haskell Algebraic Data Types}

Haskell algebraic data types can be modelled by combinatorial constructions,
The complexity of the term is the number of terminal nodes in the Haskell term
(i.e. the combinataorial complexity, section \ref{complexgen}).
If the construction is admissible,
then there are a finite number of terms of that type for any given complexity,
and the type can be enumerated.
The approach is straight-forward,
requiring only that:
\begin{itemize}
\item Haskell data constructors with multiple arguments
be treated as combinatorial products,
\item type constructors with multiple disjoint unions be evaluated pairwise, 
\item and nullary constructors (constants) be treated as an atom over a distinct $1$ element sort.
\end{itemize}

\begin{tabular}[b]{llll}\label{countseq}
Haskell Type Operator & Construction & Complexity & Comment\\
constant & $\spec{X}_{a}$ & $1$ &atom over sort of single element $a$ \\
singleton & $\spec{X}_{T}$ & $\setcard{T}$ & single element of type $T$ \\
$T_1 \mid \dots \mid T_n $ & $\spec{T}_{1} \splus \dots \splus \spec{T}_{n}$ & $\sum_{k=1}^{n} $ & sum of disjoint terms \\
$C\ T_1\ T_2 $ & $ \spec{T}_{1} \sprod \spec{T}_{2}$ & $\setcard{\spec{T}_{1}} \sprod\setcard{\spec{T}_{2}} $ &  
                   product of constructions for each argument  \\
$C\ T_1\ \dots\ T_n $ & $ \spec{T}_{1} \sprod \spec{S}$ & $\setcard{\spec{T}_{1}} \sprod\setcard{\spec{S}} $ &  
                   where  $\spec{S} = \spec{T}_{2}\sprod \dots \sprod \spec{T}_{n}$ for $n > 2$ , $\spec{S} = \spec{T}_{2}$ for $n = 2$
\end{tabular}

\noindent
This formulation guarantees that all constructions, 
including nullary constants,
are of size at least one,
and therefore satisfies the ``epsilon-freeness'' constraint for composing constructions.
The consequence of this weighting of type constants 
is that certain data types will have a non-intuitive complexity, or size.
For example,
an empty list will be of size $1$, not $0$,
as might be expected at first glance.
Evaluating a property over an empty list, however,
does entail at least one operation,
and so this is a reasonable assessment of the complexity
with respect to generating test cases.

The alternative approach of allowing constants to be of size 0
allows certain valid Haskell types to be inadmissible constructions,
i.e. to have infinitely many constructions of the same size.
Consider this somewhat unusual but allowable definition
of a binary tree with either a null or element bearing leaf node:

\begin{lstlisting}
data T a = L | N a | Br (T a) (T a)

t1 = Br L (N 1)
t2 = Br L (Br L (N 1))
t3 = Br L (Br (Br L L) (Br L (N 1))
t4 = Br (Br (Br L L) (Br L L)) ((Br L (N 1)) (Br L L))
\end{lstlisting}

The complexity of each of these trees would be $1$ if
the nullary leaf constructor is counted as weight $0$,
instead of $ 2, 3, 5, 8$ respectively when counted as $1$.
Although unusual it is still a valid Haskell type and therefore must be handled by
any \pbt system generating test cases automatically.
Any confusion over the size of containers with no elements
is outweighed by the ability to correctly generate all valid Haskell algebraic data types.

\section{Algorithms for Enumeration}\label{sec:iterenumalg}
%\section{Iterative Algorithm for Enumerating Algebraic Data Types}

Combinatorial structures were introduced to provide 
a mathematical proof for the existence of 
ranked enumerations of recursive algebraic data types.
\cite{FlajoletZC94} also provides \emph{iterative} algorithms 
for constructing and efficiently computing 
the enumeration count and selector functions.
A summary of the algorithm is provided here,
restricted to the systems of polynomial constructions of interest,
(i.e. excluding the constructions set and cycle
which are not implementable in Haskell).
This algorithm was used in \GC to create computable enumerations of Haskell types
for use in automated test case generation,
presented in chapter \ref{chap:source}.

\subsection{Iterative Counting Sequence}

A class of constructions can be considered 
the limit of a sequence of approximations to the class
restricted to structures over sets of elements up to a maximum size.
The iterative solution begins with computing the number of structures of each size, starting from $0$.
The count function $c_\spec{F} : \nat \rightarrow \nat$,
where $\specF, \specG$ are combinatorial classes,
F and G are their counting sequences, 
and n is the size, is constructed inductively:

\begin{equation}
\begin{aligned}
c_{\epsilon}(n) & =  \begin{cases} 1 & n = 0  \\  0 & \text{otherwise}   \end{cases} \\
c_{\specX}(n)  & =  \begin{cases} 1 & n = 1  \\  0 & \text{otherwise}   \end{cases} \\
c_{(\specF \splus \specG)}(n) & =  c_{\specF}(n) +  c_{\specG}(n)\\
c_{\specF \sprod \specG}(n) & = \sum_{k=0}^{n} c_{\specF}(k)  c_{\specG}(n-k)
\end{aligned}
\end{equation}

\noindent 
The full presentation is given in  \cite{FlajoletZC94}, 
and is shown to be an $O(n^{2})$ cost computation
(it is assumed that the counting sequences will be computed once).

The count function of a triangular (non-recursive) construction follows from the specification.
For general constructions, the count function is the solution of a non-linear set of equations,
such as for a binary tree with both singleton and nullary leaves:

\begin{align*}
\spec{T} & = \specI \splus \specX \splus (\spec{T} \sprod \spec{T}) \\
\begin{split}
c_{\spec{T}}(n) & = c_{\specI}(n) + c_{\specX}(n) + c_{\spec{T} \sprod \spec{T}} \\
 & = c_{\specI}(n) + c_{\specX}(n) + \sum_{k=0}^{n} c_{\spec{T}}(k) c_{\spec{T}}(n-k)
\end{split}
\end{align*}

\noindent
In this case allowing $0$ size $\epsilon$ constructions in the specification
causes the term $c_{\spec{T}}(n)$ to appear on both sides of the equation,
and the count function must be solved as an implicit equation.

The system of questions is simplified if products that permit component structures of size 0 are not allowed,
i.e. every system is $\epsilon$-free.
The counting functions for the products 
(\cite{FlSaZi91}, pgs 18-20, not to be confused with the count function for \emph{labelled} structures)
are given by:

$$ c_{\specF \sprod \specG}(n) = \sum_{k=1}^{n-1} c_{\specF}(k) \sprod c_{\specG}(n-k) $$

\noindent
and so form a triangular system of equations.

The decision to assign Haskell nullary constructors a weight of $1$ instead of $0$
guaranteed that the counting function is explicit,
and the construction of the counting and selection functions throughout this thesis
and the \GC package assume that there are no $0$ weight constructions.

The implementation of the count function in \GC (presented below)
demonstrates the inductive construction of the count functions.
Note that disjoint unions and products with more than two arguments
are handled explicitly instead of using pairwise interpretations;
also note that constants are given a rank of $1$,
as if they were singletons over a sort of a single element,
to ensure epsilon-freeness.

\begin{lstlisting}
type Counter      = Rank -> Count

cConst, cNode :: Counter
cConst r = if r == 1 then (1::Count) else (0::Count)
cNode  r = if r == 1 then (1::Count) else (0::Count)

cSum, cProd :: Counter -> Counter -> Counter
cSum c1 c2 r = c1 r + c2 r
cSum3, cProd3 :: Counter -> Counter -> Counter -> Counter
cSum3 c1 c2 c3 r = c1 r + c2 r + c3 r
cSum4, cProd4 :: Counter -> Counter -> Counter -> Counter -> Counter
cSum4 c1 c2 c3 c4 r = c1 r + c2 r + c3 r + c4 r

cProd c1 c2 r = sum $ map product (sizes r)
    where sizes r' = map (zipWith ($) [c1, c2]) (compositions1 2 r')
cProd3 c1 c2 c3 r = sum $ map product (sizes r)
    where sizes r' = map (zipWith ($) [c1, c2, c3]) (compositions1 3 r')
cProd4 c1 c2 c3 c4 r = sum $ map product (sizes r)
    where sizes r' = map (zipWith ($) [c1, c2, c3, c4]) (compositions1 4 r')

\end{lstlisting}

\subsection{Iterative Index}

The selector function of an enumeration is based on 
indices over the constructions of a common size.
Each index is a total ordering of the constructions of a given size,
denoted here as $ \combdex{\spec{F}}(n) :: \specn{F}{n} \ra [1 .. c_{\spec{F}}(n)], n \ge 0$.
The selector function is the inverse of the indexing function for each size.
If there are no admissible constructions of a given size for the specification,
then the index is empty ($[\ ]$) and the selector function is defined nowhere for that size.
The indexing of the constructions is an abstract function and never implemented,
but is the implicit basis for the inductive definition of the selector function to follow.

The index function used in \cite{FlajoletZC94},
and in \GC enumerations,
is the canonical left to right, top to bottom, approach to counting the elements of unions and Cartesian products,
defined inductively based on the class specification:

\begin{equation}
\begin{aligned}
\combdex{\specI}(n)   & =  \begin{cases} [1] & n = 0  \\  [\ ] & \text{otherwise}   \end{cases} \\
\combdex{\specX}(n)   & =  \begin{cases} [1] & n = 1  \\  [\ ] & \text{otherwise}   \end{cases} \\
\combdex{(\specF \splus \specG)}(n) & =  \combdex{\specF}(n) \concat  \combdex{\specG}(n) \\
     & =  [1\ \ldots\ (c_{\specn{F}{n}} + c_{\specn{G}{n}})] \\
\combdex{(\specF \sprod \specG)}(n)   
    & = \combdex{\specn{F}{0} \sprod \specn{G}{n}}(n) \concat\ \combdex{\specn{F}{1} \sprod \specn{G}{n-1}}(n) \concat\
            \ldots \concat \combdex{\specn{F}{n} \sprod \specn{G}{0}}(n) \\
    & = [1 \ldots\ (\sum_{k=0}^{n} (c_{\specn{F}{k}} c_{\specn{G}{n-k}}) ]
\end{aligned}
\end{equation}

\noindent where $\concat$ is the concatenation of two indices
with an appropriate renumbering of the second index.

\noindent
When no structures of size 0 are allowed in products,
the index over products is given by

\begin{equation}
\begin{aligned}
\combdex{(\specF \sprod \specG)}(n)   
    & = \combdex{\specn{F}{1} \sprod \specn{G}{n-1}}(n) \concat\ \combdex{\specn{F}{2} \sprod \specn{G}{n-2}}(n) \concat\
            \ldots \concat \combdex{\specn{F}{n-1} \sprod \specn{G}{1}}(n) \\
    & = [1 \ldots\ (\sum_{k=1}^{n-1} (c_{\specn{F}{k}} c_{\specn{G}{n-k}}) ]
\end{aligned}
\end{equation}

\noindent
As with the counting function,
when the guarantee of $\epsilon$-freeness is not available,
the combinatorial class must first be proven admissible before 
applying the more general indexing / selection of products.

\subsection{Iterative Selector Function}

The selection function for a specification 
($ s_{\specF}(n) : \combdex{\spec{F}}(n) \ra \specn{F}{n} $) inverts the above algorithm
to provide the mapping of the index interval to the constructs of the specified size.
The selection function for each size of construct 
can be constructed inductively 
from the counts and selection functions of the smaller constructs,
following the class specification to determine 
which operand from each product or disjoint union
the desired element is contained in.

This ``de-referencing'' of the index value requires 
some rather unattractive but conceptually simple bookkeeping.
The algorithm below provide an inductive definition of
the selector function of a combinatorial class,
based on the known counts of the classes in the system.
Letting $\specF, \specG$ be constructions specifications:

\begin{df}[Iterative Selector Function]


\begin{equation}
\begin{aligned}
s_{\epsilon} (n, i) & = 
     \begin{cases} \epsilon & n = 0 \text{ and } i = 1 \\ 
                               \text{undefined } & \text{otherwise}
                               \end{cases} \\
s_{\specX} (n, i)   & =  
     \begin{cases} \specX & n = 1 \text{ and } i = 1 \\ 
                               \text{undefined } & \text{otherwise}
                               \end{cases} \\
s_{\specF \splus \specG} (n, i) & =
    \begin{cases} s_{\specF} (n, i) & 1 \le i \le c_{\specF}(n) \\
                              s_{\specG} (n, i - c_{\specF}(n)) & (c_{\specF}(n) + 1) \le i \le c_{\spec{G}}(n)
                              \end{cases} \\
s_{\specF \sprod \specG}(n,i) & = s_{\specn{F}{k} \sprod \specn{G}{n-k}}(n, i -  \sum_{j=0}^{k-1}\ c_{\specn{F}{j} \specn{G}{n-j}}(n)) \\
& \text{where } \sum_{j=0}^{k-1}\ c_{\specn{F}{j} \spec{G}{n-j}}(n) < i \le \sum_{j=0}^{k}\ c_{\specn{F}{j}  \specn{G}{n-j}}(n)
\end{aligned}
\end{equation}
\end{df}


\noindent
Dereferencing the index of a product treats it as a disjoint union of pairs of fixed size specifications,
requiring the count of each of the product components,
then picking the appropriate $k$ and using the selection function of those components.
Where $\epsilon$ constructions are not allowed,
the same algorithm is used but ignoring $0$ sized constructions in the pairs.

The implementation of the selection function in \GC (presented below)
demonstrates the inductive construction.
Disjoint unions and products with more than two arguments
are again handled explicitly,
and constants are given a rank of $1$.
The selector combinators use the |interval| and |deref| functions
to determine which operand of a product or sum the desired value inhabits;
|selector_error| handles the exception of an out of range selection.

\begin{lstlisting}
data Enumeration c a = Enum Counter (Selector c a)
counter :: Enumeration c a -> Counter
counter (Enum c _) = c
selector :: Enumeration c a -> Selector c a
selector (Enum _ s) = s

sConst,sNode :: (Num a, Num b, Eq a, Eq b) => t x -> a -> b -> t x
sConst x r n = if (r==1) && (n==1) then x else selector_error
sNode  x r n = if (r==1) && (n==1) then x else selector_error
sSum :: Enumeration c a -> Enumeration c a -> Selector c a
sSum (Enum c1 s1) (Enum c2 s2) r n = 
  let n1 = c1 r in if (n <= n1) then s1 r n 
                   else if (n <= c2 r) then s2 r (n - n1)
                   else selector_error

sSum3 :: Enumeration c a -> Enumeration c a -> Enumeration c a -> Selector c a
sSum3 (Enum c1 s1) (Enum c2 s2) (Enum c3 s3) r n = 
  let n1 = c1 r 
      n2 = c2 r
      in if          (n <= n1)   then s1 r n 
           else   if (n <= n2)   then s2 r (n - n1)
           else   if (n <= c3 r) then s3 r (n - (n1+n2))
           else   selector_error

sSum4 :: Enumeration c a -> Enumeration c a -> 
		Enumeration c a -> Enumeration c a -> Selector c a
...

sProd ::  (a x -> b x -> c x) -> Enumeration a x -> 
		Enumeration b x ->Selector c x
sProd con (Enum c1 s1) (Enum c2 s2) r n =
   let ([r1,r2],n') = interval n $ enumProd c1 c2 r
       [n1,n2]      = deref [c1 r1, c2 r2] n'
   in con (s1 r1 n1) (s2 r2 n2)

sProd3 ::  (a x -> b x -> c x -> d x) -> Enumeration a x -> Enumeration b x -> 
		Enumeration c x -> Selector d x
sProd3 con (Enum c1 s1) (Enum c2 s2)  (Enum c3 s3) r n =
   let ([r1,r2,r3],n') = interval n $ enumProd3 c1 c2 c3 r
       [n1,n2,n3]      = deref [c1 r1, c2 r2, c3 r3] n'
   in con (s1 r1 n1) (s2 r2 n2) (s3 r3 n3)

sProd4 ::  (a x -> b x -> c x -> d x -> e x) -> Enumeration a x -> 
    Enumeration b x -> Enumeration c x -> Enumeration d x -> Selector e x
...

\end{lstlisting}

Enumerations index $n$-products as $n$-dimensional cubes.
\emph{interval} takes an index into a rank $r$ $n$-product and determines the 
ranks $(r_1, ..., r_n)$ of each of the $n$ components of the $n$-product where 
$\sum r_{i} = r$, and an index into just the $n$-products of that particular rank
distribution.

\emph{deref} converts the one dimensional index into the $n$-product,
along with the ranks of each of the components of the product,
into an $n$-tuple index into each component.

\begin{lstlisting}
interval :: (Integral k) => k -> [(k,a)] -> (a, k)
interval k' ps = intv k' 0 ps where
  intv _ _ [] = error "Trying to partition an empty interval"
  intv k _ _ | k < 1 = error "Trying to partition a one element interval"
  intv k acc ((p, x) : ps') = 
     let acc' = acc + p 
     in if (k <= acc') then (x, (k-acc)) else intv k acc' ps'

deref :: (Integral k) => [k] -> k -> [k]
deref ds k'  | k' >= 1 =  drf (dimd ds) k'
             | otherwise  = error "Dereferencing below 1"
  where
    drf (dmax:[1]) k | k <= dmax = [k]
    drf (dmax:(dm:dims)) k | k <= dmax =
      let (c,j) = ((k-1) `divMod` dm)
          ds' = drf (dm:dims) (j+1)
      in (c+1) : ds'
    drf _ _ = error "impossible dereference"
    dimd [] = [1]
    dimd (d:ds') = let (md:mds) = dimd ds' in (md*d : (md:mds))
\end{lstlisting}

\subsection{Iterative Enumerations}
Given the combinator functions for count and selector functions,
enumerations for a type are constructed by mirroring the type construction
with these combinators.
The constant |A| is from a type |Label|, and represents an atom
which will be replaced with values in a \GC substitution generator.

A \GC enumeration for a simple tree structure demonstrates the simplicity of constructing the enumeration.

\begin{lstlisting}
data BinTree a = BTNode a | BTBr (BinTree a) (BinTree a)
instance Enumerated BinTree where
  enumeration = eBinTree

eBinTree :: Enumeration BinTree Label
eBinTree = eMemoize $ e
  where e = eSum (eNode (BTNode A)) 
                 (eProd BTBr eBinTree eBinTree)

\end{lstlisting}


The \GC implementation bundles the count and selector combinators in an |Enumeration| structure for convenience.
Note that the pattern matching on the sums must be lazy (irrefutable) to allow the
manual counting of the possible structures of that size to terminate the
recursive construction of the patterns.
Also note that enumerations are memoized to ensure the count and selector functions
are not continuously recalculated.

\begin{lstlisting}
data Enumeration c a = Enum Counter (Selector c a)
data Label = A | B | C | D | E | F | G 
  deriving (Show, Eq, Enum, Bounded)

eMemoize :: Enumeration c a -> Enumeration c a
eMemoize (Enum c s) =  mkEnum (memoize c) (memoize s)

eConst, eNode :: c a -> Enumeration c a
eConst x = mkEnum cConst (sConst x)
eNode  x = mkEnum cNode  (sNode  x)

eSum :: Enumeration c a -> Enumeration c a -> Enumeration c a
eSum  e1@(~(Enum c1 _)) e2@(~(Enum c2 _)) = 
		mkEnum (cSum c1 c2) (sSum e1 e2)
eSum3 :: Enumeration c a -> Enumeration c a -> 
		Enumeration c a -> Enumeration c a
eSum3 e1@(~(Enum c1 _)) e2@(~(Enum c2 _)) e3@(~(Enum c3 _)) 
          = mkEnum (cSum3 c1 c2 c3) (sSum3 e1 e2 e3)
eSum4 :: Enumeration c a -> Enumeration c a -> 
		Enumeration c a -> Enumeration c a -> Enumeration c a
...

eProd :: (a x -> b x -> c x) -> Enumeration a x -> 
		Enumeration b x -> Enumeration c x
eProd con e1@(Enum c1 _) e2@(Enum c2 _) = 
		mkEnum (cProd c1 c2) (sProd con e1 e2)
eProd3 :: (a x -> b x -> c x -> d x) -> Enumeration a x -> 
		Enumeration b x -> Enumeration c x -> Enumeration d x
eProd3 con e1@(Enum c1 _) e2@(Enum c2 _) e3@(Enum c3 _) = 
         mkEnum (cProd3 c1 c2 c3) (sProd3 con e1 e2 e3)
...

\end{lstlisting}






\section{Substitution vs. Expanded Enumeration}
%section{Substitution in Enumeration Generators}

The study of combinatorial constructions is generally concerned
about the number of ``shapes'' satisfying a specification,
with the shape of the construct consisting of products of atoms and epsilon values.
A test case generator, however,
must produce concrete terms that result from substituting values for the atoms,
i.e. treating the atoms as singleton constructors encapsulating a value of the appropriate sort.
There are two choices for dealing with each of the atoms in a specification 
to construct a term generator from the enumeration:

\begin{enumerate}
\item Extend the construct's specification, and hence enumeration, 
to include all allowable value substitutions from the underlying sort ,
replacing the atom with a disjoint union of each of those values.
The counting, index and selector functions naturally extended
by replacing the counts for the atoms with a totally ordered set of the substituting values,
in the same way the ordinary generating functions are extended,
as discussed in section \ref{sec:combconstructs}.
\item Allow the generator to sample the enumerated \emph{shapes} of the constructions,
the unpopulated constructs with the atoms as placeholders.
These will be called \emph{structures} to be distinct from concrete terms.
This generator will subsequently be combined with a generator for values of the appropriate type
to produce the concrete terms for use in testing.
This kind of generator is called a substitution generator.
\end{enumerate}

\noindent
This choice can be made independently for each of the atoms in the specification,
so the resulting enumeration can be over concrete terms,
or structures with some or all of the distinct atom types 
that must later be substituted.

The choice of expanding an enumeration to include the values
or post-sampling substitution is pragmatic, 
depending on the nature of the underlying sorts,
and the significance of the data structures relative to the actual values in the test cases generated.
In particular,
if an atom represents a small number of allowable constants,
such as a Boolean value,
it may be preferable to include that substitution in the enumeration.
If there are restrictions on nature of the substitution,
such as if the values are selected without duplicates (replacement)
or there are conditions on the elements of the substitution set,
it may be easier to enumerate the them separately.
On the other hand, it would not be desirable to include all of the values of a large Int class
in the enumeration when generating test cases for a sort algorithm,
as the actual values in each node make little impact on the test case.
Note that the substitution approach may also be used if the values for the substitutions
are themselve structures by ignoring their rank in the enumeration.
For example, if a sort algorithm is to be tested by comparing strings,
the length of the string may be ignored in the enumeration and 
strings of different length selected as part of the value sampling methodology.
Both approaches result in a sample of the constructions
populated with the elements of the underlying set (or type),
but the properties of the test data seta
and the sampling methods used may be quite different.

\subsection{Substitution Generators and Strategies}

The \GC implementation of substitution generators is provided here
as an illustration of how substitution combinators can be used
to provide more options for test case generation.
The |Structure| class defines a |substitution| method, the
mechanical details of replacing the atoms in the structure with a single sort of elements.
Classes for two sorted structures, etc. are simple mechanical extensions of the single sort substitution.
Instances of Structure can be generated mechanically from the data constructor
definitions via Template Haskell.

The subst, substn and substAll functions below are generator combinators,
creating substitution generators from structure and element generators.
The ordered collection of values drawn from the element generator for insertion into a structure
is referred to as a substitution tuple.

\begin{description}
\item[subst] populates each generated structure with one generated substitution tuple
\item[substN] populates each generated structure with n substitution tuples
\item[substAll] populates each structure with every substitution tuple
        until exhausting the element generator (non-terminating for infinite generators)
\end{description}


\begin{lstlisting}
class Structure c where
  substitute    :: c a -> [b] -> (Maybe (c b), [b])

subst :: Structure c => Generator (c a) -> Generator b -> Generator (c b)
subst gfx gy r =  
  let fxs = (gfx r)
      ys  = gy 1
  in gsub fxs ys

gsub :: Structure c => [c a] -> [b] -> [c b]
gsub [] _       = []
gsub (fx:fxs) ys = 
   let (mfy, ys') = substitute fx ys
   in case mfy of
        Nothing -> []
        Just fy -> fy : (gsub fxs ys')

substN :: Structure c => Int -> Generator (c a) -> Generator b -> Generator (c b)
substN n gfx gy r = gsubN n n (gfx r) (gy 1)

gsubN :: Structure c => Int -> Int -> [c a] -> [b] -> [c b]
gsubN n _ _  _ | n < 1 = []
gsubN _ 0 _  _         = []
gsubN _ _ [] _         = []
gsubN n k fxs@(fx:fxss) ys = 
   let (mfy, ys') = substitute fx ys
   in if k > 1 then maybe [] (\fy -> fy : (gsubN n (k-1) fxs ys')) mfy
               else maybe [] (\fy -> fy : (gsubN n n fxss ys')) mfy 

substAll :: Structure c => Generator (c a) -> Generator b -> Generator (c b)
substAll gfx gy r = 
  let ys = gy 1
      fxs = gfx r
  in gsubAll fxs ys

gsubAll :: Structure c => [c a] -> [b] -> [c b]
gsubAll [] _ = []
gsubAll l@(fx:fxs) ys = 
   let (mfy, ys') = substitute fx ys  -- subst elements into first structure
       -- sub same ys into remaining list of structures, guaranteed finite
       fys = catMaybes $  map (fst.((flip substitute) ys)) fxs
   in maybe [] (\x -> x:(fys ++ (gsubAll l ys'))) mfy
\end{lstlisting}

The value of combinatorial complexity as the rank of a structure
is most apparent in a substitution generator.
For a structure with a single class of atoms,
the number of values required to complete the substitution 
is the rank of the structure.
Where there are multiple classes of atoms,
the rank will be composed of the number of required elements of each sort;
any elements that were included in an expanded enumeration
will also be part of this rank but will not require substitution.
In either case, the substitution will be a function 
providing a concrete term from a structure
and an ordered tuple of values of each sort.

Substitution of values for atoms introduces an new kind of choice 
as part of the generator sampling strategy.
The simplest approach to substitution is to 
replace the atoms in each structure with generated values
to replace the instances of the atoms, creating a single term for each,
the approach described as ``step-by-step'' in section \ref{adtgen}.
Other alternatives are:
\begin{itemize}
\item populate each structure with a fixed number of substitution tuples;
\item generate a finite number of both structures and substitution tuples, 
and populate all structures with all tuples;
\item populate each structure with all permutations of the substitution tuple
(this is particularly useful when testing symmetrical and anti-symmetrical properties).
\end{itemize}
The choice of substitution strategy will be based on 
the goal of the test,
the number of possible substitution sets and
the cost of generating the structures.
Note that substitution strategies are not specific to enumerative generators,
and can be used for any substitution generator.

%\subsection {Enumerative Test Strategy}
%The availability of datatype generic sampling methods over enumerations
%can be extended to produce entire \emph{test strategies} that are independent of the underlying data types.
%A test strategy is the implementation of one or more sampling methods,
%including stratified sampling methods and substitution strategies for ranked constructs,
%that results in an executable test data set for a property based test.
%An example of a compound sampling strategy that is suitable for property based testing
%over a domain consisting of a general tree of integer values is :
%
%    \begin{enumerate}
%    \item exhaustively select smallest / simplest test cases up to a certain size,
%    uniformly sampling 10 integer values including the boundary values (largest positive and negative values)
%    \item from where exhaustive testing is infeasible
%    up to the maximum complexity that will be well tested:
%    \begin{itemize}
%    \item use uniform selection to establish coverage over all parts
%    \item select the boundary structures, namely the empty tree and slngle leaf trees
%    \item randomly select additional values to
%    avoid any possible bias introduced from the uniform and boundary selection criteria
%    \end{itemize}
%     substituting an arbitrary ordering of 0, the boundary values, and random integer values for each tree
%    \item randomly select a small number of values from random ranks up to some arbitrary maximum complexity,
%    using the same integer value substitution strategy
%    \end{enumerate}
%
%This particular strategy depends only on the exhaustive and well-tested complexity sizes
%and a fairly generic integer test value generator,
%and could be used for any Haskell algebraic data type.
%The appropriateness of the strategy must be determined,
%but it is distinctly superior to any single sampling methodology applied to the entire population of general integer trees.
%A library of such datatype generic test strategies can be provided
%to leverage the availability of complexity ranked enumerations
%and substitution generators.






\section{\FEAT Enumerations}

There are a number of differences between the \FEAT (\cite{Duregard2012} ) enumerations
and those described in this chapter:

\begin{enumerate}
\item \FEAT uses the number of constructors embedded in the term as it's size,
instead of the number of atoms in the construct.
\item the enumeration is over concrete terms, meaning that all element types are incorporated into the overall enumeration
as opposed to supporting post enumeration substitution;
\item  a single sampling method is applied to the enumeration to create the test data set
(although the test can be repeated independently with each of the three supported sampling methods);
\item \FEAT does not support explicit substitution of independently sampled values into structures;
\item a different enumeration algebra is supported - 
they explicitly discuss \cite{FlSa95} and their reasons for not using this approach.
\end{enumerate}

Their approach, while providing a single source of random, uniform and exhaustive test case generators,
does not allow for more complicated hybrid sampling methodologies
nor for specialized generators for specific domains (e.g. a ``person name generator'' or ``physical address generator'')
as an alternate sampling methodology.
The main focus of \FEAT was to provide computationally efficient enumerations,
so further analysis of their enumerations is left to the chapter \ref{chap:source}.
\gordon{I better make sure I actually did this.}

One interesting detail is that the issue of epsilon-freeness in the theory of combinatorial structures
presents itself as the need to assign an explicit cost to recursion in the calculation of term size
(in the paper this is referred to as the part identifier).
In their enumeration part cost function (equivalent to rank in this thesis),
constants (called singletons) are given a cost of $0$.
The authors point to a representation of the natural numbers

\begin{code}
data Nat = Z | S Nat
\end{code}

and note that without an explicit cost for each recursive call,
all of the natural numbers would end up with the same cost,
thus violating the requirement that each part of the population must be finite.
This specification, as stated, cannot be represented in the theory of combinatorial constructions,
but must instead be altered so |S| is a product :

$$ \spec{N} = \specI \splus (\spec{S} \sprod \spec{N}) $$

where $\specI, \spec{S}$ are constants with weight $1$.
In effect, the natural numbers are modelled as lists of constants,
and the need to provide a weight of $1$ to the $\spec{S}$ constant to keep the structure epsilon free
is equivalent to the need for the recursion cost in the \FEAT enumeration.







%------------------------------------------------------------------------------
\setcounter{figure}{0}
\setcounter{equation}{0}
\setcounter{table}{0}
\chapter{Requirements for Property Based Testing Systems}\label{chap:requirements}

The sampling based \pbt software discussed in chapter \ref{pbtsystems}
have a number of similarities in their design
while adopting a variety of test selection and generation strategies.
One of our main criticisms of the ``*Check'' family
is their tight coupling of a test case generation, evaluation and reporting
(section \ref{pbtsystems}).
Rather a system should allow different sampling generators to be combined with
modules that evaluate and report test results in the most appropriate way for
the situation at hand.
This would be best implemented as a \emph{framework} 
based on common interfaces for generating, evaluating and reporting.

This chapter provides the requirements that a generic \pbt system framework
should satisfy, guided by the critiques from chapter \ref{pbtsystems},
by the analysis of test context supporting hypotheses (section \ref{pbt}), and 
sampling methodologies (section \ref{sec:sampling_theory}).
The requirements cover:
\begin{enumerate}
\item the definition of the properties to be tested,
\item the organization of test cases,
\item the automatic generation of test cases,
\item scheduling the evaluation of properties over test cases,
\item reporting the results of the test.
\end{enumerate}
\noindent
To remain concrete and be able to compare with \GC, we assume that the code
to be tested and the framework will be written in Haskell.  However, the
requirements should apply to other (functional) programming languages.
In places, \emph{non-requirements} will also be explicitly spelled out.

\section {Properties} \label{reqprop}

Properties need to be defined by the user, and usable by the framework.

It is generally assumed that properties are (unary) predicates which
are total on their domain of definition (section~\ref{formal_pbt}).
Optionally, one can also give an additional predication which is a decision
procedure for domain-membership --- typically used when only a subset of
a type is the natural domain of a function to be tested.

It should be possible to \emph{combine} predicates, so that multiple
properties (on the same data) can be combined.

The requirement that properties be unary predicates does seem to be
quite drastic.  However, if instead of requiring as input a predicate,
instead we require an \emph{eventual} predicate --- i.e. a function
which when saturated with all its (curried) arguments, \emph{is}
boolean-valued, that requirement is no longer a burden.  This is what
\QC does.

However, this sequential approach to multivariate testing does not allow
the effective use of systematic samplics methods, as the domain is not
effectively know at the time of test case generation. This is also
a problem with random sampling for enumerative generators,
which rely on having the complete definition of the test domain to
establish the probability distributions for selection.

Instead, properties should be uncurried, to give access to the entire
domain at once. Of course, there is no suggestion the the functions being
tested shold be uncurried --- just the property.

The domain of a property is thus a single type (but which can be a 
product). However the property may not be defined over all values of that
type.  For example, a property might apply only to ordered lists,
but the condition of ordered cannot (easily) be represented in the type system.
It is acceptable for a property to be undefined outside of its domain.
Thus we may need a \emph{precondition}, another predicate on the same
type (as that of the property function) which represents the 
property's domain.  \QC and other packages incorporate these preconditions
into the definition of properties via a conditional combinator.

\emph{Existential variables} are, however, not a requirement for our \pbt.
\SC and \GAST support properties that incorporate \emph{existential} variables
with the goal of ensuring that at least (or exactly) one value exists that
satisfies the property.  This is an appealing feature,
but solving for the existence of a value is substantially different than
evaluating a property, and as such is \emph{not} included as a
requirement.

Existential properties can be emulated by testing the negated property
i.e. $\exists x . p (x) \iff \neg (\forall x . \neg p (x))$.

\section{Organizing test cases} \label{reqtestsuite}

A core feature of a \pbt is to evaluate properies over a
collection of test cases, i.e. a \emph{test suite}.
Attributes of test cases can be useful to organize the test suite.
For example, complexity and strata can be used to order sampling
strategies. Independence can be used for parallel evaluat on distributed
systems. Meta-data about the functions being test can be used to support
coverage analysis in the reporting of results. Lastly, attributes can
be used as a filter to report verdicts on just part of the test cases.

\subsection{Test Cases}

Each test case will consist of:

\begin{description}
    \item[datum] the value over which the property will be evaluated,
    \item[meta-data] additional information about that value, including the rank
\end{description} 
\noindent
A test framework should allow different representations of test cases,
to enable attaching different kinds of meta-data,
by defining a standard interface to access the datum.

\subsection{Rank as a Complexity Measure }\label{sub:reqrank}

As mentioned before, term complexity plays an important role
in the selection, generation and evaluation priority of test cases.
Using a complexity measure to partition values into finite subsets
ties test results back to uniformity and regularity hypotheses,
required for arguing validity of the testing process.

Thus all terms should come with some complexity measure, which we will call
the \emph{rank} of the term. We will thus \textbf{require} that a rank
be available as test case meta-data.

Recall that section~\ref{complexgen} discussed the various complexity measures
implemented in the systems we surveyed. We will allow any complexity
measure as a ranking as long as it satisfies the followind requirements:
\begin{enumerate}
    \item is in $\mathbb{N}$
    \item $>0$ for any defined term; $\bot$ may have rank 0 if the test program supports such test values (e.g. \LSC)    
    \item $1$ for base types (such as |Int| or |Char|)
    \item induces a partial ordering of the terms
    \item is strictly monotonic with respect to term inclusion,
    i.e. if the term |x| contains the term |y|, the rank of |x| should be strictly greater than that of |y|
    \item a partition of the terms, with a finite number of terms for any given rank
\end{enumerate}
\noindent Most of these requirements are straightforward. The only non-trivial
requirement is the last one, that ranking induces a partition of the terms where
each class is finite.  In practice, this is not onerous.

The system should be mainly polymorphic over complexity measures. Certainly
multiple complexity measures should be usable over the same test suite (unlike
for most current \pbt which only support a single one, or a single family).

\subsection{Test Suites}

A test suite is simple a \emph{collection} of test cases. However,
for the purposes of scheduling test cases and reporting their verdict,
such a collection must have a few attributes:

\begin{description}
\item[Partitioned] allow an arbitrary partitioning of the test cases
\item[Labels] permit arbitrary labels for the parts
\item[Ordered] maintain the order of test cases within each part
\item[Immutable] neither test suite nor individual test cases should be modified by test evaluation
\item[Persistent] storable and recoverable so tests can be repeated
\item[Duplicates] duplicate test values should be allowed as separate test cases
(thus the name ``suite'' instead of the theoretical test set from chapter \ref{chp:propertytesting})
\item[Merging] it should be possible to merge test suites (of like structure)
with the partitions and labels respected by the merge.
\item[Source Independent] the test suite should allow test cases from multiple sources,
whether automated generators, loaded from files or otherwise provisioned.
\end{description}
\noindent
Note that 
these attributes are independent of the property being tested and the evaluation strategy.

The labeling may be based on 
any meta-data associated with the stratification strategy,
but generally include the rank of the test cases within the part.

A test suite can be identified as \emph{valid} for a property if 
all of its test cases are known to be valid for a property test,
i.e. are within the property's domain.
This attribute of a property / test suite pair 
should be accepted by the framework and made available to the evaluation engine
to determine if the property's precondition should be applied before evaluating test cases.

\subsection{Multiple Properties and Shared Test Suites}

Properties over a common type
should be able to use the same test cases, to avoid the overhead of regenerating them.
This is a weakness of \QC et al:
test cases are generated during test execution, and
a new test suite is generated with every test pass
(test cases can only be shared by testing a conjunction of properties).
Sharing test suites requires them to be accessible from outside the context of
a single property test.  This enables scheduling multiple properties to be
evaluated within the running of a single test.

Repeatability is also an important requirement, thus
it must be possible to store and retrieve a test suite to allow tests to be repeated.


\section{Automated Test Generation} \label{reqtestgen}

Automated test case generation for property based testing consists of :

\begin{enumerate}
    \item defining some sampling strategy defining the count, complexity and sampling methods to be used
    \item the application of sampling selection criterion, possibly including preconditions, to decide which values to evaluate and in what order
    \item building the concrete instances of test datum of the appropriate type
\end{enumerate}

A property based testing framework should provide a
standardized but powerful set of tools to generate test cases
for all representable types and standard sampling methods (section \ref{sec:sampling_theory}).
This should include both a library of generators of standard types and sampling methods,
and the ability to building new or customizing generators,
including the use of non-standard sampling methods.
As shown in chapter \ref{chp:enumgen}, 
automated test case generation in Haskell is amenable to standardizing
because generators for algebraic data types which form the majority of the terms
may be created using datatype generic construction techniques.

\subsection{Generators}

A test case \emph{generator} is a function that produces an ordered collection of test cases of a fixed type.
These were examined at length in chapter \ref{chp:enumgen}
where scalar, parameterized and complexity ranked generators were defined;
the requirements are repeated in summary form here for convenience.

Generators are functions returning lists of terms of a given type.
\begin{description}
    \item[scalar] generators return values of a scalar type and do not require rank as an input argument
    \item[complexity ranked] generators produce terms of a specified rank
    \item[parameterized] generators that accept arguments other than rank
\end{description}

A generator will apply a sampling method to select and order the values to be returned.
For recursive or infinitely populated types (e.g. unlimited Integers),
the sampling method will be applied to the values of a given complexity / rank.
The framework should support the use of standard sampling techniques:
\begin{enumerate}
    \item exhaustive
    \item random
    \item uniform
    \item extreme / boundary
\end{enumerate}
In addition, it should be possible to use alternative sampling methods in generators
and incorporate those new methods into the test strategies in the system.

The framework must accept a wide range of behaviors from generators, which may:
\begin{enumerate}
    \item produce a finite or unbounded sequence of values,
    \item guarantee unique values or permit duplicates,
    \item have a range that does or does not include all of the type's values,
    \item order values according to their priority for testing,
\end{enumerate}


\subsection{ Generator Substitution and Composition } \label{sub:reqsubcomp}

\gordon{Is this really a requirement, or is it just part of the GenCheck implementation?}

Mutual recursion, substitution and composition are all used
to define data types in Haskell programs.
Any of these could be complicated relationships,
and it should be possible to choose different sampling strategies
for components, elements and mutual recursive structures.

One of the most significant shortcomings of the single sampling strategy packages
is their inability to generate efficient test suites of data structures populated with base elements.
The problem is that exhaustive testing of algebraic data types 
is a good strategy for the \emph{shapes} of the data types,
but generally not efficient for the base type elements such as integers in the structures.
Random generators do well producing the elements,
but then don't provide the same confidence and efficiency when testing
the small / simple structure shapes that exhaustive testing would provide.

The testing framework should allow 
structures to be generated just as shapes first,
and then substitute independently generated sets of elements to populate them.
It should also allow the elements to be generated as 
part of the structure, i.e. allow the enumeration or traversal to \emph{include element values}.
Structures and their elements are fundamentally different
so the test generators should allow different sampling strategies 
to select structures and their of elements.
Generators could then populate the structures substituting 
one or more sets of elements into each structure.

Generators should be able to use at least the following five \emph{substitution} strategies:

\begin{enumerate}
\item one distinct substitution per structure
\item $n$ distinct substitutions per structure
\item linearly partition the set of elements by the structure size
and populate as many copies of the structure as there are partitions (finite generators only)
\item all  possible permutations of a set of elements per structure
\item all combinations of a set of elements per structure
\end{enumerate}

Composition is similar to substitution but replaces the elements of the data structure
with other data structures; the newly composed structure can then 
be populated with elements through substitution.
In substitution, the rank of the substitution set is ignored,
and all of the values are assumed to be of rank 1.
In composition the substitution set values are structures with a variable rank,
and the total rank (or complexity) of the composed structure
is the rank of the initial structure plus the sum of the ranks of the composing structures.

\gordon{picture of substitution and composition, composition showing the sum of ranks}


\subsection{Sampling Strategies}

Evaluation of the existing property based testing software showed that 
the most significant distinguishing feature was the test selection criterion.
Each of these sampling methods has advantages and disadvantages,
proponents and detractors, and a role to play in proving the test hypothesis.
Other sampling methods are often presented, usually as heuristics,
and there is no definitive evaluation process to determine which is best.
It is reasonable to conclude that test selection criterion are still
an area of research and that any test framework should support
all of them and other methodologies not discussed here.

It is also reasonable to demand that a test framework support
multiple sampling methods in a single test data set.
Most practitioners would agree that testing a module
using both \QC and \SC would provide greater confidence
than using either alone; being able to use the same specification
and implementation to do both tests is convenient.
It would be even more convenient if the default for testing a property
was to incorporate multiple sampling methods without
having to maintain multiple \pbt systems and separate test results.

There is a ``strategy'' in the construction of a test suite
that can be defined independently of the specific type of the test case.
The strategy will specify one or more parts to the test,
with each part defined by a sampling method, 
a number of test cases (perhaps as a percentage of the total test case count)
and a range of term complexity for ranked types.
A \pbt framework should allow the definition of such a strategy independently of the test type,
and be able to either retrieve or construct the generators required
where the sampling methods are known (e.g the standard sampling methods above).


\section{Evaluation and Scheduling}

A test evaluates some or all of the test cases in a test suite,
using an evaluation function.
There are different kinds of tests depending on 
the characteristics of the evaluation function,
and the termination conditions for the test.

A test for a property consists of evaluating a property
over the data contained in a test suite.
Assuming that the test cases are independent 
the test program must decide how to schedule test cases for evaluation
and collect the results.


\begin{df}[Test]
A test is a function from a test suite to a result based on an evaluation function:

$$\test{\eval{\property}} : \testsuite{\alpha} \ra \resultset{\alpha}$$
\end{df}


\begin{df}[Conditional Test]
A conditional test is based on a conditional evaluation function $\eval{\property}$ 
and may include test cases assigned the verdict $\invalid$ where
the test data is not in the property's domain.
\end{df}
\begin{df}[Unconditional Test]
An unconditional test is based on an unconditional evaluation function $\evaluncon{\property}$, 
so all test data is assumed to be in the property's domain
and will not be identified as $\invalid$.
\end{df}

\begin{df}[Complete Test]
A test is complete if all of the test cases are evaluated,
so no test cases will have the verdict $\noteval$.
\end{df}
\begin{df}[Partial Test]
A partial test may terminate without evaluating all of the test cases,
so some test cases may have the verdict $\noteval$.
\end{df}

\begin{df}[Time Limited]
A test is time limited if an exception is generated when
the evaluation of a property (or a precondition in the form of a domain's characteristic function)
fails to evaluated in a set period of time,
resulting in a verdict of $\nonterm$ for that case.
\end{df}

A test function can be characterized by these three concepts
(conditional / unconditional, complete / partial, time limited or not) independently.

\subsection{Evaluation function}

An evaluation function for the property P
applies the property function to the datum of a test case.
The following factors should be considered:

\begin{description}
\item[Conditional] evaluation functions will accept as input the characteristic function of a property's domain,
identify any test cases that are not valid and return $\invalid$ without evaluating the property at that value.
\item[Unconditional] evaluation functions does not perform such a check,
and are used if the test cases are guaranteed to be in the domain,
or if it is more efficient to evaluate the invalid test cases
than to prevent them from being tested.
\item[Time-limited] evaluation functions will interrupt the evaluation of a test case
if it exceeds a time limite and produce a $\nonterm$ verdict.
This could also be caused by the characteristic function not terminating,
if domain membership is being tested,
but these conditions will not be distinguished.
A $\nonterm$ result could be interpreted as a $\fail$ verdict,
but this is not an accurate reporting of the results.
\end{description}

Note that if the characteristic function for the domain is available,
an evaluation function maybe constructed by composing
the characteristic function of the domain (precondition) and 
an unconditional evaluation function:


\subsubsection{Scheduling}
Scheduling refers to providing the order in which test cases in a test suite are evaluated.
An implementation of a test function will
determine how the evaluation function will be applied
to the test cases in the test suite.
If it is a partial test, the scheduling will include
how termination conditions for the test are implemented.

The simplest form of scheduling is 
a |map| of the evaluation function
over the test suite container.
This allows the compiler to assign any order
to the evaluations and so may be more efficient.
This is a pure computation (i.e. not monadic) 
so has the property of referential transparency.
In an Haskell implementation,
this would require the test suite container to be an instance of the |Functor| class.

\QC, \SC and the related packages reviewed in chapter \ref{pbtsystems}
scheduled test cases to be generated and evaluated sequentially,
using a monadic test function (or a similar approach).
This kind of scheduling can be accomplished using 
a |fold| over the test suite container.
In an Haskell implementation,
this would require the test suite container to be an instance of the |Foldable| class;
this class will take advantage of the interior monoidal construction of the test suite
described above in section \ref{reqtestsuite}.

The evaluation of a test case is the process of computing 
whether the property holds at that value.
and possibly their distribution amongst different threads, nodes or systems as required;
on distributed systems, it would also include collecting the results for reporting.
These are generally simple steps,
but some care to make sure
Haskell's lazy evaluation environment does not confuse 
when and how that this happens.

\subsection{Execution Mode}

The programming interface should function without assumed access to the IO environment,
except for components that exist solely to perform IO
(or are use to test implementations that perform IO).
A test of a pure implementation should be a pure computation itself,
with only a thin layer of monadic control provided for displaying the results.
When the property being tested incorporates monadic functions,
the property should be written to resolve the monad.
The IO monad prevents this, so then the property must be tested in a monadic way.

\gordon{think about this some more, testing monadic functions, look up papers}

Reporting of course must be an IO function,
as is retrieving test cases from a file,
but otherwise the test components should not use an IO monad

\subsubsection{Parallelism}
Testing is one of the few intrinsically parallel tasks software engineers face,
as test cases are (at least meant to be) independent.
It is important to have the opportunity to take advantage of the increase resources this offers.

\subsection{Termination Conditions}

\QC, \SC, etc. terminate testing when a fixed number (defaulting to 1) of errors are found.
This is a reasonable approach for quick checks during coding and debugging,
and results in a small number of failure cases being reported so is easy to interpret.
It does require that the test cases are evaluated sequentially,
or perhaps results from that approach to testing.

More formal software development environments
will often complete the test and report on all test cases,
as the process for correcting, building and testing the system is much longer,
and so gathering as much information about failed test cases as possible is desirable.
Of course, it is only useful if the test reports use sensible approaches to 
consolidating or summarizing the results when there are a large numbers of errors.

Testing programs should allow time limits to be set on 
the evaluation of an individual test case,
and on the overall amount of time devoted to a test suite.
This both identifies non-terminating test cases
and allows very large test suites to be defined
and evaluated for the timeframe desired
instead of trying to guess how many test cases should be run.
This approach will work well with Haskell's lazy evaluation.


\section{Results and Reporting}
The results of a test - the evaluation of a property over a test suite - consist of the overall verdict,
but also the evaluation of each test case 
and optionally information about the performance of the system during that evaluation.
Like the test case,
a result should be a class of data structure,
with a method to retreive the verdict,
but optional meta-data that can be captured by a specialized evaluation engine
and made available to a reporting module.

Although the properties are boolean valued functions,
the verdict of a test must be expanded beyond a simple success / fail
to capture the possible outcomes associated with a test case:

\begin{df}[Verdicts]
    The \emph{verdict} of a test is a set of labels describing the combined results of one or more test evaluations:
    
    $\verdictset = \{ \fail, \nonterm, \success, \invalid, \noteval \}$ where
    
    \begin{description}
        \item[$\fail$] the property does not hold over the valid test case
        \item[$\nonterm$] the property did not evaluate in the allowed time
        \item[$\success$] the property holds over the valid test case
        \item[$\invalid$] the test case was not in the domain of P, so was not evaluated
        \item[$\noteval$] the test case was not evaluated by the test (may or may not be valid)
    \end{description}
\end{df}

Not all verdicts will be possible under all models of evaluation.
The $\nonterm$ verdict in particular will only occur if the evaluation engine 
sets some kind of time or resource limit on execution.
The other states are all used by at least one of the packages from chapter \ref{pbtsystems}.

These verdicts are actually in a precedence order that can be used when combining 
the verdicts of two result sets,
e.g. a fail verdict overrides non-termination, non-termination overrides success, etc.
\begin{df}[Verdict Monoid]
    identity : \noteval \\
    concat : v1 \^ v2 = greater of the two\\
\end{df}

The result container consists of the verdict and evaluations of each test case.
Making the result container a labeled partitioned container like the test suite
provides good support for reporting coverage analysis,
and provides for low overhead evaluation strategies (such as map),
but also allows the result set to be monoidal in both the container and the verdict.

\begin{df}[Result Class]
    defines a labelled partioned container as defined for the test suite,
    but the monoid is extended to include the calculation of a verdict.
    
    verdict is available for any part or single test result.
\end{df}
Note that while there may be some confusion in a Haskell implementation
caused by having verdicts at both the test case and test suite levels,
the monoidal nature of the verdict computation makes the meaning consistent.

Reporting modules will present the contents of a results class,
optionally using the partitioning and labeling of those parts
to provide more detailed analyses.
They may be specialized to use extra information about the evaluations.

%
%\section{Summary}
%
%It is impossible to generalize the needs of any significant project size,
%and it is equally unnecessary to trap the tester in any given paradigm.
%Data generation should not be restricted to any particular theory, 
%container type, monad or sampling strategy.
%It should allow a significant level of choice and customization by the tester.
%This is can best be managed by developing an architecture
%and framework for assembling test programs from components providing
%generators, test suites, evaluation and reporting functions through
%minimal programming interfaces. The testing software should provide support for :
%
%\begin{enumerate}
%\item quickly building default generators from the property definitions
%\item quickly build default test suites using mixed heuristic strategies
%\item allow test suites to be combined and separated, stored and recovered 
%\item allow test suite strategies consisting of sampling strategies over known generators
%\item allow generators to be customized using hybrid sampling strategies 
%for element substitution, structure composition or mutual recursion
%\item allow fully customgenerators to be included at any point,
%\item support inclusion of manually coded test cases
%\item integration with different execution models by decoupling the
%test scheduling from execution and generation, except where absolutely necessary
%\item integration of different reporting modules that allow varying degrees of 
%coverage analysis and interactivity
%\end{enumerate}
%


\setcounter{figure}{0}
\setcounter{equation}{0}
\setcounter{table}{0}
\chapter{Specification for PBT Systems}

The specification consists of interfaces isolating each of these stages
and a suggested library of components that should be provided as a base implementation.
In Haskell interfaces can be implemented as ad hoc classes,
with each new variant an instance, 
or by defining data types and allowing values of that type to be defined.
A high level description of the mandatory interfaces and 
their interactions during the testing process are addressed in this chapter.


\subsection{generators}
Test generators exhibit a lot of the structure of the data they generate,
so are suitable for generic programming techniques.
Generators should be constructed using combinators for 
disjoint unions (sums) and products of base types,
mirroring the Haskell type constructor algebra.
The generator combinators provided by \QC, \SC, etc.
are a good example of how generator combinators can be used
to develop generators for algebraic data types in Haskell.



%\section{Definitions}
%%\section{Formally define property-based testing systems}

% for convenience
\newcommand{\hf}{\ensuremath{\mathtt{f}}\xspace}
\newcommand{\Bool}{\ensuremath{\mathtt{Bool}}}

Testing a Haskell function \hf, using properties, consists of four, largely
independent, steps:
\begin{enumerate}
\item construct a property $p$ that \hf should have,
\item generate test cases (valid inputs for \hf),
\item evaluate the property over (a selection of) test cases,
\item report the results.
\end{enumerate}

A \emph{test system} is a framework to perform these steps.  As each
step admits a number of useful variations, such a framework is only 
useful if it provides a variety of capabilities.  To better establish the
requirements for a \pbt, we first refine the definition and scope of each
step.

\subsection{Properties}

Recall that in section~\ref{pbt}, we defined an \emph{implementation} $M$ as a
computable \emph{interpretation} of a specification $S$, where the specification
defines a set of symbols and their semantics as a set of axioms.
The symbols are simply the names of the functions exported from a module.
The purpose of a \pbt is to supply evidence supporting or refuting
the correctness of this implementation.

A \emph{property}  $ p: \alpha \ra \Bool$ is the implementation (using $M$) of
a predicate $P : A \ra \boolean$ in the specification.  Note
that, as it is an implementation, it is necessarily computable; without
loss of generality, we can restrict properties to be univariate.
It is important to note that there is a concretization step happening here,
not just between predicates and properties, but also between data values:
$\dom{P} \subseteq A$ and $\alpha$.  As is usual in computer science, 
we will assume that $A$ and $\alpha$ are isomorphic, and will not explicitly
worry about this isomorphism.  We can thus talk about $\dom{p}$ as well,
which is a ``subset'' of the values of type $\alpha$.

We will, from now on, talk interchangeably between predicates and properties.
Furthermore, we will make some use of set notation over types; in the context
of PBT, this actually causes no harm since all data we represent will
eventually be concrete, and thus has a Set model.

We say that a property $p$ holds if, over all values in the domain ($\dom{p}$),
the property evaluates to $\True$.  Note that on $\alpha \setminus \dom{p}$, it
is not required to hold, nor even be defined.  If $p$ is partial, we will call
it a \emph{conditional} property.  For these, a \pbt must be careful to not
attempt to evaluate the property outside its domain.

If we have a (total) function $f : \alpha \ra \Bool$ such that 
$f a = \True \Leftrightarrow a \in \dom{p}$, we call $f$ a
\emph{characteristic function} for $p$.  Such functions are very
useful to deal with partial properties.

The collection of properties implementing the predicates of the specification
will be called the testable specification,
or just the specification where it is clear through context that
the reference is to the properties and not the abstract predicates.

\jacques{I deleted all the ``formal'' definitions, are they were completely
equivalent to what is already above.  The above is sufficiently formal.}

\subsection{Test Cases and Test Suites}

For this subsection, fix a predicate $p : \alpha \ra \Bool$.

Recall that a \emph{test case} contains a single value (the \emph{datum}) of
type $\alpha$, and that a \emph{test suite} is a collection of test
cases.

Each test case may include additional information (\emph{meta-datum})
that can be used to improve the evaluation of the test cases or reporting of the results.
A datum will not necessarily be in the property's domain,
in which case the test case can be characterized as \emph{invalid}.
The meta-datum can be arbitrary, but their type must be uniform over a test suite.

Given a test case $\tau : \beta$, we assume that we have a projection function $\datum$
to extract the data.

\begin{df}[Datum Function]
$$\datum(\tau) : \beta \ra \alpha$$
\end{df}

\begin{df}[Valid Test Case]
A test case $\tau$ is \emph{valid} for a 
property $p$ if the datum is guaranteed to be in its domain:

$$ \datum(\tau) \in \dom{p} $$
\end{df}

\subsection{Evaluation, Selection}

We first define what happens when evaluating a single test case.
Second, we define some items relating to test suites.
Then we deal with actually running a test suite.

\subsubsection{Verdicts and Results}

The evaluation of a property at a datum does not merely produce a boolean 
value.  Other data, such as the execution time or system resources used,
can also be tracked as part of the \emph{result} of running a test.
Of course, if the property evaluates to $\True$, we say that the 
property holds (is a successful test evaluation), while for $\False$,
a failure.  But this is not the only method by which a test can ``fail'':
it is also possible that the test case was not valid, or that the 
property evaluation took too long.  Lastly, the framework may decide to
not evaluate every test case in a test suite, and this can be reported as well.

In other words, rather than just pass/fail, we define a finer-grained notion
of a \emph{verdict}, which encapsulates the outcome of a test case evaluation.

\begin{df}[Verdict]
A \emph{verdict} describes the result of a test evaluation.  We use the label
set $\verdictset = \{ \success, \fail, \nonterm, \invalid, \noteval \}$
as verdicts, and these should be intererpreted as:

\begin{description}
\item[$\success$] the property holds over a valid test case,
\item[$\fail$] the property does not hold over a valid test case,
\item[$\nonterm$] the property evaluation did not terminate in the allotted time,
\item[$\invalid$] the test case was not in the property's domain, so was not evaluated,
\item[$\noteval$] the test case was not evaluated and its validity is unknown.
\end{description}
\end{df}

Note that the verdict of $\nonterm$ can also be used to indicate that an
exception was raised during the evaluation of the test case,
assuming the evaluation function is capable of trapping it.

In general, different evaluation functions used.  They generally differ
in what kinds of verdicts they can return.  \emph{Unbounded} evaluation
assumes that $\nonterm$ is impossible; \emph{unconditional} evaluation
can be used when $\invalid$ is impossible.  

A \emph{result} consists of a test case, the verdict of the evaluation of the
property for that case, and any additional information about the evaluation of
the property at that value.  It must be possible to extract the verdict from
a result.

\begin{df}[Verdict Function]
$$\verdict : \gamma \ra \verdictset$$
\noindent where $\gamma$ is a result type.
\end{df}

\subsubsection{Test suites}

A test suite $T$ is a collection of test cases.  $T$ can be 
considered to be a set of test cases, although we will rarely 
implement it that way.  For concreteness, we will use $t$ to
denote a \emph{container}, $\beta$ a type of test cases, 
and thus $t \beta$ will be the type of a test suite.  $t$
will generally be assumed to be a \texttt{Functor} as well
as \texttt{Traversable}.

We can lift definitions previously made on test cases to test suites.
For example, a test suite is \emph{valid}, if all test cases it contains
are valid.

A test suite may be partitioned (and labelled) -- to 
assist in prioritizing the evaluation of test cases, and reporting the results.
The labels could be based on the test data, meta-data or 
other evaluation information.

\begin{df}[Labelled Partition]
A labelled partition, with label set $\labels$, is an ordered sequence of test suites
$$ \Pi :: \nat \ra (\labels, t \beta)$$
\end{df}
\noindent
 
The verdict over a test suite is determined by interpreting the collective
verdicts of each result; partial verdicts can also be determined for any part
of the result set.

\subsubsection{Execution of a suite}

While a test suite is any collection of tests, this does not mean that each
of these will actually be run.  We can use different methods of choosing 
when to stop testing, for example:
\begin{itemize}
\item terminate testing when a fixed number of errors are found (\QC, \SC),
\item terminate after a fixed period of time, or
\item complete all tests.
\end{itemize}

We will naturally say that an execution of a test suite 
is \emph{complete} if all cases were evaluated (even if some resulted
in $\invalid$ verdicts); \emph{partial} if some verdicts were $\noteval$;
and \emph{time limited} if some were $\nonterm$. These can be combined.

\subsubsection{Summary Verdicts}

The verdict of a test suite is based on the combined verdicts of each individual result.
We do this by
\begin{itemize}
\item putting an order structure on verdicts:
$$ \fail > \nonterm > \success > \invalid > \noteval$$
\item using the Foldable structure of the container to fold
\item the induced join-semilattice structure (aka join, $\wedge$)
fold the join ($\wedge$) 
\end{itemize}
\noindent over the verdicts.

\subsection{Reporting}

\jacques{you were doing the $4$ steps up to now, so logically you should 
be talking about reporting at this point.}

A report presents the results of a test suite, in a way useful to the
(human) tester.
Reports provide the verdict of a test and some level of detail about the results,
organizing them in different ways depending on the goal of the report.
Test reports should highlight any failed (or non-terminating) test cases.

Report components should use the generic interface into 
the results to get the verdicts and test case values.
Optional evaluation information should be provided by 
\emph{specializing} the results data structure, the report component,
and either the test case generator (for test metadata) or
the execution component (for test evaluation information).

\section{Test Systems}

\jacques{Made this into a section, as it is not about the $4$ steps anymore}

A test system provides additional functionality over the $4$ steps described above.
In particular, it can

\begin{enumerate}
\item obtain actual test cases from a generator,
\item schedule the evaluation of test cases,
\item perform evaluation of test cases (with potential early termination),
\item organizes test results,
\item report verdict and optionally a summary and/or details of the results.
\end{enumerate}

\noindent
These components should adhere to a common interface
but tcan be specialized to exchange additional information.

Scheduling refers to providing the order in which test cases in a test suite are evaluated.
Scheduling may be implicit,
such as when applying |map| with an evaluation function over the test suite container.
For example,
\QC, \SC and the other packages reviewed in section \ref{pbt_soft}
generate and evaluate test cases sequentially,
terminating on the first error (by default).
On distributed systems, scheduling would include 
the allocation of evaluations to different nodes
and collecting the results into the common result set.

Test \emph{results} should be considered distinct from \emph{reports}, as discussed above.

%
%\section{Sampling and Test Strategies}
%%\section{Sampling and Test Strategies}
%\gordon{Formally define 'strategy' and how it relates to sampling}

\subsection{Generators}
\gordon{ defined by type, sampling method, (finite or infinite) }
\gordon{ already gave formal definitions of generators, but should have done that here I think }

Test case generators were described in section \ref{???}
as the basic mechanism for generating test cases.
A \emph{generator} is a function that instantiates 
a subset of the possible input values for a property for use as test cases.
The selection of test cases forms a \emph{sample} of the property's domain,
or a part of the property's domain,
and possibly invalid test cases from outside the domain for conditional properties.
The generator also provide an order to the test cases they generate,
interpreted as the priority of the values for testing,
although it may simply be an artifact of the implementation of the generator.


\begin{df}[Valid Generator]
A generator for a conditional property is valid if it only creates valid test cases,
%i.e. $\gen{\dom{\property}}$ :: $\[\powerset{\dom{\property}}\]$
\end{df}


It was recommended earlier that the test framework support
the construction of generators using the four standard strategies
(exhaustive, boundary / extreme, uniform and random);
a test suite strategy can be defined as a collection of these strategies.
These should be implemented in a data type generic way.

Should also be possible to construct custom generators that 
are arbitrarily assigned as satisfying one of those sampling strategies.

Should be able to add other generator sampling strategies 
to be included in the test suite strategies.

There should be no requirement that generators produce finite lists of test cases,
as Haskell's lazy evaluation permits infinite lists to be defined.
For example, the random generators from \QC are not finite,
and as can be seen in the discussion of substitution and composition below,
infinite generators can be quite useful.
Test suite building should function with both finite and infinite generators,
and test programs should ensure termination through 
carefully selecting a fixed number of values from a generator,
or only selecting all values when there is an expectation of finiteness.

�%base generator_(type,sample): C(a), ranked generator_(type,sample): Nat -> C(a)

\subsection{Test strategy}
%base = \{(sampling method, num. cases)\}, ranked = \{(sampling method, rank, num. cases)\}

\begin{df}[Sampling Strategy]
blah
\end{df}

\subsection{Generator Dictionaries}
Access generators for a type given a sampling method
Standardized sampling methods allows standardized strategies.

%generator dictionary: sample -> generator_(type,sample) (base or ranked)
%	� �standard sampling strategies = {Exhaust, Random, Uniform n, Boundary}, could add others

\subsection{Test Suite Builders}
%Build (Test strategy (sample, (rank), cases), sample -> generator_(type, sample))�
There is a ``strategy'' in the construction of a test suite,
and that strategy will consist of instructions to automated test case generators,
loading cases from a file, interactively requesting cases etc.
Test suites might be composed of 
multiple collections of test cases from different sources.
It should be possible to merge these sources into a single test suite
without biasing the scheduling and evaluation of these test cases.
There should be a language to construct test suites
in the same way there is a language to construct test case generators.
The language will decouple the strategy for building the test suite
from the provision of the actual test cases,
to allow test suite strategies to be researched independently.

\subsubsection{Traversing a Test Suite}
\gordon{must define these standard operations}


%


%------------------------------------------------------------------------------
\setcounter{figure}{0}
\setcounter{equation}{0}
\setcounter{table}{0}
\chapter{\GC Implementation}\label{chap:source}
%% GenCheck implementation

This chapter contains the literate source code for the GenCheck implementation.

\input{../GenCheck/GenCheck.lhs}
\input{../GenCheck/Base/Base.lhs}

\section{PureTest: an example of a \GC \pbt system}
\input{../GenCheck/PureTest.lhs}
\input{../GenCheck/SimpleCheck.lhs}

\section{Test Suites and Results}
\input{../GenCheck/System/TestSuite.lhs}
\input{../GenCheck/Base/LabelledPartition.lhs}
\input{../GenCheck/Base/Datum.lhs}
\input{../GenCheck/Base/Verdict.lhs}
\input{../GenCheck/System/Result.lhs}

\section{Generators}
\input{../GenCheck/Generator/Generator.lhs}
\input{../GenCheck/Generator/Substitution.lhs}
\gordon{Need the Composition module}
%\input{../GenCheck/BaseGens.lhs}

\section{Enumerations}
\input{../GenCheck/Generator/Enumeration.lhs}
\input{../GenCheck/Generator/BaseEnum.lhs}
\input{../GenCheck/Generator/EnumStrat.lhs}
%


The framework will be explained w.r.t. the requirements from the previous chapter,
from the software engineering perspective not the theoretical perspective.

The \GC tutorials have been included as an appendix to this document.
It may be valuable for the reader to go through the tutorials first to get a
friendly user level view of the package before reading this section.

% The tutorial needs to go here, but it can't be linked from GitHub so I'll copy it over later.
\appendix
\setcounter{figure}{0}
\setcounter{equation}{0}
\setcounter{table}{0}
\chapter{\GC Tutorials}\label{chap:tutorial}
%%\section{Tutorial 1: The simplest case.}

\input {reverse/TestReverseList.lhs}

The examples in this module have properties over standard Haskell types, where
the default generators were already available.  How are test cases generated
when the module contains a new type?  How are the test cases chosen for
inclusion in the test suite, and how can that be controlled?  Surely there is
a better way to display the results?  These questions and more are answered
in the next part of the tutorial.  

\section{Tutorial 2: Testing with New Data Types}

The first tutorial covered testing properties over standard Haskell types.
Default test case generators were already supplied by GenCheck
as instances of the Testable class, but what about new data types?
In this tutorial, we explain how to use the type constructor definition and 
GenCheck enumeration combinators to construct the standard test generators
based on the type constructor definitions.

There are three steps in testing a module against 
a specification using the GenCheck framework:

\begin{enumerate}
\item define and code the properties that make up the specification
\item select the test system that provides the most appropriate
test scheduling and reporting schema for the current phase of the development
\item define the test suite for those properties,
including the test case generators for each.
\end{enumerate}

In the previous section the details of the test system
and test suite generation were hidden by the SimpleCheck API.
In this section, we look at each of these steps in detail 
and discuss how to generate new data types and customize test suites
for a particular set of properties.

For this tutorial, we'll look at a list zipper implementation,
taken from the ListZipper package (1.2.0.2).  Only snippets
of the code will be provided as the package is rather large
(see tutorial/list\_zipper/ListZipper.hs  for the module source).
The Zipper data type is

\begin{verbatim}

data Zipper a = Zip ![a] ![a] deriving (Eq,Show)

\end{verbatim}

where the ``current'' element or ``cursor'' is the head of the second list.
The specification for ListZipper includes empty, cursor, start, end, left, right, foldrz, foldlz, 
and many more functions.  The example below shows how to test (some)
of the specification for this module, with a focus on the generators for the new type.

\subsection{Specifications and Properties}
A GenCheck specification is the collection of properties
that any implementation of the specification must satisfy.
Properties are univariate Boolean valued Haskell functions;
this is not a restriction as the input value may be of an arbitrarily complicated type 
to support uncurried predicates. The GenCheck test programs accept 
a single property and test it over the defined suite of test cases.
Only one property is tested per call because different properties 
may have different input types or require different test suites
The specification module would normally be separate from the implementation,
but may include functions that test the properties using GenCheck.

\subsubsection{List Zipper Properties}

\input {list_zipper/PropListZip.lhs}

\subsection{Test Suites and Standard Generators}
The module System.TestSuite contains functions that assemble test suites 
from one or more test case generators, including three that should seem familiar: 
stdSuite, deepSuite and baseSuite.  These suite builders take a collection of 
``standard'' generators and a set of instructions in the form of  a list of rank, count pairs.  
The allocation of test cases to ranks through these instructions is called a ``test strategy'' 
and can be used independently of the type of data being generated, 
assuming that the standard generators are available for that type.
The System.SimpleCheck module test suites are test cases stored in 
a Haskell Map (as in Data.Map) indexed by rank, called a MapRankSuite.

A GenCheck generator is any function from a rank (positive integer) to a list of values.
Generators may be finite or infinite, may include duplicates of the same value,
and may be missing values from the total population of the type.  Generators
can be manually coded or built from a GenCheck enumeration of the type using
a sampling strategy called an ``enumerative strategy''.  These strategies can be
applied to any data type that can be enumerated, i.e. ordered, and indexed by a finite rank.

The so called standard generators are just four different sampling strategies
that can be applied to an enumeration of the generated type.  These are:

\begin{description}
\item[exhaustive] {Produces all of the values of the specified rank in a sequence}
\item[extreme]{Starts with the boundary cases, then the middle, then progressively splits the intervals}
\item[uniform]{Select the specified number elements at uniform intervals through the enumeration}
\item[random]{Use the random generator to pick an infinite list of values}
\end{description}

These are not the only valuable sampling strategies that could be used, but these were
selected as the ``standard'' generator strategies to be used by the GenCheck API.
The standard generator set is the basis for the test suites used by SimpleCheck test programs,
the TestSuite module, and most of the higher level API functions as the default generators.
A type is an instance of Testable if a set of standard generators is available for it;
these generators can be constructed for any type that has an Enumeration associated with it. 

\subsubsection{Enumerating New Data Types}
A GenCheck enumeration is a way of ordering the values of a type.
For base types such as Int and Char, this is called a base enumeration (BaseEnum)
and is effectively the same as the Haskell Enum class;
the GenCheck class is called EnumGC to highlight that it is an extension of Enum.
For structure types, the enumeration is first partitioned into finite subsets
by the number of elements in the structure (called the rank),
and then the subset is ordered and indexed.
Structure types that are enumerated are instances of the Enumerated class.

Enumerations for standard Haskell data types such as list and tuples are provided.
For new data types, enumerations can be easily made using the enumeration combinators 
|eConst, eNode, eSum, eProd, etc.| in the Generator.Enumeration module.
These combinators simply mirror the structure of the type constructors,
but produce the required enumerations.

\subsubsection{List Zipper Enumeration and Generators}

\input {list_zipper/ListZipper_GC.lhs}

Generators are created from enumerations and enumerative strategies
using Generator.enumGenerator for structure types and BaseGen.baseEnumGen for base types.
The standard generator set, and therefore the Testable instance, is built using stdEnumGens.

\subsection{The Testing Process}
Typically a GenCheck user would start with the default test suites
and only reporting failures during the development cycle.
Nearing completion, s/he would switch to the test programs 
that provide more control over the test suite generation,
and report all of the test cases in the results to ensure good coverage.

The objective of GenCheck is to provide a testing framework that
scales in scope from a very simple QuickCheck like interface,
through progressively more thorough test suites and reporting,
during the course of the development towards production.
The highly modular GenCheck API allows an arbitrarily deep level of control,
with a commensurate level of complexity,
supporting customization and new development as required.

\subsubsection{Testing the List Zipper Module}
\input {list_zipper/TestListZipper.lhs}




%\section{Tutorial 3: Customizing the Test Programs}
%
%??? Everything below this is currently junk, keeping it for review / reuse
%
%Test suite and Generators
%
%The System.SimpleCheck module test suites are test cases stored in 
%a Haskell Map (as in Data.Map) indexed by rank, called a MapRankSuite.
%Storing test results this way may or may not be meaningful in 
%any given situation, but it is the only information guaranteed to be available.
%The GenCheck framework allows test suite containers to be customized using
%a class called LabelledPartition that abstracts the interface required by the test programs,
%but all of the discussion in this tutorial assumes the use of MapRankSuites.
%
%need to identify the domain of each property
%start with the input type, and then consider any restrictions over the values of that type
%a test suite is a partitioned collection of test cases
%  the partions are labelled to improve scheduling and reporting
%  could use a single partition, or multi-level labelled partitions
%
%test cases can be manually coded or automatically generated
%manually coded test cases just written into the partitioned collection
%
%Input types are base types or structures composed with other types as ``elements''.
%
%generators are functions from positive Int rank to (possibly empty) list of values
%  intent is that rank is in inverse ``priority'' of test cases i.e. 1 is highest.
%  there is generally no guarantee of uniqueness / completeness
%  either use the generators provided, use purely custom generators,
%     or use generator combinators provided
%
%  base type generators only have values for rank 1, other ranks are empty lists
%  some generators are provided in:
%    Test.GenCheck.Generator.BaseGen:
%       Int,Char,String,Ratio,Double,Float
%       baseGen - creates a base type generator from a list
%
%  polymorphic structures are generated in two stages:
%    first generate the structures using a constant type as the element
%        (e.g. (), or |data Label = A \vbar B \vbar C ...| for multi-sort structures)
%    use composition with a base type generator to fill in the elements
%      Test.GenCheck.Generator.Composition has composition functions
%    some polymorphic structure generators in Test.GenCheck.Generator.StructureGen:
%        lists ([]), pairs (,)
%
%enumerative generators provide a useful approach to generators
%an enumeration of a type is an ordering of the type values with (rank,index)
%   guarantees uniqueness and totality
%the Enumeration is a : count of the number of elements at each rank
%                       select function from rank,index to a value
%
%flat (rank 1) enumerations - Test.GenCheck.Generator.BaseTypeEnum module
%Haskell (Bounded, Enum) instances are rank 1 only enumerations, 
% includes Int, Char, any finite list of values
% enumBaseAll is a template for this, copy and hard code the specific type
%    (wanted to optimize Int generation, so didn't create general instance of Enum,Bounded)
% enumBaseRng to enumerate a range of values, don't need Bounded instance
% enumList enumerates finite list of values (all rank 1)
%
%ranked enumerations -   Test.GenCheck.Generator.Enumeration
%Haskell polymorphic types which are regular polynomial structures 
%  have enumerations where the rank is the number of elements (size)
%  use combinatorial theory to index regular polynomial Haskell structures
%
%structure generators composed from combinators that mirror constructor
%eConst c, eNode x, eSum e1 e2, eProd t e1 e2 (and eSum3,eProd3,etc.)
%
%       
%
%
%
%
%Enumerations are total orderings over a type.
%
%Base Type Enumerations
%are flat so all values have rank 1
%any instance of the Enum class in Haskell is enumerable, assuming it is also bounded
%Integer is not bounded, so is not enumerable in this way.
%
%Structure Enumerations
%rank is the number of elements in the structure
%value of a structure is intended to be the structure holes for each element;
%can't do ``holes'' in Haskell so generally use either |()| or a sort label identifying the element type,
%e.g. data Label = A | B | C etc.
%
               % you can include your appendix if you have any!
%

\setcounter{figure}{0}
\setcounter{equation}{0}
\setcounter{table}{0}
\chapter{Extra Material}\label{chap:tutorial}
\gordon{Extra material looking for a home}

\section{Comments about strategies}
%\section{Sampling for Property Based Testing}\label{sec:samplingstrat}

An early example of a sampling strategy for property based testing
is provided by \cite{Cartwright1981}:

\begin{quote}
One possible approach to generation test data ... 
at each predicate forcing a variable binding,
the generator makes a ``non-deterministic'' choice ... 
from a small set of heuristically generated values...
If the variable belongs to inductively defined type T, 
then the base cases and first level constructions are the obvious choices.... 
generation of test cases proceeds until all possible non-deterministic choices
have been tried (given the generator sets a small bound on the maximum recursion depth...
\end{quote}

\noindent
This suggests exhaustively testing of all possible terms up to a certain complexity,
organizing and prioritizing the smaller test cases first,
but randomizing the order across the values at each depth of the recursive structure.
This strategy is based on the heuristic that errors will, on average,
be found sooner by randomly traversing the domain instead of linearly,
emphasizing that the selection criterion is only one of the roles of the test strategy.

It is sensible to combine different sampling methods in a test,
particularly within a stratified sampling strategy.
An example of a compound sampling strategy that is suitable for property based testing
over a domain consisting of a recursive algebraic data type is :

    \begin{enumerate}
    \item exhaustively test the smallest / simplest test cases up to a certain size of structure
    \item from where exhaustive testing is infeasible
    up to the maximum complexity that will be tested:
    \begin{itemize}
    \item use uniform selection to establish coverage over all parts
    \item select additional cases to over-represent the boundary structures,
    those which represent an unusually high proportion of one choice of a disjoint union
    \item randomly select additional values of random complexity to
    avoid any possible bias introduced from the uniform and boundary selection criteria
    \end{itemize}
    \end{enumerate}

\noindent
This kind of strategy is supplied in the \GC framework as the standard \emph{test suite}
(as explained in chapter \ref{chap:source}).

\subsection{Evaluating Test Strategies}

The choice of sampling strategy for property based testing,
and software testing in general,
remains an open area of research
(\cite{ZhuHallMay1997}, \cite{Hieronsetal2009}).
These works suggest that there is not a canonical choice,
but instead suggest that a variety of sampling methods 
should be available and combined into software strategies.
Random sampling is simple to use and has very good statistical properties (e.g. unbiased),
but systematic sampling ensures a diversity of values and avoids duplicates.
Stratification allows different sampling methods to be used 
over different subsets of the domain,
and more coverage of simple cases versus more complex terms.
Purposive sampling such as static analysis of the property
or expertise in the problem domain take advantage of human experience.
The benefit of each will depend on 
the nature of the values being sampled and 
the software system under development;

The value of a strategy as evidence toward the tested hypothesis
can be evaluated independently of any individual test.
Evaluating sampling strategies is quite a subtle and tricky task.
Testing strategies must be evaluated by comparing their error detection rate,
the confidence a successful test will generate,
and the cost of developing and evaluating the test.
\cite{WeyukerEtal1991} presents various ideas for evaluating sampling strategies
without the use of statistical models:

\begin{itemize}
\item testing ``mutations'' of a correct program to ensure errors are found
(\cite{BuddEtal1980} is cited for this approach and provides a good explanation of the approach)
\item deliberate error injection
\item comparing sampled test results to exhaustive tests
\item comparing different test strategies over many different software projects
\end{itemize}

\noindent
Mutating correct programs and deliberate error injections
provide an explicit way to test the value of a specific data set
in uncovering errors in a program.
The other two evaluation techniques compare the effectiveness
of test data selection criterion over multiple test contexts to determine
their effectiveness.

There are two kinds of costs in a test:
the cost to develop the test and the cost to evaluate it.
The cost of software testing is largely in the development of the test,
while the cost of performing or repeating each test is often very low
(when compared to testing crumple zones in automobiles for example).
Reducing the cost of \emph{developing} the test cases is 
therefore the priority for sampling methods,
as opposed to the cost of evaluating them.
Automatic test case generation using sampling techniques 
allows test cases to be created inexpensively,
but loses the benefit of static analysis and domain expertise.

The confidence associated with a successful test 
is based on the assertions necessary to use the evidence gathered
in the argument for program correctness.
The stronger the hypotheses required to justify program correctness  from the test results,
the weaker the overall argument.
Different sampling strategies require different hypotheses to form such an argument,
and offer varying levels of support for these assertions.
The arguments supporting the additional assertions are subjective,
as is the interpretation of the support provided by a test,
so any arguments drawn from this aspect of the test strategy will be heuristic.

All of the sampling methods described above have their place in software testing,
and should be incorporated into any tool that will support property based testing
and automated test case generation.
Evaluating sampling strategies is an area of ongoing research
that can be supported by automated test case generation,
since many sample data sets can be generated and compared automatically.
The remainder of this chapter demonstrates how 
sampling methods and strategies can be implemented
across the types likely to be found in algebraic specifications,
with a focus on the Haskell language as a practical example.

Any test framework  should include at least 
the four \emph{standard} sampling strategies below,
but as sampling is an open area of research, 
should also allow new strategies to be easily incorporated.

\subsubsection{Exhaustive}

Exhaustive testing is not a true ``sampling'' technique because 
it includes all of the values of the population.
Exhaustive sampling, however, refers to testing all of the values
from some part of the population values.
For example, in \SC all of the recursive structures up to a certain depth were tested,
while in \FEAT an exhaustive sample includes all structures up to a given number of elements.
As noted, while these two packages produce different exhaustive test data sets,
both partition the domain of test values by a complexity measure,
and then exhaustively test all of the elements in the parts up to a set complexity.
Any property based test framework should include this kind of exhaustive sampling,
remaining agnostic to the choice of complexity measure.

Exhaustive sampling is significantly different from other strategies
in that it does not use of the order of the values in each part when
selecting the values to test, since all of the values of a part are tested.
An exhaustive test generator will still require the values of the part are organized in some way
to ensure that all of the values are selected, and only once,
and the order may impact the scheduling of the test case evaluations.
Exhaustive generators will often be programmed differently than other sampled generators
to take advantage of there not being a selection criterion,
optimizing the construction of test cases,
and reinforceing the need for a test framework to allow customize, optimized, generators
to be arbitrarily labelled as the ``exhaustive'' generator.

\subsubsection{Uniform}

Uniform sampling selects values that are equidistant from each other
in some ordering over the values, such as an enumeration or traversal algorithm.
In other words, uniform distribution takes every $ (n / k)_th $ element where
$n$ is the total number of values and $ k $ is the sample size.
It is usually used to support a uniformity argument
for complexity levels where exhaustive testing is impractical.
This is similar to random sampling,
but with a better guarantee of diversity from the population.

It is possible to introduce a \emph{bias} to the test case selection
with a uniform sampling interval over a fixed index of the type's values.
Selecting values that are precisely equidistant in an enumeration / order
makes the technique susceptible to systemic bias caused by correlations between
the enumeration / traversal order and patterns in algebraic operations over the types.
A ``slightly randomized'' or near-uniform sampling which 
would take a uniform sampling and add 
a small amount of variability to which value was selected to ensure 
no correlation between the sampling interval of the structure of the type being sampled
and the implementation's processing of that type.

The testing framework should support uniform and ``near-uniform'' sampling strategies.

\subsubsection{ Extreme / Boundary}

\gordon{References, support?}

Boundary, or extreme, values are generally recognized as having higher probability of errors.
For example, for any base type the upper and lower bounds of the values
of any bounded, enumerated type represent likely sources of error
in process because the program push the value outside of the bounds without checking.
For recursively defined algebraic data types
the extreme values are those that represent the most unbalanced set of choices,
considering the disjoint unions in the constructors as decision points,
e.g. a binary tree that is just a single left or right branch.
These unbalanced choices are likely to be special cases in
an implementation to handle recursive types,
so a possible source of error.
These are heuristics, and are provided without proof,
but it is part of the ``common sense'' of software engineering.

The test framework should support a datatype generic scheme
for generating extreme samples from as many types as possible,
and in particular from recursive algebraic types as described above.
Since the definition of a boundary or extreme value is type specific,
particularly for base types, 
it should also allow extreme samples to be 
explicitly supplied for any given type.
For example,
the extreme values of  |Double| should include negative infinity,
positive infinity, not a number, 0, the largest positive and negative numbers,
the smallest possible absolute values, etc.
User defined types might also include values that are extreme because
of the semantic interpretation of a model,
so it should be possible to include these extreme values in a customized sample.


\subsubsection{Random}

Random sampling selects a sequence of values``at random'' 
from a population based on a probability distribution assigned over the values.
Random sampling produces an infinite stream of values:
repetitions are allowed and 
there is no change in the probability of any given value being selected
regardless of how many times it has already been selected.

A pseudo-random value generator function,
such as those found in Haskell's  |Random| module,
produces a generic sequence of random values within a given interval.
based on an input ``seed'' value.
The values are pseudo-random because the generator function
is deterministic with respect to the initial input ``seed'';
the sequence for each seed simulates a random, but fixed, sequence of values.
The generator maps the pseudo-random values from the sequence
to values of the generated type through the probability distribution.
Each computation results in a test value and a new seed,
which must be passed through to the next computation.
For example, \QC pulls the seed from the IO monad environment using |newStdGen|,
and then test values are generated using Haskell's |Random| module within the |Gen| monad, 
which manages the updating random seed for the generators.

One very important principle of testing is that any test should be repeatable.
Truly ``random'' sampling would not allow a test to be repeated exactly,
unless the test cases were stored and reused.
Any random sampling facility provided by a property based test framework
should be deterministic, based on an initial seed that can be stored with the test results.
Fortunately, Haskell's random generators (and random generators in general) are not actually random,
so beginning from the same seed value should provide the same sequence of pseudo-random test cases.
A ``random'' starting seed could be provided as an optional part of an interface,
to initiated the random sampling,
providing a choice between deterministic and random testing.

Two different approaches to random generation were used in the reviewed packages.
\QC approaches the problem by treating 
each disjoint union in a type constructor as a decision point,
randomly selecting a constructor and then randomly populating it,
``building'' the structure one constructor at a time.
The constructors in a disjoint union are given equal or specified weighting
to alter the probability distribution of the choices.
The probability distribution is difficult to control this way
and does not guarantee termination unless
the generator is explicitly keeping track of the number of recursions,
so this approach requires a significant planning effort by the test developer.
However, the constructive approach is easy to implement if the exact distribution is not important and
is quite efficient for very large values since the cost of each decision is constant (i.e. $O(n)$).

The other approach to random sampling is to create an \emph{enumeration}
over the values of the type being generated,
and use that information to create a uniform probability distribution.
This approach was explored extensively in \cite{FlajoletZC94,FlSa95}
and provides a much more reliable distribution of values for a given size.
This approach is distinctly more expensive, ( $O (n log n)$),
so for larger structures may not be the best choice.
This approach is used in \FEAT and the \GC packages,
both released in June, 2012.

\gordon{reference and briefly discuss Boltzmann algorithm for large random value generation}

Any test framework should support both enumerative and constructive styles of random generation.
In addition, different random selection strategies should be allowed by 
the test framework to support research and
allow the most appropriate implementation to be selected for any given situation.
The sampling strategy should receive a the generating seed as an input,
so the sample can be repeated later or given a random seed.


\subsection{Static Analysis of Properties and Implementations}

Not all test case selection uses sampling strategies.
For example, \HOLTG does \emph{not} use a sampling strategy -
it partitions the specification into subdomains, or equivalence classes,
and derives a representative for each part.
This is a \emph{complementary} strategy to sampling.

It would be interesting to consider a hybrid approach which
derived the partitions based on the specification,
but then used (over)sampling strategies within those regions to
test the assertion that the subdomains are the equivalence classes of
the implementation, while also providing greater confidence in the test.
In any case, a test framework should allow test cases to be generated
externally to the test package and incorporated into the test data sets.



\bibliographystyle{natbib}
\bibliography{test_ref,species-refs,genericprog-refs}      % your list of references

\label{NumDocumentPages}

\end{document}
