%\section{Enumerating Haskell Algebraic Data Types}

Haskell algebraic data types can be modelled by combinatorial constructions,
The complexity of the term is the number of terminal nodes in the Haskell term
(i.e. the combinataorial complexity, section \ref{complexgen}).
If the construction is admissible,
then there are a finite number of terms of that type for any given complexity,
and the type can be enumerated.
The approach is straight-forward,
requiring only that:
\begin{itemize}
\item Haskell data constructors with multiple arguments
be treated as combinatorial products,
\item type constructors with multiple disjoint unions be evaluated pairwise, 
\item and nullary constructors (constants) be treated as an atom over a distinct $1$ element sort.
\end{itemize}

\begin{tabular}[b]{llll}\label{countseq}
Haskell Type Operator & Construction & Complexity & Comment\\
constant & $\spec{X}_{a}$ & $1$ &atom over sort of single element $a$ \\
singleton & $\spec{X}_{T}$ & $\setcard{T}$ & single element of type $T$ \\
$T_1 \mid \dots \mid T_n $ & $\spec{T}_{1} \splus \dots \splus \spec{T}_{n}$ & $\sum_{k=1}^{n} $ & sum of disjoint terms \\
$C\ T_1\ T_2 $ & $ \spec{T}_{1} \sprod \spec{T}_{2}$ & $\setcard{\spec{T}_{1}} \sprod\setcard{\spec{T}_{2}} $ &  
                   product of constructions for each argument  \\
$C\ T_1\ \dots\ T_n $ & $ \spec{T}_{1} \sprod \spec{S}$ & $\setcard{\spec{T}_{1}} \sprod\setcard{\spec{S}} $ &  
                   where  $\spec{S} = \spec{T}_{2}\sprod \dots \sprod \spec{T}_{n}$ for $n > 2$ , $\spec{S} = \spec{T}_{2}$ for $n = 2$
\end{tabular}

\noindent
This formulation guarantees that all constructions, 
including nullary constants,
are of size at least one,
and therefore satisfies the ``epsilon-freeness'' constraint for composing constructions.
The consequence of this weighting of type constants 
is that certain data types will have a non-intuitive complexity, or size.
For example,
an empty list will be of size $1$, not $0$,
as might be expected at first glance.
Evaluating a property over an empty list, however,
does entail at least one operation,
and so this is a reasonable assessment of the complexity
with respect to generating test cases.

The alternative approach of allowing constants to be of size 0
allows certain valid Haskell types to be inadmissible constructions,
i.e. to have infinitely many constructions of the same size.
Consider this somewhat unusual but allowable definition
of a binary tree with either a null or element bearing leaf node:

\begin{lstlisting}
data T a = L | N a | Br (T a) (T a)

t1 = Br L (N 1)
t2 = Br L (Br L (N 1))
t3 = Br L (Br (Br L L) (Br L (N 1))
t4 = Br (Br (Br L L) (Br L L)) ((Br L (N 1)) (Br L L))
\end{lstlisting}

The complexity of each of these trees would be $1$ if
the nullary leaf constructor is counted as weight $0$,
instead of $ 2, 3, 5, 8$ respectively when counted as $1$.
Although unusual it is still a valid Haskell type and therefore must be handled by
any \pbt system generating test cases automatically.
Any confusion over the size of containers with no elements
is outweighed by the ability to correctly generate all valid Haskell algebraic data types.