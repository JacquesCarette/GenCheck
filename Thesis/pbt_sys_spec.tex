
The specification consists of interfaces isolating each of these stages
and a suggested library of components that should be provided as a base implementation.
In Haskell interfaces can be implemented as ad hoc classes,
with each new variant an instance, 
or by defining data types and allowing values of that type to be defined.
A high level description of the mandatory interfaces and 
their interactions during the testing process are addressed in this chapter.


\subsection{generators}
Test generators exhibit a lot of the structure of the data they generate,
so are suitable for generic programming techniques.
Generators should be constructed using combinators for 
disjoint unions (sums) and products of base types,
mirroring the Haskell type constructor algebra.
The generator combinators provided by \QC, \SC, etc.
are a good example of how generator combinators can be used
to develop generators for algebraic data types in Haskell.

