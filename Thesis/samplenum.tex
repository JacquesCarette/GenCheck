%section{Substitution in Enumeration Generators}

The study of combinatorial constructions is generally concerned
about the number of ``shapes'' satisfying a specification,
with the shape of the construct consisting of products of atoms and epsilon values.
A test case generator, however,
must produce concrete terms that result from substituting values for the atoms,
i.e. treating the atoms as singleton constructors encapsulating a value of the appropriate sort.
There are two choices for dealing with each of the atoms in a specification 
to construct a term generator from the enumeration:

\begin{enumerate}
\item Extend the construct's specification, and hence enumeration, 
to include all allowable value substitutions from the underlying sort ,
replacing the atom with a disjoint union of each of those values.
The counting, index and selector functions naturally extended
by replacing the counts for the atoms with a totally ordered set of the substituting values,
in the same way the ordinary generating functions are extended,
as discussed in section \ref{sec:combstruct}.
\item Allow the generator to sample the enumerated \emph{shapes} of the constructions,
the unpopulated constructs with the atoms as placeholders.
These will be called \emph{structures} to be distinct from concrete terms.
This generator will subsequently be combined with a generator for values of the appropriate type
to produce the concrete terms for use in testing.
This kind of generator is called a substitution generator.
\end{enumerate}

\noindent
This choice can be made independently for each of the atoms in the specification,
so the resulting enumeration can be over concrete terms,
or structures with some or all of the distinct atom types 
that must later be substituted.

The choice of expanding an enumeration to include the values
or post-sampling substitution is pragmatic, 
depending on the nature of the underlying sorts,
and the significance of the data structures relative to the actual values in the test cases generated.
In particular,
if an atom represents a small number of allowable constants,
such as a Boolean value,
it may be preferable to include that substitution in the enumeration.
If there are restrictions on nature of the substitution,
such as if the values are selected without duplicates (replacement)
or there are conditions on the elements of the substitution set,
it may be easier to enumerate the them separately.
On the other hand, it would not be desirable to include all of the values of a large Int class
in the enumeration when generating test cases for a sort algorithm,
as the actual values in each node make little impact on the test case.
Note that the substitution approach may also be used if the values for the substitutions
are themselve structures by ignoring their rank in the enumeration.
For example, if a sort algorithm is to be tested by comparing strings,
the length of the string may be ignored in the enumeration and 
strings of different length selected as part of the value sampling methodology.
Both approaches result in a sample of the constructions
populated with the elements of the underlying set (or type),
but the properties of the test data seta
and the sampling methods used may be quite different.

\subsection{Substitution Generators and Strategies}

The \GC implementation of substitution generators is provided here
as an illustration of how substitution combinators can be used
to provide more options for test case generation.
The |Structure| class defines a |substitution| method, the
mechanical details of replacing the atoms in the structure with a single sort of elements.
Classes for two sorted structures, etc. are simple mechanical extensions of the single sort substitution.
Instances of Structure can be generated mechanically from the data constructor
definitions via Template Haskell.

The subst, substn and substAll functions below are generator combinators,
creating substitution generators from structure and element generators.
The ordered collection of values drawn from the element generator for insertion into a structure
is referred to as a substitution tuple.

\begin{description}
\item[subst] populates each generated structure with one generated substitution tuple
\item[substN] populates each generated structure with n substitution tuples
\item[substAll] populates each structure with every substitution tuple
        until exhausting the element generator (non-terminating for infinite generators)
\end{description}


\begin{lstlisting}
class Structure c where
  substitute    :: c a -> [b] -> (Maybe (c b), [b])

subst :: Structure c => Generator (c a) -> Generator b -> Generator (c b)
subst gfx gy r =  
  let fxs = (gfx r)
      ys  = gy 1
  in gsub fxs ys

gsub :: Structure c => [c a] -> [b] -> [c b]
gsub [] _       = []
gsub (fx:fxs) ys = 
   let (mfy, ys') = substitute fx ys
   in case mfy of
        Nothing -> []
        Just fy -> fy : (gsub fxs ys')

substN :: Structure c => Int -> Generator (c a) -> Generator b -> Generator (c b)
substN n gfx gy r = gsubN n n (gfx r) (gy 1)

gsubN :: Structure c => Int -> Int -> [c a] -> [b] -> [c b]
gsubN n _ _  _ | n < 1 = []
gsubN _ 0 _  _         = []
gsubN _ _ [] _         = []
gsubN n k fxs@(fx:fxss) ys = 
   let (mfy, ys') = substitute fx ys
   in if k > 1 then maybe [] (\fy -> fy : (gsubN n (k-1) fxs ys')) mfy
               else maybe [] (\fy -> fy : (gsubN n n fxss ys')) mfy 

substAll :: Structure c => Generator (c a) -> Generator b -> Generator (c b)
substAll gfx gy r = 
  let ys = gy 1
      fxs = gfx r
  in gsubAll fxs ys

gsubAll :: Structure c => [c a] -> [b] -> [c b]
gsubAll [] _ = []
gsubAll l@(fx:fxs) ys = 
   let (mfy, ys') = substitute fx ys  -- subst elements into first structure
       -- sub same ys into remaining list of structures, guaranteed finite
       fys = catMaybes $  map (fst.((flip substitute) ys)) fxs
   in maybe [] (\x -> x:(fys ++ (gsubAll l ys'))) mfy
\end{lstlisting}

The value of combinatorial complexity as the rank of a structure
is most apparent in a substitution generator.
For a structure with a single class of atoms,
the number of values required to complete the substitution 
is the rank of the structure.
Where there are multiple classes of atoms,
the rank will be composed of the number of required elements of each sort;
any elements that were included in an expanded enumeration
will also be part of this rank but will not require substitution.
In either case, the substitution will be a function 
providing a concrete term from a structure
and an ordered tuple of values of each sort.

Substitution of values for atoms introduces an new kind of choice 
as part of the generator sampling strategy.
The simplest approach to substitution is to 
replace the atoms in each structure with generated values
to replace the instances of the atoms, creating a single term for each,
the approach described as ``step-by-step'' in section \ref{{adtgen}}.
Other alternatives are:
\begin{itemize}
\item populate each structure with a fixed number of substitution tuples;
\item generate a finite number of both structures and substitution tuples, 
and populate all structures with all tuples;
\item populate each structure with all permutations of the substitution tuple
(this is particularly useful when testing symmetrical and anti-symmetrical properties).
\end{itemize}
The choice of substitution strategy will be based on 
the goal of the test,
the number of possible substitution sets and
the cost of generating the structures.
Note that substitution strategies are not specific to enumerative generators,
and can be used for any substitution generator.

%\subsection {Enumerative Test Strategy}
%The availability of datatype generic sampling methods over enumerations
%can be extended to produce entire \emph{test strategies} that are independent of the underlying data types.
%A test strategy is the implementation of one or more sampling methods,
%including stratified sampling methods and substitution strategies for ranked constructs,
%that results in an executable test data set for a property based test.
%An example of a compound sampling strategy that is suitable for property based testing
%over a domain consisting of a general tree of integer values is :
%
%    \begin{enumerate}
%    \item exhaustively select smallest / simplest test cases up to a certain size,
%    uniformly sampling 10 integer values including the boundary values (largest positive and negative values)
%    \item from where exhaustive testing is infeasible
%    up to the maximum complexity that will be well tested:
%    \begin{itemize}
%    \item use uniform selection to establish coverage over all parts
%    \item select the boundary structures, namely the empty tree and slngle leaf trees
%    \item randomly select additional values to
%    avoid any possible bias introduced from the uniform and boundary selection criteria
%    \end{itemize}
%     substituting an arbitrary ordering of 0, the boundary values, and random integer values for each tree
%    \item randomly select a small number of values from random ranks up to some arbitrary maximum complexity,
%    using the same integer value substitution strategy
%    \end{enumerate}
%
%This particular strategy depends only on the exhaustive and well-tested complexity sizes
%and a fairly generic integer test value generator,
%and could be used for any Haskell algebraic data type.
%The appropriateness of the strategy must be determined,
%but it is distinctly superior to any single sampling methodology applied to the entire population of general integer trees.
%A library of such datatype generic test strategies can be provided
%to leverage the availability of complexity ranked enumerations
%and substitution generators.



