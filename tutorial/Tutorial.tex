\section{Tutorial 1: The simplest case.}

\input {reverse/TestReverseList.lhs}

The examples in this module have properties over standard Haskell types, where
the default generators were already available.  How are test cases generated
when the module contains a new type?  How are the test cases chosen for
inclusion in the test suite, and how can that be controlled?  Surely there is
a better way to display the results?  These questions and more are answered
in the next part of the tutorial.  

\section{Tutorial 2: Testing with New Data Types}

The first tutorial covered testing properties over standard Haskell types.
Default test case generators were already supplied by GenCheck
as instances of the Testable class, but what about new data types?
In this tutorial, we explain how to use the type constructor definition and 
GenCheck enumeration combinators to construct the standard test generators
based on the type constructor definitions.

There are three steps in testing a module against 
a specification using the GenCheck framework:

\begin{enumerate}
\item define and code the properties that make up the specification
\item select the test system that provides the most appropriate
test scheduling and reporting schema for the current phase of the development
\item define the test suite for those properties,
including the test case generators for each.
\end{enumerate}

In the previous section the details of the test system
and test suite generation were hidden by the SimpleCheck API.
In this section, we look at each of these steps in detail 
and discuss how to generate new data types and customize test suites
for a particular set of properties.

For this tutorial, we'll look at a list zipper implementation,
taken from the ListZipper package (1.2.0.2).  Only snippets
of the code will be provided as the package is rather large
(see tutorial/list\_zipper/ListZipper.hs  for the module source).
The Zipper data type is

\begin{verbatim}

data Zipper a = Zip ![a] ![a] deriving (Eq,Show)

\end{verbatim}

where the ``current'' element or ``cursor'' is the head of the second list.
The specification for ListZipper includes empty, cursor, start, end, left, right, foldrz, foldlz, 
and many more functions.  The example below shows how to test (some)
of the specification for this module, with a focus on the generators for the new type.

\subsection{Specifications and Properties}
A GenCheck specification is the collection of properties
that any implementation of the specification must satisfy.
Properties are univariate Boolean valued Haskell functions;
this is not a restriction as the input value may be of an arbitrarily complicated type 
to support uncurried predicates. The GenCheck test programs accept 
a single property and test it over the defined suite of test cases.
Only one property is tested per call because different properties 
may have different input types or require different test suites
The specification module would normally be separate from the implementation,
but may include functions that test the properties using GenCheck.

\subsubsection{List Zipper Properties}

\input {list_zipper/PropListZip.lhs}

\subsection{Generators and Enumerations}

A GenCheck generator is any function from a rank (positive integer) to a list of values.
Generators may be finite or infinite, may include duplicates of the same value,
and may be missing values from the total population of the type.  Generators
can be manually coded or built from a GenCheck enumeration of the type using
a sampling strategy called an ``enumerative strategy''.  These strategies can be
applied to any data type that can be enumerated, i.e. ordered, and indexed by a finite rank.

One way to build generators is to start with an enumeration of the type to be generated.
An enumeration of the type is a total order over the type's values.  For base types,
any list of values is a de facto (base) enumeration, as is any type that is an instance
of Haskell's Enum and Bounded classes.  Regular recursive algebraic data types
are enumerated by first partitioning the values by size (i.e. the number of elements)
and then indexing each partition.  Base and structure type enumerations are
distinguished in GenCheck as the structure enumerations use the rank, or size,
as the first part of the index, but base type enumerations have no rank.
A collection of common base type enumerations is supplied in the |Generator.BaseEnum|.
|Generator.Enumeration| contains a collection of enumeration combinators
that mirror the sum and product construction of the Haskell algebraic data types,
allowing structure enumerations to be created mechanically.

An enumerative strategy is a type independent approach to sampling
the values of an enumeration to create a list of values for a generator.
These strategies are implemented as lists of |Integer| indices;
the selector function from the enumeration is mapped across the strategy,
at a specified rank, to get the ordered values to be returned by the generator.
For example, |EnumStrat.uniform| takes every $k^th$ value from an enumeration,
so the strategy for | uniform 5 ->  [1, 6, 11, 16, ...] |, and if this were applied to 
an enumeration of | Char | would produce:

\begin{code}
genUni5LowChar = enumGenerator (uniform 5) enumLowChar 
-> ['a', 'f', 'k', ...]
\end{code}

The so called standard generators are just four different sampling strategies
that can be applied to an enumeration of the generated type.  These are:

\begin{description}
\item[exhaustive] {Produces all of the values of the specified rank in a sequence,}
\item[extreme]{Alternating lowest and highest values in the enumeration,}
\item[uniform]{Select the specified number elements at uniform intervals through the enumeration,}
\item[random]{Use the random generator to pick an infinite list of values.}
\end{description}

\subsubsection{List Zipper Enumeration and Generators}

\input {list_zipper/ListZipper_GC.lhs}
	

\subsection{Test Suites and Standard Generators}
The module System.TestSuite contains functions that assemble test suites 
from one or more test case generators. These suite builders take a collection of 
``standard'' generators and a set of ``generator instructions'', a list of rank, count pairs.  
The allocation of test cases to ranks through these instructions is called a ``test strategy'' 
and can be used independently of the type of data being generated, 
assuming that the standard generators are available for that type.

The basis for the test suites used by SimpleCheck test programs
are the standard generator set described above.
These are not the only valuable sampling strategies that could be used, but these were
selected as the ``standard'' generator strategies to be used by the GenCheck API.
A type is an instance of Testable if a set of standard generators is available for it;
these generators can be constructed for any type that has an Enumeration associated with it. 

A test suite is a collection of test cases selected from the ranks of one or more generators.
Each test suite uses a sampling ``strategy'' to decide which generators and
at what rank to select the test cases; this mix is called the test strategy.  
The SimpleCheck test strategies draw test cases from 
the four standard generator strategies to provide a good sampling 
of the test cases relative to the number of tests performed.
The strategies are heuristic of course, and the optimal composition of a test strategy
depends on the nature of the structures and properties  being tested,
so these provided test suite strategies are really just a starting point.
The TestSuite module contains the test suite building functions:

\begin{description}
\item[stdSuite] exhaustive testing for smaller structures, then the boundary elements
and a mix of uniformly and randomly selected values to the specified maximum rank 
\item[deepSuite] similar to standard but for ``narrow'' structures with fewer elements per rank
(i.e. a binary tree or list), so the exhaustive testing goes to a much higher rank
\item[baseSuite] a test suite for base types, where the rank is always 1
\end{description}

The System.SimpleCheck module test suites are test cases stored in 
a Haskell map (as in Data.Map) indexed by rank, called a MapRankSuite.


\subsection{The Testing Process}
Typically a GenCheck user would start with the default test suites
and only reporting failures during the development cycle.
Nearing completion, s/he would switch to the test programs 
that provide more control over the test suite generation,
and report all of the test cases in the results to ensure good coverage.


\subsubsection{Testing the List Zipper Module}
\input {list_zipper/TestListZipper.lhs}

\subsection{Generating Simple Structures with Significant Elements}

Generating test cases is generally decomposed into two phases: 
generating structures, and then populating them with base type elements.
For some properties, such as the ListZipper example above,
the structure of the type is the most significant aspect of the values.
However, in other situations, the values of the elements in the structure
are very important to the properties being tested.
Of course, some properties just require a single base type, 
but if there are even two base type inputs to the property
the test cases are at least a product of the two.
GenCheck provides a rich API to control the generation of both 
the structural and content aspects of the test cases.

The pair tuple is an example of a type where the content is more important
than the structure.  There is only one ``shape'' of a pair, namely $(A, B)$,
so the structure generator consists of a single value at rank 2 and
A and B are labels that represent the sorts of the elements of the pair.
The pair is the simplest instance of the Structure2 class, 
which identifies the substitution2 method for populating pairs from two generators,
one of each type of elements for the pair.  The implementation of the tuple
in | Generator.StructureGens | is shown below:

\begin{code}
instance Structure2 (,) where
  substitute2 _ (x:xs) (y:ys) = (Just (x,y), xs, ys)
  substitute2 _ xs ys = (Nothing, xs, ys)
  
genTpl :: Generator (Label, Label)
genTpl r | r == 2    = [(A,B)]
genTpl _ | otherwise = []

genABTpl = subst2N
\end{code}

The tuple template is populated by substituting values for the Labels
using the substitution combinators from the | Generator.Substitution | module.
These combinators provide several different strategies to combine the two sorts of elements. 
An example the two is given here, where :

\begin{description}
\item[genABTpl\_20x20] { is a generator of up to 400 elements taken as 
the Cartesian product of the first 20 elements of each generator, or as many exist,}
\item[genABTpl\_All] {assuming genB at least is finite, provides all of the pairs in 
A dominant order}
\end{description}

\begin{code}
genABTpl_20x20 = subst2N 20 genA genB

genABTpl_All = subst2All genA genB
\end{code}

The choice and result of the substitution strategies are clearly impacted by
the nature of the element generators, specifically whether they are finite
and the order they impose upon their values.  

Note that the elements substituted are drawn from rank 1 of the generator.
This could be because they are base types or through a generator combinator
that flattens the ranks of the values arbitrarily to 1; either way they are treated
as ``flat'' or rank-less values.  If the rank of the elements was relevant,
composition combinators from the |Generator.Composition| module should be used
instead of substitution.  In principle, composed structures may still receive substituted elements
but substituted structures are complete.

\subsubsection {Decimal Example}

\input{decimal/Decimal_GC.lhs}

%\subsubsection{Enumerations in Detail}
%
%Haskell structure types can be recursively defined and therefore arbitrarily large,
%even infinitely large with lazy evaluation.  To allow any sort of ordering of these infinite sets,
%the values are first partitioned into finite subsets by the number of elements in the structure, 
%which is called the rank of the structure.  The common sized elements in the subset 
%are then ordered and indexed using a depth first, sum of products view of the values.
%This allows most (any regularly recursive polynomial) structure type to be enumerated
%using a two tiered (rank, integer index) pair.  Structure types that are enumerated are 
%instances of the Enumerated class, which provides methods to count and select (construct)
%values of the specified rank and index.
%
%A base type, such as Int or Char, do not have a rank and are enumerated if there is a
%single ordering over all of the elements of that type.  This is called a base enumeration (BaseEnum)
%and is effectively the same as the Haskell Enum class; the GenCheck class is called EnumGC 
%to highlight that it is an extension of Enum.
%
%A generator may be constructed from an enumeration by mapping the selector method
%over a sorted list of integer indices.  The GenCheck standard generators are created
%by applying the four standard strategies - exhaustive, boundary (extreme), uniform and random -
%over an enumeration of the type.  These sampling strategies abstract the test plan from
%the concrete construction of particular instances of the test values, leaving the details
%of the specified type being tested to the enumeration methods.  
%
%Enumerations for standard Haskell data types such as list and tuples are provided.
%For new data types, enumerations can be easily made using the enumeration combinators 
%|eConst, eNode, eSum, eProd, etc.| in the Generator.Enumeration module.
%These combinators simply mirror the structure of the type constructors
%to produce the required enumerations, and can be constructed mechanically.
%
%A pure structure enumeration / generator constructs structures that are over a constant value
%(i.e. unit), so some method is required for substituting other values into the data nodes.
%One approach is to define the structure over a particular kind of element,
%and have the enumeration count all of those element types as distinct values of the structure.
%From a testing perspective, this creates an unfortunate imbalance in the number of values
%for any given size of structure vs. the relevance of the values in the structures' nodes.
%For example, when testing a heap sort insert, it would be nice to exhaustively test
%all \emph{shapes} of heaps up to a certain size, but not necessarily exhaustively test
%all possible integer node values in each of those heap shapes.  A better solution 
%is to create a structure generator and sample it using the appropriate strategies,
%and then populate each of those shapes with one or more sets of values, 
%also chosen from an appropriately sampling generator.  GenCheck supports this
%kind of substitution at the generator level; a generator of structures $Foo a$ is
%created by applying one of the substitution functions over a $Foo$ generator and an $a$ generator.
%A collection of substitution functions is available in the Generator.Substitution module,
%and provides the ability to populate each structure with one or more sets of elements,
%using different strategies such as all permutations, all combinations, or just unique elements.
%Substitution allows a great deal of control over the generation of complex data structures
%by decoupling the sampling strategy of the structures, the elements, and their substitution;
%this is one of the most important benefits of GenCheck over QuickCheck and SmallCheck.
%
%Generators are created from enumerations and enumerative strategies
%using Generator.enumGenerator for structure types and BaseGen.baseEnumGen for base types.
%The standard generator set, and therefore the Testable instance, is built using stdEnumGens.
%

%\section{Tutorial 3: Customizing the Test Programs}
%%
% Putting this on hold so Jacques can change the class definitions
%
%The objective of GenCheck is to provide a testing framework that
%scales in scope from a very simple QuickCheck like interface,
%through progressively more thorough test suites and reporting,
%during the course of the development towards production.
%The highly modular GenCheck API allows an arbitrarily deep level of control,
%with a commensurate level of complexity,
%supporting customization and new development as required.
%
%\subsection{Generalizing test data: Datum , Verdict, and LabelledPartition Classes}
%
%\begin{itemize}
%\item decouples test data, test result, meta data and storage of test suite / results
%\item define a structure for carrying data value as instance of Datum
%\item define a structure for carrying individual result as instance of Verdict
%\item define a structure for holding the collection of data and results as
%an instance of LabelledPartition
%\item arbitrary meta data about the test case or the evaluation of the result
%can then be carried through the GenCheck system
%\end{itemize}
%
%\subsubsection{Labelled Partitions}
%The test suites are test cases in the earlier tutorials are stored in
%a Haskell Map (as in Data.Map) indexed (or ``labelled'') by rank (called a MapRankSuite).
%This indexing mirrors the way the test cases are generated, namely by structure size,
%and is useful in unit testing structure manipulation functions such as reverse and fold,
%where the size of the structure is highly relevant, and is the only information 
%guaranteed to be available in general.
%
%Storing test results this way may or may not be meaningful in other situations,
%in particular where the values of an element play a key role in defining the outcome
%of the functions being tested.  In these cases, the tester will want to modify
%the partitioning to group (and label) test cases by some other useful value.
%The GenCheck framework allows test suite containers to be customized using
%a class called LabelledPartition that abstracts the interface required by the test programs.
%For example, a test over $Char$ might be grouped by the labels $ASCII$,
%$CONTROL$ and $Extended$ to help focus on unusual characters in a printing scheme.
%Multiple levels of partitioning can also be used, by embedding labelled partitions
%inside of labelled partitions, to allow a more granular level of test case analysis.
%
%The abstract structure of a labelled partition can also be leveraged to
%optimize the evaluation of tests by the test scheduling and evaluation.
%Testing is an excellent target for parallel execution, and very large tests
%might be carried out on multiple processors or even different systems.
%The structure of the labelled partition can be modified to support decentralized 
%test systems.
%
%\subsection{Generalizing Test Evaluation and Reporting}
%
%\begin{itemize}
%\item tests can be pure or monadic computations
%\item tests can be serialized or parallel
%\item might want to stop after some number of failures,
%or complete entire test suite
%\item the same test suite can be used in different modes
%\end{itemize}
%
%\subsection{The SimpleCheck System}
%The System.SimpleCheck module contains a simple set of test programs,
%each composed of a test suite building strategy and data structure, 
%an execution strategy, and a reporting strategy.  
%
%
